\documentclass[preprint]{sigplanconf}

% The following \documentclass options may be useful:

% preprint      Remove this option only once the paper is in final form.
% 10pt          To set in 10-point type instead of 9-point.
% 11pt          To set in 11-point type instead of 9-point.
% numbers       To obtain numeric citation style instead of author/year.

\usepackage{datafun}
%% \renewcommand{\todo}[1]{}


%% ---------- Setup ----------
\begin{document}

\special{papersize=8.5in,11in}
\setlength{\pdfpageheight}{\paperheight}
\setlength{\pdfpagewidth}{\paperwidth}

\conferenceinfo{ICFP '16}{18--24 September, 2016, Nara, Nara, Japan}
\copyrightyear{2016}
\copyrightdata{978-1-nnnn-nnnn-n/yy/mm}
\copyrightdoi{nnnnnnn.nnnnnnn}

% Uncomment the publication rights you want to use.
%\publicationrights{transferred}
%\publicationrights{licensed}     % this is the default
%\publicationrights{author-pays}


%% ---------- The title ----------
% These are ignored unless 'preprint' option specified.
\titlebanner{preprint}
\preprintfooter{Datafun: a Functional Datalog (PREPRINT)}

\title{Datafun: a Functional Datalog}
\subtitle{}

%% TODO: get my bham email to work?
\authorinfo{Michael Arntzenius\and Neelakantan R. Krishnaswami}
           {University of Birmingham}
           {daekharel@gmail.com, N.Krishnaswami@cs.bham.ac.uk}

\maketitle


%% ---------- The abstract ----------
\begin{abstract}
  Datalog may be considered either an unusually powerful query language or a
  carefully limited logic programming language. Datafun is declarative,
  expressive, and optimizable, and has been applied successfully in a wide
  variety of problem domains. However, most use-cases require extending Datalog
  in an application-specific manner. In this paper we define Datafun, an
  analogue of Datalog supporting higher-order functional programming. The key
  idea is to \emph{track monotonicity via types}.

  \todo{Explain why functional language may not need application-specific
    extensions?}
\end{abstract}

\todo{categories? keywords?}

% \category{CR-number}{subcategory}{third-level}
%
% % general terms are not compulsory anymore,
% % you may leave them out
% \terms
% term1, term2
%
% \keywords
% keyword1, keyword2


%% ---------- Paper body ----------
%% Section 1: Introduction
\input{introduction}

%% Section 2: Datafun, informally
%% FIGURE: CORE SYNTAX
\begin{figure}
  \[\begin{array}{ccl}
    %% types
    A, B     &\bnfeq& \bool \pipe \N \pipe \str \pipe \Set{A}
    \pipe A + B \pipe A \x B\\
    \textsf{types} && A \uto B \pipe A \mto B
    \vspace{0.5em}\\
    %% semilattice types
    L, M         &\bnfeq& \bool \pipe \N \pipe \Set{A}
    \pipe L \x M \pipe A \uto L \pipe A \mto L\\
    \textsf{semilattice types}
    \vspace{0.5em}\\
    %% equality types
    \eq{A}, \eq{B} &\bnfeq& \bool \pipe \N \pipe \str \pipe \Set{\eq{A}}
    \pipe \eq{A} + \eq{B} \pipe \eq{A} \x \eq{B}\\
    \textsf{eqtypes} &&
    \vspace{0.5em}\\
    %% finite equality types
    \fineq{A},\fineq{B}
    &\bnfeq& \bool \pipe \Set{\fineq{A}}
             \pipe \fineq{A} + \fineq{B} \pipe \fineq{A} \x \fineq{B}\\
    \textsf{finite eqtypes}
    \vspace{0.5em}\\
    %% contexts
    \GD &\bnfeq& \cdot \pipe \GD, x\of A\\
    \GG &\bnfeq& \cdot \pipe \GG{},\m{x}\of A\\
    \textsf{contexts}
    \vspace{0.5em}\\
    %% expressions
    e &\bnfeq& x \pipe \m{x} %% \pipe n \pipe s
    \pipe \fn\bind{x} e \pipe \fn\bind{\m{x}} e
    \pipe e\;e\\
    \textsf{terms}
    && (e,e) \pipe \pi_1\;e \pipe \pi_2\;e
    \pipe \ms{in}_1\;e \pipe \ms{in}_2\;e\\
    && \case{e}{x}{e}{x}{e}\\
    && \case{e}{\m{x}}{e}{\m{x}}{e}\\
    && \ms{true} \pipe \ms{false} \pipe \ifthen{e}{e}{e}\\
    && \singleset{e} \pipe \unit \pipe e \vee e \pipe \letin{x}{e}{e}\\
    && \fix{\m{x}}{e} \pipe \fixle{\m{x}}{e}{e}
    %% \vspace{0.5em}\\
    %% x, \m{x} && \text{variables}\\
    %% n && \text{numerals}\\
    %% s && \text{string literals}
  \end{array}\]

  %% \todo{Remove $\N$ from semilattice types \& give it the discrete order? We
  %%   never use it as a semilattice type.}

  \caption{Syntax of core Datafun}
  \label{fig:syntax}
\end{figure}


\section{Datafun, informally}
\label{sec:informally}

We give the core syntax of Datafun in Figure \ref{fig:syntax}. Datafun is a
simply-typed $\lambda$-calculus extended in four major ways:

\begin{enumerate}
\item We add a type of finite sets, $\Set{A}$.

  %% \todo{Describe why sets are useful?}

  %% We use finite sets to represent Datalog predicates; one might also think of
  %% them as tables or views in a database setting.

\item We add a type of \emph{monotone functions}, $A \mto B$. Consequently
  Datafun has two flavors of variable, \emph{ordinary} and \emph{monotone}. We
  write ordinary variables in $script$ and monotone variables in \m{bold}.

  In order for ``monotone'' to have meaning, our types are implicitly partially
  ordered:
  \begin{itemize}
  \item Booleans $\bool$ are ordered $\ms{false} < \ms{true}$.
  \item Natural numbers $\N$ have the usual order: $0 < 1 < 2 < ...$.
  \item We have no particular use-case for comparing strings $\str$ in
    this paper, so we order them discretely; $a \le b$ iff $a = b$.
    %% \todo{Better explanation?}
  \item Pairs and functions are ordered pointwise:
    \begin{itemize}
    \item $(a, x) \le (b, y)$ iff $a \le b \wedge x \le y$
    \item $f \le g$ iff $\forall \bind{x} f(x) \le g(x)$
    \end{itemize}
  \item Sum types are ordered disjointly: $\ms{in}_i\; a \le
    \ms{in}_i\; b$ iff $a \le b$, but $\ms{in}_1\; a$ and $\ms{in}_2\; b$ are
    never comparable.
  \item Sets are ordered by inclusion: $a \le b$ iff $a \subseteq b$.
  \end{itemize}

  %% TODO: explain why fixed points are useful?
\item We add a term $(\fix{\m{x}}{e})$ denoting the least fixed point of the
  monotone function $(\fn\bind{\m{x}} e)$. This is computed (modulo
  optimizations) by iteration, starting from the smallest value of the desired
  type and halting once a fixed point is found. This strategy constrains the
  types of \ms{fix} terms in several ways:
  \begin{itemize}
  \item The type must have a smallest value. We enforce this using semilattice
    types (see item \ref{item:semilattice-types}, below).

  \item The type must support equality tests, to determine when a fixed point
    has been reached. We call a type supporting equality tests an \emph{eqtype}.
    \todo{CITE}

  \item To ensure termination, the type must have finite height.\footnote{The
    height of a poset is the cardinality of its largest chain (totally-ordered
    subset).} We conservatively approximate this property by limiting \ms{fix}
    to finite types.
  \end{itemize}

  In summary, \ms{fix} may only be used at \emph{finite semilattice eqtypes}.

  %% \todo{TODO: connection to Datalog via finiteness of predicates}

  %% \todo{explain $\fixle{\m{x}}{e_1}{e_2}$?}

\item\label{item:semilattice-types} Generalizing the empty set $\emptyset$ and
  union $\cup$, we identify a subset of types that have a \emph{least element}
  $\unit$ and a \emph{least upper bound} operator $\vee$. We call these
  \emph{semilattice types}\footnote{Technically, the partial orderings on these
    types form \emph{join-semilattices with a least element}. For brevity's
    sake, we call these structures simply ``semilattices.''}, and denote them by
  the metavariables $L,M$.

  Semilattice types serve two purposes. First, as already mentioned, they
  guarantee the presence of a least element, needed to compute \ms{fix} terms.

  %% \todo{Explain how products of semilattice being semilattice + monotone
  %%   fixed-points account for mutual recursion.}

  Second, they provide a natural eliminator for sets. Given $e : \Set{A}$, we
  write $\letin{x}{e'}{e_x}$ for the least upper bound, over all elements $x \in
  e'$, of $e_x$, for $e_x$ of some semilattice type $L$. If $e_x$ is a set, for
  example, this provides the set type's monadic ``bind'' operation. For example,
  $\forin{x \in \setlit{1,2,3}} \{10 \cdot x, x^2\}$ denotes the set $\{1, 4, 9,
  10, 20, 30\}$.

\end{enumerate}


%% Section 3: examples
%% FIGURE: SYNTAX SUGAR
\begin{figure}
  \[\begin{array}{lccl}
  %% expressions
  \textsf{terms} &
  e &\bnfeq& ... \pipe e \isin e \pipe \setlit{\vec{e}}
             \pipe \setfor{e}{\mc{L}}
             \pipe \forin{\mc{L}}{e}\\
  &&& \mathcal{C}\;\vec{e} \pipe \rawcase{e}{[{p} \cto {e}]^*}
  \vspace{0.5em}\\
  %% patterns
  %%
  %% TODO: maybe remove the pattern-matching stuff? since we don't explain how
  %% to translate it & we also use various other sugar we don't explain how to
  %% translate, why do we include only pattern-matching here?
  \textsf{patterns} &
  p &\bnfeq& \pwild \pipe x \pipe (p,p)
             \pipe \ms{true} \pipe \ms{false} \pipe \mathcal{C}\;\vec{p}
  \vspace{0.5em}\\
  \textsf{constructors} & \mathcal{C} && \text{are abstract identifiers}
  \vspace{0.5em}\\
  %% loop clauses
  \textsf{loops} &
  \mc{L} &\bnfeq& \mc{L}, \mc{L} \pipe p \in e \pipe e
  \end{array}\]

  %% the desugaring syntax-expansion itself
  \begin{eqnarray*}
    \setlit{} &\expandsto& \unit\\
    \setlit{e,\vec{e_i}} &\expandsto& \setlit{e} \vee \setlit{\vec{e_i}}\\
    \setfor{e}{\mc{L}}       &\expandsto& \forin{\mc{L}}{\{e\}}\\
    \forin{\mc{L}_1,\mc{L}_2}{e}
    &\expandsto& \forin{\mc{L}_1}{\forin{\mc{L}_2}{e}}\\
    \forin{p\in e_1}{e_2} &\expandsto&
    \letin{x}{e_1}{\rawcase{x}{p \cto e_2;\,\pwild \cto \unit}}\\
    \forin{e_1}{e_2} &\expandsto& \ifthen{e_1}{e_2}{\unit}
    %% \ms{let}~x = e_1 ~\ms{in}~ e_2
    %% &\expandsto& (\fn\bind{x} e_2)\; e_1\\
    %% \ms{let}~[x_i = e_i]^* ~\ms{in}~ e
    %% &\expandsto& [\ms{let}~x_i = e_i~\ms{in}]^* e\\
    %% \rawcase{e}{[p \cto e]^*} &\expandsto& \text{(omitted, see \todo{CITE})}
  \end{eqnarray*}
  \caption{Syntax sugar}
  \label{fig:sugar}
\end{figure}


\section{Examples}

For purposes of these examples, we use a simple Haskell-like syntax for
top-level type and function definitions. We also permit ourselves infix
notation, \ms{let}-binding, $n$-ary tuples, $n$-ary sum types with named
constructors, pattern-matching (including non-linear patterns), and additional
syntax sugar given in Figure \ref{fig:sugar}. All of these conveniences are
supported (with slightly different concrete syntax) in our implementation.

%% \todo{(TODO: mention \& cite monadic query syntax)}

For clarity, we set the names of top-level variables in \textsf{sans-serif};
ordinary variables in $script$ or \mi{italic} (for long variable names); and
monotone variables in \m{bold}.

Although Datafun as presented does not have polymorphism, we give our examples
their most general possible type schemes.

%% IDEAS FOR MORE EXAMPLES:
%%
%% \begin{itemize}
%% \item \texttt{make}-style topological sort?
%% \item SQL-style examples? SQL vs Datalog vs Datafun?
%% \item translating relational algebra into datafun?
%% \end{itemize}


\subsection{Filtering, mapping, and cross products}

Armed with the syntactic sugar given in Figure \ref{fig:sugar}, basic set
operations such as map, filter, and cross-product are easy first examples:
\[\begin{array}{l}
\fname{map} ~:~ (A \uto B) \uto \Set{A} \mto \Set{B}\\
\fname{map}\;f\;\m{A} = \setfor{f\;x}{x \in \m{A}}\\
\\
\fname{filter} ~:~ (A \uto \bool) \mto \Set{A} \mto \Set{A}\\
\fname{filter}\;\m{f}\;\m{A} = \setfor{x}{x \in \m{A}, \m{f}\; x}\\
\\
(\times) ~:~ \Set{A} \mto \Set{B} \mto \Set{A \x B}\\
\m{A} \times \m{B} = \setfor{(a,b)}{a \in \m{A}, b \in \m{B}}
\end{array}\]

Worth noting here are the subtleties of monotonicity typing. For example,
\ms{map} is not monotone in its function argument, while \fname{filter} is.
Recalling that sets are ordered by inclusion, this is straightforward enough ---
observe, for example, that:
\begin{eqnarray*}
 \fname{map}\;(\le 0)\;\setlit{0,1}
 &\not\subseteq& \fname{map}\;(\le 1)\;\setlit{0,1}\\
 \fname{filter}\;(\le 0)\;\setlit{0,1}
 &\subseteq& \fname{filter}\;(\le 1)\;\setlit{0,1}
\end{eqnarray*}

However, it is perhaps unclear how Datafun's type system ``knows''
\fname{filter} is monotone in \m{f}. In brief, Datafun knows that application
$(\m{f}\;x)$ is monotone in the function, and moreover, testing a boolean guard
$(\m{f}\;x)$ in a set-comprehension such as $\setfor{x}{x \in \m{A}, \m{f}\;x}$
is monotone in the guard expression. A fuller explanation is given in Section
\ref{sec:typing-rules}.

%% Consider the desugaring of \ms{filter}:
%% \[\begin{array}{l}
%% \fname{filter}\;\m{f}\;\m{A} = \forin{x \in \m{A}}
%% \ifthen{\m{f}\;x}{\singleset{x}}{\unit}
%% \end{array}\]
%% \fname{filter}'s type asserts that it uses \m{f} monotonically. This in turn
%% requires that $(\ifthen{\m{f}\;x}{\setlit{x}}{\unit})$ increases monotonically
%% as the value of \m{f} increases.


%% FIGURE: PRIMITIVES
\begin{figure}
  %% TODO: remove unused primitives.
  \[\begin{array}{cll}
  \neg &\of& \bool \uto \bool\\
  =   &\of& \eq{A} \uto \eq{A} \uto \bool\\
  \le &\of& \eq{A} \uto \eq{A} \mto \bool\\
  %% \fname{keys}     &:& \Map{A}{B} \mto \Set{A}\\
  %% \fname{entries}  &:& \Map{A}{B} \uto \Set{A \x B}\\
  %% \fname{tabulate} &:& \Set{A} \mto (A \uto B) \mto \Map{A}{B}\\
  %% \fname{getWith}  &:& \Map{\eq{A}}{B} \mto \eq{A} \uto (B \mto L) \mto L\\
  %% \fname{get}      &:& \Map{\eq{A}}{L} \mto \eq{A} \uto L\\
  %% \fname{substrings} &\of& \ms{Str} \uto \Set{\ms{Str}}\\
  %% \fname{size}     &:& \Set{\eq{A}} \mto \N\\
  \fname{range}    &:& \N \uto \N \mto \Set{\N}\\
  \fname{length}   &:& \str \uto \N\\
  \fname{substring} &:& \str \uto \N \uto \N \uto \str
  \end{array}\]
  \caption{Primitive functions and their type schemes}
  \label{fig:primitives}
\end{figure}


\subsection{Membership, intersection}

So long as the type of a set's elements supports equality, we can test whether
the set contains a value $x$ as follows:
\[\begin{array}{l}
(\isin) ~:~ \eq{A} \uto \Set{\eq{A}} \mto \bool\\
x \isin \m{A} = \forin{y \in \m{A}} x = y
\end{array}\]

The expression $\forin{y \in \m{A}} x = y$ takes the least upper bound, at
boolean type, for every $y \in \m{A}$, of the value of $x = y$. Since booleans
are ordered $\ms{false} < \ms{true}$, ``least upper bound'' is simply logical
disjunction! %% In plain English, this code says ``a set \m{A} contains an element
%% $x$ if some element $y \in \m{A}$ tests equal to $x$''.

Similarly, we can define set intersection by testing for equality:
\[\begin{array}{l}
(\cap) ~:~ \Set{\eq{A}} \mto \Set{\eq{A}} \mto \Set{\eq{A}}\\
\m{A} \cap \m{B} = \setfor{x}{x \in \m{A}, y \in \m{B}, x = y}
\end{array}\]

However, explicitly testing for equality can become tedious, so we usually use
\emph{nonlinear pattern-matching} instead --- that is, we bind the same-named
variable multiple times, which indicates it must have an equal value at each
occurrence:
\[\begin{array}{l}
(\cap) ~:~ \Set{\eq{A}} \mto \Set{\eq{A}} \mto \Set{\eq{A}}\\
\m{A} \cap \m{B} = \setfor{x}{x \in \m{A}, x \in \m{B}}
\end{array}\]

This is merely syntax sugar for an equality test, so the condition that the
set's element type support equality remains in force.

%% \todo{TODO: mention you can extend to all relational algebra? give example?}


\subsection{Composition of relations}

One extremely useful operator it is convenient to define using nonlinear pattern
matching is composition of finite relations (that is, sets of pairs):
\[\begin{array}{l}
(\bullet) : \Set{A \x \eq{B}} \mto \Set{\eq{B} \x C} \mto \Set{A \x C}\\
\m{R} \bullet \m{S} = \setfor{(a,c)}{(a,b) \in \m{R}, (b,c) \in \m{S}}
\end{array}\]

This already demonstrates a capability Datafun has that Datalog does not:
defining operators over relations. A Datalog program defining binary predicates
\texttt{r} and \texttt{s} which wished to compose those predicates would have to
define a new top-level predicate:

\begin{verbatim}
r(X,Y) :- (...).
s(X,Y) :- (...).
rs(A,C) :- r(A,B), s(B,C).
\end{verbatim}

In Datafun, we simply define $(\bullet)$ and use it inline as needed. We shall
see the use of this in later examples.

%% \todo{work phrase ``higher-order'' in here somewhere?}


\subsection{Transitive closure}

Consider the following Datalog program, authored perhaps by a J.R.R. Tolkien
aficionado wishing to trace the geneologies of their favorite work, \textit{The
  Silmarillion}:
\begin{verbatim}
parent(earendil, elrond).
parent(elrond, arwen).
ancestor(X, Y) :- parent(X, Y).
ancestor(X, Z) :- ancestor(X, Y), ancestor(Y, Z).
\end{verbatim}

%% \todo{Discuss how this works in Datalog, but not in Prolog, b/c Prolog is
%%   defined by operational semantics of unification while Datalog is denotational,
%%   least-model semantics. It also works in Datafun!}

%% \todo{Neel suggests using distinction b/w backward \& forward chaining here,
%%   rather than operational/denotational. see Logical Algorithms paper by
%%   McAllister \& co for phrasing?}

This defines a binary \texttt{parent} relation, along with its transitive
closure, \texttt{ancestor}. The Datafun equivalent is:
\[\begin{array}{l}
\mathbf{data}~\ms{person} =
\ctor{E\"arendil} ~|~ \ctor{Elrond} ~|~ \ctor{Arwen}\\
\fname{parent},~\ms{ancestor} ~:~ \Set{\ms{person} \x \ms{person}}\\
\ms{parent} =
\{(\ctor{E\"arendil}, \ctor{Elrond}), (\ctor{Elrond}, \ctor{Arwen})\}\\
\ms{ancestor} = \fix{\m{X}} \ms{parent} \vee (\m{X} \bullet \m{X})
%% \setfor{(a,c)}{(a,b) \in \m{X}, (b,c) \in \m{X}}
\end{array}\]

The type \ms{person} represents the domain of our \ms{parent} and \ms{ancestor}
relations. \ms{parent} is simply a list of parent-child pairs. \ms{ancestor} is
where the action is at: since the Datalog predicate \texttt{ancestor} is defined
recursively, \ms{ancestor} is defined as a least fixed point --- in this case,
of the the equation
\begin{equation*}
  \m{X} = \ms{parent} \vee (\m{X} \bullet \m{X})
\end{equation*}
Informally, we may read this as stating that a pair is in \m{X} if it is in
either \ms{parent} or the composition of \m{X} with itself. This requires that
\m{X} contain the transitive closure of \ms{parent}. And since we take the
\emph{least} fixed point of this equation, \ms{ancestor} contains \emph{exactly}
the transitive closure of \ms{parent}. Voil\`a!


\subsubsection{Transitive closure with an upper bound}

The above explanation glosses over one critical requirement: \ms{fix} requires
that the type at which the fixed-point is taken be a \emph{finite semilattice
  eqtype}.

The type of \ms{ancestor} is $\Set{\ms{person} \x \ms{person}}$. Does this
suffice? It's certainly a semilattice, since it's a set type. Since \ms{person}
is effectively a sum of units, it supports equality, and sets and products of
eqtypes are themselves eqtypes. Likewise, \ms{person} is finite, and products
and sets of finite types are themselves finite.

So! We find ourselves in the clear, for now. However, in practice, the
restriction of \ms{fix} to finite types can be quite limiting. So Datafun
provides an more general way to take a fixed-point: provide an \emph{upper
  bound} which the desired fixed point will not exceed. For this we write
$(\fixle{\m{x}}{e_\top} e)$, where $e_\top$ is our upper bound.

Suppose, for example, we wish to represent our \textit{dramatis personae} as
strings $\str$ rather than defining a \ms{person} type. Then, making use of
bounded fixed points, we could write:
\[\begin{array}{l}
\ms{person} ~:~ \Set{\str}\\
\ms{person} = \{\texttt{"e\"arendil"}, \texttt{"elrond"}, \texttt{"arwen"}\}\\
\ms{parent}, \ms{ancestor} ~:~ \Set{\str \x \str}\\
\ms{parent} = \{(\texttt{"e\"arendil"}, \texttt{"elrond"}),
(\texttt{"elrond"}, \texttt{"arwen"})\}\\
\ms{ancestor} = \fixle{\m{X}}{(\ms{person} \x \ms{person})}
\ms{parent} \vee (\m{X} \bullet \m{X})
\end{array}\]

Instead of a \ms{person} type, we have \ms{person} \emph{set}, which we use to
construct an upper bound on our fixed-point: $(\ms{person} \x \ms{person})$, the
complete binary relation. Since every string in \ms{parent} is also in
\ms{person}, the transitive closure of \ms{parent} cannot exceed this upper
bound.

However, this invariant is left to the programmer to check. What if a sloppy
programmer should mistakenly include a person in \ms{parent} not present in
\ms{person}? More generally, what if the ``fixed point'' $(\fix{\m{x}}{e_\top}
e)$ is trying to compute exceeds $e_\top$? (Or indeed, no such fixed point
exists?)

In that case, the value of $(\fixle{\m{x}}{e_\top} e)$ is \emph{clamped} to the
upper bound $e_\top$. This ensures Datafun programs terminate even in the
presence of sloppy programmers, and although they may not have the value you
expect, that value is at least predictable.


\subsubsection{Generic transitive closure}

Thus far we have only considered taking the transitive closure of a relation we
have already defined. But consider: for any finite eqtype $\fineq{A}$, we may
write:
\[\begin{array}{l}
\ms{trans} ~:~ \Set{\fineq{A} \x \fineq{A}} \mto \Set{\fineq{A} \x \fineq{A}}
\vspace{0.3em}\\
%% \ms{trans}\ E = \fix{X} E \vee \setfor{(a,c)}{(a,b) \in E, (b,c) \in X}
%% \ms{trans}\ \m{E} = \fix{\m{X}} \m{E} \vee %
%% \setfor{(a,c)}{(a,b) \in \m{X}, (b,c) \in \m{X}}
\ms{trans}\ \m{E} = \fix{\m{X}} \m{E} \vee (\m{X} \bullet \m{X})
\end{array}\]
Similarly, for any eqtype $\eq{A}$, we may write:
\[\begin{array}{l}
\ms{trans} ~:~
\Set{\eq{A}} \mto \Set{\eq{A} \x \eq{A}} \mto \Set{\eq{A} \x \eq{A}}
\vspace{0.3em}\\
%% \ms{trans}\ \m{V}\ \m{E} = %
%% \ms{fix}~ \m{S} \le \setfor{(a,b)}{a\in \m{V}, b \in \m{V}}\\
%% \hspace{5.35em}\ms{is}~ \m{E} \vee %
%% \setfor{(a,c)}{(a,b) \in \m{S}, (b,c) \in \m{S}}\\
\ms{trans}\ \m{V}\ \m{E} = %
\fixle{\m{S}}{(\m{V} \x \m{V})} \m{E} \vee (\m{S} \bullet \m{S})
\end{array}\]

In this way, we can abstract away from choice of underlying relation and define
transitive closure generically. Using functions as a means of abstraction is of
course familiar and unremarkable to functional programmers, but it is simply not
possible in Datalog.


\subsection{CYK parsing}
Parsing can be understood logically, with a parse tree representing a
proof that a certain string belongs to a language described by a
context-free grammar. As a result, it is possible to formulate parsing
in terms of proof search~\cite{deductive-parsing}. One of the
simplest algorithms for parsing context free grammars is the
Cocke-Younger-Kasami (CYK) algorithm for parsing with grammars in
Chomsky normal form.\footnote{In Chomsky normal form, each production
  is of the form $A \to B \cdot C$ or $A \to \vec{a}$, with $A,B,C$
  ranging over nonterminals, and $\vec{a}$ over nonempty strings of
  terminals.}  Given a grammar $G$, we begin by introducing a family
of predicates (sometimes called \emph{facts} or \emph{items}) $A(i,j)$,
with one $A$ for each nonterminal, and $i$ and $j$ representing
indices into a string. Given a word $w$, we write $w[i,n]$ for the
$n$-element substring of $w$ beginning at position $i$. Then, we can
specify the CYK algorithm with the following two inference rules:
\begin{mathpar}
  \inferrule*{B(i, j) \\ C(j, k) \\ (A \to B\; C) \in G}
             {A(i, k)}
  \and
  \inferrule*{ (A \to \vec{a}) \in G \\ w[i,n] = \vec{a} }
             {A(i,i+n)}
\end{mathpar}
Then, the predicate $A(i,j)$ means that $A$ is derivable from the
substring of $w$ running from $i$ to $j$, and so the whole word $w$ is
derivable from the start symbol $S$ if $S(0, \mathit{length}\;w)$ is
derivable.

In Datafun, this rule-based description of the algorithm can be
transliterated almost directly into code. We begin by introducing a
few basic types.
\[\begin{array}{l}
\mathbf{type}~\ms{sym} = \str\\
\mathbf{data}~\ms{rule} = \ctor{String}~\str ~|~ \ctor{Concat}~\ms{sym}~\ms{sym}\\
\mathbf{type}~\ms{grammar} = \Set{\ms{sym} \x \ms{rule}}\\
\mathbf{type}~\ms{fact} = \ms{sym} \x \N \x \N\\
\end{array}\]
The $\ms{sym}$ type is a type synonym representing nonterminal names
with strings. The $\ms{rule}$ type is the type of the right-hand-sides
of productions in Chomsky normal form -- either a string, or a pair of
nonterminals. A $\ms{grammar}$ is just a set of productions -- a set
of pairs of nonterminals paired with their rules. The type $\ms{fact}$
is the type representing the atomic facts derived by the CYK inference
system -- they are triples of the rulename, the start position, and
the end position.

With these types in hand, we can write the CYK algorithm as a fixed
point computation. In fact, it is convenient to break it into two
pieces, by first defining the function whose fixed point we take. So
we can write down the $\fname{iter}$ function, which represents one step of
the fixed point iteration.
\[\begin{array}{l}
\fname{iter} ~:~ \str \uto \ms{grammar} \mto \Set{\ms{fact}} \mto \Set{\ms{fact}}\\
\fname{iter} \;\mi{text} \;\m{G} \;\m{chart} =\\
\hspace{1em}\phantom{\vee~}
\{(a,i,k) ~|~ (a, \ctor{Concat}~b~c) \in \m{G},\\
\hspace{6.25em} (b,i,j) \in \m{chart}, (c,j,k) \in \m{chart}\}\\
\hspace{1em}\vee~ \{(a,i,i+\fname{length}\;s)\\
\hspace{2.1em}|~ (a, \ctor{String}~s) \in \m{G},\\
\hspace{2.2em}\phantom{|~} i \in \fname{range}\;0\;(n-\fname{length}\;s),\\
\hspace{2.2em}\phantom{|~}
s = \fname{substring} \;\mi{text} \;i \;(i+\ms{length}\;s)\}
\end{array}\]
This function works by taking a string $\mi{text}$ and a grammar $\m{G}$, and
then taking a set of facts $\m{chart}$, and taking a union. The first clause is
a set comprehension, saying that we return $(a, i, k)$ if $(b, i, j)$ and $(c,
j, k)$ are in $\m{chart}$ -- this corresponds to applications of the first rule.
The second clause corresponds to the second rule above, saying that $(a, i, i +
\ms{length}\;s)$ is a generated fact if $s$ is a substring of $\mi{text}$ at
position $i$.

We can then use $\fname{iter}$ to implement the $\fname{parse}$ function.
%% parse
\[\begin{array}{l}
\fname{parse} ~:~ \str \uto \ms{grammar} \mto \Set{\ms{sym}}\\
\fname{parse} \;\mi{text} \;\m{G} =\\
\hspace{1em} \ms{let}~ n = \ms{length} \;\mi{text}\\
\hspace{2.375em}\m{bound} =
  \{(a,i,j) ~|~ (a,\pwild) \in \m{G},\\
\hspace{10.5em}i \in \ms{range}\;0\;n, \\
\hspace{10.5em}j\in\fname{range}\;i\;n\}\\
\hspace{2.375em} \m{chart} = \fixle{\m{C}}{\m{bound}}
  \ms{iter} \;\mi{text} \;\m{G} \;\m{C}\\
\hspace{1em}\ms{in}~\setfor{a}{(a, 0, n) \in \m{chart}}\\
%% %% iter with \forin
%% \\
%% \fname{iter} \;\mi{text} \;\m{G} \;\m{chart} =\\
%% \hspace{1em}\phantom{\vee~}
%% (\bigvee((a, \ctor{Concat} \;b \;c) \in \m{G},\\
%% \hspace{1.25em}\phantom{\vee~ \bigvee(}
%% (b,i,j) \in \m{chart}, (c,j,k) \in \m{chart})\\
%% \hspace{1.25em}\phantom{\vee~}\, \setlit{(a,i,k)})\\
%% \hspace{1em}\vee~ (\bigvee((a, \ctor{String} \;s) \in \m{G},\\
%% \hspace{1.25em}\phantom{\vee~\bigvee(}
%% i \in \ms{range} \;0 \;(n - \ms{length} \; s),\\
%% \hspace{1.25em}\phantom{\vee~\bigvee(}
%% s = \ms{substring} \;\mi{text} \;i \;(i+\ms{length}\;s))\\
%% \hspace{1.25em}\phantom{\vee~}\, \setlit{(a,i,i+ \ms{length}\;s)})
%% \\
%% %% iter with case. I like this version best.
%% \\
%% \fname{iter} \;\mi{text} \;\m{G} \;\m{chart} =\\
%% \hspace{1em}\forin{(a,r) \in \m{G}}\\
%% \hspace{1.875em}\ms{case}~ r\\
%% \hspace{2.4em}\ms{of}~
%% %% \hspace{3.05em}\pipe
%% \ctor{Concat} \;b \;c \cto \{(a,i,k) ~|~ (b,i,j) \in \m{chart},\\
%% \hspace{14.42em}(c,j,k) \in \m{chart}\}\\
%% \hspace{3.05em}\pipe \ctor{String} \;s \cto
%% \{\,(a, i, i+\ms{length}\;s)\\
%% \hspace{9em}|~ i \in \ms{range} \;0 \;(n-\ms{length}\;s),\\
%% \hspace{9em}\phantom{|~}
%% s = \ms{substring} \;\mi{text} \;i \;(i+\ms{length}\;s)\}
%% \\
%% iter. Neel prefers this. People know set-comprehension.
\end{array}\]
This function just takes the fixed point of $\fname{iter}$ --
almost. Because facts are triples $\ms{sym} \x \N \x \N$, sets of
facts may in general grow unboundedly.  To ensure termination, we
construct a set $\m{bound}$ to bound the sets of facts we consider in
our fixed point computation, by bounding the symbols to names found in
the grammar $\m{G}$, and the indices to positions of the string. Since
all of these are finite, we know that the computation of $\m{chart}$
as a bounded fixed point will terminate. Then, having computed the
fixed point, we can check chart to see if $(a, 0, \ms{length}\;\mi{text})$
is derivable.

There are three things worth noting about this program. First, it is
not expressible in Datalog. Because Datalog provides no way to
represent a \emph{grammar} as a piece of data (it's compound, not an
atom), there is simply no way in Datalog to express a \emph{generic}
parser taking a grammar as an input. This demonstrates one of the key
benefits of moving to a functional language like Datafun.

Moreover, Datalog programs must be \emph{constructor-free}, to ensure all
relations are finite. Primitives such as \ms{range} and \ms{substring} violate
this restriction (as relations, they are infinite); it is not immediately
obvious that Datalog programs extended with these primitives remain terminating.
Our use of bounded fixed-points to guarantee termination is robust under such
extensions; as long as all primitive functions are total, Datafun programs
always terminate.

Finally, having computed a set via a fixed point, we can test whether
or not an element is in that set \emph{or not} -- the ability to test
for negative information after the fixed point computation completes
corresponds to a use of stratified negation in Datalog.


\subsection{Dataflow analysis}
In this section, we show how some simple dataflow analyses can be expressed in
Datafun. We begin with the types in these programs.
\[\begin{array}{l}
\textbf{type}~\ms{var} = \str\\
\textbf{type}~\ms{label} = \N\\
\textbf{data}~\ms{oper} = \ctor{Eq} \pipe \ctor{Le}
\pipe \ctor{Add} \pipe \ctor{Sub} \pipe\ctor{Mul}\pipe\ctor{Div}\\
\textbf{data}~\ms{atom} = \ctor{Var}\;\ms{var} \pipe \ctor{Num}\;\N\\
\textbf{data}~\ms{expr} = \ctor{Atom}\;\ms{atom}
\pipe \ctor{Apply}\;\ms{oper}\;\ms{atom}\;\ms{atom}\\
\textbf{data}~\ms{stmt} =
\ctor{Assign} \;\ms{var} \;\ms{expr}
\pipe \ctor{If} \;\ms{expr} \;\ms{label}\;\ms{label} \\
\textbf{type}~\ms{program} = \Set{\ms{label} \x \ms{stmt}}

\end{array}\]
The basic idea is that we represent a program as a kind of control
flow graph. Each node of this graph has a $\ms{label}$, which is a
natural number, and contains a statement of type $\ms{stmt}$, which is
either an assignment of an expression (of type $\ms{expr}$) to a
variable (of type $\ms{var}$), or a conditional jump.  A program is
then just the set of nodes -- i.e., a set of label, statement pairs --
with the invariant that the relation is functional (i.e., if $(l, s)$
and $(l,s')$ are both in a program, then $s = s'$).

In what follows, we use a few trivial functions whose definitions are omitted
for space reasons.
\[\begin{array}{l}
%% omitted functions
\ms{labels} ~:~ \ms{program} \uto \Set{\ms{label}}\\
\ms{vars} ~:~ \ms{program} \uto \Set{\ms{var}}\\
\ms{uses} ~:~ \ms{stmt} \uto \Set{\ms{var}}\\
\ms{defines} ~:~ \ms{stmt} \uto \Set{\ms{var}}
\end{array}\]
The $\ms{labels}$ function returns the set of labels in a program. The
$\ms{vars}$ function returns the set of variables used in a program (both in
expressions and as targets for assignments). The $\ms{uses}$ function
returns the set of variables used by the expressions in a statement. The
$\ms{defines}$ function returns the set of variables defined by a statement
(i.e., at most one variable -- the target of the assignment).

Given a program, we define the 1-step control flow graph with the $\ms{flow}$
function.
\[\begin{array}{l}
%% control flow
%% TODO: use long variable name for argument.
\textbf{type}~\ms{flow} = \Set{\ms{label} \x \ms{label}}\\
\fname{flow} ~:~ \ms{program} \uto \ms{flow}\\
\fname{flow}\;c = \forin{(i,s) \in c}\\
\hspace{4em}\ms{case}~ s ~\ms{of}~
\ctor{If} \;\pwild \;j \;k \cto \setlit{(i,j),(i,k)}\\
\hspace{7.45em}\pipe\pwild \cto \setfor{(i,i+1)}{i+1 \isin \ms{labels}\;c}
\end{array}
\]
It says that if $(i, s)$ is a node of the program, then if $s$ is a conditional
jump $\ctor{If} \;\pwild \;j \;k$, then control can flow from $i$ to $j$, and
from $i$ to $k$ -- i.e., we add both $(i, j)$ and $(i, k)$ to the set of edges.
Otherwise, it's an assignment, and control flows to the next statement (i.e., we
add $(i, i+1)$ to the set of edges).

Now, we can define liveness analysis, one of the classic ``backwards'' dataflow
analyses. The type of $\ms{live}$ say that given a program and its flow graph,
it returns a set of label/variable pairs, which determine a relation saying
for each label which variables are live.
%% live code analysis
\[\begin{array}{l}
\ms{live} ~:~ \ms{program} \uto \ms{flow} \uto \Set{\ms{label} \x \ms{var}}\\
\ms{live} \;\mi{code} \;\mi{flow} =\\
\hspace{2em} \fixle{\m{Live}}{ %
  \ms{labels}\;\mi{code} \x \ms{vars}\;\mi{code}}\\
\hspace{2em}\forin{(i,\mi{stmt}) \in \mi{code}}\\
\hspace{2.875em} (\phantom{\vee~}\setfor{(i,v)}{v \in \ms{uses}\;\mi{stmt}}\\
\hspace{3.2em} \vee~ \{(i,v) ~|~ (i,j) \in \mi{flow},\\
\hspace{7.4em}(j,v) \in \m{Live},\\
\hspace{7.4em}\neg (v \isin \ms{defines}\; \mi{stmt})\})
\end{array}\]
For a statement $\mi{stmt}$ at label $i$, we say that the variable
$v$ is live at $i$ if $v$ is used by $\mi{stmt}$. The variable $v$
is also live at $i$ if control flows from $i$ to $j$, and and $v$
is live at $j$, assuming that $\mi{stmt}$ isn't a definition site for $v$.

When computing this analysis, we again need to use a bounded fixed
point, which we do by taking the Cartesian product of the labels and
variables occuring in the program.


Next, we give one of the classic forwards dataflow analyses,
reaching definitions. This analysis is used to figure out whether
an assignment (a ``definition'') can influence the value of later
expressions or not.
%% reaching definitions analysis
\[\begin{array}{l}
\ms{reachingDefinitions} ~:~ \ms{program} \uto \ms{flow}
\uto \Set{(\ms{label} \x \ms{var}) \x \ms{label}}\\
\ms{reachingDefinitions} \;\mi{code} \;\mi{flow} =\\
\hspace{2em}\fixle{\m{RD}}{%
  (\ms{labels}\;\mi{code} \x \ms{vars}\;\mi{code}) \x \ms{labels}\;\mi{code} }\\
\hspace{2em}\forin{(i,\mi{stmt}) \in \mi{code}}\\
\hspace{2.875em} (
\phantom{\vee~}\setfor{((i,v), i)}{v \in \ms{defines}\;\mi{stmt}}\\
\hspace{3.2em} \vee~ \{((l,v), i) ~|~ (j,i) \in \mi{flow},\\
\hspace{8.95em}((l,v), j) \in \m{RD},\\
\hspace{8.95em}\neg(v \isin \ms{defines}\;\mi{stmt})\})
\end{array}\]
We define a function $\fname{reachingDefinitions}$ which takes a
program and a set of flows as arguments, and returns a relation of
type $\Set{(\ms{label} \x \ms{var}) \x \ms{label}}$. An entry $((l,v),
i)$ in this relation means the definition of $v$ at $l$ reaches program
point $i$.

This is then computed as a fixed point of two clauses. First, if there
is a definition $v$ at program point $i$, then $i$ is reached by that
definition. Second, if $(l,v)$ reaches $j$, and $j$ flows to $i$, then
$(l,v)$ reaches $i$ as long as $v$ is not re-defined at $i$.

As \citet{whaley-lam} observed, Datalog makes it very easy to express
dataflow analyses, and it is similarly easy in Datafun.


%% Section 4: Typing rules
\input{typing-rules}

%% Section 5: Semantics
\documentclass[nomarginums]{rntz}
%\usepackage[b5]{rntzgeometry}
\usepackage[phone]{fantasy}
%\usepackage[baskerville,newmath]{rntzfont}
\usepackage[pt]{rntzfont}
\usepackage{anyfontsize}
\usepackage[spacing=true,stretch=10]{microtype}
\frenchspacing

\usepackage{amssymb,amsmath,amsthm} % \square etc.
%\usepackage{array}                  % >{blah}, <{blah} in array formats.
%\usepackage{booktabs}               % \midrule
\usepackage{mathpartir}             % \begin{mathpar}, \infer, etc.
\usepackage{stmaryrd}               % \shortrightarrow, \llbracket, etc.
\usepackage[b]{esvect}              % \vv for wide vector arrows
\usepackage{nccmath}                % fix align* (etc.) spacing

%% Commands
\newcommand\todo[1]{{\color{Rhodamine}#1}}

\newcommand\naive{na\"ive}
\newcommand\Naive{Na\"ive}
\newcommand\cat\textbf
\newcommand\strong\textbf
\newcommand\CP{\cat{ChangePoset}}
\newcommand\Poset{\cat{Poset}}
\newcommand\initO{\ensuremath{\mathbold{0}}}
\newcommand\termO{\ensuremath{\mathbold{1}}}
\newcommand\initE{\mathop{\text{\textexclamdown}}}
\newcommand\termI{\mathop{!}}

\newcommand\G\Gamma
\newcommand\D\Delta
\newcommand\x\times
\newcommand\dee\delta
\newcommand\tuple[1]{\left({#1}\right)}
\newcommand\triv{\star}
\newcommand\injc{\mathrm{in}}
\newcommand\inj[1]{\injc_{#1}\,}
\newcommand\zero{\ensuremath{\mathbold{0}}}

\newcommand\iso{\texorpdfstring{\ensuremath{\square}}{iso}}
\newcommand\isof[1]{\iso {#1}}
\newcommand\fname[1]{\textit{#1}}
\newcommand\id{\fname{id}}
\newcommand\dummy{\fname{dummy}}

%% TODO: look at other latex arrows
\newcommand\validarrow{{\to}}
\newcommand\longvalidarrow{{\longrightarrow}}
\newcommand\valid[1]{\mathrel{\overset{#1}{\validarrow}}}
\newcommand\longvalid[1]{\mathrel{\overset{#1}{\longvalidarrow}}}
\newcommand\vals[1]{#1^v} % other options: _v, _o, _\iota
\newcommand\valfn{\vals{-}}
\newcommand\chgs[1]{\D{#1}}

\newcommand\fork[1]{\langle{#1}\rangle}
\newcommand\krof[1]{[{#1}]}
\newcommand\Fork[1]{\left\langle{#1}\right\rangle}
\newcommand\Krof[1]{\left[{#1}\right]}
\renewcommand\fork\Fork
\renewcommand\krof\Krof

\renewcommand\vals{\mathcal{V}}
\renewcommand\valfn\vals

\newcommand\funct[1]{\vals{#1}}
\newcommand\deriv[1]{#1'}

\newcommand\bindsp{~\,}
\newcommand\fa[1]{\forall #1.\bindsp}
\newcommand\ex[1]{\exists #1.\bindsp}
%% \renewcommand\ex[1]{(\exists #1)\bindsp}
%% \renewcommand\fa[1]{(\forall #1)\bindsp}


\title{Change Semantics for Semi\naive{} Datafun}
\author{Michael Arntzenius}
\date{5 November 2018}

\begin{document}

\maketitle

\begin{abstract}
  We define the category \CP{} of posets equipped with \emph{increasing
    changes}, whose morphisms are monotone maps equipped with \emph{derivatives}
  taking input changes to output changes. We give a semantics for Datafun in
  \CP.
\end{abstract}

\todo{TODO: should I be using posets or preorders?}


\section{Notation and conventions}

$\initE_A : \initO \to A$ is the unique map out of an initial object; $\termI_A
: A \to \initO$ is the unique map into a terminal object.

\todo{TODO: Explain $\iso : \Poset \to \Poset$.}


\section{The category \CP}

\newcommand\pto\rightharpoonup
\newcommand\upd\oplus

Objects $A$ of \CP{} are tuples $(\vals A, \chgs A, \upd_A)$, where
\begin{enumerate}
\item $\vals A \in \Poset$ is the poset of values.
\item $\chgs A \in \Poset$ is the poset of changes.
\item $(\upd) : \vals A \x \chgs A \pto \vals A$ is a \strong{partial} map
  taking a value and a change to an updated value. If $x \upd dx$ is defined, we
  say $dx$ is a \emph{valid} change to $x$.

  $x \valid{dx} y$ denotes that $x \upd dx$ is defined and equal to $y$.
  Otherwise, propositions involving $\upd$ should be understood to hold only
  when it is defined.
\end{enumerate}

\noindent
These must satisfy the following conditions:\footnote{There are other
  conditions one might sensibly impose here, most notably, that $x \upd dx$ is
  monotone in $dx$. I have omitted this because I have not needed it, but the
  lack of an obvious ``correct'' set of conditions bothers me.}

\begin{align*}
  x \le y &~\implies \ex{dx} x \valid{dx} y
  & \text{complete for increases}\\
  x \le y &\impliedby~ \ex{dx} x \valid{dx} y
  & \text{sound for increases}
\end{align*}

\noindent
A morphism $f \in \CP(A, B)$ is a value map $\funct f \in \Poset(\vals A, \vals
B)$ eq\-uip\-ped with a derivative $\deriv f \in \Poset(\isof{\vals A} \x \chgs
A, \chgs B)$ such that:

\[ x \valid{dx} y \implies \funct f\,x \longvalid{\deriv f(x,dx)} \funct f\,y\]

\noindent Or, equivalently:
\[ \funct f(x \upd dx) = \funct f\,x \upd \deriv f(x, dx) \]

\noindent
Two morphisms are equal iff they have equal value maps and their derivatives'
behavior coincides on \emph{valid} changes. In other words, we \strong{quotient}
equality of derivatives as follows:

\[ (\fa{x \valid{dx} y} \deriv f(x,dx) = \deriv g(x,dx))
\implies \deriv f = \deriv g \]

\noindent
Composition of derivatives follows the chain rule:

\begin{align*}
  %% \funct\id\, x &= x &
  %% \funct{(f \circ g)}\, x &= \funct f (\funct g\,x)\\
  \funct\id &= \id & \funct{(f \circ g)} &= \funct f \circ \funct g\\
  \deriv\id(x,dx) &= dx
  & \deriv{(f \circ g)} (x,dx) &= \deriv f (g \;x,\, \deriv g(x,dx))
\end{align*}

\noindent The only interesting case of associativity is the one involving the
chain rule:
\begin{align*}
  \deriv{(f \circ (g \circ h))} (x,dx)
  &= \deriv f((g \circ h) \;x,\ \deriv{(g \circ h)}(x,dx))\\
  &= \deriv f (g(h \;x),\ \deriv g(h \;x,\ \deriv h(x,dx)))\\
  &= \deriv{(f \circ g)} (h \;x,\ \deriv h(x,dx))\\
  &= \deriv{((f \circ g) \circ h)} (x,dx)
\end{align*}


\subsection{Zero changes and change composition}

Applying the axiom of choice to soundness \& completeness for increases, we can
recover zero-change and composition operators $\zero$, $(\cdot)$ such that $x
\upd \zero_x = x$ and $x \upd (dx \cdot dy) = (x \upd dx) \upd dy$. These
do not necessarily satisfy the identity and associativity laws of a category,
but can be useful nonetheless. \todo{TODO: talk about non-constructivity and why
  it won't be a problem. maybe should go in ``strategy'' section at the top?}


\subsection{\CP{} refines \Poset{}}
\label{sec:refines}

There is a straightforward forgetful functor $\valfn : \CP \to \Poset$ taking
objects $A$ to $\vals{A}$ and morphisms $f$ to $\funct f$. The structures we
will build in \CP{} all \emph{refine} the corresponding structures in \Poset{},
meaning they commute with $\valfn$. For example, for cartesian products,
$\vals{(A \x B)} = \vals{A} \x \vals{B}$ and $\funct{\fork{f,g}} = \fork{\funct
  f, \funct g}$. For brevity's sake, we omit the ``value components'' of our
constructions when they can be derived from this fact.


\subsection{Cartesian structure}

\begin{theorem}
  \CP{} has all finite products and sums.
\end{theorem}

\begin{proof}
  The value- and change-poset structure is inherited from \Poset{}, as are the
  value components of the universal, projection, and injection maps (see
  \S\ref{sec:refines}), while the projection and injection's derivatives operate
  pointwise on changes:

  \begin{align*}
    \vals{\left(\prod_{i \in I} A_i\right)} &= \prod_{i \in I} \vals A_i &
    \vals{\left(\sum_{i \in I} A_i\right)} &= \sum_{i \in I} \vals A_i
    \\[1em]
    \chgs{\prod_{i \in I} A_i} &= \prod_{i \in I} \chgs A_i &
    \chgs{\sum_{i \in I} A_i} &= \sum_{i \in I} \chgs A_i
    \\[.5em]
    \funct{\Fork{\vv f}} &= \Fork{\vv{\funct f}} &
    \funct{\Krof{\vv{f}}} &= \Krof{\vv{\funct f}}
    \\
    \funct\pi_i &= \pi_i & \funct \injc_i &= \injc_i
    \\
    \deriv\pi_i(x,dx) &= \pi_i\;dx & \deriv\injc_i(x,dx) &= \inj i dx
  \end{align*}

  \noindent Updates are pointwise, noting that $(\inj i x \upd \inj j dx)$ is
  undefined for $i \ne j$:

  \begin{align*}
    (x_i)_i \upd (dx_i)_i &= (x_i \upd dx_i)_i &
    \inj i x \upd \inj i dx &= \inj i (x \upd dx)
  \end{align*}

  We calculate $\deriv{\fork{\vv f}}$ from its universal property. Imagine $g$
  such that $\deriv f_i = (\pi_i \circ g)'$ for $i \in I$. Then:

  \begin{align*}
    \deriv f_i(x,dx) &= \deriv{(\pi_i \circ g)} (x,dx)\\
    &= \deriv\pi_i(\funct g\,x,\, \deriv g(x,dx))\\
    &= \pi_i (\deriv g(x,dx))
  \end{align*}

  \noindent
  So by the universal property of products in \Poset{}, we have $\deriv g =
  \fork{\vv{\deriv f}}$. Thus $\deriv{\fork{\vv f}} = \fork{\vv{\deriv f}}$.

  Now let's try calculating $\deriv{\krof{\vv f}}$ from its universal property.
  Imagine $g$ such that $\deriv f_i = \deriv{(g \circ \injc_i)}$. Then:

  \begin{align*}
    \deriv f_i(x,dx) &= \deriv{(g \circ \injc_i)}(x,dx)\\
    &= \deriv g(\funct\injc_i(x),\, \deriv\injc_i(x,dx))\\
    &= \deriv g(\inj i x,\, \inj i dx)
  \end{align*}

  This defines $g'(\inj i x,\, \inj j dx)$ when $i = j$. What about $i \ne j$?
  Here we appeal to our quotient: derivatives are equal iff they agree on
  \emph{valid} changes. By construction, all valid changes to $\inj i x$ are of
  the form $\inj i dx$. So it \strong{does not matter} what $\deriv{\krof{\vv
      f}}(\inj i x, \inj j dx)$ does for $i \ne j$, so long as it is defined.
  One satisfactory definition is:

  \begin{equation*}
    \deriv{\Krof{\vv f}}(\inj i x,\, \inj j dx) = 
    \begin{cases}
      \deriv f_i(x,dx) & \text{if}~ i=j\\
      \fname{magic}(x) & \text{otherwise}
    \end{cases}
  \end{equation*}

  \noindent\todo{TODO}

  \pagebreak
  We show this by constructing initial and terminal objects and binary sums and
  products. In each case, both the underlying and the change posets come from
  the corresponding structure in \Poset{}:

  %% \begin{align*}
  %%   \vals \termO = \chgs \termO &= \termO \in \Poset &
  %%   \vals \initO = \chgs \initO &= \initO \in \Poset\\
  %%   \triv \upd \triv &= \triv
  %%   \\
  %%   \vals{(A \x B)} &= \vals A \x \vals B &
  %%   \vals{(A + B)} &= \vals A + \vals B
  %%   \\
  %%   \chgs{(A \x B)} &= \chgs A \x \chgs B &
  %%   \chgs{(A + B)} &= \chgs A + \chgs B \\
  %%   (a,b) \upd (da,db) &= (a \upd da, b \upd db) &
  %%   \inj i x \upd \inj i dx &= \inj i (x \upd dx)
  %% \end{align*}

  \begin{align*}
    \vals \initO = \chgs \initO &= \initO \in \Poset &
    \vals \termO = \chgs \termO &= \termO \in \Poset \\
    \vals{(A \x B)} &= \vals A \x \vals B &
    \vals{(A + B)} &= \vals A + \vals B \\
    \chgs{(A \x B)} &= \chgs A \x \chgs B &
    \chgs{(A + B)} &= \chgs A + \chgs B
  \end{align*}

  \noindent The update maps are given by:

  \begin{align*}
    \triv \upd \triv &= \triv\\
    (a,b) \upd (da,db) &= (a \upd da, b \upd db)\\
    \inj i x \upd \inj i dx &= \inj i (x \upd dx)
  \end{align*}

  \noindent Here, $\triv$ is the single inhabitant of $\termO \in \Poset$. We
  give no cases for $\initO$ because it is empty. Note that $(\inj i x \upd \inj
  j dx)$ is undefined for $i \ne j$.

  The ``value components'' of the morphisms $!$, \textexclamdown, $\pi_i$,
  $\injc_i$, $\fork{f, g}$, and $[f,g]$ are inherited from \Poset{} along
  $(\vals{-})$ (see \S\ref{sec:refines}). The derivatives of $!$, \textexclamdown,
  $\pi_i$, and $\injc_i$ simply operate pointwise on their second argument:

  \begin{align*}
    {\deriv{!}}(x,dx) &= {\funct{!}}\,dx &
    \deriv{\text{\textexclamdown}}(x,dx) &= \funct{\text{\textexclamdown}}\,dx\\
    \deriv\pi_i(x,dx) &= \funct{\pi_i}\,dx &
    \deriv\injc_i(x,dx) &= \funct{\injc_i}\,dx
  \end{align*}

  \noindent
  Note that $!$ and {\textexclamdown} inherit their universal properties from
  \Poset{}.

  We can calculate $\deriv{\fork{f_1,f_2}}$ from its universal property.
  Consider some $g$ such that $\deriv f_i = \deriv{(\pi_i \circ g)}$ for $i \in
  \{1,2\}$. Then:

  \begin{align*}
    \deriv f_i(x,dx)
    &= \deriv{(\pi_i \circ g)}(x,dx)\\
    &= \deriv\pi_i (\funct g\,x,\, \deriv g(x,dx))\\
    &= \funct\pi_i(\deriv g(x,dx))
  \end{align*}

  \noindent
  So by the universal property of products in \Poset{}, we have $g' =
  \fork{\deriv f_1, \deriv f_2}$. Thus $\deriv{\fork{f_1,f_2}} = \fork{\deriv
    f_1, \deriv f_2}$.

  Finally, let's attempt to calculate $\deriv{[f_1,f_2]}$ from its universal
  property. Consider some $g$ such that $\deriv f_i = \deriv{(g \circ
    \injc_i)}$. Then:

  \begin{align*}
    \deriv f_i(x,dx)
    &= \deriv{(g \circ \injc_i)}(x,dx)\\
    &= \deriv g (\funct\injc_i \,x,\, \deriv\injc_i \,(x,dx))\\
    &= \deriv g (\inj i x,\, \inj i dx)
  \end{align*}

  \noindent
  This defines $g'(\inj i x,\, \inj j dx)$ when $i = j$. What about $i \ne j$?
  Here we appeal to our quotient: derivatives are equal iff they agree on
  \emph{valid} changes. By construction of the validity relation for sums, all
  valid changes to $\inj i x$ are of the form $\inj i dx$. So it \strong{does
    not matter} what $\deriv{[f_1,f_2]}(\inj i x, \inj j dx)$ does for $i \ne
  j$, so long as it is defined. One satisfactory definition is:

  \begin{equation*}
    \deriv{\Krof{f_1,f_2}} (\inj{i} x,\, \inj{j} dx)
    = \begin{cases}
      \deriv f_i(x,dx) & \text{if}~ i=j\\
      \fname{magic}(x) & \text{otherwise}
    \end{cases}
  \end{equation*}

  \noindent
  Here $\fname{magic} : \vals A \to \chgs A$ is given by the axiom of choice as
  follows. Fix $x : \vals A$. By reflexivity $x \le x$, so by soundness,
  $\exists dx : \chgs A$ (indeed, one such that $x \valid{dx} x$, but this is
  irrelevant here). Let \fname{magic} be a choice function that picks such a
  $dx$.

  I omit the proofs that the derivatives are monotone in their second argument.
\end{proof}

\noindent
NB. \fname{magic} is effectively dead code: necessary to define $\deriv{[f,g]}$,
but irrelevant to its behavior. We will exploit this later to avoid needing the
axiom of choice in the context of Datafun.


\subsection{Exponential objects}

\newcommand\expO[2]{#1 \Rightarrow #2}
\newcommand\df{d\kern-0.1em f\kern-.1em} % fucking kerning.

The change poset of the exponential object $\expO A B \in \CP$ consists of
monotone maps $\df, dg \in \expO{\isof{\vals A}}{\expO{\chgs A}{\chgs B}}$,
partially ordered pointwise by their effect on valid changes and quotiented to
satisfy antisymmetry:

\begin{equation*}
  \df \le dg : \chgs{(\expO A B)}
  \iff \fa{x \valid{dx} y} \df \;x \;dx \le dg \;x \;dx 
\end{equation*}

\noindent
The validity relation is:

\begin{equation*}
  f \valid{\df} g : \expO A B
  \iff \fa{x \valid{dx} y} f\;x \longvalid{df \;x \;dx} g\;y
\end{equation*}

\noindent
This is sound \& complete for increases: \todo{TODO}
\begin{align*}
  f \le g : \vals{(\expO A B)}
  &\iff f \le g : \expO{\vals A}{\vals B}\\
  &\iff \fa{x \le y} f\;x \le g\;y\\
  &\iff \fa{x,y} (\ex{dx} x \valid{dx} y)
  \implies \ex{dx} f\;x \valid{dx} g\;y\\
  &\iff \todo{???}\\
  &\iff \ex{\df} f \valid{\df} g
\end{align*}

\newcommand\fn\lambda
\todo{TODO: define \fname{eval}, compute $\fn$ from its universal property}


\subsection{The lifted \iso{} comonad}

\subsection{Semilattice objects}

\subsection{Fixed points}

\end{document}


%% Section 6: Datafun vs Datalog
\input{datalog-comparison}

%% Section 7: Implementation
\section{Implementation}
We have built a proof-of-concept implementation of Datafun in Racket, available
at \texttt{[link omitted for double-blind review]}. In addition to core Datafun,
it supports pattern-matching, variant types, record types, dictionaries,
subtyping, antitone functions, and unbounded (potentially nonterminating) fixed
points. It performs no optimizations whatsoever.

\paragraph{Type inference}
As a practical matter, type-checking needs to distinguish between ordinary and
monotone $\lambda$, application, \ms{case}, \ms{let}, and \ms{if}. In our
implementation we solve this in two ways:
\begin{enumerate}
\item Bidirectional type inference \todo{CITE} determines whether $\fn$s and
  applications are ordinary or monotone.
\item For $\ms{if}$, $\ms{case}$, and $\ms{let}$, the programmer annotates which
  form is intended; for example, $(\ifthen{e}{e_1}{\unit})$ is written
  (\texttt{when e then e1}) to indicate the rule $\ms{if}^+$ applies.
\end{enumerate}

It remains an open question how much annotation is necessary. We speculate that
bidirectional inference could be replaced by a Damas-Milner \todo{CITE} style
algorithm, which infers a principal type for any term without any annotation at
all, \emph{if} we add polymorphism, tone-polymorphism, and subtyping---so that,
for example, $\fn\bind{f}\fn\bind{x} f\;x$ can be assigned the principal type
$\forall\bind{o\of\ms{tone}}\forall\bind{\alpha,\beta \of \ms{type}} (\alpha
\overset{o}\to \beta) \mto (\alpha \overset{o}\to \beta)$, where
$\overset{o}\to$ indicates a function of tone $o$; a tone may be empty (for an
ordinary function) or ${+}$ for a monotone function.

%% \todo{explain subtyping?}
%% \todo{explain antitonicity?}
%% \todo{explain ordering on dictionaries?}


%% Section (omitted): Tradeoffs, etc.
%% not sure how many sections to split this up into.
\section{Tradeoffs, limitations, and design decisions}

\paragraph{Finite and bounded fixed-points} \TODO Discuss
Datalog's constructor restriction (name?), not being sure how to encode it in a
type theory, its disadvantages, and the disadvantages of our approaches ---
finite \& bounded fixed-points.

bounded fixed-points strictly more general than finitary-type fixed-points, but
at possible runtime cost, but finite types very restrictive in practice

\TODO In particular, could bounded fix-points have adverse performance
implications?

\paragraph{Termination} \TODO nontermination and declarativeness.

For example, one powerful optimization technique is \emph{loop reordering} (in
SQL terminology, \emph{join reordering}), that is, taking advantage of the
equation
\begin{eqnarray*}
  \forin{x \in e_1} \forin{y \in e_2} e
  &=& \forin{y \in e_2} \forin{x \in e_1} e
\end{eqnarray*}
when $x,y \notin \ms{FV}(e_1) \cup \ms{FV}(e_2)$. (\TODO Explain why join
reordering is powerful). But this equation does not always hold in the presence
of nontermination; for example, if $e_1 = \unit$ and $e_2$ diverges. \TODO
finish up

\paragraph{Type inference} Typechecking needs to distinguish between ordinary
and monotone $\lambda$, application, \ms{case}, \ms{let}, and \ms{if}. In our
implementation we solve this in two ways:
\begin{enumerate}
\item Bidirectional type inference \todo{CITE} determines whether $\fn$s and
  applications are ordinary or monotone.
\item For $\ms{if}$, $\ms{case}$, and $\ms{let}$, the programmer annotates which
  form is intended; for example, $(\ifthen{e}{e_1}{\unit})$ is written
  (\texttt{when e then e1}) to indicate the rule $\ms{if}^+$ applies.
\end{enumerate}

It remains an open question how much annotation is necessary. See
Section \ref{sec:futurework}, \emph{Related and future work}.

\paragraph{User-defined usls}
The two fundamental usl types Datafun provides are booleans and sets; products
and functions merely preserve usl structure where they find it. One might
contemplate allowing the programmer to define their own usl structures using
something like Haskell's \texttt{newtype}/\texttt{instance}. Unfortunately, this
\TODO
\begin{itemize}
\item User-defined functions give the compiler no guarantee that they are
  commutative, associative, and idempotent.
\item Without further information it would be impossible to
\end{itemize}

Unfortunately, there is no simple characterization of the space of all possible
computable usls.

\paragraph{Lexical types}
The category \cPoset{} has a much richer structure than \cSet{} (which may
indeed be seen as a subcategory of \cPoset{}).

\todo{\begin{itemize}
\item Pro: allow expressing many things more generally, for example, map lookup
\item Con: have no good general monotone elimination rule
\item Con: complicate the type theory
\end{itemize}}


%% Section 8: Related & future work
\section{Other Related and Future Work}
\label{sec:futurework}

\paragraph{Deletion} \citet{logical-algorithms} showed how
forward-chaining logic programming permits concise and elegant
expression of a wide variety of algorithms, including a natural cost
semantics. However, they noted that there were some algorithms (such
as union-find and greedy algorithms) which could be formulated in this
style, \emph{if} there were additionally support for deleting facts
from a database. Later, \citet{linear-logical-algorithms} went on to
show how deletion could be given a logical interpretation by
formulating in terms of linear logic programming.

This naturally raises the question of whether we could identify a
``linear Datafun'' corresponding to this style of programming, where
we might linear types to model features like deletion. There are many
nontrivial semantic issues (e.g., how to define monotonicity), but
it seems a promising question for future work.

\paragraph{User-Defined Posets and Semilattices}
The two fundamental semilattice types Datafun provides are booleans and sets;
products and functions merely preserve semilattice structure where they find
it. One might contemplate allowing the programmer to define their own
semilattice structures using something like Haskell's
\texttt{newtype}/\texttt{instance}. In general, this is a difficult
problem, because we may need to do serious mathematical
reasoning to prove that a datatype has a partial ordering, or can be
equipped with a semilattice structure which is commutative, associative
and idempotent.

One example of such a family of types are the \emph{lexicographic sum
  types}. Given two posets $P$ and $Q$, their disjoint union $P + Q$
is also a poset, with left values compared by the $P$-ordering, and
right values compared by the $Q$-ordering, and no ordering between
left and right values. However, this is not the only way that the
disjoint union could be equipped with an order structure.

For example, we could define the \emph{lexicographic} sum $P \lhd Q$,
which has the same elements as the sum, but extending the coproduct order
relation with the additional
facts that $\ms{in}_1(p) \leq \ms{in}_2(q)$. Indeed, we already have a
special case of this: as we noted earlier, our boolean type is not $1
+ 1$, but it \emph{is} $1 \lhd 1$.

But as our Booleans already show, giving good syntax for their
eliminators is difficult, because we have to show that not just a term
is monotone, but that the different branches of a lexicographic case
expression are ordered with respect to \emph{each other}. For the case
of ordered Booleans, we were able to give a special eliminator which
guaranteed it, but in general it requires proof.

One natural direction for future work is to extend the syntax of
Datafun with support for these kinds of proofs, perhaps taking
inspiration from dependent type theory.

\paragraph{Relational Algebra} Datalog has sometimes been described
as ``relational algebra plus fixed points'', and as Datafun
generalizes Datalog, so too does it generalize relational algebra.
However, one of the benefits of having a denotational semantics is
that it lets offer precise comparisons.

In particular


\paragraph{Optimization} Given 




\begin{itemize}
\item \TODO the datalog literature
\end{itemize}




%% ---------- End matter ----------

%% \acks
%% Acknowledgments, if needed.

% We recommend abbrvnat bibliography style.
\bibliographystyle{abbrvnat}
\bibliography{datafun}

%% \appendix
%% \section{Appendix Title}

%% This is the text of the appendix, if you need one.


\end{document}

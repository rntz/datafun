
\section{Other Related and Future Work}
\label{sec:futurework}

\paragraph{Deletion} \citet{logical-algorithms} showed how
forward-chaining logic programming permits concise and elegant
expression of a wide variety of algorithms, including a natural cost
semantics. However, they noted that there were some algorithms (such
as union-find and greedy algorithms) which could be formulated in this
style, \emph{if} there were additionally support for deleting facts
from a database. Later, \citet{linear-logical-algorithms} went on to
show how deletion could be given a logical interpretation by
formulating in terms of linear logic programming.

This naturally raises the question of whether we could identify a
``linear Datafun'' corresponding to this style of programming, where
we might linear types to model features like deletion. There are many
nontrivial semantic issues (e.g., how to define monotonicity), but
it seems a promising question for future work.



\paragraph{Optimization}
\begin{itemize}
\item \TODO the datalog literature
\end{itemize}

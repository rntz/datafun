\section{Introduction}

%% Lorem ipsum dolor sit amet, consectetur adipiscing elit. Nec vero alia sunt
%% quaerenda contra Carneadeam illam sententiam. Cum ageremus, inquit, vitae beatum
%% et eundem supremum diem, scribebamus haec. Quid sequatur, quid repugnet, vident.

%% Minime id quidem, inquam, alienum, multumque ad ea, quae quaerimus, explicatio
%% tua ista profecerit. Qua ex cognitione facilior facta est investigatio rerum
%% occultissimarum. Quis animo aequo videt eum, quem inpure ac flagitiose putet
%% vivere? Primum quid tu dicis breve? Ita enim vivunt quidam, ut eorum vita
%% refellatur oratio.

\TODO

\paragraph{Contributions}
\begin{itemize}
\item We describe Datafun, a typed language capturing the expressive power of
  Datalog and extending it to support higher-order functional programming.
  Datafun's key feature is to \emph{track monotonicity with types}.

\item We present examples illustrating the expressive power of Datafun,
  including relational-algebra-style operations, transitive closure, CYK
  parsing, and dataflow analysis.

\item We identify the semantic structures underpinning Datalog, and use this to
  give a denotational semantics for Datafun in terms of a pair of adjunctions
  between \ms{Set}, \ms{Poset}, and the category of join-semilattices with a
  least element.

\item We have a prototype implementation of Datafun in Racket. \todo{(CITE)}
\end{itemize}

%% Contributions (as summarized by Michael):
% - Datafun, like Datalog but functional
% - examples, incl. both datalog examples & things datalog can’t do
% - key ingredient is monotonicity; ``found'' semantics by analyzing
%   datalog: two adjunctions, three categories
% - prototype implementation

%% Contributions (as written by Neel):

% - We describe Datafun, a type theory for a language capturing the expressive
%   power of Datalog and extends it to both relax the constructor term
%   restriction and to support higher-order functional programming.

% - We give a variety of examples that illustrate the expressive power of
%   Datafun, such as CYK parsing, dataflow analysis, and transitive closure on
%   graphs, etc. Many of these examples are traditional examples of Datalog,
%   but we are also able to support things like first-class relations (eg,
%   generic transitive closure) and higher-order functions (example using
%   monotonicity and HO?). (doing a fix-point code analysis / parsing something
%   & dispatching on result?)

% - We identify the semantic structures underpinning Datalog, and use this to
%   give a denotational semantics for Datafun in terms of a pair of adjunctions
%   between Set, Poset, and the category of semilattices with finitary joins.

% - We have a prototype implementation of Datafun in Racket.

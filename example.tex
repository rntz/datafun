% ---- The rntz class ----
% Based on extarticle, and supports most of its options. Notable differences:
%
% 1. Section & sub-section numbers go into the left margin.
% 2. Sections & sub-sections headings are smaller.
% 3. Sub-sub-sections are un-numbered; I use them sparingly if at all.
% 4. Redefines \maketitle and the `abstract' environment.

% It also has a somewhat random grab-bag of other features I happen to use:
%
% 5. Requires and configures hyperref, url, and cleveref.
%
% 6. Requires amsmath & amsthm, and defines theorem, conjecture, lemma,
% definition, and corollary environments. It sets them to share a single running
% counter.
%
% 7. Defines some colors, taken from acmart.cls:
% ACM{Blue,Yellow,Orange,Red,LightBlue,DarkBlue,Green,Purple}.
\documentclass{rntz}

% ---- rntzgeometry ----
%
% rntzgeometry sets the page geometry to have "reasonable" margins. It uses
% narrow.sty to produce a single column of text; the figure* and fullwidth
% environments expand to a larger width.
%
% a5, b5, a4, letter
%   Paper size. Default is b5.
%
% width=LEN
%   Text width. Default is 345pt.
%
% fullwidth=LEN
%   Width for figure* and fullwidth environments. Default varies by paper size.
\usepackage{rntzgeometry}

% ---- rntzfont ----
%
% rntzfont chooses from a set of paired text & math fonts, with appropriate
% scalings and line spacings.
% 
% The options are charter, cochineal, palatino, libertine, cm.
% Default is palatino.
\usepackage{rntzfont}

% Some small typographic improvements.
\usepackage{microtype}
\frenchspacing

\usepackage{lipsum} % for example text.

\title{An example document}
\author{Lore M. Ipsum}
\date{\today}

\begin{document}
\maketitle

\begin{abstract}
  Nam dui ligula, fringilla a, euismod sodales, sollicitudin vel, wisi. Morbi
  auctor lorem non justo. Nam lacus libero, pretium at, lobortis vitae,
  ultricies et, tellus. Donec aliquet, tortor sed accumsan bibendum, erat ligula
  aliquet magna, vitae ornare odio metus a mi. Morbi ac orci et nisl hendrerit
  mollis.
\end{abstract}

\section{A section}
\lipsum[4]

\subsection{Some mathematics}

Nulla malesuada porttitor diam. Donec felis erat, congue non, volutpat at,
tincidunt tristique, libero.

\[ \sum_{i=1}^n n = \frac{n(n+1)}{2} \]

Vivamus viverra fermentum felis. Donec nonummy pellentesque ante. Phasellus
adipiscing semper elit. Proin fermentum massa ac quam.
%
\begin{eqnarray}
  e^{i\tau} &=& 0\\
  (x+y)^n &=& \sum_{k=0}^n \binom{n}{k} x^{n-k} y^k
\end{eqnarray}

Sed diam turpis, molestie vitae, placerat a, molestie nec, leo. Maecenas lacinia. Nam ipsum ligula, eleifend at, accumsan nec, suscipit a, ipsum. Morbi blandit ligula feugiat magna. Nunc eleifend consequat lorem. Sed lacinia nulla vitae enim. Pellentesque tincidunt purus vel magna. Integer non enim. Praesent euismod nunc eu purus. Donec bibendum quam in tellus. Nullam cursus pulvinar lectus. Donec et mi. Nam vulputate metus eu enim. Vestibulum pellentesque felis eu massa.

\section{Another section}
\subsection{Immediately followed by a subsection}

\lipsum[6]

\subsubsection{I usually avoid sub-subsections, but here one is anyway}

\lipsum[7-8]

\end{document}

%% Laptop-optimized.
\documentclass{rntz}
\usepackage[laptop,]{fantasy}
\usepackage[scaled=1.036,scaled=1.0417,]{rntzfont}

%% %% A5 paper.
%% \documentclass{rntz}
%% \usepackage[a5]{rntzgeometry}
%% \usepackage[width=\textwidth,fullwidth=120mm,]{narrow}
%% \usepackage[newmath,scaled=1.02]{rntzfont}
%% %% \providecommand\mathvar[1]{\ensuremath{\mathit{#1}}}
%% %% \providecommand\mathbold[1]{{\boldsymbol#1}}

%% %% Compromise between screen viewing & printing one-sided on A4 or letter.
%% \documentclass{rntz}
%% \usepackage[hscale=4/5,
%%   aspect=1.4142135623730951, vscale=6/7, % A4 aspect ratio.
%%   %aspect=11/8.5, vscale=.85, % US letter aspect ratio.
%% ]{aspect}
%% \geometry{vratio=2:3}\pagestyle{plain}
%% %\usepackage[width=\textwidth,fullwidth=380pt,]{narrow}
%% \usepackage[scaled=1.0417,]{rntzfont}


\usepackage{seminaive}

%\definecolor{VibrantRed}{cmyk}{0, 0.85, 1, 0}

%\definecolor{Morph}{cmyk}{0, 1, 0.75, 0.33} % reddish
%\definecolor{Morph}{cmyk}{0, 1, 0.5, 0.5} % mahogany
%\definecolor{Morph}{cmyk}{0, 0.8, 1, 0.5} % brown
%\definecolor{Morph}{cmyk}{1, 0.75, 0, 0} % dark blue
\definecolor{Morph}{cmyk}{1, 0.5, 0, 0.2} % darkened Royal Blue

% DarkOrchid, Sepia, RoyalBlue, Blue.
\renewcommand\morph[1]{\ensuremath{\mathrm{\color{Sepia}#1}}}
\renewcommand\tuple[1]{\ensuremath{(#1)}}


\title{Semi\naive\ Datafun via tangent bundles in \CP}
\author{\scshape Michael Arntzenius}
\date{\today}

\begin{document}

\maketitle


\section{The category \CP}

An object $A : \CP$ consists of:
\begin{mathpar}
  \vals A,\, \chgs A : \Poset
  \and
  {\validto_A} \subseteq \chgs A \x \vals A \x \vals A
  \and
  \dummy : \vals A \to \chgs A
\end{mathpar}

\noindent
A morphism $f : A \to B$ consists of $\vals f : \vals A \to \vals B$ and $f' :
\iso \vals A \to \chgs A \to \chgs B$ such that
%
\begin{equation}\label{eqn:derivative}
  \validat A \dx x y \implies \validat B {f'\<x \<\dx}{f\<x}{f\<y}
\end{equation}

\noindent
Identity and composition are straightforward:
%
\begin{align*}
  \vals \id &= \id
  & \vals (g \circ f) &= \vals g \circ \vals f
  \\
  \id' \<x \<\dx &= \dx
  & (g \circ f)' \<x \<\dx &= g' \<(\vals f\<x) \<(f' \<x \<\dx)
\end{align*}

\noindent The identity law for composition is inherited from \Poset\ on the
value-level, and also holds for derivatives:

\begin{gather*}
  (\id \circ f)' \<x \<\dx = \id' \<(f\<x) \<(f'\<x\<\dx)
  = f'\<x\<\dx
  \\
  (f \circ \id)' \<x \<\dx = f' \<(\id\<x) \<(\id'\<x\<\dx)
  = f' \<x \<\dx
\end{gather*}

\noindent
Associativity of derivatives also holds:

\begin{align*}
  ((h \circ g) \circ f)' \<x \<\dx
  &= (h \circ g)' \<(f\<x) \<(f'\<x\<\dx)
  \\
  &= (h \circ g)' \<(f\<x) \<(f'\<x\<\dx)
  \\
  &= h' \<(g\<(f\<x)) \<(g' \<(f\<x) \<(f'\<x\<\dx))
  \\
  &= h' \<((g \circ f)\<x) \<((g \circ f)'\<x\<\dx)
  \\
  &= (h \circ (g \circ f))' \<x \<\dx
\end{align*}


\section{The semi\naive\ model of Datafun}

A Datafun model consists of a bicartesian closed category \catC\ equipped with:
a ``discreteness'' comonad \iso; for every object $A$, a ``finite powerset''
object $\pfin A$; families of equality objects $\eqt A$, semilattice objects
$L$, and fixed point objects $\fixt L$ which interpret the corresponding Datalog
type families; and the following morphism families: \nopagebreak[2]
\begin{mathpar}
  \textstyle
  \discox : \prod_i \iso A_i \to \iso \prod_i A_i

  \discosum : \iso \sum_i A_i \to \sum_i \iso A_i

  \morph{empty?} : \iso \pfin \termO \to \termO + \termO

  \morph{single} : \iso A \to \pfin A

  \infer{f : \G \x \iso A \to L}{\pcollect f : \G \x \pfin A \to L}

  \morph{eq} : \iso \eqt A \x \iso \eqt A \to \pfin \termO

  \morph{join}_n : L^n \to L

  \morph{fix} : \iso (\expO{\fixt L}{\fixt L}) \to \fixt L
\end{mathpar}

\noindent
In the remainder of this section we construct a Datafun model in \CP\ that
corresponds to semi\naive\ evaluation.


\subsection{\boldmath The tangent bundle comonad \tangfn}

For our discreteness comonad, we use the ``tangent bundle'' comonad $\tangfn$ on
$\CP$ is defined by:
%
\begin{mathpar}
  \vals\tang A = \iso \vals A \x \chgs A

  \chgs\tang A = \termO

  \dummy_{\tang A} \<(x,dx) = ()

  \infer{\validat A \dx x x}{\validat{\tang A}{()}{(x,\dx)}{(x,\dx)}}
\end{mathpar}

\noindent The functorial action $\tang f$, extraction $\extract_A : TA \to A$,
and duplication $\dup_A : \tang A \to \tang{\tang A}$ are:
\begin{align*}
  \tang{f} \<(x,\dx) &= (f\<x,\, f'\<x\<\dx)
  &
  (\tang{f})' \<(x,\dx) \<() &= ()
  \\
  \extract_A \<(x,\dx) &= x
  &
  \extract_A' \<(x,\dx) \<() &= \dx
  \\
  \dup_A \<(x,\dx) &= ((x,\dx),())
  &
  \dup_A' \<(x,dx) \<() &= ()
\end{align*}

\noindent
Correctness of $(\tang f)'$ follows from correctness of $f'$:
%
\nopagebreak[3]
\begin{align*}
  &\mathrel{\phantom{\implies}}
  \validat{\tang A}{()}{(x,\dx)}{(x,\dx)}
  \\
  &\implies \validat{A}{\dx} x x &\text{inversion}
  \\
  &\implies \validat{B}{f'\<x\<\dx}{f\<x}{f\<x}
  &\text{by \cref{eqn:derivative}}
  \\
  &\implies \validat{\tang B}{()}{\tang f \<(x,dx)}{\tang f \<(x,dx)}
\end{align*}

\noindent
We show $T$ respects \id\ and composition, ignoring the derivatives since they
are trivial:
%
\nopagebreak[3]
\begin{align*}
  \tang{\id} \<(x,\dx) &= (\id \<x,\, \id'\<x\<\dx) = (x,\dx)
  = \id\<(x,\dx)
  \\[1ex]
  \tang{(g \circ f)} \<(x,\dx)
  &= ((g \circ f) \<x,\, (g \circ f)' \<x \<\dx)
  \\
  &= (g \<(f\<x),\, g' \<(f\<x) \<(f'\<x\<\dx))
  \\
  &= (\tang g \circ \tang f) \<(x,\dx)
\end{align*}

\noindent
The derivatives of $\extract$ and $\dup$ are correct: \XXX
%
\noindent
Naturality of $\extract$ and $\dup$: \XXX
%
\noindent
The comonad laws hold: \XXX
%
\todo{I've checked these things on paper.}


\subsection{Finite powersets}

Finite powerset objects in \CP\ are defined using the analogous finite powerset
functor on \Poset: \nopagebreak[3]
\begin{mathpar}
  \vals \pfin A = \pfin \vals A

  \chgs \pfin A = \pfin \vals A

  \dummy\<x = \emptyset

  \infer{}{\validat{\pfin A}{\dx}{x}{x \cup \dx}}
\end{mathpar}

\noindent
Although we won't need this fact, \pfin\ is also an endofunctor on \CP.

\subsection{Equality objects}

\subsection{Semilattice objects}

\subsection{Fixed point objects}

Any semilattice object satisfying the ascending chain condition is a fixed point
object. \todo{TODO: define derivative of \prim{fix}.}


\section{Reading off semi\naive\ evaluation as a static transformation}

\XXX

\end{document}

%% Laptop-optimized.
\documentclass{rntz}
%\usepackage[laptop]{fantasy}
\usepackage[a5]{rntzgeometry}\usepackage[width=\textwidth,fullwidth=120mm,]{narrow}
%\usepackage[scaled=1.036,scaled=1.0417,scaled=1.02,]{rntzfont}

\usepackage{xfp}
\makeatletter
\edef\@scaled{1.016}
\newcommand\CrimsonProInitials@scale{1}
\edef\zeu@Scale{\@scaled}
\usepackage[p,lf,sb,scaled=\fpeval{\@scaled*1.0957269037956552}]{CrimsonPro}
\usepackage[scaled=\fpeval{\@scaled*1.002183406113537}]{AlegreyaSans}
\usepackage{eulervm}
\usepackage[cal=cm]{mathalfa}
\linespread{\fpeval{\@scaled*1.0661}}
\makeatother

\providecommand\mathvar\ensuremath
\providecommand\mathbold\boldsymbol

%% %% A5 paper.
%% \documentclass{rntz}
%% \usepackage[a5]{rntzgeometry}
%% \usepackage[width=\textwidth,fullwidth=120mm,]{narrow}
%% \usepackage[newmath,scaled=1.02]{rntzfont}
%% %% \providecommand\mathvar[1]{\ensuremath{\mathit{#1}}}
%% %% \providecommand\mathbold[1]{{\boldsymbol#1}}


\usepackage{seminaive}
\captionsetup{font={it},labelfont={}}

%\definecolor{VibrantRed}{cmyk}{0, 0.85, 1, 0}
%\definecolor{Morph}{cmyk}{0, 1, 0.75, 0.33} % reddish
%\definecolor{Morph}{cmyk}{0, 1, 0.5, 0.5} % mahogany
%\definecolor{Morph}{cmyk}{0, 0.8, 1, 0.5} % brown
%\definecolor{Morph}{cmyk}{1, 0.75, 0, 0} % dark blue
\definecolor{Morph}{cmyk}{1, 0.5, 0, 0.2} % darkened Royal Blue

% DarkOrchid, Sepia, RoyalBlue, Blue.
%\renewcommand\morph[1]{\ensuremath{\mathrm{\color{Sepia}#1}}}
\renewcommand\tuple[1]{\ensuremath{(#1)}}


\title{Semi\naive\ Datafun via the category \CP}
\author{\scshape Michael Arntzenius}
\date{\today}

\begin{document}

\maketitle


\section{The category \CP}

An object $A : \CP$ consists of:
\begin{mathpar}
  \vals A,\, \chgs A : \Poset
  \and
  {\validto_A} \subseteq \chgs A \x \vals A \x \vals A
  \and
  \dummy : \vals A \to \chgs A
\end{mathpar}

\noindent
A morphism $f : A \to B$ consists of $\vals f : \vals A \to \vals B$ and $f' :
\iso \vals A \to \chgs A \to \chgs B$ such that
%
\begin{equation}\label{eqn:derivative}
  \validat A \dx x y \implies \validat B {f'\<x \<\dx}{f\<x}{f\<y}
\end{equation}

\noindent
Identity and composition are straightforward:
%
\begin{align*}
  \vals \id &= \id
  & \vals (g \circ f) &= \vals g \circ \vals f
  \\
  \id' \<x \<\dx &= \dx
  & (g \circ f)' \<x \<\dx &= g' \<(\vals f\<x) \<(f' \<x \<\dx)
\end{align*}

\noindent The identity law for composition is inherited from \Poset\ on the
value-level, and also holds for derivatives:

\begin{gather*}
  (\id \circ f)' \<x \<\dx = \id' \<(f\<x) \<(f'\<x\<\dx)
  = f'\<x\<\dx
  \\
  (f \circ \id)' \<x \<\dx = f' \<(\id\<x) \<(\id'\<x\<\dx)
  = f' \<x \<\dx
\end{gather*}

\noindent
Associativity of derivatives also holds:

\begin{align*}
  ((h \circ g) \circ f)' \<x \<\dx
  &= (h \circ g)' \<(f\<x) \<(f'\<x\<\dx)
  \\
  &= (h \circ g)' \<(f\<x) \<(f'\<x\<\dx)
  \\
  &= h' \<(g\<(f\<x)) \<(g' \<(f\<x) \<(f'\<x\<\dx))
  \\
  &= h' \<((g \circ f)\<x) \<((g \circ f)'\<x\<\dx)
  \\
  &= (h \circ (g \circ f))' \<x \<\dx
\end{align*}


\section{The semi\naive\ model of Datafun}

A Datafun model consists of a bicartesian closed category \catC\ equipped with:
a ``discreteness'' comonad \iso; for every object $A$, a ``finite powerset''
object $\pfin A$; families of equality objects $\eqt A$, semilattice objects
$L$, and fixed point objects $\fixt L$ which interpret the corresponding Datalog
type families; and the following morphism families: \nopagebreak[2]
\begin{mathpar}
  \textstyle
  \discox : \prod_i \iso A_i \to \iso \prod_i A_i

  \discosum : \iso \sum_i A_i \to \sum_i \iso A_i

  \morph{empty?} : \iso \pfin \termO \to \termO + \termO

  \morph{single} : \iso A \to \pfin A

  \infer{f : \G \x \iso \eqt A \to L}{\pcollect f : \G \x \pfin \eqt A \to L}

  \morph{eq} : \iso \eqt A \x \iso \eqt A \to \pfin \termO

  \morph{join}_n : L^n \to L

  \morph{fix} : \iso (\expO{\fixt L}{\fixt L}) \to \fixt L
\end{mathpar}

\noindent
In the remainder of this section we construct a Datafun model in \CP\ that
corresponds to semi\naive\ evaluation.


\subsection{Cartesian closedness}
\XXX

\subsection{Weak coproducts}

\XXX


\subsection{The comonad \ciso}

\begin{figure*}[pth]
  \begin{mathpar}
    \vals\ciso A = \iso(\vals A \x \chgs A)

    \chgs\ciso A = \termO

    \dummy_{\ciso A} \<\eboxd{x,\dx} = ()

    \infer{\vldat{A}{\dx} x x}
          {\vldat{\ciso A}{()}{\ebox{x, \dx}}{\ebox{x,\dx}}}
  \end{mathpar}

  \begin{align*}
    \ciso f \<\eboxd{x,\dx} &= \ebox{f\<\dvar x, f'\<\eboxd x\<\dvar\dx}
    &
    (\ciso f)' \<\eboxd{x,\dx} \<() &= ()
    \\
    \extract_A \<\eboxd{x,\dx} &= \dvar x
    &
    \extract_A' \<\ebox{\eboxd{x,\dx}} \<() &= \dvar\dx
    \\
    \dup_A \<\eboxd{x,\dx} &= \ebox{\eboxd{x,\dx},()}
    &
    \dup_A' \<\ebox{\eboxd{x,\dx}} \<() &= ()
    \\
    \discox \<(\eboxd{x_i, \dx_i})_i &= \ebox{(\dvar{x_i})_i, (\dvar{\dx_i})_i}
    &
    \discox{}' \<\pwild \<(())_i &= ()
    \\
    \discosum \<\ebox{\inj i \dvar x, \inj j \dvar\dx}
    &= 
    \begin{cases}
      \inj i \eboxd{x,\dx} & \text{if}\ i = j
      \\
      \inj i \ebox{\dvar x, \dummy\<\dvar x} & \text{if}\ i \ne j
    \end{cases}
    &
    \discosum{}' \<\ebox{\inj i \dvar x, \inj j \dvar\dx} &= \inj i ()
  \end{align*}
  \caption{The distributive comonad $\ciso : \CP \to \CP$}
  \label{fig:ciso}
\end{figure*}

\begin{figure*}[pth]
  \begin{align*}
    \morph{collect}(f) \<(\gamma, s)
    &= \bigvee_{\dvar x \in s} f\<(\gamma,\ebox{\dvar x, \zero_{\dvar x}})
    \\
    \morph{collect}(f)' \<\eboxd{\gamma, s} \<(\dgamma, \ds)
    &= \left(\bigvee_{\dvar x \in \ds} f \<(\dvar\gamma, \eboxd{x, \zero_x})\right)
    \vee
    \left(\bigvee_{\dvar x \in {\dvar s} \cup \ds}
    f' \<\ebox{\dvar\gamma, \eboxd{x,\zero_x}} \<()\right)
  \end{align*}

  \begin{align*}
    \morph{empty?} \<\eboxd{s, \pwild} &= 
    \begin{cases}
      \inj 1 () & \text{if}\ \dvar s = \emptyset\\
      \inj 2 () & \text{else}
    \end{cases}
    & \morph{empty?}' \<\ebox{\eboxd{s, \ds}} \<()
    &= \morph{empty?}\<\eboxd{s,\ds}
    \\
    \morph{join} \<(x_i)_i &= \bigvee_i x_i
    &
    \morph{join}' \<\pwild \<(\dx_i)_i &= \bigvee_i \dx_i
    \\
    \morph{eq} \<(\eboxd{x,\pwild}, \eboxd {y,\pwild})
    &= \begin{cases}
      \eset{()} &\text{if}\ \dvar x = \dvar y\\
      \emptyset &\text{else}
    \end{cases}
    &
    \morph{eq}' \<\pwild \<\pwild &= \emptyset
    \\
    \morph{single} \<\eboxd{x, \pwild} &= \esetd{x}
    &
    \morph{single}' \<\pwild \<\pwild &= \emptyset
    \\
    \morph{fix} \<\eboxd{f,\df} &= \fastfix \<\dvar f \<\dvar\df
    &
    \morph{fix}' \<\pwild \<\pwild &= \bot
  \end{align*}

  \caption{Datafun model morphism families in \CP}
  \label{fig:morphisms}
\end{figure*}


For our discreteness comonad, we use \ciso, defined in \cref{fig:ciso} along with
its functorial action, extraction, duplication, and distributive maps.
\noindent
\todo{TODO: write up proofs (done on paper) that $\ciso f$, $\extract$, $\dup$
  are correct.}

The derivative of $\discox$ is correct because from $\fa{i}
\vldat{A_i}{\dx_i}{x_i}{x_i}$ it follows that $\vldat{\prod_i
  A_i}{(\dx_i)_i}{(x_i)_i}{(x_i)_i}$. The correctness of $\discosum{}'$ is more
complicated. Suppose that $\vldat{\sum_i A_i}{\inj j \dx}{\inj i x}{\inj i x}$.
Then by inversion $i = j$ and $\vldat{A_i}{\dx}{x}{x}$. Thus we have
$\vldat{\ciso\<A_i}{()}{\ebox{x,\dx}}{\ebox{x,\dx}}$ and consequently
$\vldat{\sum_i \ciso A_i}{\inj i ()}{\inj i \ebox{x,\dx}}{\inj i \ebox{x,\dx}}$.

\todo{TODO: prove functoriality}

As for the comonad laws, $\extract_{\ciso A} \circ \dup_A = \id = \ciso\extract_A
\circ \dup_A$ because:
%
\begin{align*}
  \extract \<(\dup \<\ebox{x,\dx})
  &= \extract \<\ebox{\ebox{x,\dx},()}
  = \ebox{x,\dx}
  \\
  \ciso\extract \<(\dup \<\ebox{x,\dx})
  &= \ciso\extract \<\ebox{\ebox{x,\dx}, ()}
  = \ebox{\extract\<\ebox{x,\dx},\, \extract' \<\ebox{\ebox{x,\dx}} \<()}
  = \ebox{x,\dx}
  \\
  (\extract \circ \dup)' \<\ebox{\ebox{x,\dx}} \<()
  &= \extract' \<\ebox{\dup\<\ebox{x,\dx}} \<(\dup' ...)
  = \extract' \<\ebox{\ebox{x,\dx},()} \<()
  = ()
  \\
  (\ciso\extract \circ \dup)' \<\ebox{\ebox{x,\dx}} \<()
  &= (\ciso\extract)' \<...
  = ()
\end{align*}

\noindent
Second, $\dup_{\ciso A} \circ \dup_A = \ciso\dup_A \circ \dup_A$ because:
%
\nopagebreak[3]
\begin{align*}
  \dup \<(\dup \<\ebox{x,\dx})
  &=
  \dup \<\ebox{\ebox{x,\dx}, ()}
  =
  \ebox{\ebox{\ebox{x,\dx},()},()}
  \\
  \ciso\dup \<(\dup \<\ebox{x,\dx})
  &=
  \ciso\dup \<\ebox{\ebox{x,\dx},()}
  =
  \ebox{\dup\<\ebox{x,\dx},\, \dup' \<...}
  = 
  \ebox{\ebox{\ebox{x,\dx},()},()}
  \\
  (\dup \circ \dup)' \<\ebox{x,\dx} \<()
  &= \dup' \<... = ()
  \\
  (\ciso\dup \circ \dup)' \<\ebox{x,\dx} \<()
  &= (\ciso\dup)' \<... = ()
\end{align*}

%% Likewise, $\ciso \extract_A \circ \dup_A = \id$ because:

%% \noindent
%% Finally, $\extract \circ \ciso f = f \circ \extract$ because:
%% %
%% \nopagebreak[3]
%% \begin{gather*}
%%   \extract \<(\ciso f \<\ebox{x,\dx})
%%   = \extract \<\ebox{f\<x,\, f'\<\ebox x\<\dx}
%%   = f\<x
%%   = f \<(\extract \<\ebox{x,\dx})\\
%%   (\extract \circ \ciso f)' \<\ebox{x,\dx} \<()
%%   = \extract' \<\ebox{f\<x,\, f'\<\ebox x\<\dx} \<()
%%   = f'\<\ebox x\<\dx = (f \circ \extract)' \<\ebox{x,\dx} \<()
%% \end{gather*}


%% \subsection{\boldmath The tangent bundle comonad \tangfn}

%% For our discreteness comonad, we use the ``tangent bundle'' comonad $\tangfn$ on
%% $\CP$ is defined by:
%% %
%% \begin{mathpar}
%%   \vals\tang A = \iso \vals A \x \chgs A

%%   \chgs\tang A = \termO

%%   \dummy_{\tang A} \<(\eboxd{x}, dx) = ()

%%   \infer{\validat A \dx x x}{\validat{\tang A}{()}{(\eboxd x,\dx)}{(\eboxd x,\dx)}}
%% \end{mathpar}

%% \noindent The functorial action $\tang f$, extraction $\extract_A : TA \to A$,
%% and duplication $\dup_A : \tang A \to \tang{\tang A}$ are: \todo{FIXME: $\dup$
%%   isn't monotone! uh-oh, is $\tangfn$ not a comonad?}
%% %
%% \nopagebreak[3]
%% \begin{align*}
%%   \tang{f} \<(\eboxd x,\dx) &= (f\<\dvar x,\, f' \<\eboxd x \<\dx)
%%   &
%%   (\tang{f})' \<(\eboxd x,\dx) \<() &= ()
%%   \\
%%   \extract_A \<(\eboxd x,\dx) &= \dvar x
%%   &
%%   \extract_A' \<(\eboxd x,\dx) \<() &= \dx
%%   \\
%%   \dup_A \<(\eboxd x,\dx) &=
%%   \color{red} ((x,\dx),()) % uh-oh
%%   &
%%   \dup_A' \<(\eboxd x,dx) \<() &= ()
%% \end{align*}

%% \noindent
%% Correctness of $(\tang f)'$ follows from correctness of $f'$:
%% %
%% \nopagebreak[3]
%% \begin{align*}
%%   &\mathrel{\phantom{\implies}}
%%   \validat{\tang A}{()}{(x,\dx)}{(x,\dx)}
%%   \\
%%   &\implies \validat{A}{\dx} x x &\text{inversion}
%%   \\
%%   &\implies \validat{B}{f'\<x\<\dx}{f\<x}{f\<x}
%%   &\text{by \cref{eqn:derivative}}
%%   \\
%%   &\implies \validat{\tang B}{()}{\tang f \<(x,dx)}{\tang f \<(x,dx)}
%% \end{align*}

%% \noindent
%% We show $T$ respects \id\ and composition, ignoring the derivatives since they
%% are trivial:
%% %
%% \nopagebreak[3]
%% \begin{align*}
%%   \tang{\id} \<(x,\dx) &= (\id \<x,\, \id'\<x\<\dx) = (x,\dx)
%%   = \id\<(x,\dx)
%%   \\[1ex]
%%   \tang{(g \circ f)} \<(x,\dx)
%%   &= ((g \circ f) \<x,\, (g \circ f)' \<x \<\dx)
%%   \\
%%   &= (g \<(f\<x),\, g' \<(f\<x) \<(f'\<x\<\dx))
%%   \\
%%   &= (\tang g \circ \tang f) \<(x,\dx)
%% \end{align*}

%% \noindent
%% The derivatives of $\extract$ and $\dup$ are correct: \XXX
%% %
%% \noindent
%% Naturality of $\extract$ and $\dup$: \XXX
%% %
%% \noindent
%% The comonad laws hold: \XXX
%% %
%% \todo{I've checked these things on paper.}

%% The distributive morphisms $\discox$ and $\discosum$ are:
%% \nopagebreak[3]
%% \begin{align*}
%%   \discox \<(\eboxd{x_i}, \dx_i)_i &=
%% \end{align*}


\subsection{Other Datafun model structures}

\begin{description}
\item[Finite powersets] Finite powerset objects in \CP\ are defined using the
  analogous finite powerset functor on \Poset: \nopagebreak[3]
  \begin{mathpar}
    \vals \pfin A = \pfin \vals A

    \chgs \pfin A = \pfin \vals A

    \dummy\<x = \emptyset

    \infer{}{\validat{\pfin A}{\dx}{x}{x \cup \dx}}
  \end{mathpar}

  \noindent
  Although we won't need this fact, \pfin\ is also an endofunctor on \CP.
  
\item[Equality objects] Equality objects $\eqt A$ are those in which every value
  $x : \vals \eqt A$ possesses a zero change $\zero_x : \chgs \eqt A$
  such that that $\vldat{\eqt A}{\zero_x} x x$.

\item[Semilattice objects] Semilattice objects $L$ are those such that $\vals L
  = \chgs L$ is a semilattice and $\vldat{L} \dx x y \iff x \vee \dx = y$. (This
  condition implies every semilattice object is an equality object by letting
  $\zero_x = \bot$.)

\item[Fixed point objects] A fixed point object $\fixt L$ is a semilattice
  object such that $\vals\fixt L$ has no infinite strictly ascending sequences
  (the \emph{ascending chain condition}).
\end{description}


\noindent
The remaining morphism families are given in \cref{fig:morphisms}. \todo{TODO:
  write up correctness proofs (done on paper, except for
  $\morph{collect}(f)'$).}


\subsection{The function \fastfix}

The function $\fastfix_L \<f\<f'$ is defined for any semilattice $L$ satisfying
the ascending chain condition, given monotone maps $f : L \to L$ and $f' : \iso
L \to L \to L$. It yields the limit of the ascending sequence $(x_i)_i$ defined by:

\begin{align*}
  x_0 &= \bot & x_{i+1} &= x_i \vee \dx_i\\
  \dx_0 &= f\<\bot & \dx_{i+1} &= f'\<\ebox{x_i}\<\dx_i
\end{align*}

\noindent
If $f'$ is a derivative of $f$, then $\fastfix\<f\<f'$ will be the least fixed
point of $f$. \todo{TODO: Import proof here that I've given elsewhere.}


%% \section{Reading off semi\naive\ evaluation as a static transformation}
%% \XXX

\end{document}

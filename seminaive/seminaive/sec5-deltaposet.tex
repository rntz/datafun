%% ---------- Incremental semantics via change posets ----------
\section{The category \CP}
\label{sec:changeposets}

Objects $A$ of \CP{} are tuples $\tuple{\vals A, \chgs A, \updfn_A,
  \composefn_A}$, where:

\begin{itemize}
\item $\vals A$ is a poset of values.

\item $\chgs A$ is a poset of changes.

\item $\updfn_A : \vals A \x \chgs A \pto \vals A$ is a \emph{partial} map
  taking a value and a change to an updated value. When $x \upd \dx$ is defined,
  we say \dx\ is a \emph{valid} change to $x$. It is often convenient to notate
  $x \upd \dx = y$ relationally, as $\valid\dx x y$ (``\dx\ sends $x$ to $y$''),
  or more briefly as $\vld\dx x y$.

\item $\composefn_A : \chgs A \x \chgs A \to \chgs A : \Poset$ is a monotone
  change composition operator.
\end{itemize}

\noindent
These must satisfy the following conditions:

\nopagebreak[2]
\begin{align*}
  x \le y &\impliedby \ex{\dx} \vld{\dx} x y
  & \text{sound for increases}\\
  x \le y &\implies \ex{\dx} \vld{\dx} x y
  & \text{complete for increases}\\
  \vld\dx x y \wedge \vld\dy y z &\implies \vld{\dx\compose\dy} x z
  & \text{correctness of composition}
\end{align*}

\noindent Completeness for increases is equivalent (via choice) to having an
operator $x \changeto y : \chgs A$, defined for $x \le y : \vals A$, such that
$\valid{x \changeto y} x y$.\footnote{Our $x \changeto y$ is effectively the
  incremental $\fn$-calculus' $y \ominus x$, but defined only when $x \le y$.}
In particular, every $x$ has a \emph{zero change}, $\zero_x = x \changeto x$.
However, changes (including zero changes) are not necessarily unique: for a
given $x,y$ there may be many $\dx$ such that $\valid\dx x y$.

Morphisms $f : \CP(A, B)$ are differentiable maps $f : \Poset(\vals A, \vals
B)$. A map is differentiable if it has a derivative $\deriv f : \vals A \to
\Poset(\chgs A, \chgs B)$ such that:

\nopagebreak[2]
\[ \vld{\dx} x y \implies \vld{\deriv f\<x\<\dx}{f\<x}{f\<y}\]

\noindent Or, equivalently (when $x \upd \dx$ is defined):

\nopagebreak[2]
\[ f\<(x \upd \dx) = f\<x \upd \deriv f\<x\<\dx \]

\noindent
The same morphism may have many different derivatives; all that matters is that
one exist.

Identity and composition morphisms in \CP{} are inherited from \Poset{}, so we
need only exhibit their derivatives. These follow a variant of the chain rule:

\nopagebreak[2]
\begin{align*}
  \deriv\id \<x\<\dx &= \dx
  & \deriv{(g \circ f)} \<x\<\dx &= \deriv g \<(f\<x) \<(\deriv f\<x\<\dx)
\end{align*}

\noindent
It's easy to see these functions are monotone in \dx\ as required.


\subsection{The forgetful functor \valfn}
\label{sec:CP-vals}

The functor $\valfn : \CP \to \Poset$ takes objects $A$ to $\vals A$ and
morphisms $f$ to themselves. This can be seen as a process of ``forgetting''. On
objects, it forgets a poset's associated change-structure; on morphisms, it
forgets their differentiability.
%
Our ``first-order'' constructions in \CP{} --- products, sums, the finite
powerset functor --- commute with \valfn, in a sense \emph{inheriting} from
those in \Poset. For example, products have $\vals{(A \x B)} = \vals A \x \vals
B$ and $\vals{\fork{f,g}} = \fork{\vals f, \vals g}$.
%
Where this applies, we omit for brevity this inherited structure.
%
However, exponentials will prove the necessary exception to this rule: since all
maps in \CP{} are differentiable, so are the elements of $\vals{(\expO A B)}$.

%% The structures we will build in \CP{} will
%% usually \emph{refine} those in \Poset, in the sense of commuting with \valfn.
%% For example, for cartesian products, $\vals{(A \x B)} = \vals A \x \vals B$ and
%% $\vals{\fork{f,g}} = \fork{\vals f, \vals g}$.
%% %
%% For brevity, we omit the ``value components'' of our constructions when they are
%% inherited in this way from \Poset.


%% \subsection{Products}
%% \label{sec:CP-products}

%% \CP\ has all finite products. $\valfn$ and $\chgfn$ distribute across them:

%% \nopagebreak[2]
%% \begin{align*}
%%   \vals{\prod_{i \in I} A_i} &= \prod_{i \in I} \vals A_i &
%%   \chgs{\prod_{i \in I} A_i} &= \prod_{i \in I} \chgs A_i
%% \end{align*}

%% \noindent
%% Updates are pointwise:

%% \nopagebreak[2]
%% \[ (x_i)_i \upd (\dx_i)_i = (x_i \upd \dx_i)_i
%% \quad\text{when the RHS is defined}\]

%% \noindent Soundness and completeness follow pointwise from soundness and
%% completeness of $A_i$; in particular, $(x_i)_i \changeto (y_i)_i = (x_i
%% \changeto y_i)_i$. Composition is also pointwise:

%% \[ (\dx_i)_i \compose (\dy_i)_i = (\dx_i \compose \dy_i)_i \]

%% It's easy to see

%% \noindent \todo{TODO: soundness, completeness, composition, correctness of composition}

%% The projection and fork maps are inherited along \valfn, with their derivatives
%% being:

%% \nopagebreak[2]
%% \begin{align*}
%%   \deriv\pi_i\<x\<\dx &= \pi_i\<\dx &
%%   \deriv{\fork{f_i}_i} &= \fork{\deriv f_i}_i
%% \end{align*}

%% \noindent Correctness of $\deriv\pi_i$ is straightforward; if
%% $\valid{(\dx_1,\dx_2)}{(x_1,x_2)}{(y_1,y_2)}$ then by definition
%% $\valid{\dx_i}{x_i}{y_i}$. As for correctness of $\deriv{\fork{f_i}_i}$:

%% \nopagebreak[2]
%% \begin{equation*}
%%   \fork{f_i}_i \<(x \upd \dx)
%%   = (f_i(x \upd \dx))_i
%%   = (f_i\<x \upd \deriv f_i \<x \<\dx)_i
%%   = \fork{f_i}_i \<x \upd \fork{\deriv f_i}_i \<x \<\dx
%% \end{equation*}

%% \noindent
%% Finally, the universal property of $\fork{f_i}_i$ follows from its universal
%% property in \Poset\ --- requiring differentiability merely makes the property
%% weaker.


%% \subsection{Sums}
%% \label{sec:CP-sums}

%% \CP\ has all finite sums. \valfn\ and \chgfn\ distribute across them:

%% \nopagebreak[2]
%% \begin{align*}
%%   \vals{\sum_{i \in I} A_i} &= \sum_{i \in I} \vals A_i &
%%   \chgs{\sum_{i \in I} A_i} &= \sum_{i \in I} \chgs A_i
%% \end{align*}

%% \noindent
%% Updates are pointwise but require that the tags match:

%% \nopagebreak[2]
%% \[ \inj i x \upd \inj j \dx =
%% \begin{cases}
%%   \inj i (x \upd \dx) & \text{if}~i=j\\
%%   \text{undefined} & \text{otherwise}
%% \end{cases}
%% \]


\subsection{Cartesian structure}
\label{sec:CP-cartesian}
\begin{figure*}
  \begin{align*}
    \vals{\prod_{i} A_i} &= \prod_{i} \vals A_i &
    (x_i)_i \upd (\dx_i)_i &= (x_i \upd \dx_i)_i
    \quad\text{when defined}
    \\[1ex]
    \chgs{\prod_{i} A_i} &= \prod_{i} \chgs A_i &
    %% (x_i)_i \changeto (y_i)_i = (x_i \changeto y_i)_i
    (\dx_i)_i \compose (\dy_i)_i &= (\dx_i \compose \dy_i)_i
    \\[1ex]
    \vals{\sum_{i} A_i} &= \sum_{i} \vals A_i &
    \inj i x \upd \inj i \dx &= \inj i (x \upd \dx) \quad\text{when defined}
    \\
    \chgs{\sum_{i} A_i} &= \sum_{i} \chgs A_i &
    \inj i \dx \compose \inj j \dy &= 
    \begin{cases}
      \inj i (\dx \compose \dy) & \text{if}~i=j\\
      \inj i \dx & \text{otherwise}
    \end{cases}
  \end{align*}

  \begin{align*}
    \deriv\pi_i \<x \<\dx &= \pi_i\<\dx &
    \deriv{\fork{f_i}_i} &= \fork{\deriv f_i}_i &
    \deriv\injc_i \<x \<\dx &= \inj i \dx
  \end{align*}

  \begin{equation*}
    \deriv{\krof{f_i}_i} \<(\inj i x) \<(\inj j \dx)
    =
    \begin{cases}
      \deriv f_i \<x \<\dx & \text{if}~i = j\\
      \deriv f_i \<x \<\zero_x & \text{otherwise}
    \end{cases}
  \end{equation*}
  \vspace{-1em}

  \caption{Products and sums in \CP}
  \label{fig:CP-cartesian}
\end{figure*}


\CP\ has all finite products and sums (\cref{fig:CP-cartesian}). Their
value-level structure (including $\pi_i$, $\injc_i$, $\fork{f_i}_i$, and
$\krof{f_i}_i$) is inherited from \Poset; \valfn\ and \chgfn\ distribute across
both products and sums. Soundness and completeness follow directly from the
pointwise ordering of products and sums in \Poset{} and soundness \&
completeness for each factor/summand $A_i$. In particular, \changetofn\ can be
defined by:

\nopagebreak[2]
\begin{align*}
  (x_i)_i \changeto (y_i)_i &= (x_i \changeto y_i)_i &
  \inj i x \changeto \inj i y &= \inj i (x \changeto y)
\end{align*}

\noindent The universal properties of $\fork{f_i}_i$ and $\krof{f_i}_i$ follow
from the corresponding universal properties in \Poset\ --- requiring
differentiability merely weakens them.
%
As for correctness of composition and the derivatives of $\pi_i$/$\injc_i$ and
$\fork{f_i}_i$/$\krof{f_i}_i$, we will show these in turn for products and sums.

\subsubsection{Products}

Correctness of change composition is straightforward: if
%
\[ \vld{(\dx_i)_i}{(x_i)_i}{(y_i)_i}
\quad\text{and}\quad
\vld{(\dy_i)_i}{(y_i)_i}{(z_i)_i} \]
%
then for each $i$ we have $\valid{\dx_i \compose \dy_i}{x_i}{z_i}$ and therefore
%
\[\valid{(\dx_i)_i \compose (\dy_i)_i = (\dx_i \compose
  \dy_i)_i}{(x_i)_i}{(z_i)_i}\]

\noindent
Likewise for correctness of $\deriv\pi_i$; if $\valid{\dx_i}{x_i}{y_i}$ for each
$i$ then by definition
\[\valid{\pi_i\<(\dx_i)_i = \dx_i}{x_i}{y_i}\]

\noindent As for correctness of $\deriv{\fork{f_i}_i}$:

\nopagebreak[2]
\begin{equation*}
  \fork{f_i}_i \<(x \upd \dx)
  = (f_i(x \upd \dx))_i
  = (f_i\<x \upd \deriv f_i \<x \<\dx)_i
  = \fork{f_i}_i \<x \upd \fork{\deriv f_i}_i \<x \<\dx
\end{equation*}

\subsubsection{Sums}

Change composition for sums is interesting, because it must handle changes
$(\inj i \dx \compose \inj j \dy)$ with disjoint tags $i \ne j$. Recall that
composition is correct if, when $\valid\dx x y$ and $\valid\dy y z$, we have
$\valid{\dx\compose\dy} x z$. But if $i \ne j$, then
%
there \emph{are} no $x,y,z$ such that $\valid{\inj i \dx} x y$ and $\valid{\inj j \dy} y z$.
%
This case is \emph{dead code}; the only guarantee we need is monotonicity with
respect to \dx\ and \dy, so it suffices to simply return the left argument.

Correctness of the other case is straightforward: since $i = j$, from the
definition of $\updfn_{\sum_i A_i}$ we know all the tags must agree and we have:

\nopagebreak[2]
\[ \inj i x \longvalidarrow{\inj i \dx} \inj i y
\longvalidarrow{\inj i \dy} \inj i z \]

\noindent From this it follows that \( {x} \longvalidarrow{\dx} {y}
\longvalidarrow{\dy} {z} \) and therefore $\vld{\dx\compose\dy} x z$ and so
finally

\nopagebreak[2]
\[ \valid{\inj i \dx \compose \inj i \dy = \inj i (\dx \compose \dy)}{\inj i x}{\inj i z} \]

\noindent
Correctness of $\deriv\injc_i$ is immediate: suppose $\vld\dx x x$. Then
\(\vld{\inj i \dx}{\inj i x}{\inj i x}\). As for split, its derivative must obey

\nopagebreak[2]
\begin{equation*}
  \krof{f_i}_i\<(\inj i x) \upd \deriv{\krof{f_i}_i} \<(\inj i x) \<(\inj i \dx)
  = \krof{f_i}_i\<(\inj i x \upd \inj i \dx)
  = f_i \<(x \upd \dx)
\end{equation*}

\noindent It's easy to see that the $i = j$ case of the definition in
\cref{fig:CP-cartesian} satisfies this property. As with $\composefn_{\sum_i
  A_i}$, the $i \ne j$ case is dead code.


%% \subsection{Cartesian structure}
%% \label{sec:CP-cartesian}

%% \todo{TODO: Neel suggests splitting this into products \& sums. It's easier to
%%   understand the quirks of sums on their own rather than while also trying to
%%   understand products.}

%% \CP\ has all finite products and sums. The value-level structure is inherited
%% from \Poset\ along \valfn. The change-poset structure is also inherited from
%% \Poset, while the projection and injection's derivatives operate pointwise on
%% changes:

%% \nopagebreak[2]
%% \begin{align*}
%%   \chgs{\prod_{i \in I} A_i} &= \prod_{i \in I} \chgs A_i &
%%   \chgs{\sum_{i \in I} A_i} &= \sum_{i \in I} \chgs A_i
%%   %\label{eqn:delta-distributes-over-sums-and-products}
%%   \\
%%   \deriv\pi_i\<x\<\dx &= \pi_i\<\dx & \deriv\injc_i\<x\<\dx &= \inj i \dx
%% \end{align*}

%% \noindent \todo{TODO: derivatives of fork \& split!}

%% \noindent Updates are pointwise, noting that $(\inj i x \upd \inj j \dx)$ is
%% undefined for $i \ne j$:

%% \nopagebreak[2]
%% \begin{align*}
%%   (x_i)_i \upd (\dx_i)_i &= (x_i \upd \dx_i)_i &
%%   \inj i x \upd \inj i \dx &= \inj i (x \upd \dx)
%% \end{align*}

%% \noindent
%% Soundness \& completeness for increases follow directly from the pointwise
%% orderings of products \& coproducts in \Poset{}. In particular, the $\changetofn$
%% operator is given by:

%% \nopagebreak[2]
%% \begin{align*}
%%   (x_i)_i \changeto (y_i)_i &= (x_i \changeto y_i)_i &
%%   \inj i x \changeto \inj i y &= \inj i (x \changeto y)
%% \end{align*}

%% \noindent Finally, change composition is also pointwise --- mostly:

%% \nopagebreak[2]
%% \begin{align*}
%%   (\dx_i)_i \compose (\dy_i)_i &= (\dx_i \compose \dy_i)_i &
%%   \inj i \dx \compose \inj j \dy &=
%%   \begin{cases}
%%     \inj i (\dx \compose \dy) &\text{if}~i=j\\
%%     \inj i \dx &\text{otherwise}
%%   \end{cases}
%% \end{align*}

%% \noindent
%% However, for sums we must also handle changes with distinct tags. Recall that
%% composition is correct if, when $\valid\dx x y$ and $\valid\dy y z$, we have
%% $\valid{\dx\compose\dy} x z$. But if $i \ne j$, then
%% %
%% there \emph{are} no $x,y,z$ such that $\valid{\inj i \dx} x y$ and $\valid{\inj j \dy} y z$.
%% %
%% This case is \emph{dead code}; the only guarantee we need is monotonicity with
%% respect to \dx\ and \dy, so it suffices to simply return the left argument.

%% Correctness of the other cases follows directly from correctness of each $\composefn_{A_i}$.


\subsection{Exponentials}
\label{sec:CP-exponentials}

The values $\vals{(\expO A B)}$ of the exponential object are the
\emph{differentiable} monotone maps $\CP(A, B)$, ordered pointwise. Because we
require differentiability, \valfn\ does \emph{not} commute with exponentials:
$\vals{(\expO A B)} \ne \expO{\vals A}{\vals B}$.
%
Meanwhile, the changes $\chgs{(\expO A B)}$ are monotone maps $\isof{\vals A}
\to \chgs A \to \chgs B$, ordered pointwise. The update relation and change
composition must be very carefully defined, so we examine them at length below.

\subsubsection{Updating functions}

The update relation is:

\nopagebreak[2]
\begin{equation}\label{eqn:function-update}
  \vld{\df}{f}{g} : \expO A B
  \iff \fa{\vld\dx x y} \vld{\df\<x\<\dx} {f\<x} {g\<y}
\end{equation}

\noindent However, recall that $\upd$ must be a partial function: from
$\valid\df f g$ and $\valid\df f h$, we must show $g = h$. So consider any $x :
A$. \Cref{eqn:function-update} applied to $\valid{\zero_x} x x$ yields

\nopagebreak[2]
\[ \vld{\df \<x \<\zero_x}{f \<x}{g\<x}
\quad\text{and}\quad \vld{\df \<x \<\zero_x}{f \<x}{h\<x} \]

\noindent
Then by functionality of $\updfn_B$ we have $g\<x = h\<x$, and thus $g = h$.

Although this shows that \cref{eqn:function-update} defines a partial function,
one may wonder what it actually \emph{does} (when it is defined). So let's
rewrite \cref{eqn:function-update} in terms of $\upd_{\expO A B}$:

\nopagebreak[2]
\[ f \upd \df = g \iff \fa{x \upd \dx = y} f\<x \upd \df\<x\<\dx = g\<y \]

\noindent Substituting away $g$ and $y$, this implies that for all valid
\dx, \df:

\nopagebreak[2]
\[
f\<x \upd \df\<x\<\dx = (f \upd \df) \<(x \upd \dx)
\]

\noindent In particular, letting $\dx = \zero_x$ and swapping sides, we get:

\nopagebreak[2]
\[ (f \upd \df) \<x = f\<x \upd \df\<x\<\zero_x \]

\noindent
\todo{TODO: explain why this does not suffice as a \emph{definition} of
  $\updfn_{\expO A B}$ --- in particular, they don't explain when $f \upd \df$
  is defined! It's \emph{not} simply when $x \mapsto f\<x \upd \df\<x\<\dx$ is
  monotone, for example. Can I find a counterexample?}

\subsubsection{Soundness and completeness}

Soundness follows from completeness at $A$ and soundness at $B$: if $\valid\df f
g$ then for any $x$ we have $\valid{\df\<x\<\zero_x}{f\<x}{g\<x}$ (using
completeness of $A$ to find $\zero_x$), and by soundness at $B$ we have $f\<x
\le g\<x$, so $f \le g$.

We'll show completeness by constructing \changetofn\ explicitly:

\nopagebreak[2]
\[ (f \changeto_{\expO A B} g) \<x \<\dx = (f\<x \changeto_B g\<x) \compose_B \deriv g\<x\<\dx\]

\noindent Observe that $f\<x \changeto_B g\<x$ is defined because $f \le g$
implies $f\<x \le g\<x$, and the RHS is monotone in \dx\ by monotonicity of
$\deriv g \<x$ and of $\composefn_B$.
%
To prove completeness, assuming $\valid\dx x y$, we need to show \( \valid{(f
  \changeto g) \<x \<\dx}{f\<x}{g\<y} \). This follows from correctness of
$\composefn_B$, since \( \valid{f\<x \changeto g\<x}{f\<x}{g\<x} \) and
\(\valid{\deriv g \<x \<\dx}{g\<x}{g\<y}\) by definition.

\subsubsection{Function change composition}

Change composition is defined by:

\nopagebreak[2]
\[ (\df \compose_{\expO A B} \dg) \<x \<\dx = \df \<x \<\zero_x \compose_B \dg \<x \<\dx\]

\noindent
This is monotone in \dx\ by monotonicity of $\dg\<x$ and $\composefn_B$. To show
this valid, suppose $\vld\df f g$, $\vld\dg g h$ and $\vld\dx x y$. Then
%
\(\vld{\df\<x\<\zero_x}{f\<x}{g\<x}\)
\text{and}
\(\vld{\dg\<x\<\dx}{g\<x}{h\<y} \).
%
\noindent So by correctness of $\composefn_B$ we have \(\valid{(\df\<x\<\zero_x
  \compose \dg\<x\<\dx)}{f\<x}{h\<y}\) as desired.

\todo{TODO: discuss unavoidable asymmetry in definitions of \(\changetofn_{\expO
    A B}\) and \(\composefn_{\expO A B}\).}

\subsubsection{\eval\ and \fn}

To show that $\expO A B$ is an exponential object, we need evaluation and
currying maps, \eval\ and $\fn f$. Modulo differentiability, these are
inherited from \Poset{}, with the following derivatives:

\begin{align*}
  %% \eval &: (\expO A B) \x A \to B &
  %% \fn (f : \G \x A \to B) &: \G \to \expO A B
  %% \\
  \vals{\eval} \<(f,x) &= f\<x &
  \vals{(\fn f)} \<x &= y \mapsto f\<(x,y)\\
  \deriv{\eval} \<(f,x) \<(\df,\dx) &= \df \<x \<\dx &
  \deriv{(\fn f)} \<x \<\dx &= y\<\dy \mapsto \deriv f \<(x,y) \<(\dx,\dy)
\end{align*}

\noindent
$\deriv{\eval}$ is correct by the definition of $\updfn_{\expO A B}$, for if
$\valid\df f g$ and $\valid\dx x y$ then

\nopagebreak[2]
\[ \eval\<(f,x) = {f\<x} \longvalidarrow{\df\<x\<\dx} {g\<y} = \eval\<(g,y) \]

\noindent
Similarly, \(\deriv{(\fn f)}\) is correct, for suppose $\valid\dx {x_1}{x_2}$
and $\valid\dy{y_1}{y_2}$. Then by differentiability of $f$ we have

\nopagebreak[2]
\[ \fn f \<x_1 \<y_1 =
f\<(x_1,y_1) \longvalidarrow{\deriv f \<(x_1,y_1) \<(\dx,\dy)} f\<(x_2,y_2)
= \fn f \<x_2 \<y_2
\]

\noindent
Finally, $\fn f$ is unique such that $(\fn f \x \id) \then \eval = f$ by the
same argument as in \Poset{} (or indeed \cat{Set}): supposing $\eval \<(\fn f
\<x, y) = f\<(x,y) = \eval\<(g\<x,y)$ for all $x,y$, then we have $\fn f \<x \<
y = g \<x \<y$ and so $\fn f = g$.


\subsection{The pseudo-exponential}

\newcommand\pseudoexp{\rightrightarrows}
%\renewcommand\pseudoexp\rightarrowtriangle
%\renewcommand\pseudoexp{\overset{?}{\Rightarrow}}

You might have expected changes $\chgs{(\expO A B)}$ to be given pointwise, as
maps $\isof{\vals A} \to \chgs B$. Intriguingly, this does yield a valid change
poset, one rather simpler than $\expO A B$, but it does not satisfy the
properties of an exponential. I call this the \emph{pseudo-exponential}, $A
\pseudoexp B$. Pseudo-exponentials are irrelevant to Datafun, but are an
interesting and surprising aspect of \CP, so I describe them here. They are
constructed as follows:

\vspace{-1.2ex}
\begin{mathpar}
  \vals{(A \pseudoexp B)} = \CP(A,B)

  \chgs{(A \pseudoexp B)} = \isof{\vals A} \to \chgs B

  \vld{\df} f g : A \pseudoexp B \iff \fa{x} \vld{\df\<x}{f\<x}{g\<x}

  (\df \compose_{A \pseudoexp B} \dg) \<x = \df\<x \compose_B \dg\<x

  (f \changeto g) \<x = f\<x \changeto g\<x
\end{mathpar}

\noindent
I leave it as an exercise for the reader to verify soundness \& completeness and
correctness of composition. \todo{TODO: explain why they aren't the exponential
  object and aren't isomorphic to $\expO A B$.}

%% Besides $\expO A B$, there is another, simpler, change poset which one might
%% expect to be the exponential, but which is not, which I term the
%% \emph{pseudo-exponential}, written $A \pseudoexp B$.


\subsection{The lifted \iso\ comonad}
\label{sec:CP-iso}

Now we construct a comonad $\iso$ on \CP\ such that $\valfn\iso_\CP =
\iso_\Poset\valfn$. This requirement defines the value-level structure, leaving
only the change poset, update map, and composition. Since $\isof A$ is ordered
discretely, values \emph{cannot} increase. So the simplest definition is:

\nopagebreak[2]
\begin{align*}
  \chgs{\isof A} &= \termO &
  x \upd_{\isof A} \tuple{} &= x &
  \tuple{} \compose_{\isof A} \tuple{} &= \tuple{}
\end{align*}

\noindent
This trivially satisfies soundness, completeness, and correctness of
composition.
%
Functoriality, extraction, duplication, and product- and sum-distribution
inherit from the corresponding properties of $\iso_\Poset$, and their
derivatives are fairly straightforward:

\nopagebreak[2]\vspace{-1ex}
\begin{mathpar}
  \deriv{\iso(f)} \<x \<\tuple{} = \tuple{}

  \deriv\delta_A \<x \<\tuple{} = \tuple{}

  \deriv\varepsilon_A \<x \<\tuple{} = \zero_x
  \\
  \deriv{{\discox}} \<x \<\dx = \tuple{}

  \deriv{{\discosum}} \<(\inj i x) \<\tuple{} = \inj i {\tuple{}}
\end{mathpar}


\subsubsection{The alternative \altiso}

Although we will not use it, there is another refinement of $\iso$ in \CP{},
which I'll call $\altiso$ for clarity, defined like so:

\nopagebreak[2]
\begin{align*}
  \chgs{\altiso A} &= \isof{\chgs A} &
  %% x \upd_{\altiso A} \dx = y &\iff x = y \wedge x \upd_A \dx = y
  \vld{\dx} x x : \altiso A &\iff \vld{\dx} x x : A
  &
  \composefn_{\altiso A} &= \composefn_A
\end{align*}

\noindent
So $\altiso A$ inherits all of $A$'s changes, but ordered discretely and with
\emph{only zero-changes still valid}. Soundness and completeness for increases
and correctness of composition are all implied by the corresponding properties
of $A$. The derivatives of functoriality, duplication, and extraction are
different:

\nopagebreak[2]
\begin{align*}
  \deriv{\altiso(f)} &= \deriv f
  & \deriv\delta_A \<x\<\dx &= \dx
  & \deriv\varepsilon_A \<x\<\dx &= \dx
\end{align*}

\noindent
Observe that we no longer need the non-constructive $\zero_x$ to define comonad
extraction $\varepsilon$! However, for our purposes this will not be an issue.
Meanwhile, product- and sum-distribution become identity maps, because $\prod_i
\altiso A_i = \altiso \prod_i A$ and $\sum_i \altiso A_i = \altiso \sum_i
A_i$ (checking this is left as an exercise for the reader).


\subsection{The lifted \pfin\ functor}
\label{sec:CP-pfin}

We lift the finite powerset functor \pfin\ to \CP\ as follows:

\nopagebreak[2]
\begin{align}
  \vals{\pfinof A} &= \pfinof{\vals A} &
  \chgs{\pfinof A} &= \pfinof{\vals A} &
  (\updfn_{\pfinof A}) = (\composefn_{\pfinof A}) = ({\cup})
  \label{eqn:CP-pfin}
\end{align}

\noindent Since $\updfn$ and \composefn\ are just $\cup$, soundness and
completeness come from a standard property of union (or more generally of
semilattice join):

\nopagebreak[2]
\[ x \le z \iff \ex{y} x \cup y = z \]

\noindent Meanwhile, correctness of composition is simply associativity of
$\cup$:

\nopagebreak[2]
\[ (x \cup \dx = y) \wedge (y \cup \dy = z) \implies (x \cup \dx \cup \dy = z) \]

\noindent
Since the arguments to \morph{singleton} and \morph{isEmpty} cannot change
(having $\iso$ type), their derivatives always return zero-changes:

\nopagebreak[2]
\begin{align*}
  \deriv{\morph{singleton}}_A \<x \<\tuple{} &= \emptyset &
  \deriv{\morph{isEmpty}} \<x \<\tuple{} &= \morph{isEmpty} \< x
\end{align*}



\subsection{Equality objects}
\label{sec:CP-eq}

Every object in \CP{} is an equality object, with \morph{eq} and its derivative
defined by:

\nopagebreak[2]
\begin{align*}
  \morph{eq}\<\tuple{x,y} &= 
  \begin{cases}
    \{\tuple{}\} & \text{if}~ x = y\\
    \emptyset & \text{otherwise}
  \end{cases}
  &
  \deriv{\morph{eq}} \<\pwild \<\pwild &= \emptyset
\end{align*}

\noindent Note that $\deriv{\morph{eq}}$ is correct because $\morph{eq}$'s
arguments cannot change, having $\iso$ type.

Since \emph{every} object is an equality object, every equality type
is trivially interpreted into an equality object. \todo{TODO more explanation
  --- why isn't every type an equality type, then?}


\subsection{Semilattice objects and collecting morphisms}
\label{sec:CP-semilattice}

A semilattice object $L$ in \CP\ is one where $\vals L = \chgs L$ is a
semilattice and $x \upd_L y = x \vee y$. \todo{(TODO: we might want to require
  $\composefn_L = {\vee}$ as well, if we need it.)} This immediately justifies
letting $\deriv{\morph{join}} \<x \<\dx = \morph{join}\<\dx$, because:

\nopagebreak[2]
\begin{align*}
  \morph{join}\<(x \upd \dx) &= \bigvee_i \pi_i \<(x \vee \dx)\\
  &= \left(\bigvee_i \pi_i\<x\right) \vee \left(\bigvee_i \pi_i\<\dx\right)\\
  &= \morph{join}\<x \vee \morph{join}\<\dx\\
  &= \morph{join}\<x \upd \deriv{\morph{join}}\<x\<\dx
\end{align*}

\begin{restatable}{lemma}{lemSemi}
  \label{lem:semi} $\incdens L$ is a semilattice object. \todo{FIXME: haven't
  introduced $\incdenfn$ yet!}
\end{restatable}
\begin{proof}
  By induction on $L$. \XXX
  %\todo{See \cpageref{proof:semi}.}
\end{proof}


\subsection{Collecting morphisms}
\label{sec:CP-collect}

Let $\G, A$ be objects, $L$ be a semilattice object, and $f : \G \x \isof A \to
L$. Our derivative for $\pcollect f : \G \x \pfinof A \to L$ is:

\nopagebreak[2]
\begin{align*}
  \deriv{\pcollect f} \<\tuple{\gamma,s} \<\tuple{\dgamma,\ds}
  &= \bigvee_{x \in \ds} f\<\tuple{\gamma,x} \vee
  \bigvee_{x \in s \cup \ds} \deriv f \<\tuple{\gamma,x} \<\tuple{\dgamma, \zero_x}
\end{align*}

\noindent To establish that this is correct, suppose
$\vld{\dgamma}{\gamma}{\gamma'} : \G$. From this we have:

\nopagebreak[2]
\begin{align*}
  &\phantom{{}={}}\pcollect{f} \<\tuple{\gamma', s \cup \ds}
  \\
  &= \bigvee_{x \in s \cup \ds} f \<\tuple{\gamma', x}
  \\
  &\overset{\footnotemark}{=} \bigvee_{x \in s \cup \ds} \bigl(f \<\tuple{\gamma,x}
  \vee \deriv f \<\tuple{\gamma,x} \<\tuple{\dgamma,\zero_x}\bigr)
  \\
  &= \bigvee_{x \in s} f\<\tuple{\gamma,x}
  \upd \bigvee_{x \in \ds} f\<\tuple{\gamma,x}
  \vee \bigvee_{x \in s \cup \ds} \deriv f \<\tuple{\gamma,x} \<\tuple{\dgamma,\zero_x}
  \\
  &= \pcollect f \<\tuple{\gamma,s}
  \upd \deriv{\pcollect f} \<\tuple{\gamma,s} \<\tuple{\dgamma,\ds}
\end{align*}

\nopagebreak% for footnote to be on same page as mark
\noindent
Which is what we wished to show.\footnotetext{This equality follows from the
  correctness of $\deriv f$ applied to
  \(\vld{\tuple{\dgamma,\zero_x}}{\tuple{\gamma,x}}{\tuple{\gamma',x}}\).}


\subsection{Fixed point objects}
\label{sec:CP-fix}

A fixed point object $\fixt L$ in \CP\ is a semilattice object such that
$\vals{\fixt L}$ satisfies the ascending chain condition. \todo{TODO more
  explanation.} Since the argument to \morph{fix} is boxed and cannot change,
$\deriv{\morph{fix}}$ needs to produce a zero change; and at any semilattice
object, $\bot$ is a zero change, so \( \deriv{\morph{fix}} \<x \<\tuple{} = \bot
\).

\begin{lemma}\label{lem:fixobject}
  \(\incdens{\fixt L}\) is a fixed point object.
\end{lemma}
\begin{proof}
  By induction on $\fixt L$. \todo{TODO}
\end{proof}


\section{Incrementalization}
\label{sec:incremental}

%% The \emph{incremental semantics} for Datafun, notated $\incden{-}$, is given by
%% the Datafun model in \CP{} constructed in \cref{sec:changeposets}, letting
%% $\disco = \iso_\CP$ and $\pfin = \pfin_\CP$. The obvious question is: how does
%% this semantics relate to the standard \Poset\ semantics, $\den{-}$? \todo{TODO:
%% Explain how they are mostly the same if you apply \valfn, but exponentials
%% crucially aren't. So we tried to use a logical relation, like that
%% in \cref{fig:agrees}.}

%% Recalling
%% \cref{sec:CP-vals}, since most of our constructions inherit along $\valfn$,
%% restricted to the \Poset-level they are very nearly the same!
%% %
%% However, $\vals{\incden{A \to B}} \ne \den{A \to B}$ because of the
%% differentiability restriction on \CP's exponentials.
%% %
%% \todo{TODO: more explanation}

\todo{TODO: Explain that \CP\ gives as incremental semantics for Datafun. We
suspect (but haven't yet proven; back-burner) that you can connect this to the
regular semantics for Datafun using a logical relation (\cref{fig:agrees}).

However, ultimately the motivation for \CP\ was to give a semantic foundation
for semi\naive{} evaluation. But it's not clear how to do this, because
the \emph{morphisms} and \emph{values} in \CP\ are ``hereditarily
differentiable'', but the \emph{changes} (eg. of the exponential) are \emph{not}
necessarily differentiable, and this causes no end of pain.
%
In particular, there seems to be no way to re-use the validity relations from
objects in \CP\ to prove correctness of the semi\naive{} transform. Because
$\chgs(\expO A B) = \isof{\vals A} \to \chgs A \to \chgs B$ mixes differentiable
and non-differentiable things, we can't use it either on values we get out of
the \Poset{}-semantics or out of the the \CP-semantics.}

\newcommand\agrees\approx
\begin{figure*}
  \begin{align*}
    {\agrees_A} &\subseteq \vals{\incdens{A}} \x \den{A} &
    x \agrees_{\tseteq A} y &\iff x = y\\
    %% \tuple{x_1,x_2} \agrees_{A_1 \x A_2} \tuple{y_1,y_2} &\iff \fa{i} x_i \agrees_{A_i} y_i &
    \tuple{x_i}_i \agrees \tuple{y_i} &\iff \fa{i} x_i \agrees y_i &
    \inj i x \agrees \inj j y &\iff i = j \wedge x \agrees y\\
    x \agrees_{\isof A} y &\iff x \agrees y &
    f \agrees_{A \to B} g &\iff \fa{x \agrees_A y} f\<x \agrees_B g\<y
  \end{align*}

  \todo{TODO: Add explanation that $\agrees_{\tseteq A}$ makes sense because of
    \cref{thm:agree-first-order}}

  \caption{Agreement relation between $\incdenfn$ and $\den{-}$}
  \label{fig:agrees}
\end{figure*}


\begin{restatable}[First-order semantic agreement]{conjecture}{thmAgreeFirstOrder}
  \label{thm:agree-first-order}
  \( \vals{\incdens{\eqt A}} = \den{\eqt A} \).
\end{restatable}
\begin{proof}
  \todo{TODO, by induction on types}
\end{proof}

\begin{restatable}[First-order agreement is equality]{conjecture}{thmAgreeEqualsEquals}
  \(x \agrees_{\eqt A} y \iff x = y\).
\end{restatable}
\begin{proof}
  \label{thm:agree-equals-equals}
  \todo{TODO, by induction on eqtypes.}
\end{proof}

\begin{restatable}[Semantic agreement]{conjecture}{thmAgree}
  If $\J e \G A$
  %and $\g \agrees_\G \sigma$ then $\incdens{e}\<\g \agrees_A \den{e}\<\sigma$.
  then $\incdens{e} \agrees_{\G \vdash A} \den{e}$.
\end{restatable}
\begin{proof}
  \todo{TODO by induction on typing derivations}
\end{proof}

%% \noindent
%% The change poset $\D\incdens A$ of a Datafun type $A$ is internally definable
%% via a type translation given in \cref{fig:PhiDelta}, also named $\D$, despite
%% the risk of confusion. \todo{TODO: move this \P\ into \cref{sec:seminaive}.}

%% \begin{restatable}{theorem}{thmDeltaDen}
%%   \label{thm:delta-den}
%%   \(\D\incdens A = \vals{\incdens{\D A}}\).
%% \end{restatable}
%% \nopagebreak[2]
%% \begin{proof}
%%   \label{proof:delta-den}
%%   By induction on types. A starred equality, $\stareq$, marks equalities given by
%%   an inductive hypothesis; additionally, $\daggereq$ indicates an appeal to
%%   \cref{thm:agree-first-order}.
%%   \begin{description}
%%     \item[Case $\isof A$:] \(\D\incdens{\isof A} = \D{\isof{\incdens A}} =
%%       \termO_\Poset = \vals\termO_\CP = \vals{\incdens{\tunit}} \).

%%     \item[Case $\tseteq A$:]
%%       \(\D\incdens{\,{\tseteq A}\,} = \D\pfinof{\incdens{\eqt A}}
%%       = \pfinof{\vals{\incdens{\eqt A}}}
%%       = \vals{\pfinof{\incdens{\eqt A}}}
%%       = \vals{\incdens{\,{\tseteq A}\,}}
%%       = \vals{\incdens{\D{\tseteq A}}}
%%       \).

%%     \item[Case $A \x B$:]
%%       \(\D\incden{A \x B} = \D(\incdens A \x \incdens B)
%%       = \D\incdens A \x \D\incdens B
%%       \stareq \vals{\incdens{\D A}} \x \vals{\incdens{\D B}}
%%       = \vals{\incden{\D A \x \D B}}
%%       = \vals{\incdens{\D(A \x B)}}
%%       \).

%%     \item[Case $A + B$:]
%%       \(\D\incden{A + B} = \D(\incdens A + \incdens B)
%%       = \D\incdens A + \D\incdens B
%%       \stareq \vals{\incdens{\D A}} + \vals{\incdens{\D B}}
%%       = \vals{\incden{\D A + \D B}}
%%       = \vals{\incdens{\D(A + B)}}
%%       \).

%%     \item[Case $A \to B$:]
%%       \begin{align*}\D\incden{A \to B}
%%         &= \D(\expO{\incdens A}{\incdens B})\\
%%         &= \expO{\isof{\vals{\incdens
%%               A}}}{\expO{\D\incdens{A}}{\D\incdens{B}}}\\
%%         &= ...\\
%%         &= \CP(\isof{\incdens A},\ {\expO{\incdens{\D A}}{\incdens{\D B}}})\\
%%         &= \vals{(\expO{\isof{\incdens{A}}}{\expO{\incdens{\D A}}{\incdens{\D B}}})}\\
%%         &= \vals{\incden{\isof A \to \D A \to \D B}}\\
%%         &= \vals{\incdens{\D(A \to B)}}
%%       \end{align*}

%%     \item[Case $\tunit$:]
%%   \end{description}
%% \end{proof}

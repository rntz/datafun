\title{Change Semantics for Semi\naive{} Datafun}
\author{\scshape Michael Arntzenius}
\date{\today}

\maketitle

\begin{abstract}
  We generalize semi\naive{} evaluation from Datalog to Datafun and prove it
  correct.
\end{abstract}

\section{Outline}

The key insight behind \emph{semi\naive{} evaluation} is to find fixed points
faster by computing only what changes between iterations (\cref{sec:fastfix}).
To apply this in Datafun, we first need to understand what \emph{change} means.
%
To this end, we generalize Datafun's \Poset{} semantics to any category with
appropriate structures (\cref{sec:datafun-models}).

\todo{We construct such a category, \CP{}, whose objects are posets equipped
with a notion of increasing changes, and whose morphisms are monotone maps
possessing \emph{derivatives} relating input and output changes
(\cref{sec:changeposets}). This yields an incremental semantics for Datafun
(\cref{sec:incremental}).}

Nonetheless, we use \CP\ as a rough guide when constructing a semi\naive{}
evaluation static transformation (\cref{sec:seminaive}). This transformation
reinterprets the discreteness comonad to compute derivatives, and uses these to
speed up fixed point computations. Finally, we prove this transformation correct
using logical relations (\cref{sec:logical-relations}).

% TODO: more section links once I finish them.


\section{Notation and conventions}

\begin{enumerate}
\item Adjacency $f\<x$ denotes the function $f$ applied to an argument $x$.

\item I write $f \then g$ for the composition of morphisms $f : A \to B$ and $g
  : B \to C$. This is the reverse of the usual composition operator, $g \circ
  f$.

\item I write $\subst{e}{x \substo f}$ to mean ``$e$ with all free occurrences of $x$
  replaced by $f$ ($\alpha$-varying as necessary to avoid capture)''.

  \newcommand\iexpr{\phi}
\item I write $(\iexpr_i)_i$ for the tuple or sequence $(\iexpr_1,
  ..., \iexpr_n)$, leaving $n$ unspecified. For example,

  \nopagebreak[2]
  \[ (2^ix + y_i)_i \quad\text{means}\quad
  (2^1x + y_1,~ 2^2x + y_2,~ ...,~ 2^nx + y_n) \]

  \noindent
  I also use this notation with other braces, eg. $\fork{\iexpr_i}_i =
  \fork{\iexpr_0, ..., \iexpr_n}$.

\item $\expO A B$ denotes the exponential $B^A$ in a cartesian closed category.

\item Given a family of morphisms $f_i : A \to B_i$ in a category with finite
  products, their \emph{``fork''} $\fork{f_i}_i : A \to \textstyle\prod_{i} B_i$
  uniquely satisfies $\fork{f_i}_i \then \pi_i = f_i$. Dually, given $f_i : A_i
  \to B$ in a category with finite sums, their \emph{``split''} $\krof{f_i}_i :
  \sum_i A_i \to B$ uniquely satisfies $\injc_i \then \krof{f_i}_i = f_i$.
  %
  So in particular, $\termI : A \to \termO$ is the unique map into the terminal
  object and $\initE : \initO \to B$ the unique map out of the initial object.

\item The comonad $\iso : \Poset \to \Poset$ takes a poset to the
  \emph{discrete} poset on the same elements, defined:

  \begin{eqnarray*}
    x \in \iso A &\iff& x \in A\\
    x \le y : \iso A &\iff& x = y\\
    \iso(f : A \to B) &=& f : \iso A \to \iso B
  \end{eqnarray*}
\end{enumerate}


\section{Fixed points and \fastfix}
\label{sec:fastfix}

\todo{TODO: more explanation.} Let $L$ be a semilattice satisfying the ascending
chain condition. Fix monotone maps $f : L \to L$ and $\deriv f : L \to L \to L$,
and consider the sequences $x_i$, $\dx_i$ defined by:

\nopagebreak[2]
\begin{align*}
  x_0 &= \bot
  &
  x_{i+1} &= x_i \vee \dx_i\\
  \dx_0 &= f\< x
  &
  \dx_{i+1} &= \deriv f \<x_i \<\dx_i
\end{align*}

\noindent
Let $\fastfix\<(f,\deriv f) = \bigvee_i x_i$. The existence of this least upper
bound is guaranteed by the ascending chain condition, since $x_i \le x_i \vee
\dx_i = x_{i+1}$.

\begin{restatable}[\prim{fix} and \fastfix]{theorem}{thmFastFix}\label{thm:fast-fix}
  If $\fa{x,\dx} f \<x \vee \deriv f \<x \<\dx = f \<(x \vee \dx)$,
  then \(\efix f = \fastfix\<(f,\deriv f)\).
\end{restatable}
\begin{proof}
  Assume $\fa{x,\dx} f\<x \vee \deriv f \<x \<\dx = f\<(x \vee \dx)$. It suffices
  to show $\fa{i} x_{i+1} = f \<x_i$, which we do by induction on $i$. The base
  case is $x_1 = x_0 \vee \dx_0 = \bot \vee f\<\bot = f\<x_0$. Inductively,
  assuming $x_{i+1} = f\<x_i$, we have:

  \nopagebreak[2]
  \begin{align*}
    x_{i+2} &= x_{i+1} \vee \dx_{i+1} && \text{definition of $x_{i+2}$}\\
    &= f\<x_i \vee \deriv f\<x_i\<\dx_i && \text{inductive hypothesis, definition of $\dx_{i+1}$}\\
    &= f\<(x_i \vee \dx_i) && \text{assumption}\\
    &= f\<x_{i+1} && \text{definition of $x_{i+1}$}
  \end{align*}
\end{proof}


\begin{figure*}\centering
  \begin{mathpar}
    \setlength\arraycolsep{.33em}\begin{array}{r@{\hskip 1em}ccl}
      \text{types} & A,B &\bnfeq& \iso A \bnfor \tset{\eqt A}
      \bnfor \tunit \bnfor A \x B \bnfor A + B \bnfor A \to B
      \\
      \text{eqtypes} & \eqt A, \eqt B &\bnfeq&
      %% TODO: revamp to make \iso an eqtype. this requires \Phi(\iso{\eqt A})
      %% = \iso{\Phi{\eqt A}} = \iso{\eqt A} and changing \phi & \delta
      %% accordingly.
      %\iso{\eqt A} \bnfor
      \tset{\eqt A} \bnfor
      \tunit \bnfor \eqt A \x \eqt B \bnfor \eqt A + \eqt B
      \\
      \text{semilattices} & L,M &\bnfeq& \tset{\eqt A} \bnfor \tunit \bnfor L \x M
      \\
      \text{finite eqtype} & \fint A, \fint B &\bnfeq&
      \tset{\fint A} \bnfor \tunit \bnfor \fint A \x \fint B
      \\
      \text{fixtypes} & \fixt L, \fixt M &\bnfeq&
      \tset{\fint A} \bnfor \tunit \bnfor \fixt L \x \fixt M
      \\[.5em]
      \text{terms} & e,f,g &\bnfeq& x \bnfor \dvar x \bnfor \fnof{x} e
      \bnfor e\<f \bnfor \etuple{} \bnfor \etuple{e,f} \bnfor \prj i e\\
      &&& \inj i e \bnfor \ecase{e} (\inj i x \caseto f_i)_{i\in\{1,2\}}\\
      &&& \eboxd e \bnfor \elet{\eboxd x = e} f\\
      &&& \bot \bnfor e \vee f \bnfor \esetd{e_i}_i \bnfor \eford x e f\\
      &&& \eeqd e f \bnfor \eisempty e \bnfor \esplit e \bnfor \efix e
      \\[.5em]
      \text{contexts} & \G &\bnfeq& (H_i)_i\\
      \text{hypotheses} & H &\bnfeq& \h x A \bnfor \hd x A
    \end{array}
    \\
    \stripcxd\G = (\hd x A)_{\hd x A \in \G}
    \\
    \infer{\h x A \in \G}{\J x \G A}

    \infer{\hd x A \in \G}{\J {\dvar x} \G A}

    \infer{\J e {\G,\h x A} B}{\J {\fnof x e} \G {A \to B}}

    \infer{\J e \G {A \to B} \\ \J f \G A}{\J {e\<f} \G B}

    \infer{\quad}{\J {\etuple{}} \G \tunit}

    \infer{(\J{e_i}\G{A_i})_i}{\J{\etuple{e_1,e_2}} \G {A_1 \x A_2}}

    \infer{\J e \G {A_1 \x A_2}}{\J{\prj i e}\G{A_i}}

    \infer{\J e \G A_i}{\J{\inj i e}\G{A_1 + A_2}}

    \infer{\J e \G {A_1 + A_2} \\
      (\J {f_i} {\G,\h {x_i} {A_i}} {B})_i
    }{
      \J {\ecase{e} (\inj i x_i \caseto f_i)_i} \G B
    }

    \infer{\J {\isocolor e} {\stripcxd\G} A}{\J{\eboxd e} \G {\iso A}}

    \infer{\J e \G {\iso A} \\ \J f {\G,\hd x A} B}{
      \J {\elet{\eboxd x = e} f} \G B}

    \infer{\quad}{\J\bot\G {\eqt L}}

    \infer{(\J{e_i} \G {\eqt L})_i}{\J{e_1 \vee e_2}\G {\eqt L}}

    %% \infer{\J e \G {\eqt A}}{\J {\edown e} \G {\tdown {\eqt A}}}
    \infer{(\J {\isocolor e_i} {\stripcxd\G} {\eqt A})_i}{
      \J {\esetd{e_i}_i} \G {\tset{\eqt A}}}

    %% \infer{\J e \G {\tdown {\eqt A}} \\
    %%   \J f {\G,\h x {\eqt A}} L
    %% }{\J {\ebigvee x e f} \G L}
    %%
    \infer{
      \J e \G {\tset A} \\
      \J f {\G,\hd x A} {\eqt L}
    }{\J {\eford x e f} \G {\eqt L}}

    %%\infer{\J e \G {\iso{(\eqt A \x \eqt A)}}}{\J{\prim{eq}\<e} \G {\tdown\tunit}}
    \infer{(\J {\isocolor e_i} {\stripcxd\G} {\eqt A})_i}
          {\J {\eeqd{e_1}{e_2}} \G \tbool}

    \infer{\J {\isocolor e} {\stripcxd\G} {\tset\tunit}}{
      \J {\eisempty e} \G {\tunit + \tunit}}

    \infer{\J e \G {\iso{(A + B)}}}{\J{\esplit e} \G {\iso A + \iso B}}

    \infer{\J e \G {\iso{(\fixt L \to \fixt L)}}}{\J{\prim{fix}\< e} \G {\fixt L}}
  \end{mathpar}

  \caption{Datafun core syntax and typing rules}
  \label{fig:core-datafun}
\end{figure*}

\begin{figure*}
  \[\setlength\arraycolsep{.33em}\begin{array}{r@{\hskip 1em}ccl}
    \text{types} & A,B &\bnfeq& ... \bnfor \tbool\\
    \text{terms} & e,f,g &\bnfeq&
    ... \bnfor \elet{x = e} f \bnfor \efixisd x e\\
    &&& \efalse \bnfor \etrue \bnfor \eif e f g \bnfor \ewhen e f
    % NB. no more \edcase because we just use pattern-matching elaboration
    %\\ &&& %\edcase{e} (\inj i {\dvar{x_i}} \caseto f_i)_{i\in\{1,2\}}
  \end{array}\]

  \begin{mathpar}
    \infer{\quad}{\J\bot\G L}

    \infer{(\J{e_i} \G L)_i}{\J{e_1 \vee e_2}\G L}

    \infer{\J e \G {\tseteq A} \\ \J f {\G,\hd x A} L}{
      \J {\eford x e f} \G L}

    \infer{\J e \G A \\ \J f {\G,\h x A} B}{\J{\elet{x = e} f} \G B}

    \infer{\J {\isocolor e} {\stripcxd{\G}, \h x {\fixt L}} {\fixt L}}
          {\J {\efixisd x e} \G {\fixt L}}

    \infer{\quad}{\J \etrue \G \tbool}

    \infer{\quad}{\J \efalse \G \tbool}

    \infer{\J {\isocolor e} {\stripcxd\G} \tbool \\ (\J{f_i}\G B)_i}
          {\J {\eifd e{f_1}{f_2}} \G {B}}

    \infer{\J e \G \tbool \\ \J f \G L}{\J {\ewhen e f} \G L}
  \end{mathpar}

  \caption{Surface syntax and typing rules}
  \label{fig:surface-syntax}
\end{figure*}

\begin{figure}\centering
  \begin{align*}
    \tbool &\desugars \tset{\tunit}
    \\
    %% \eeqd e f &\desugars \prim{eq}\< \eboxd{\etuple{e, f}}\\
    %% \esetd{e_i}_i &\desugars \edown\eboxd{e_0} \vee ... \vee \edown\eboxd{e_n} \quad \text{(or $\bot$ if $n=0$)}\\
    %% \eford x e f &\desugars \ebigvee {\freshvar y} e {\elet{\eboxd x = \freshvar y} f}\\
    \eifd e {f_1}{f_2} &\desugars
    \ecase{\eisEmpty e} (\inj i {\pwild} \caseto f_i)_i
    %% %% NB. This is mostly redundant with the pattern-matching elaboration in fig:
    %% \edcase e (\inj i {\dvar{x_i}} \caseto f_i)_i
    %% &\desugars \ecase{\esplit \eboxd e}
    %% (\inj i {\freshvar y} \caseto \elet{\eboxd{x_i} = \freshvar y} f_i)_i
    %% \\
  \end{align*}

  \begin{align*}
    \efalse &\desugars \esetd{} &
    \etrue &\desugars \esetd{\etuple{}}
    \\
    \elet{x = e} f &\desugars (\fnof{x} f)\<e &
    \efixisd x e &\desugars \efix \eboxd{\fnof {\color{black} x} e}
  \end{align*}

  \caption{Desugaring of surface syntax}
  \label{fig:desugaring}
\end{figure}

%\begin{figure*}\centering
  \begin{tabular}{@{}rll@{}}
    Surface term & expanded at $A \to L$ & expanded at $L \x M$\\\midrule
    $\bot$
    & \(\fnof{\freshvar\pwild} \bot\)
    & \(\etuple{\bot,\bot}\)
    \\
    \(e \vee f\)
    & \(\fnof{\freshvar x} e\<{\freshvar x} \vee f\<{\freshvar x}\)
    & \(\etuple{\pi_1\<e \vee \pi_1\<f,\, \pi_2\<e\vee\pi_2\<f}\)
    \\
    \(\eford x e f\)
    & \(\fnof{\freshvar y} \eford x e {f\<\freshvar y}\)
    & \(\Etuple{\bigl(\eford x e {\pi_1\<f}\bigr), \bigl(\eford x e {\pi_2\<f}\bigr)}\)
  \end{tabular}

  \caption{Desugaring higher-order semilattice operations}
  \label{fig:desugaring-higher-order-semilattice}
\end{figure*}

%% TODO: check that we have every rule we need.
%% TODO: are any of these redundant/unnecessary?
\begin{figure}\centering
  \[\setlength\arraycolsep{.15em}\begin{array}{rcl}
    \ecase{e} x \caseto f &\desugars& \elet{x = e} f\\[.33em]
    \ecase{e} \eboxd x \caseto f &\desugars& \elet{\eboxd x = e} f
    \\[.33em]
    %% A * B
    \ecase{e} (\etuple{p_i,q_i} \caseto f_i)_i
    &\desugars& \elet{\freshvar y = e}\\
    && \ecase{\pi_1\< {\freshvar y},\, \pi_2\< {\freshvar y}} (p_i,q_i \caseto f_i)_i
    \\[.33em]
    %% A + B
    \ecase{e} (\inj i p_{i,j} \caseto f_{i,j})_{i,j}
    &\desugars& \ecase{e} (\inj i {\freshvar y}
    \caseto \ecase{\freshvar y} (p_{i,j} \caseto f_{i,j})_j)_i
    \\[.33em]
    %% [](A * B)
    \ecase{e} (\eboxd{\etuple{p_i,q_i}} \caseto f_i)_i
    &\desugars& \elet{\eboxd {\freshvar y} = e}\\
    && \ecase{\eboxd{\pi_1\<\freshvar y}, \eboxd{\pi_2\<\freshvar y}}
    (\eboxd{p_i}, \eboxd{q_i} \caseto f_i)
    \\[.33em]
    %% [](A + B)
    \ecase{e} (\eboxd{\inj i p_i} \caseto f_i)_i
    &\desugars& \ecase{\esplit e} (\inj i \eboxd{p_i} \caseto f_i)_i
    \\[.33em]
    %% multi-case -> nested case
    \ecase{e,\vec f} (p_i, \vec q_{i,j} \caseto g_{i,j})_{i,j}
    &\desugars& \ecase{e} (p_i \caseto \ecase{\vec f} (\vec q_{i,j} \caseto g_{i,j})_j)_i
  \end{array}\]

  {\small Fresh variables are named $\freshvar y$ and are \freshvar{pink}.}

  \caption{Some pattern-matching elaboration rules}
  \label{fig:desugaring-pattern-matching}
\end{figure}



\section{Surface syntax and its desugaring}
\label{sec:syntax}

We present Datafun in two layers: a simpler \emph{core}
(\cref{fig:core-datafun}) and a more liberal \emph{surface}
(\cref{fig:surface-syntax}) whose extensions are desugared into core terms
(\crefrange{fig:desugaring}{fig:desugaring-pattern-matching}).
%
The simpler core terms are easier to \emph{analyse}, so our semantics and static
transformations consume core terms; conversely, surface terms are easier to
\emph{construct}, so we use them in example programs and in the outputs of our
static transformations.

The specific surface features we add are:
\begin{enumerate}
\item Let-bindings.

\item Booleans, ordered $\efalse < \etrue$. Since sum types are ordered
  disjointly, booleans are instead desugared into sets of empty tuples, with
  $\efalse = \emptyset$ and $\etrue = \eset{\etuple{}}$. \todo{TODO: explain
    $(\eisempty e)$}.

\item Pattern matching on tuples and boxes. \todo{TODO: explain $\prim{split}$.}

%% \item In the core language, semilattice operations ($\bot$, $e \vee f$, and
%%   \kw{for}-loops) are permitted only at \emph{first-order} semilattice types
%%   $\eqt L$. This limitation is lifted in the surface language using
%%   $\eta$-expansion (\cref{fig:desugaring-higher-order-semilattice}).

\item In the core language, we express fixed point computations using what
  amounts to a higher-order operator $\prim{fix} : \iso{(\fixt L \to \fixt L)}
  \to \fixt L$. In the surface language there is a more convenient binding form,
  $\efixisd x e$.

\end{enumerate}


\section{Datafun models}
\label{sec:datafun-models}

The \emph{``standard''} model of Datafun interprets types as posets and terms as
monotone maps. Generalizing this, we can interpret Datafun terms into any
bicartesian closed category \catC{} equipped with the following structures:

\begin{enumerate}
\item A \emph{``discreteness''} comonad $\tuple{\disco{}, \varepsilon, \delta}$
  equipped with distributive morphisms:

  \nopagebreak[2]
  \begin{align*}
    \discox &: \prod_i \discof{A_i} \to \disco \prod_i A_i
    &
    \discosum &: \disco \sum_i A_i \to \sum_i \discof{A_i}
  \end{align*}

\item A \emph{``finite powerset''} functor \(\pfin : \catC \to \catC\) equipped
  with morphism families:

  \nopagebreak[2]
  \begin{align*}
    \morph{single} &: \iso A \to \pfinof{A} &
    \morph{isEmpty} &: \discof{\pfinof{\termO}} \to \termO + \termO
  \end{align*}

  %% \noindent
  %% and, moreover, for any objects $\G,A$, semilattice object $L$, and morphism
  %% $f : \G \x \iso A \to L$, a ``collecting'' morphism \( \pcollect{f} : \G \x
  %% \pfinof A \to L \).

\item \emph{Equality}, \emph{semilattice}, and \emph{fixed point objects}, which
  must interpret first-order, semilattice, and fixed point types respectively.
  Letting $\eqt A$ be an equality object, $L$ be a semilattice object, and
  $\fixt L$ be a fixed point object, these must have morphisms:

  \nopagebreak[2]
  \begin{align*}
    \morph{eq} &: \iso{\eqt A} \x \iso{\eqt A} \to \pfinof\termO
    &
    \morph{join}_n &: L^n \to L
    &
    \morph{fix} &: \iso{(\expO{\fixt L}{\fixt L})} \to \fixt L
  \end{align*}

\item To interpret \kw{for}-loops, for any objects $\G, A$ and any
  semilattice object $L$ we require a family of morphisms $\pcollect{f}$:

  \nopagebreak[2]
  \[
    \infer{f : \G \x \iso A \to L}
          {\pcollect{f} : \G \x \pfinof{A} \to L}
  \]

%% , Given objects $\G, A$, a semilattice equality type $\eqt L$, and
%%   a morphism $f : \G \x \iso{A} \to \den{\eqt L}$, we require a morphism
%%   $\pcollect{f} : \G \x \pfinof{A} \to \den{\eqt L}$. Or, abusing inference rule
%%   notation:
%%   %
%%   \[
%%     \infer{f : \G \x \iso A \to \den{\eqt L}}
%%           {\pcollect{f} : \G \x \pfinof{A} \to \den{\eqt L}}
%%   \]

\end{enumerate}

%% ---- Semantics in a Datafun Model ----
\begin{figure*}
  \figsectionname{Types and Contexts}

  \begin{align*}
    \den{\tunit} &= \termO & \den{A \to B} &= \expO{\den{A}}{\den{B}}
    \\
    \den{\tseteq A} &= \pfinof{\den{\eqt A}}
    & \den{A \x B} &= \den{A} \x \den{B}
    \\
    \den{\iso A} &= \discof{\den{A}} & \den{A + B} &= \den{A} + \den{B}
  \end{align*}

  \begin{align*}
    \den{\G} &= \prod_{H \in \G} \den{H} &
    \den{\h x A} &= \den{A} & \den{\hd x A} &= \discof{\den{A}} &
    \den{\G \vdash A} &= \catC(\den\G, \den A)
  \end{align*}
  \vspace{0pt} % yes, this matters.

  \figsectionname{Terms}

  \begin{align*}
    \den{\dvar x} &= \pi_{\dvar x} \then \varepsilon \quad \text{($\dvar x$ discrete)}
    & \den{x} &= \pi_x \quad \text{($x$ monotone)}
    \\
    \den{\fnof x e} &= \lambda_x\den{e}
    & \den{e\<f} &= \fork{\den{e}, \den{f}} \then \eval
    \\
    \den{\etuple{e_i}_i} &= \fork{\den{e_i}}_i
    & \den{\pi_i\<e} &= \den{e} \then \pi_i
    \\
    \bigden{\eboxd e} &= \boxd e
    & \bigden{\elet{\eboxd x = e} f} &=
    \fork{\id, \den{e}} \then \den{f}
    \\
    \den{\bot} &= \termI \then \morph{join}_0
    &
    \den{e \vee f} &= \fork{\den{e}, \den{f}} \then \morph{join}_2
    \\
    %\den{\color{red}\prim{isEmpty}\<e} &= \den{e} \then \morph{isEmpty}
    \den{\eisempty e} &= \boxd{e} \then \morph{isEmpty}
    & \den{\esplit e} &= \den{e}\then \discosum
    \\
    %\den{\color{red}\prim{eq}\<e} &= \den{e} \then \morph{eq}
    \den{\eeqd{e_1}{e_2}} &= \fork{\boxd{e_i}}_i \then \morph{eq}
    & \den{\efix e} &= \den{e} \then \morph{fix}
    \\
    \den{\esetd{e_i}_i} &= \fork{\boxd{e_i} \then \morph{single}}_i \then \morph{join}
    &
    \den{\eford x e f} &=
    \fork{\id,\den{e}} \then \pcollect{\den{f}}
    %\den{\ebigvee x e f} &=
    %\fork{\id,\den{e}} \then \morph{strong} \then \downof{\den{f}} \then \morph{collect}
  \end{align*}

  \begin{align*}
    \den{\inj i e} &= \den{e} \then \injc_i
    \\
    \den{\ecase{e} (\inj i{x_i} \caseto f_i)_i} &=
    \fork{\id, \den{e}} \then \morph{dist}^\x_+ \then \bigkrof{\den{f_i}}_i
    %\bigkrof{\den{f_i}}_i \circ \morph{dist}^\x_+ \circ \fork{\id, \den{e}}
  \end{align*}
  \vspace{0pt} % yes, this matters

  \figsectionname{Lemmas}

  \begin{align*}
    \morph{box} &: (\stripcxd\G \vdash A) \to \den{\G \vdash \iso A}
    %\den\G \to_\catC \discof{\den{\stripcxd\G}} % of \strip
    \\
    \boxd e &= \fork{\pi_{\dvar x} \then \delta}_{\hd x A \in \G} \then \discox
    \then \iso \den{\isocolor e}
    \\[1ex]
    \morph{dist}^\x_+ &: A \x (B_1 + B_2) \to_\catC (A \x B_1) + (A \x B_2)
    \\
    % this could be simpler if it distributed in the opposite direction.
    \morph{dist}^\x_+ &= \fork{\pi_2 \then \krof{\lambda (\fork{\pi_2,\pi_1} \then \injc_i)}_i, \pi_1}
    \then \eval
  \end{align*}

  %% \raggedright\footnotesize Strictly speaking, the cases for
  %% $\den{\esetd{e_i}_i}$, $\den{\eisempty e}$, and $\den{\eeqd e f}$ are not
  %% structurally inductive; nonetheless $\den{e}$ is well-defined.

  \caption{Datafun semantics in a Datafun model $\tuple{\catC, \disco, \pfin}$}
  \label{fig:semantics}\label{def:strip}
\end{figure*}


\noindent
\Cref{fig:semantics} shows how to interpret Datafun into any Datafun
model. The standard semantics is an instance of this, letting \catC{} be
\Poset{}, \iso{} be \iso{}, and \pfin{} be the finite powerset functor.
\todo{TODO: explain in more detail?}

Since we have not imposed any laws, there is no useful general equational theory
of Datafun models. Instead, our approach is to relate the models we construct to
the standard model and use \emph{its} equational theory.


\section{Semi\naive{} evaluation}
\label{sec:seminaive}

\begin{figure}\centering
  \figsectionname{On Types}

  %% \begin{align*}
  %%   \D\tunit &= \tunit & \Phi\tunit &= \tunit
  %%   \\
  %%   \D \iso A &= \tunit & \Phi \iso A &= \iso{(\Phi A \x \DP A)}
  %%   \\
  %%   \D\tseteq A &= \tseteq A &
  %%   \Phi\tseteq A &= \tset{\Phi{\eqt A}}
  %%   \qquad\text{(see \cref{thm:phi-eqtype})}
  %%   \\
  %%   \D(A \x B) &= \D A \x \D B & \Phi(A \x B) &= \Phi A \x \Phi B\\
  %%   \D(A + B) &= \D A + \D B & \Phi(A + B) &= \Phi A + \Phi B\\
  %%   \D(A \to B) &= \iso A \to \D A \to \D B &
  %%   \Phi(A \to B) &= \Phi A \to \Phi B
  %% \end{align*}

  \begin{align*}
    \D\tunit &= \tunit &
    \D(A \x B) &= \D A \x \D B
    \\
    \D \iso A &= \tunit &
    \D(A + B) &= \D A + \D B
    \\
    \D\tseteq A &= \tseteq A &
    \D(A \to B) &= \iso A \to \D A \to \D B
  %\\[1em]
  \end{align*}\par\begin{align*}
    \Phi\tunit &= \tunit &
    \Phi(A \x B) &= \Phi A \x \Phi B
    \\
    \Phi \iso A &= \iso{(\Phi A \x \DP A)}
    &
    \Phi(A + B) &= \Phi A + \Phi B
    \\
    \Phi\tseteq A &= \tset{\Phi{\eqt A}}
    \quad\text{(see \cref{thm:phi-eqtype})}
    &
    \Phi(A \to B) &= \Phi A \to \Phi B
  \end{align*}
  \vspace{0pt} % yes, this matters

  \figsectionname{On Contexts}

  \begin{align*}
    \D(\h x A) &= \h \dx {\D A} & \D(\hd x A) &= \emptycx\quad\text{(the empty context)}
    \\
    \Phi(\h x A) &= \h x {\Phi A} & \Phi(\hd x A) &= \hd x {\Phi A}, \hd \dx {\DP A}
    \\
    \iso{(\h x A)} &= \hd x A & \iso{(\hd x A)} &= \hd x A
  \end{align*}

  \caption{The $\D$, $\Phi$, and $\iso$ type and context translations}
  \label{fig:PhiDelta}
\end{figure}

%% ---- Seminaive ("go faster") term translation ----
\begin{figure}\centering
  %% TODO: think about this translation and syntax sugar. for example, the
  %% case-split I use in \phi(\esplit e) is only valid *because* \esplit exists.
  %%
  %% Would be nice to give the syntax sugar explicitly, but maybe more effor
  %% than its worth? Might be clearer to mostly avoid syntax sugar? Not sure.
  \figsectionname{Interesting cases}

  \begin{align*}
    \phi(\prim{fix}\<e) &= \fastfix\<\phi e\\
    \phi\eboxd{e} &= \eboxd{\etuple{\phi e, \delta e}}
    \\
    \phi(\elet{\eboxd x = e} f) &= \elet{\eboxd{\etuple{x,\dx}} = \phi e} \phi f
    \\
    \phi(\eford x e f)
    &= \eford x {\phi e} {\phi \substd{f}{dx \substo \zero\<x}}
    \\
    %% TODO: check this lines up with \delta(if then else), because that's where
    %% it comes up.
    \phi(\esplit e) &= \ecase{\phi e}\\
    &\phantom{{}={}}\quad
    \left(\eboxd{\etuple{\inj i x, \inj i \dx}}
    \caseto \inj i \eboxd{\etuple{x,\dx}}\right)_{i\in\{1,2\}}
    \\
    &\phantom{{}={}}\quad
    \left(\eboxd{\etuple{\inj i x, \inj j \pwild}}
    \caseto \inj i \eboxd {\etuple{x, \dummy\<x}} \right)_{i\ne j\in\{1,2\}}
    %% \phi(\esplit e) &= \color{red} \elet{\eboxd{x} = \phi e}\\
    %% &\phantom{{}={}} \esplit [\isocolor \ecase x\\
    %% &\phantom{= \esplit}\quad\isocolor
    %%   (\etuple{\inj i y, \inj i \dy} \caseto \inj i \etuple{y,\dy})_i\\
    %% &\phantom{= \esplit}\quad
    %%   {\isocolor(\etuple{\inj i y, \pwild}
    %%     \caseto \inj i \etuple{y, \dummy\<y})_i}]
  \end{align*}
  \vspace{0pt} % yes, this matters

  \figsectionname{Distributive cases}

  \begin{align*}
    %% TODO: I should mark discrete vs. monotone variables in some
    %% (colorblind/greyscale-printing)-safe way.
    \phi x &= x & \phi \dvar x &= \dvar x\\
    \phi(\fnof x e) &= \fnof x \phi e & \phi(e\<f) &= \phi e\<\phi f\\
    \phi\etuple{e_i}_i &= \etuple{\phi e_i}_i &
    \phi(\prj i e) &= \prj i \phi e\\
    \phi(\inj i e) &= \inj i \phi e\\
    \phi\bot &= \bot &
    \phi(e \vee f) &= \phi e \vee \phi f\\
    \phi(\esetd{e_i}_i) &= \esetd{\phi e_i}_i &
    \phi(\eeqd e f) &= \eeqd {\phi e} {\phi f}
  \end{align*}

  \begin{align*}
    \phi(\ecase e (\inj i x \caseto f_i)_i)
    &= \ecase{\phi e} (\inj i x \caseto \phi f_i)_i
    \\
    \phi(\eisempty e) &= \eisempty {\phi e}
  \end{align*}

  \noindent
  \todo{TODO: Explain the use of weakening in semi\naive{} $\delta(e\<f)$,
  hilighted in {\color{Rhodamine}pink}.}

  \todo{TODO: explain implementation of \zero{} via \dummy{}.}

  \caption{Semi\naive{} speed-up translation, $\phi$}
  \label{fig:seminaive-phi}
\end{figure}

%% ----- Figure: Seminaive δ derivative -----
\begin{figure}\centering
  \[ \delta\bot = \delta\esetd{e_i}_i = \delta(\eeqd e f) = \delta(\efix e) = \bot \]

  \begin{align*}
    \delta x &= \dx &
    \delta \dvar x &= \dvar\dx\\
    \delta(\fnof{x} e) &= \fnof{\eboxd x} \fnof\dx \delta e
    & \delta(e\<f) &= \delta e \<\eboxd{\color{Rhodamine}\phi e} \<\delta f\\
    \delta\etuple{e_i}_i &= \etuple{\delta e_i}_i
    & \delta(\prj i e) &= \prj i \delta e\\
    \delta(\inj i e) &= \inj i {\delta e} &
    \delta(e \vee f) &= \delta e \vee \delta f\\
    \delta\eboxd{e} &= \etuple{} &
    \delta(\elet{\eboxd x = e} f)
    &= \elet{\eboxd{\etuple{x,\dx}} = \phi e} \delta f
  \end{align*}

  \begin{align*}
    %% TODO: double-check this!
    \delta(\ecase e (\inj i x \caseto f_i)_i)
    &= \ecase{\esplit{\eboxd{\color{Rhodamine} \phi e}},\, \delta e}\\
    &\qquad ({\inj i \eboxd x,\, \inj i \dx} \caseto \delta f_i)_{i\in\{1,2\}}\\
    &\qquad ({\inj i \eboxd x,\, \inj j \pwild}
      \caseto \subst{\delta f_i}{\dx \substo \dummy\<\dvar x})_{i\ne j \in\{1,2\}}
    %% &= \ecase{\esplit{\eboxd{\phi e}}}\\
    %% &\qquad (\inj i \eboxd x
    %% \caseto \subst{\delta f_i}{\dx \substo \extract_i\<x\<\delta e})_i\\
    %% \phi(\eif e f g) &= \eif{\phi e}{\phi f}{\phi g}\\
    %% \delta(\eif e f g) &= \eif{\phi e}{\delta f}{\delta g}
    \\
    %% \delta(\ebigvee x e f)
    %% &= (\ebigvee x {\delta e} {\color{Rhodamine}\phi f})
    %% \vee
    %% (\ebigvee x {{\color{Rhodamine}\phi e} \vee \delta e} {\delta f})
    %% TODO: is weakening being used here?
    \delta(\eford x e f)
    &= (\eford x {\delta e} \substd{\phi f}{\dx \substo \zero\<x}) \\
    &\vee (\eford x {{\phi e} \vee \delta e} %\elet{\eboxd\dx = \eboxd{\zero\<x}}
    \substd{\delta f}{\dx \substo \zero\<x})
    \\
    \delta(\eisempty e) &= \eisempty {\color{Rhodamine} \phi e}
    \\
    \delta(\esplit e) &= \ecase{\phi e}
    (\eboxd{\etuple{\inj i \pwild, \pwild}}
    \caseto \inj i \etuple{})_i
  \end{align*}
  %\vskip -.5\baselineskip

  \caption{Semi\naive{} derivative translation, $\delta$}
  \label{fig:seminaive-delta}
\end{figure}

%% ----- Figure: Dummy function -----
\begin{figure*}
  \[\setlength\arraycolsep{.3em}\begin{array}{lcl}
    \dummy &:& A \to \D A\\
    \dummy \<(\pwild : \tbool) &=& \efalse\\
    \dummy \<(\pwild : \tset{A}) &=& \eset{}\\
    \dummy \<\eboxd{x} &=& \eboxd{\dummy\<x}\\
    \dummy \<(f : A \to B) &=& \fnof{x} \dummy\<(f\<x)\\
    \dummy \<\etuple{x,y} &=& \etuple{\dummy\<x, \dummy\<y}\\
    %% \dummy \<\etuple{x_i}_i &=& \etuple{\dummy\<x_i}_i\\
    \dummy \<(\inj i x) &=& \inj i (\dummy\<x)
  \end{array}\]
  \caption{The \dummy\ function}
  \label{fig:dummy}
\end{figure*}


\todo{TODO: explain \fastfix. explain $\Phi$ type transformation as
  ``decorating'' $\iso A$ values with their zero changes so that \prim{fix}
  gets its argument's derivative and can be implemented via \fastfix.
  Give semantics for \fastfix. Explain that we need $\Phi{\eqt A}$ to be
  an eqtype for $\Phi{\tseteq{A}} = \tset{\Phi{\eqt A}}$ to be well-formed, and
  in fact we can be more specific, as in \cref{thm:phi-eqtype}.}

\begin{restatable}[$\Phi$ leaves equality types alone]{theorem}{thmPhiEqtype}
  \label{thm:phi-eqtype}
  $\Phi{\eqt A} = \eqt A$.
\end{restatable}

\begin{proof}
  \XXX
\end{proof}

\Cref{fig:seminaive-phi,fig:seminaive-delta} define two mutually recursive
static transformations: $\phi e$, which computes $e$ \emph{semi\naive{}ly},
speeding up fixed point computation by using derivatives; and $\delta e$, which
computes the \emph{derivative} of $\phi e$.
%
To describe the types of $\phi e$ and $\delta e$, we'll need three new operators
on contexts: $\iso\G$, $\D\G$, and $\Phi\G$. \Cref{fig:PhiDelta} defines these
pointwise by their action on hypotheses. For example,
%\par\nopagebreak[2]\vspace{-.5\baselineskip}
\begin{mathpar}
  \iso{(\h x A, \hd y B)} = \hd x A, \hd y B

  \D(\h x A, \hd y B) = \h \dx {\D A}

  \Phi(\h x A, \hd y B) = \h x {\Phi A}, \hd y {\Phi B}, \hd \dy {\DP B}
\end{mathpar}

\noindent
The idea is that $\iso\G$, $\D\G$, and $\Phi\G$ mirror the effect of $\iso$,
$\D$, and $\Phi$ on the semantics of $\G$. In particular, $\den{\iso\G} =
\iso{\den\G}$, and:

\nopagebreak[2]
\begin{align*}
  \den{\D(\h x A)} &\cong \den{\D A}
  &
  \den{\Phi(\h x A)} &\cong \den{\Phi A}
  \\
  \den{\D(\hd x A)} &\cong \den{\D \iso A}
  &
  \den{\Phi(\hd x A)} &\cong \den{\Phi \iso A}
\end{align*}

\noindent
These defined, we can state the sense in which $\phi$ and $\delta$ are
type-correct:

\begin{restatable}[Type-correctness]{theorem}{thmTypeCorrect}
  \label{thm:type-correct}
  If $\J e \G A$, then
  \[ \J {\phi e} {\Phi\G} {\Phi A}
  \qquad\text{and}\qquad
  \J {\delta e} {\iso{\Phi\G}, \DP\G} {\DP A}
  \]
\end{restatable}
\begin{proof}
  By induction on typing derivations; see \XXX. %\cpageref{proof:typeCorrect}.
\end{proof}

\noindent To get the hang of these context and type transformations, suppose $\J
e {\h x A, \hd y B} C$. Then \cref{thm:type-correct} tells us:

\nopagebreak[2]
\begin{align*}
  \Jalign {\phi e} {\h x {\Phi A}, \hd y {\Phi B}, \hd \dy {\DP B}} {\Phi C}
  \\
  \Jalign {\delta e} {\hd x {\Phi A}, \h\dx{\DP A}, \hd y {\Phi B}, \hd \dy {\DP B}} {\DP C}
\end{align*}


\section{Correctness via logical relations}
\label{sec:logical-relations}

\begin{definition}\label{def:weird}
  Inductively on types $A$, given $a,b : \den{A}$, $x,y : \den{\Phi A}$, and
  $\dx : \den{\D\Phi A}$, we define $\weirdat{A}{\dx}{x}{a}{y}{b}$ as the
  relation generated by these inference rules:

  %% \nopagebreak[2]
  %% \begin{align*}
  %%   \weirdat{\tunit}{\tuple{}}{\tuple{}}{\tuple{}}{\tuple{}}{\tuple{}}
  %%   \\
  %%   \weirdat{\tseteq A}{\dx}{x}{x}{x \cup \dx}{x \cup \dx}
  %%   \\
  %%   \weirdat{\isof A}{\tuple{}}{(x,\dx)}{a}{(x,\dx)}{a}
  %%   &\impliedby \weirdat{A}{\dx}{x}{a}{x}{a}
  %%   \\
  %%   \weirdat{A \to B}{\df}{f}{f_s}{g}{g_s}
  %%   &\impliedby
  %%   \fa{\weirdat{A}{\dx}{x}{a}{y}{b}}
  %%   \\
  %%   &\phantom{{}\impliedby{}}
  %%   \weirdat{B}{\df\<x\<\dx}{f\<x}{f_s\<a}{g\<y}{g_s\<b}
  %%   \\
  %%   \weirdat{A_1 \x A_2}{\vec{\dx}}{\vec x}{\vec a}{\vec y}{\vec b}
  %%   &\impliedby
  %%   \fa{i} \weirdat{A_i}{\dx_i}{x_i}{a_i}{y_i}{b_i}
  %%   \\
  %%   \weirdat{A_1 + A_2}{\inj i \dx}{\inj i x}{\inj i a}{\inj i y}{\inj i b}
  %%   &\impliedby
  %%   \weirdat{A_i}{\dx}{x}{a}{y}{b}
  %% \end{align*}

  \nopagebreak[2]\vspace{-.5\baselineskip}
  \begin{mathpar}
    \weirdat{\tunit}{\tuple{}}{\tuple{}}{\tuple{}}{\tuple{}}{\tuple{}}

    \weirdat{\tseteq A}{\dx}{x}{x}{x \cup \dx}{x \cup \dx}

    \infer{\weirdat{A}{\dx}{x}{a}{x}{a}}{
      \weirdat{\isof A}{\tuple{}}{(x,\dx)}{a}{(x,\dx)}{a}}

    \infer{\fa{\weirdat{A}{\dx}{x}{a}{y}{b}}
      \weirdat{B}{\df\<x\<\dx}{f\<x}{f_s\<a}{g\<y}{g_s\<b}
    }{
      \weirdat{A \to B}{\df}{f}{f_s}{g}{g_s}}

    \infer{\fa{i} \weirdat{A_i}{\dx_i}{x_i}{a_i}{y_i}{b_i}}{
      \weirdat{A_1 \x A_2}{\vec{\dx}}{\vec x}{\vec a}{\vec y}{\vec b}}

    \infer{\weirdat{A_i}{\dx}{x}{a}{y}{b}}{
      \weirdat{A_1 + A_2}{\inj i \dx}{\inj i x}{\inj i a}{\inj i y}{\inj i b}}
  \end{mathpar}
\end{definition}


%%% ---------- Definitions of logical relations for correctness proof ----------

%% TODO: put these in files so we can include them in figures.tex!
\begin{definition} \label{def:changes}
  Inductively on types $A$, we define $(\changesat{A}{\dx}{x}{y})$ for $\dx :
  \den{\D A}$ and $x,y : \den{A}$, glossed as ``$\dx$ changes $x$ into $y$'', as
  the least relation such that:

  \begin{align*}
    \changes{(\dx_i)_i}{(x_i)_i}{(y_i)_i}
    &\impliedby \fa{i} \changes{\dx_i}{x_i}{y_i}
    \\
    \changes{\inj i \dx}{\inj i x}{\inj i y}
    &\impliedby \changes{\dx}{x}{y}
    \\
    \changesat{A \to B}{\df}{f}{g}
    &\impliedby \fa{\changes\dx x y} \changes{\df\<x\<\dx}{f\<x}{g\<y}
    \\
    \changesat{\iso A}{\tuple{}}{x}{x}\\
    \changesat{\tseteq A}{\dx}{x}{(x \cup \dx)}
  \end{align*}

  \noindent This extends to contexts $\G$, letting $\dgamma : \den{\D\G}$ and
  $\g,\g' : \den{\G}$, as follows:

  \begin{align*}
    \changesat{\G}{\dgamma}{\g}{\g'}
    &\iff \fa{\h x A \in \G} \changesat{A}{\dgamma_x}{\g_x}{\g'_x}
    \\
    &\hspace*{1.8em}\wedge
    %\wedge
    \fa{\hd x A \in \G}
    \changesat{\iso A}{\tuple{}}{\g_{\dvar x}}{\g'_{\dvar x}}
  \end{align*}

  \noindent Observe that the condition $\changesat{\iso A}{\tuple{}}{\g_{\dvar
      x}}{\g'_{\dvar x}}$ is equivalent to $\g_{\dvar x} = \g'_{\dvar x}$.

  %% Finally, this extends to \Poset-morphisms $f,g : \den{\G} \to \den{A}$ and
  %% $\df : \den{\iso \G \x \D\G} \to \den{\D A}$:
  %% \begin{align*}
  %%   \changesat{\G \vdash A}{\df}{f}{g} \impliedby&
  %%   \fa{\changesat{\G}{\dgamma}{\g}{\g'}}
  %%   \changesat{A}{\df\<\tuple{\g,\dgamma}}{f\<\g}{g\<\g'}
  %% \end{align*}
\end{definition}

\begin{definition}\label{def:impls}
  Inductively on types $A$, we define ${\impls_A} \subseteq \den{\Phi A} \x
  \den{A}$ as the least relation such that:
  %\par\nopagebreak[2]\vspace{-1ex}
  \begin{mathpar}
    %% \impls
    \infer{}{a \impls_{\tseteq A} a}

    \infer{x \impls_A a \\ \changesat{\Phi A}\dx x x}{
      (x,\dx) \impls_{\iso A} a}

    \tuple{} \impls_\tunit \tuple{}

    \infer{x \impls_A a \\ y \impls_B b}{
      (x,y) \impls_{A \x B} (a,b)}

    \infer{x \impls_{A_i} a}{\inj i x \impls_{A_1 + A_2} \inj i a}

    \infer{\fa{x \impls_A a} f\<x \impls_B g\<a}{f \impls_{A \to B} g}
  \end{mathpar}

  \noindent
  This lifts to contexts, ${\impls_\G} \subseteq \den{\Phi\G} \x \den\G$, as
  follows:

  \nopagebreak[2]
  \begin{align*}
    \gamma \impls_\G \rho
    &\iff \fa{\h x A \in \G} \gamma_x \impls \rho_x
    %\wedge
    \\&\hspace*{1.8em}\wedge
    \fa{\hd x A \in \G}
    (\gamma_{\dvar x}, \gamma_{\dvar \dx}) \impls_{\iso A} \rho_{\dvar x}
  \end{align*}

  \noindent Observe that the last condition is equivalent to \( (\gamma_{\dvar
    x} \impls_A \rho_{\dvar x}) \wedge (\changesat{A}{\gamma_{\dvar
      \dx}}{\gamma_{\dvar x}}{\gamma_{\dvar x}}) \).
\end{definition}


%% \begin{restatable}[Correctness]{theorem}{thmCorrect}\label{thm:correct}
%%   If $\J e \G A$ then

%%   \begin{enumerate}
%%   \item If $\gamma \impls_\G \rho$ then \(\den{\phi e} \<\gamma \impls_A \den{e}
%%     \<\rho\).
%%   \item If $\changesat{\G}{\dgamma}{\g}{\g'}$ then \(\changesat{A}{\den{\delta
%%       e} \<\tuple{\gamma,\dgamma}}{\den{\phi e} \<\gamma}{\den{\phi e}
%%     \<\gamma'}\).
%%   \end{enumerate}

%%   %% \nopagebreak[2]
%%   %% \begin{gather*}
%%   %% \text{(1)}~~ \den{\phi e}\<\gamma \impls_A \den{e}\<\rho
%%   %% \qquad\text{and}\qquad
%%   %% \text{(2)}~~
%%   %% \vld{\den{\delta e} \tuple{\gamma,\dgamma}}{\den{\phi e} \<\gamma}{\den{\phi e} \<\gamma'}
%%   %% : \incdens{\Phi{A}}
%%   %% \end{gather*}

%%   \noindent
%%   Note that $\tuple{\gamma,\dgamma}$ is a slight abuse of notation here;
%%   \todo{TODO explain that \emph{monotone} variables in $\g$ when we use it as
%%     $\den{\phi e}\<\g$ turn into \emph{discrete} variables in $\extend\g\dgamma$
%%     when we use it in $\den{\delta e} \<\extend\g\dgamma$}.
%% \end{restatable}
%% \begin{proof}
%%   By induction on the derivation of $\J e \G A$. See \XXX.
%%   %\cpageref{proof:correct}.
%% \end{proof}


%%% ---------- Change posets ----------
%% We're not really using this section at present, and we don't need the full
%% generality of the section on ``Datafun models'' either.
%% ---------- Incremental semantics via change posets ----------
\section{The category \CP}

\todo{This section is outdated; it describes a version of \CP\ that we no longer
  use, because although it forms a Datafun model, it doesn't let us prove the
  $\phi$/$\delta$ transformations correct. Still, it's interesting that the
  model works, and it suggests the possibility of defining a category of
  \emph{smooth} (infinitely differentiable) morphisms.}

Objects $A$ of \CP{} are tuples $\tuple{\vals A, \chgs A, \updfn_A,
  \composefn_A}$, where:

\begin{itemize}
\item $\vals A$ is a poset of values.

\item $\chgs A$ is a poset of changes.

\item $\updfn_A : \vals A \x \chgs A \pto \vals A$ is a \emph{partial} map
  taking a value and a change to an updated value. When $x \upd \dx$ is defined,
  we say \dx\ is a \emph{valid} change to $x$. It is often convenient to notate
  $x \upd \dx = y$ relationally, as $\valid\dx x y$ (``\dx\ sends $x$ to $y$''),
  or more briefly as $\vld\dx x y$.

\item $\composefn_A : \chgs A \x \chgs A \to \chgs A : \Poset$ is a monotone
  change composition operator.
\end{itemize}

\noindent
These must satisfy the following conditions:

\nopagebreak[2]
\begin{align*}
  x \le y &\impliedby \ex{\dx} \vld{\dx} x y
  & \text{sound for increases}\\
  x \le y &\implies \ex{\dx} \vld{\dx} x y
  & \text{complete for increases}\\
  \vld\dx x y \wedge \vld\dy y z &\implies \vld{\dx\compose\dy} x z
  & \text{correctness of composition}
\end{align*}

\noindent Completeness for increases is equivalent (via choice) to having an
operator $x \changeto y : \chgs A$, defined for $x \le y : \vals A$, such that
$\valid{x \changeto y} x y$.\footnote{Our $x \changeto y$ is effectively the
  incremental $\fn$-calculus' $y \ominus x$, but defined only when $x \le y$.}
In particular, every $x$ has a \emph{zero change}, $\zero_x = x \changeto x$.
However, changes (including zero changes) are not necessarily unique: for a
given $x,y$ there may be many $\dx$ such that $\valid\dx x y$.

Morphisms $f : \CP(A, B)$ are differentiable maps $f : \Poset(\vals A, \vals
B)$. A map is differentiable if it has a derivative $\deriv f : \vals A \to
\Poset(\chgs A, \chgs B)$ such that:

\nopagebreak[2]
\[ \vld{\dx} x y \implies \vld{\deriv f\<x\<\dx}{f\<x}{f\<y}\]

\noindent Or, equivalently (when $x \upd \dx$ is defined):

\nopagebreak[2]
\[ f\<(x \upd \dx) = f\<x \upd \deriv f\<x\<\dx \]

\noindent
The same morphism may have many different derivatives; all that matters is that
one exist.

Identity and composition morphisms in \CP{} are inherited from \Poset{}, so we
need only exhibit their derivatives. These follow a variant of the chain rule:

\nopagebreak[2]
\begin{align*}
  \deriv\id \<x\<\dx &= \dx
  & \deriv{(g \circ f)} \<x\<\dx &= \deriv g \<(f\<x) \<(\deriv f\<x\<\dx)
\end{align*}

\noindent
It's easy to see these functions are monotone in \dx\ as required.


\subsection{\texorpdfstring{\boldmath}{}The forgetful functor \valfn}
\label{sec:CP-vals}

The functor $\valfn : \CP \to \Poset$ takes objects $A$ to $\vals A$ and
morphisms $f$ to themselves. This can be seen as a process of ``forgetting''. On
objects, it forgets a poset's associated change-structure; on morphisms, it
forgets their differentiability.
%
Our ``first-order'' constructions in \CP{} --- products, sums, the finite
powerset functor --- commute with \valfn, in a sense \emph{inheriting} from
those in \Poset. For example, products have $\vals{(A \x B)} = \vals A \x \vals
B$ and $\vals{\fork{f,g}} = \fork{\vals f, \vals g}$.
%
Where this applies, we omit for brevity this inherited structure.
%
However, exponentials will prove the necessary exception to this rule: since all
maps in \CP{} are differentiable, so are the elements of $\vals{(\expO A B)}$.

%% The structures we will build in \CP{} will
%% usually \emph{refine} those in \Poset, in the sense of commuting with \valfn.
%% For example, for cartesian products, $\vals{(A \x B)} = \vals A \x \vals B$ and
%% $\vals{\fork{f,g}} = \fork{\vals f, \vals g}$.
%% %
%% For brevity, we omit the ``value components'' of our constructions when they are
%% inherited in this way from \Poset.


%% \subsection{Products}
%% \label{sec:CP-products}

%% \CP\ has all finite products. $\valfn$ and $\chgfn$ distribute across them:

%% \nopagebreak[2]
%% \begin{align*}
%%   \vals{\prod_{i \in I} A_i} &= \prod_{i \in I} \vals A_i &
%%   \chgs{\prod_{i \in I} A_i} &= \prod_{i \in I} \chgs A_i
%% \end{align*}

%% \noindent
%% Updates are pointwise:

%% \nopagebreak[2]
%% \[ (x_i)_i \upd (\dx_i)_i = (x_i \upd \dx_i)_i
%% \quad\text{when the RHS is defined}\]

%% \noindent Soundness and completeness follow pointwise from soundness and
%% completeness of $A_i$; in particular, $(x_i)_i \changeto (y_i)_i = (x_i
%% \changeto y_i)_i$. Composition is also pointwise:

%% \[ (\dx_i)_i \compose (\dy_i)_i = (\dx_i \compose \dy_i)_i \]

%% It's easy to see

%% \noindent \todo{TODO: soundness, completeness, composition, correctness of composition}

%% The projection and fork maps are inherited along \valfn, with their derivatives
%% being:

%% \nopagebreak[2]
%% \begin{align*}
%%   \deriv\pi_i \<x\<\dx &= \prj i \dx &
%%   \deriv{\fork{f_i}_i} &= \fork{\deriv f_i}_i
%% \end{align*}

%% \noindent Correctness of $\deriv\pi_i$ is straightforward; if
%% $\valid{(\dx_1,\dx_2)}{(x_1,x_2)}{(y_1,y_2)}$ then by definition
%% $\valid{\dx_i}{x_i}{y_i}$. As for correctness of $\deriv{\fork{f_i}_i}$:

%% \nopagebreak[2]
%% \begin{equation*}
%%   \fork{f_i}_i \<(x \upd \dx)
%%   = (f_i(x \upd \dx))_i
%%   = (f_i\<x \upd \deriv f_i \<x \<\dx)_i
%%   = \fork{f_i}_i \<x \upd \fork{\deriv f_i}_i \<x \<\dx
%% \end{equation*}

%% \noindent
%% Finally, the universal property of $\fork{f_i}_i$ follows from its universal
%% property in \Poset\ --- requiring differentiability merely makes the property
%% weaker.


%% \subsection{Sums}
%% \label{sec:CP-sums}

%% \CP\ has all finite sums. \valfn\ and \chgfn\ distribute across them:

%% \nopagebreak[2]
%% \begin{align*}
%%   \vals{\sum_{i \in I} A_i} &= \sum_{i \in I} \vals A_i &
%%   \chgs{\sum_{i \in I} A_i} &= \sum_{i \in I} \chgs A_i
%% \end{align*}

%% \noindent
%% Updates are pointwise but require that the tags match:

%% \nopagebreak[2]
%% \[ \inj i x \upd \inj j \dx =
%% \begin{cases}
%%   \inj i (x \upd \dx) & \text{if}~i=j\\
%%   \text{undefined} & \text{otherwise}
%% \end{cases}
%% \]


\subsection{Cartesian structure}
\label{sec:CP-cartesian}
\begin{figure*}
  \begin{align*}
    \vals{\prod_{i} A_i} &= \prod_{i} \vals A_i &
    (x_i)_i \upd (\dx_i)_i &= (x_i \upd \dx_i)_i
    \quad\text{when defined}
    \\[1ex]
    \chgs{\prod_{i} A_i} &= \prod_{i} \chgs A_i &
    %% (x_i)_i \changeto (y_i)_i = (x_i \changeto y_i)_i
    (\dx_i)_i \compose (\dy_i)_i &= (\dx_i \compose \dy_i)_i
    \\[1ex]
    \vals{\sum_{i} A_i} &= \sum_{i} \vals A_i &
    \inj i x \upd \inj i \dx &= \inj i (x \upd \dx) \quad\text{when defined}
    \\
    \chgs{\sum_{i} A_i} &= \sum_{i} \chgs A_i &
    \inj i \dx \compose \inj j \dy &= 
    \begin{cases}
      \inj i (\dx \compose \dy) & \text{if}~i=j\\
      \inj i \dx & \text{otherwise}
    \end{cases}
  \end{align*}

  \begin{align*}
    \deriv\pi_i \<x \<\dx &= \pi_i\<\dx &
    \deriv{\fork{f_i}_i} &= \fork{\deriv f_i}_i &
    \deriv\injc_i \<x \<\dx &= \inj i \dx
  \end{align*}

  \begin{equation*}
    \deriv{\krof{f_i}_i} \<(\inj i x) \<(\inj j \dx)
    =
    \begin{cases}
      \deriv f_i \<x \<\dx & \text{if}~i = j\\
      \deriv f_i \<x \<\zero_x & \text{otherwise}
    \end{cases}
  \end{equation*}
  \vspace{-1em}

  \caption{Products and sums in \CP}
  \label{fig:CP-cartesian}
\end{figure*}


\CP\ has all finite products and sums (\cref{fig:CP-cartesian}). Their
value-level structure (including $\pi_i$, $\injc_i$, $\fork{f_i}_i$, and
$\krof{f_i}_i$) is inherited from \Poset; \valfn\ and \chgfn\ distribute across
both products and sums. Soundness and completeness follow directly from the
pointwise ordering of products and sums in \Poset{} and soundness \&
completeness for each factor/summand $A_i$. In particular, \changetofn\ can be
defined by:

\nopagebreak[2]
\begin{align*}
  (x_i)_i \changeto (y_i)_i &= (x_i \changeto y_i)_i &
  \inj i x \changeto \inj i y &= \inj i (x \changeto y)
\end{align*}

\noindent The universal properties of $\fork{f_i}_i$ and $\krof{f_i}_i$ follow
from the corresponding universal properties in \Poset\ --- requiring
differentiability merely weakens them.
%
As for correctness of composition and the derivatives of $\pi_i$/$\injc_i$ and
$\fork{f_i}_i$/$\krof{f_i}_i$, we will show these in turn for products and sums.

\subsubsection{Products}

Correctness of change composition is straightforward: if
%
\[ \vld{(\dx_i)_i}{(x_i)_i}{(y_i)_i}
\quad\text{and}\quad
\vld{(\dy_i)_i}{(y_i)_i}{(z_i)_i} \]
%
then for each $i$ we have $\valid{\dx_i \compose \dy_i}{x_i}{z_i}$ and therefore
%
\[\valid{(\dx_i)_i \compose (\dy_i)_i = (\dx_i \compose
  \dy_i)_i}{(x_i)_i}{(z_i)_i}\]

\noindent
Likewise for correctness of $\deriv\pi_i$; if $\valid{\dx_i}{x_i}{y_i}$ for each
$i$ then by definition
\[\valid{\prj i (\dx_i)_i = \dx_i}{x_i}{y_i}\]

\noindent As for correctness of $\deriv{\fork{f_i}_i}$:

\nopagebreak[2]
\begin{equation*}
  \fork{f_i}_i \<(x \upd \dx)
  = (f_i(x \upd \dx))_i
  = (f_i\<x \upd \deriv f_i \<x \<\dx)_i
  = \fork{f_i}_i \<x \upd \fork{\deriv f_i}_i \<x \<\dx
\end{equation*}

\subsubsection{Sums}

Change composition for sums is interesting, because it must handle changes
$(\inj i \dx \compose \inj j \dy)$ with disjoint tags $i \ne j$. Recall that
composition is correct if, when $\valid\dx x y$ and $\valid\dy y z$, we have
$\valid{\dx\compose\dy} x z$. But if $i \ne j$, then
%
there \emph{are} no $x,y,z$ such that $\valid{\inj i \dx} x y$ and $\valid{\inj j \dy} y z$.
%
This case is \emph{dead code}; the only guarantee we need is monotonicity with
respect to \dx\ and \dy, so it suffices to simply return the left argument.

Correctness of the other case is straightforward: since $i = j$, from the
definition of $\updfn_{\sum_i A_i}$ we know all the tags must agree and we have:

\nopagebreak[2]
\[ \inj i x \longvalidarrow{\inj i \dx} \inj i y
\longvalidarrow{\inj i \dy} \inj i z \]

\noindent From this it follows that \( {x} \longvalidarrow{\dx} {y}
\longvalidarrow{\dy} {z} \) and therefore $\vld{\dx\compose\dy} x z$ and so
finally

\nopagebreak[2]
\[ \valid{\inj i \dx \compose \inj i \dy = \inj i (\dx \compose \dy)}{\inj i x}{\inj i z} \]

\noindent
Correctness of $\deriv\injc_i$ is immediate: suppose $\vld\dx x x$. Then
\(\vld{\inj i \dx}{\inj i x}{\inj i x}\). As for split, its derivative must obey

\nopagebreak[2]
\begin{equation*}
  \krof{f_i}_i\<(\inj i x) \upd \deriv{\krof{f_i}_i} \<(\inj i x) \<(\inj i \dx)
  = \krof{f_i}_i\<(\inj i x \upd \inj i \dx)
  = f_i \<(x \upd \dx)
\end{equation*}

\noindent It's easy to see that the $i = j$ case of the definition in
\cref{fig:CP-cartesian} satisfies this property. As with $\composefn_{\sum_i
  A_i}$, the $i \ne j$ case is dead code.


%% \subsection{Cartesian structure}
%% \label{sec:CP-cartesian}

%% \todo{TODO: Neel suggests splitting this into products \& sums. It's easier to
%%   understand the quirks of sums on their own rather than while also trying to
%%   understand products.}

%% \CP\ has all finite products and sums. The value-level structure is inherited
%% from \Poset\ along \valfn. The change-poset structure is also inherited from
%% \Poset, while the projection and injection's derivatives operate pointwise on
%% changes:

%% \nopagebreak[2]
%% \begin{align*}
%%   \chgs{\prod_{i \in I} A_i} &= \prod_{i \in I} \chgs A_i &
%%   \chgs{\sum_{i \in I} A_i} &= \sum_{i \in I} \chgs A_i
%%   %\label{eqn:delta-distributes-over-sums-and-products}
%%   \\
%%   \deriv\pi_i \<x\<\dx &= \prj i \dx & \deriv\inj i\<x\<\dx &= \inj i \dx
%% \end{align*}

%% \noindent \todo{TODO: derivatives of fork \& split!}

%% \noindent Updates are pointwise, noting that $(\inj i x \upd \inj j \dx)$ is
%% undefined for $i \ne j$:

%% \nopagebreak[2]
%% \begin{align*}
%%   (x_i)_i \upd (\dx_i)_i &= (x_i \upd \dx_i)_i &
%%   \inj i x \upd \inj i \dx &= \inj i (x \upd \dx)
%% \end{align*}

%% \noindent
%% Soundness \& completeness for increases follow directly from the pointwise
%% orderings of products \& coproducts in \Poset{}. In particular, the $\changetofn$
%% operator is given by:

%% \nopagebreak[2]
%% \begin{align*}
%%   (x_i)_i \changeto (y_i)_i &= (x_i \changeto y_i)_i &
%%   \inj i x \changeto \inj i y &= \inj i (x \changeto y)
%% \end{align*}

%% \noindent Finally, change composition is also pointwise --- mostly:

%% \nopagebreak[2]
%% \begin{align*}
%%   (\dx_i)_i \compose (\dy_i)_i &= (\dx_i \compose \dy_i)_i &
%%   \inj i \dx \compose \inj j \dy &=
%%   \begin{cases}
%%     \inj i (\dx \compose \dy) &\text{if}~i=j\\
%%     \inj i \dx &\text{otherwise}
%%   \end{cases}
%% \end{align*}

%% \noindent
%% However, for sums we must also handle changes with distinct tags. Recall that
%% composition is correct if, when $\valid\dx x y$ and $\valid\dy y z$, we have
%% $\valid{\dx\compose\dy} x z$. But if $i \ne j$, then
%% %
%% there \emph{are} no $x,y,z$ such that $\valid{\inj i \dx} x y$ and $\valid{\inj j \dy} y z$.
%% %
%% This case is \emph{dead code}; the only guarantee we need is monotonicity with
%% respect to \dx\ and \dy, so it suffices to simply return the left argument.

%% Correctness of the other cases follows directly from correctness of each $\composefn_{A_i}$.


\subsection{Exponentials}
\label{sec:CP-exponentials}

The values $\vals{(\expO A B)}$ of the exponential object are the
\emph{differentiable} monotone maps $\CP(A, B)$, ordered pointwise. Because we
require differentiability, \valfn\ does \emph{not} commute with exponentials:
$\vals{(\expO A B)} \ne \expO{\vals A}{\vals B}$.
%
Meanwhile, the changes $\chgs{(\expO A B)}$ are monotone maps $\iso{\vals A}
\to \chgs A \to \chgs B$, ordered pointwise. The update relation and change
composition must be very carefully defined, so we examine them at length below.

\subsubsection{Updating functions}

The update relation is:

\nopagebreak[2]
\begin{equation}\label{eqn:function-update}
  \vld{\df}{f}{g} : \expO A B
  \iff \fa{\vld\dx x y} \vld{\df\<x\<\dx} {f\<x} {g\<y}
\end{equation}

\noindent However, recall that $\upd$ must be a partial function: from
$\valid\df f g$ and $\valid\df f h$, we must show $g = h$. So consider any $x :
A$. \Cref{eqn:function-update} applied to $\valid{\zero_x} x x$ yields

\nopagebreak[2]
\[ \vld{\df \<x \<\zero_x}{f \<x}{g\<x}
\quad\text{and}\quad \vld{\df \<x \<\zero_x}{f \<x}{h\<x} \]

\noindent
Then by functionality of $\updfn_B$ we have $g\<x = h\<x$, and thus $g = h$.

Although this shows that \cref{eqn:function-update} defines a partial function,
one may wonder what it actually \emph{does} (when it is defined). So let's
rewrite \cref{eqn:function-update} in terms of $\upd_{\expO A B}$:

\nopagebreak[2]
\[ f \upd \df = g \iff \fa{x \upd \dx = y} f\<x \upd \df\<x\<\dx = g\<y \]

\noindent Substituting away $g$ and $y$, this implies that for all valid
\dx, \df:

\nopagebreak[2]
\[
f\<x \upd \df\<x\<\dx = (f \upd \df) \<(x \upd \dx)
\]

\noindent In particular, letting $\dx = \zero_x$ and swapping sides, we get:

\nopagebreak[2]
\[ (f \upd \df) \<x = f\<x \upd \df\<x\<\zero_x \]

\noindent
\todo{TODO: explain why this does not suffice as a \emph{definition} of
  $\updfn_{\expO A B}$ --- in particular, they don't explain when $f \upd \df$
  is defined! It's \emph{not} simply when $x \mapsto f\<x \upd \df\<x\<\dx$ is
  monotone, for example. Can I find a counterexample?}

\subsubsection{Soundness and completeness}

Soundness follows from completeness at $A$ and soundness at $B$: if $\valid\df f
g$ then for any $x$ we have $\valid{\df\<x\<\zero_x}{f\<x}{g\<x}$ (using
completeness of $A$ to find $\zero_x$), and by soundness at $B$ we have $f\<x
\le g\<x$, so $f \le g$.

We'll show completeness by constructing \changetofn\ explicitly:

\nopagebreak[2]
\[ (f \changeto_{\expO A B} g) \<x \<\dx = (f\<x \changeto_B g\<x) \compose_B \deriv g\<x\<\dx\]

\noindent Observe that $f\<x \changeto_B g\<x$ is defined because $f \le g$
implies $f\<x \le g\<x$, and the RHS is monotone in \dx\ by monotonicity of
$\deriv g \<x$ and of $\composefn_B$.
%
To prove completeness, assuming $\valid\dx x y$, we need to show \( \valid{(f
  \changeto g) \<x \<\dx}{f\<x}{g\<y} \). This follows from correctness of
$\composefn_B$, since \( \valid{f\<x \changeto g\<x}{f\<x}{g\<x} \) and
\(\valid{\deriv g \<x \<\dx}{g\<x}{g\<y}\) by definition.

\subsubsection{Function change composition}

Change composition is defined by:

\nopagebreak[2]
\[ (\df \compose_{\expO A B} \dg) \<x \<\dx = \df \<x \<\zero_x \compose_B \dg \<x \<\dx\]

\noindent
This is monotone in \dx\ by monotonicity of $\dg\<x$ and $\composefn_B$. To show
this valid, suppose $\vld\df f g$, $\vld\dg g h$ and $\vld\dx x y$. Then
%
\(\vld{\df\<x\<\zero_x}{f\<x}{g\<x}\)
\text{and}
\(\vld{\dg\<x\<\dx}{g\<x}{h\<y} \).
%
\noindent So by correctness of $\composefn_B$ we have \(\valid{(\df\<x\<\zero_x
  \compose \dg\<x\<\dx)}{f\<x}{h\<y}\) as desired.

\todo{TODO: discuss unavoidable asymmetry in definitions of \(\changetofn_{\expO
    A B}\) and \(\composefn_{\expO A B}\).}

\subsubsection{\texorpdfstring{\boldmath}{}\eval\ and \fn}

To show that $\expO A B$ is an exponential object, we need evaluation and
currying maps, \eval\ and $\fn f$. Modulo differentiability, these are
inherited from \Poset{}, with the following derivatives:

\begin{align*}
  %% \eval &: (\expO A B) \x A \to B &
  %% \fn (f : \G \x A \to B) &: \G \to \expO A B
  %% \\
  \vals{\eval} \<(f,x) &= f\<x &
  \vals{(\fn f)} \<x &= y \mapsto f\<(x,y)\\
  \deriv{\eval} \<(f,x) \<(\df,\dx) &= \df \<x \<\dx &
  \deriv{(\fn f)} \<x \<\dx &= y\<\dy \mapsto \deriv f \<(x,y) \<(\dx,\dy)
\end{align*}

\noindent
$\deriv{\eval}$ is correct by the definition of $\updfn_{\expO A B}$, for if
$\valid\df f g$ and $\valid\dx x y$ then

\nopagebreak[2]
\[ \eval\<(f,x) = {f\<x} \longvalidarrow{\df\<x\<\dx} {g\<y} = \eval\<(g,y) \]

\noindent
Similarly, \(\deriv{(\fn f)}\) is correct, for suppose $\valid\dx {x_1}{x_2}$
and $\valid\dy{y_1}{y_2}$. Then by differentiability of $f$ we have

\nopagebreak[2]
\[ \fn f \<x_1 \<y_1 =
f\<(x_1,y_1) \longvalidarrow{\deriv f \<(x_1,y_1) \<(\dx,\dy)} f\<(x_2,y_2)
= \fn f \<x_2 \<y_2
\]

\noindent
Finally, $\fn f$ is unique such that $(\fn f \x \id) \then \eval = f$ by the
same argument as in \Poset{} (or indeed \cat{Set}): supposing $\eval \<(\fn f
\<x, y) = f\<(x,y) = \eval\<(g\<x,y)$ for all $x,y$, then we have $\fn f \<x \<
y = g \<x \<y$ and so $\fn f = g$.


\subsection{The pseudo-exponential}

\newcommand\pseudoexp{\rightrightarrows}
%\renewcommand\pseudoexp\rightarrowtriangle
%\renewcommand\pseudoexp{\overset{?}{\Rightarrow}}

You might have expected changes $\chgs{(\expO A B)}$ to be given pointwise, as
maps $\iso{\vals A} \to \chgs B$. Intriguingly, this does yield a valid change
poset, one rather simpler than $\expO A B$, but it does not satisfy the
properties of an exponential. I call this the \emph{pseudo-exponential}, $A
\pseudoexp B$. Pseudo-exponentials are irrelevant to Datafun, but are an
interesting and surprising aspect of \CP, so I describe them here. They are
constructed as follows:

\vspace{-1.2ex}
\begin{mathpar}
  \vals{(A \pseudoexp B)} = \CP(A,B)

  \chgs{(A \pseudoexp B)} = \iso{\vals A} \to \chgs B

  \vld{\df} f g : A \pseudoexp B \iff \fa{x} \vld{\df\<x}{f\<x}{g\<x}

  (\df \compose_{A \pseudoexp B} \dg) \<x = \df\<x \compose_B \dg\<x

  (f \changeto g) \<x = f\<x \changeto g\<x
\end{mathpar}

\noindent
I leave it as an exercise for the reader to verify soundness \& completeness and
correctness of composition. \todo{TODO: explain why they aren't the exponential
  object and aren't isomorphic to $\expO A B$.}

%% Besides $\expO A B$, there is another, simpler, change poset which one might
%% expect to be the exponential, but which is not, which I term the
%% \emph{pseudo-exponential}, written $A \pseudoexp B$.


\subsection{\texorpdfstring{\boldmath}{}The lifted \iso\ comonad}
\label{sec:CP-iso}

Now we construct a comonad $\iso$ on \CP\ such that $\valfn\iso_\CP =
\iso_\Poset\valfn$. This requirement defines the value-level structure, leaving
only the change poset, update map, and composition. Since $\iso A$ is ordered
discretely, values \emph{cannot} increase. So the simplest definition is:

\nopagebreak[2]
\begin{align*}
  \chgs{\iso A} &= \termO &
  x \upd_{\iso A} \tuple{} &= x &
  \tuple{} \compose_{\iso A} \tuple{} &= \tuple{}
\end{align*}

\noindent
This trivially satisfies soundness, completeness, and correctness of
composition.
%
Functoriality, extraction, duplication, and product- and sum-distribution
inherit from the corresponding properties of $\iso_\Poset$, and their
derivatives are fairly straightforward:

\nopagebreak[2]\vspace{-1ex}
\begin{mathpar}
  \deriv{\iso(f)} \<x \<\tuple{} = \tuple{}

  \deriv\delta_A \<x \<\tuple{} = \tuple{}

  \deriv\varepsilon_A \<x \<\tuple{} = \zero_x
  \\
  \deriv{{\discox}} \<x \<\dx = \tuple{}

  \deriv{{\discosum}} \<(\inj i x) \<\tuple{} = \inj i {\tuple{}}
\end{mathpar}


\subsubsection{The alternative \altiso}

Although we will not use it, there is another refinement of $\iso$ in \CP{},
which I'll call $\altiso$ for clarity, defined like so:

\nopagebreak[2]
\begin{align*}
  \chgs{\altiso A} &= \iso{\chgs A} &
  %% x \upd_{\altiso A} \dx = y &\iff x = y \wedge x \upd_A \dx = y
  \vld{\dx} x x : \altiso A &\iff \vld{\dx} x x : A
  &
  \composefn_{\altiso A} &= \composefn_A
\end{align*}

\noindent
So $\altiso A$ inherits all of $A$'s changes, but ordered discretely and with
\emph{only zero-changes still valid}. Soundness and completeness for increases
and correctness of composition are all implied by the corresponding properties
of $A$. The derivatives of functoriality, duplication, and extraction are
different:

\nopagebreak[2]
\begin{align*}
  \deriv{\altiso(f)} &= \deriv f
  & \deriv\delta_A \<x\<\dx &= \dx
  & \deriv\varepsilon_A \<x\<\dx &= \dx
\end{align*}

\noindent
Observe that we no longer need the non-constructive $\zero_x$ to define comonad
extraction $\varepsilon$! However, for our purposes this will not be an issue.
Meanwhile, product- and sum-distribution become identity maps, because $\prod_i
\altiso A_i = \altiso \prod_i A$ and $\sum_i \altiso A_i = \altiso \sum_i
A_i$ (checking this is left as an exercise for the reader).


\subsection{\texorpdfstring{\boldmath}{}The lifted \pfin\ functor}
\label{sec:CP-pfin}

We lift the finite powerset functor \pfin\ to \CP\ as follows:

\nopagebreak[2]
\begin{align}
  \vals{\pfinof A} &= \pfinof{\vals A} &
  \chgs{\pfinof A} &= \pfinof{\vals A} &
  (\updfn_{\pfinof A}) = (\composefn_{\pfinof A}) = ({\cup})
  \label{eqn:CP-pfin}
\end{align}

\noindent Since $\updfn$ and \composefn\ are just $\cup$, soundness and
completeness come from a standard property of union (or more generally of
semilattice join):

\nopagebreak[2]
\[ x \le z \iff \ex{y} x \cup y = z \]

\noindent Meanwhile, correctness of composition is simply associativity of
$\cup$:

\nopagebreak[2]
\[ (x \cup \dx = y) \wedge (y \cup \dy = z) \implies (x \cup \dx \cup \dy = z) \]

\noindent
Since the arguments to \morph{single} and \morph{isEmpty} cannot change
(having $\iso$ type), their derivatives always return zero-changes:

\nopagebreak[2]
\begin{align*}
  \deriv{\morph{single}}_A \<x \<\tuple{} &= \emptyset &
  \deriv{\morph{isEmpty}} \<x \<\tuple{} &= \morph{isEmpty} \< x
\end{align*}



\subsection{Equality objects}
\label{sec:CP-eq}

Every object in \CP{} is an equality object, with \morph{eq} and its derivative
defined by:

\nopagebreak[2]
\begin{align*}
  \morph{eq}\<\tuple{x,y} &= 
  \begin{cases}
    \{\tuple{}\} & \text{if}~ x = y\\
    \emptyset & \text{otherwise}
  \end{cases}
  &
  \deriv{\morph{eq}} \<\pwild \<\pwild &= \emptyset
\end{align*}

\noindent Note that $\deriv{\morph{eq}}$ is correct because $\morph{eq}$'s
arguments cannot change, having $\iso$ type.

Since \emph{every} object is an equality object, every equality type
is trivially interpreted into an equality object. \todo{TODO more explanation
  --- why isn't every type an equality type, then?}


\subsection{Semilattice objects and collecting morphisms}
\label{sec:CP-semilattice}

A semilattice object $L$ in \CP\ is one where $\vals L = \chgs L$ is a
semilattice and $x \upd_L y = x \vee y$. \todo{(TODO: we might want to require
  $\composefn_L = {\vee}$ as well, if we need it.)} This immediately justifies
letting $\deriv{\morph{join}} \<x \<\dx = \morph{join}\<\dx$, because:

\nopagebreak[2]
\begin{align*}
  \morph{join}\<(x \upd \dx) &= \bigvee_i \pi_i \<(x \vee \dx)\\
  &= \left(\bigvee_i \prj i x\right) \vee \left(\bigvee_i \prj i \dx\right)\\
  &= \morph{join}\<x \vee \morph{join}\<\dx\\
  &= \morph{join}\<x \upd \deriv{\morph{join}}\<x\<\dx
\end{align*}

\begin{restatable}{lemma}{lemSemi}
  \label{lem:semi} $\incdens L$ is a semilattice object. \todo{FIXME: haven't
  introduced $\incdenfn$ yet!}
\end{restatable}
\begin{proof}
  By induction on $L$. \XXX
  %\todo{See \cpageref{proof:semi}.}
\end{proof}


\subsection{Collecting morphisms}
\label{sec:CP-collect}

Let $\G, A$ be objects, $L$ be a semilattice object, and $f : \G \x \iso A \to
L$. Our derivative for $\pcollect f : \G \x \pfinof A \to L$ is:

\nopagebreak[2]
\begin{align*}
  \deriv{\pcollect f} \<\tuple{\gamma,s} \<\tuple{\dgamma,\ds}
  &= \bigvee_{x \in \ds} f\<\tuple{\gamma,x} \vee
  \bigvee_{x \in s \cup \ds} \deriv f \<\tuple{\gamma,x} \<\tuple{\dgamma, \zero_x}
\end{align*}

\noindent To establish that this is correct, suppose
$\vld{\dgamma}{\gamma}{\gamma'} : \G$. From this we have:

\nopagebreak[2]
\begin{align*}
  &\phantom{{}={}}\pcollect{f} \<\tuple{\gamma', s \cup \ds}
  \\
  &= \bigvee_{x \in s \cup \ds} f \<\tuple{\gamma', x}
  \\
  &\overset{\footnotemark}{=} \bigvee_{x \in s \cup \ds} \bigl(f \<\tuple{\gamma,x}
  \vee \deriv f \<\tuple{\gamma,x} \<\tuple{\dgamma,\zero_x}\bigr)
  \\
  &= \bigvee_{x \in s} f\<\tuple{\gamma,x}
  \upd \bigvee_{x \in \ds} f\<\tuple{\gamma,x}
  \vee \bigvee_{x \in s \cup \ds} \deriv f \<\tuple{\gamma,x} \<\tuple{\dgamma,\zero_x}
  \\
  &= \pcollect f \<\tuple{\gamma,s}
  \upd \deriv{\pcollect f} \<\tuple{\gamma,s} \<\tuple{\dgamma,\ds}
\end{align*}

\nopagebreak% for footnote to be on same page as mark
\noindent
Which is what we wished to show.\footnotetext{This equality follows from the
  correctness of $\deriv f$ applied to
  \(\vld{\tuple{\dgamma,\zero_x}}{\tuple{\gamma,x}}{\tuple{\gamma',x}}\).}


\subsection{Fixed point objects}
\label{sec:CP-fix}

A fixed point object $\fixt L$ in \CP\ is a semilattice object such that
$\vals{\fixt L}$ satisfies the ascending chain condition. \todo{TODO more
  explanation.} Since the argument to \morph{fix} is boxed and cannot change,
$\deriv{\morph{fix}}$ needs to produce a zero change; and at any semilattice
object, $\bot$ is a zero change, so \( \deriv{\morph{fix}} \<x \<\tuple{} = \bot
\).

\begin{lemma}\label{lem:fixobject}
  \(\incdens{\fixt L}\) is a fixed point object.
\end{lemma}
\begin{proof}
  By induction on $\fixt L$. \todo{TODO}
\end{proof}


\section{Incrementalization}
\label{sec:incremental}

%% The \emph{incremental semantics} for Datafun, notated $\incden{-}$, is given by
%% the Datafun model in \CP{} constructed in \cref{sec:changeposets}, letting
%% $\disco = \iso_\CP$ and $\pfin = \pfin_\CP$. The obvious question is: how does
%% this semantics relate to the standard \Poset\ semantics, $\den{-}$? \todo{TODO:
%% Explain how they are mostly the same if you apply \valfn, but exponentials
%% crucially aren't. So we tried to use a logical relation, like that
%% in \cref{fig:agrees}.}

%% Recalling
%% \cref{sec:CP-vals}, since most of our constructions inherit along $\valfn$,
%% restricted to the \Poset-level they are very nearly the same!
%% %
%% However, $\vals{\incden{A \to B}} \ne \den{A \to B}$ because of the
%% differentiability restriction on \CP's exponentials.
%% %
%% \todo{TODO: more explanation}

\todo{TODO: Explain that \CP\ gives as incremental semantics for Datafun. We
suspect (but haven't yet proven; back-burner) that you can connect this to the
regular semantics for Datafun using a logical relation (\cref{fig:agrees}).

However, ultimately the motivation for \CP\ was to give a semantic foundation
for semi\naive{} evaluation. But it's not clear how to do this, because
the \emph{morphisms} and \emph{values} in \CP\ are ``hereditarily
differentiable'', but the \emph{changes} (eg. of the exponential) are \emph{not}
necessarily differentiable, and this causes no end of pain.
%
In particular, there seems to be no way to re-use the validity relations from
objects in \CP\ to prove correctness of the semi\naive{} transform. Because
$\chgs(\expO A B) = \iso{\vals A} \to \chgs A \to \chgs B$ mixes differentiable
and non-differentiable things, we can't use it either on values we get out of
the \Poset{}-semantics or out of the the \CP-semantics.}

\begin{figure*}
  \begin{align*}
    {\agrees_A} &\subseteq \vals{\incdens{A}} \x \den{A} &
    x \agrees_{\tseteq A} y &\iff x = y \quad(\text{\textdagger})\\
    %% \tuple{x_1,x_2} \agrees_{A_1 \x A_2} \tuple{y_1,y_2} &\iff \fa{i} x_i \agrees_{A_i} y_i &
    \tuple{x_i}_i \agrees \tuple{y_i} &\iff \fa{i} x_i \agrees y_i &
    \inj i x \agrees \inj j y &\iff i = j \wedge x \agrees y\\
    x \agrees_{\isof A} y &\iff x \agrees y &
    f \agrees_{A \to B} g &\iff \fa{x \agrees_A y} f\<x \agrees_B g\<y
  \end{align*}

  \small \textdagger\ This is well-typed because \cref{thm:agree-first-order}
  implies $\vals{\incdens{\,\tseteq{A}\,}} = \den{\tseteq{A}}$.

  \caption{Agreement relation between $\incdenfn$ and $\den{-}$}
  \label{fig:agrees}
\end{figure*}


\begin{restatable}[First-order semantic agreement]{conjecture}{thmAgreeFirstOrder}
  \label{thm:agree-first-order}
  \( \vals{\incdens{\eqt A}} = \den{\eqt A} \).
\end{restatable}

\begin{proof}
  \todo{TODO, by induction on types}
\end{proof}

\begin{restatable}[First-order agreement is equality]{conjecture}{thmAgreeEqualsEquals}
  \(x \agrees_{\eqt A} y \iff x = y\).
\end{restatable}
\begin{proof}
  \label{thm:agree-equals-equals}
  \todo{TODO, by induction on eqtypes.}
\end{proof}

\begin{restatable}[Semantic agreement]{conjecture}{thmAgree}
  If $\J e \G A$
  %and $\g \agrees_\G \sigma$ then $\incdens{e}\<\g \agrees_A \den{e}\<\sigma$.
  then $\incdens{e} \agrees_{\G \vdash A} \den{e}$.
\end{restatable}
\begin{proof}
  \todo{TODO by induction on typing derivations}
\end{proof}

%% \noindent
%% The change poset $\D\incdens A$ of a Datafun type $A$ is internally definable
%% via a type translation given in \cref{fig:PhiDelta}, also named $\D$, despite
%% the risk of confusion. \todo{TODO: move this \P\ into \cref{sec:seminaive}.}

%% \begin{restatable}{theorem}{thmDeltaDen}
%%   \label{thm:delta-den}
%%   \(\D\incdens A = \vals{\incdens{\D A}}\).
%% \end{restatable}
%% \nopagebreak[2]
%% \begin{proof}
%%   \label{proof:delta-den}
%%   By induction on types. A starred equality, $\stareq$, marks equalities given by
%%   an inductive hypothesis; additionally, $\daggereq$ indicates an appeal to
%%   \cref{thm:agree-first-order}.
%%   \begin{description}
%%     \item[Case $\iso A$:] \(\D\incdens{\iso A} = \D{\iso{\incdens A}} =
%%       \termO_\Poset = \vals\termO_\CP = \vals{\incdens{\tunit}} \).

%%     \item[Case $\tseteq A$:]
%%       \(\D\incdens{\,{\tseteq A}\,} = \D\pfinof{\incdens{\eqt A}}
%%       = \pfinof{\vals{\incdens{\eqt A}}}
%%       = \vals{\pfinof{\incdens{\eqt A}}}
%%       = \vals{\incdens{\,{\tseteq A}\,}}
%%       = \vals{\incdens{\D{\tseteq A}}}
%%       \).

%%     \item[Case $A \x B$:]
%%       \(\D\incden{A \x B} = \D(\incdens A \x \incdens B)
%%       = \D\incdens A \x \D\incdens B
%%       \stareq \vals{\incdens{\D A}} \x \vals{\incdens{\D B}}
%%       = \vals{\incden{\D A \x \D B}}
%%       = \vals{\incdens{\D(A \x B)}}
%%       \).

%%     \item[Case $A + B$:]
%%       \(\D\incden{A + B} = \D(\incdens A + \incdens B)
%%       = \D\incdens A + \D\incdens B
%%       \stareq \vals{\incdens{\D A}} + \vals{\incdens{\D B}}
%%       = \vals{\incden{\D A + \D B}}
%%       = \vals{\incdens{\D(A + B)}}
%%       \).

%%     \item[Case $A \to B$:]
%%       \begin{align*}\D\incden{A \to B}
%%         &= \D(\expO{\incdens A}{\incdens B})\\
%%         &= \expO{\iso{\vals{\incdens
%%               A}}}{\expO{\D\incdens{A}}{\D\incdens{B}}}\\
%%         &= ...\\
%%         &= \CP(\iso{\incdens A},\ {\expO{\incdens{\D A}}{\incdens{\D B}}})\\
%%         &= \vals{(\expO{\iso{\incdens{A}}}{\expO{\incdens{\D A}}{\incdens{\D B}}})}\\
%%         &= \vals{\incden{\iso A \to \D A \to \D B}}\\
%%         &= \vals{\incdens{\D(A \to B)}}
%%       \end{align*}

%%     \item[Case $\tunit$:]
%%   \end{description}
%% \end{proof}


\title{Change Semantics for Semi\naive{} Datafun}
\author{\scshape Michael Arntzenius}
\date{\today}

\maketitle

\begin{abstract}
  We generalize semi\naive{} evaluation from Datalog to Datafun and prove it
  correct.
\end{abstract}

\section{Outline}

The key insight behind \emph{semi\naive{} evaluation} is to find fixed points
faster by computing only what changes between iterations (\cref{sec:fastfix}).
To apply this in Datafun, we first need to understand what \emph{change} means.
%
To this end, we generalize Datafun's \Poset{} semantics to any category with
appropriate structures (\cref{sec:datafun-models}).

We construct such a category, \CP{}, whose objects are posets equipped with a
notion of increasing changes, and whose morphisms are monotone maps equipped
with \emph{derivatives} relating input and output changes
(\cref{sec:changeposet}). This yields an incremental semantics for Datafun
(\cref{sec:incremental}). \todo{TODO discuss relation $\phi$/$\delta$
translation via alternate comonad $\blackiso$.}

We use \CP\ as a guide to constructing a semi\naive{} evaluation static
transformation (\cref{sec:seminaive}). This transformation reinterprets the
discreteness comonad to compute derivatives, and uses these to speed up fixed
point computations. Finally, we prove this transformation preserves the
semantics of closed first-order programs using logical relations
(\cref{sec:logical-relations}).

% TODO: more section links once I finish them.


\section{Notation and conventions}

\begin{enumerate}[flushleft,widest=0]
\item Adjacency $f\<x$ denotes the function $f$ applied to an argument $x$.

\item I write $f \then g$ for the composition of morphisms $f : A \to B$ and $g
  : B \to C$. This is the reverse of the usual composition operator, $g \circ
  f$.

\item I write $\subst{e}{x \substo f}$ to mean ``$e$ with all free occurrences of $x$
  replaced by $f$ ($\alpha$-varying as necessary to avoid capture)''.

  \newcommand\iexpr{\phi}
\item I write $(\iexpr_i)_i$ for the tuple or sequence $(\iexpr_1,
  ..., \iexpr_n)$, leaving $n$ unspecified. For example,

  \nopagebreak[2]
  \[ (2^ix + y_i)_i \quad\text{means}\quad
  (2^1x + y_1,~ 2^2x + y_2,~ ...,~ 2^nx + y_n) \]

  \noindent
  I also use this notation with other braces, eg. $\fork{\iexpr_i}_i =
  \fork{\iexpr_0, ..., \iexpr_n}$.

\item $\expO A B$ denotes the exponential $B^A$ in a cartesian closed category.

\item Given a family of morphisms $f_i : A \to B_i$ in a category with finite
  products, their \emph{``fork''} $\fork{f_i}_i : A \to \textstyle\prod_{i} B_i$
  uniquely satisfies $\fork{f_i}_i \then \pi_i = f_i$. Dually, given $f_i : A_i
  \to B$ in a category with finite sums, their \emph{``split''} $\krof{f_i}_i :
  \sum_i A_i \to B$ uniquely satisfies $\injc_i \then \krof{f_i}_i = f_i$.
  %
  So in particular, $\termI : A \to \termO$ is the unique map into the terminal
  object and $\initE : \initO \to B$ the unique map out of the initial object.

\item The comonad $\iso : \Poset \to \Poset$ takes a poset to the
  \emph{discrete} poset on the same elements, defined:

  \begin{eqnarray*}
    x \in \iso A &\iff& x \in A\\
    x \le y : \iso A &\iff& x = y\\
    \iso(f : A \to B) &=& f : \iso A \to \iso B
  \end{eqnarray*}

\end{enumerate}


\section{Fixed points and \fastfix}
\label{sec:fastfix}

\todo{TODO: more explanation.} Let $L$ be a semilattice satisfying the ascending
chain condition. Fix monotone maps $f : L \to L$ and $\deriv f : L \to L \to L$,
and consider the sequences $x_i$, $\dx_i$ defined by:

\nopagebreak[2]
\begin{align*}
  x_0 &= \bot
  &
  x_{i+1} &= x_i \vee \dx_i\\
  \dx_0 &= f\< x
  &
  \dx_{i+1} &= \deriv f \<x_i \<\dx_i
\end{align*}

\noindent
Let $\fastfix\<(f,\deriv f) = \bigvee_i x_i$. The existence of this least upper
bound is guaranteed by the ascending chain condition, since $x_i \le x_i \vee
\dx_i = x_{i+1}$.

\begin{restatable}[\prim{fix} and \fastfix]{theorem}{thmFastFix}\label{thm:fast-fix}
  If $\fa{x,\dx} f \<x \vee \deriv f \<x \<\dx = f \<(x \vee \dx)$,
  then \(\efix f = \fastfix\<(f,\deriv f)\).
\end{restatable}
\begin{proof}
  Assume $\fa{x,\dx} f\<x \vee \deriv f \<x \<\dx = f\<(x \vee \dx)$. It suffices
  to show $\fa{i} x_{i+1} = f \<x_i$, which we do by induction on $i$. The base
  case is $x_1 = x_0 \vee \dx_0 = \bot \vee f\<\bot = f\<x_0$. Inductively,
  assuming $x_{i+1} = f\<x_i$, we have:

  \nopagebreak[2]
  \begin{align*}
    x_{i+2} &= x_{i+1} \vee \dx_{i+1} && \text{definition of $x_{i+2}$}\\
    &= f\<x_i \vee \deriv f\<x_i\<\dx_i && \text{inductive hypothesis, definition of $\dx_{i+1}$}\\
    &= f\<(x_i \vee \dx_i) && \text{assumption}\\
    &= f\<x_{i+1} && \text{definition of $x_{i+1}$}
  \end{align*}
\end{proof}


\definecolor{isobg}{cmyk}{.06, 0, .06, 0}
\definecolor{isorule}{cmyk}{.36, 0, .36, .36}
%\definecolor{isorule}{cmyk}{.5, 0, .5, .5}
%\definecolor{isorule}{cmyk}{.8, 0, .8, .8}

%% \definecolor{isobg}{cmyk}{0, 0.03, .3, 0}
%% \definecolor{isorule}{cmyk}{0, 0.03, .3, 0.37}

\newcommand\eiso[1]{\adjustbox{margin=0pt 1pt,}{\adjustbox{
      cframe=isorule .25pt 2pt,
      bgcolor=isobg,
}{%
\vphantom{lp}% room for ascenders & descenders
\(#1\)}}}
\renewcommand\eisempty[1]{{\prim{empty?}\<{#1}}}

\begin{figure*}\centering
  \begin{mathpar}
    \setlength\arraycolsep{.33em}\begin{array}{r@{\hskip 1em}ccl}
      \text{types} & A,B &\bnfeq& \iso A \bnfor \tset{\eqt A}
      \bnfor \tunit \bnfor A \x B \bnfor A + B \bnfor A \to B
      \\
      \text{eqtypes} & \eqt A, \eqt B &\bnfeq&
      %% TODO: revamp to make \iso an eqtype. this requires \Phi(\iso{\eqt A})
      %% = \iso{\Phi{\eqt A}} = \iso{\eqt A} and changing \phi & \delta
      %% accordingly.
      %\iso{\eqt A} \bnfor
      \tset{\eqt A} \bnfor
      \tunit \bnfor \eqt A \x \eqt B \bnfor \eqt A + \eqt B
      \\
      \text{semilattices} & L,M &\bnfeq& \tset{\eqt A} \bnfor \tunit \bnfor L \x M
      \\
      \text{finite eqtype} & \fint A, \fint B &\bnfeq&
      \tset{\fint A} \bnfor \tunit \bnfor \fint A \x \fint B
      \\
      \text{fixtypes} & \fixt L, \fixt M &\bnfeq&
      \tset{\fint A} \bnfor \tunit \bnfor \fixt L \x \fixt M
      \\[.5em]
      \text{terms} & e,f,g &\bnfeq& x \bnfor \dvar x \bnfor \fnof{x} e
      \bnfor e\<f \bnfor \etuple{} \bnfor \etuple{e,f} \bnfor \prj i e
      \bnfor \inj i e
      \\&&&%\bnfor
      \ecase{e} (\inj i x \caseto f_i)_{i\in\{1,2\}}
      \bnfor%\\&&&
      \eiso{\ebox e} \bnfor \elet{\eboxd x = e} f
      \\&&&
      \bot \bnfor e \vee f \bnfor \eford x e f
      \\&&&
      \eiso{\eset{e_i}_i} \bnfor
      \eiso{\eeq e f}
      \bnfor \eiso{\eisempty e}
      \bnfor \esplit e
      \bnfor \efix e
      \\[.5em]
      \text{contexts} & \G &\bnfeq& (H_i)_i\\
      \text{hypotheses} & H &\bnfeq& \h x A \bnfor \hd x A
    \end{array}
    \\
    \stripcxd\G = (\hd x A)_{\hd x A \in \G}
    \\
    \infer{\h x A \in \G}{\J x \G A}

    \infer{\hd x A \in \G}{\J {\dvar x} \G A}

    \infer{\J e {\G,\h x A} B}{\J {\fnof x e} \G {A \to B}}

    \infer{\J e \G {A \to B} \\ \J f \G A}{\J {e\<f} \G B}

    \infer{\quad}{\J {\etuple{}} \G \tunit}

    \infer{(\J{e_i}\G{A_i})_i}{\J{\etuple{e_1,e_2}} \G {A_1 \x A_2}}

    \infer{\J e \G {A_1 \x A_2}}{\J{\prj i e}\G{A_i}}

    \infer{\J e \G A_i}{\J{\inj i e}\G{A_1 + A_2}}

    \infer{\J e \G {A_1 + A_2} \\
      (\J {f_i} {\G,\h {x_i} {A_i}} {B})_i
    }{
      \J {\ecase{e} (\inj i x_i \caseto f_i)_i} \G B
    }

    \infer{\J {e} {\stripcxd\G} A
    }{
      \J{\eiso{\ebox e}} \G {\iso A}
    }

    \infer{\J e \G {\iso A} \\ \J f {\G,\hd x A} B}{
      \J {\elet{\eboxd x = e} f} \G B}

    \infer{\quad}{\J\bot\G {\eqt L}}

    \infer{(\J{e_i} \G {\eqt L})_i}{\J{e_1 \vee e_2}\G {\eqt L}}

    %% \infer{\J e \G {\eqt A}}{\J {\edown e} \G {\tdown {\eqt A}}}
    \infer{(\J {e_i} {\stripcxd\G} {\eqt A})_i}{
      \J {\eiso{\eset{e_i}_i}} \G {\tset{\eqt A}}
    }

    %% \infer{\J e \G {\tdown {\eqt A}} \\
    %%   \J f {\G,\h x {\eqt A}} L
    %% }{\J {\ebigvee x e f} \G L}
    %%
    \infer{
      \J e \G {\tset A} \\
      \J f {\G,\hd x A} {\eqt L}
    }{\J {\eford x e f} \G {\eqt L}}

    %%\infer{\J e \G {\iso{(\eqt A \x \eqt A)}}}{\J{\prim{eq}\<e} \G {\tdown\tunit}}
    \infer{(\J {e_i} {\stripcxd\G} {\eqt A})_i}
          {\J {\eiso{\eeq{e_1}{e_2}}} \G \tbool}

    \infer{\J {e} {\stripcxd\G} {\tset\tunit}}{
      \J {\eiso{\eisempty e}} \G {\tunit + \tunit}}

    \infer{\J e \G {\iso{(A + B)}}}{\J{\esplit e} \G {\iso A + \iso B}}

    \infer{\J e \G {\iso{(\fixt L \to \fixt L)}}}{\J{\prim{fix}\< e} \G {\fixt L}}
  \end{mathpar}

  \caption{Datafun core syntax and typing rules}
  \label{fig:core-datafun}
\end{figure*}

\input{seminaive/fig-surface-syntax}
\input{seminaive/fig-desugaring}
%\begin{figure*}\centering
  \begin{tabular}{@{}rll@{}}
    Surface term & expanded at $A \to L$ & expanded at $L \x M$\\\midrule
    $\bot$
    & \(\fnof{\freshvar\pwild} \bot\)
    & \(\etuple{\bot,\bot}\)
    \\
    \(e \vee f\)
    & \(\fnof{\freshvar x} e\<{\freshvar x} \vee f\<{\freshvar x}\)
    & \(\etuple{\pi_1\<e \vee \pi_1\<f,\, \pi_2\<e\vee\pi_2\<f}\)
    \\
    \(\eford x e f\)
    & \(\fnof{\freshvar y} \eford x e {f\<\freshvar y}\)
    & \(\Etuple{\bigl(\eford x e {\pi_1\<f}\bigr), \bigl(\eford x e {\pi_2\<f}\bigr)}\)
  \end{tabular}

  \caption{Desugaring higher-order semilattice operations}
  \label{fig:desugaring-higher-order-semilattice}
\end{figure*}

\input{seminaive/fig-desugaring-pattern-matching}


\section{Surface syntax and its desugaring}
\label{sec:syntax}

We present Datafun in two layers: a simpler \emph{core}
(\cref{fig:core-datafun}) and a more liberal \emph{surface}
(\cref{fig:surface-syntax}) whose extensions are desugared into core terms
(\crefrange{fig:desugaring}{fig:desugaring-pattern-matching}).
%
The simpler core terms are easier to \emph{analyse}, so our semantics and static
transformations consume core terms; conversely, surface terms are easier to
\emph{construct}, so we use them in example programs and in the outputs of our
static transformations.
%
The specific surface features we add are:

\nopagebreak[2]
\begin{enumerate}
\item Let-bindings.

\item Booleans, ordered $\efalse < \etrue$. Since sum types are ordered
  disjointly, booleans are instead desugared into sets of empty tuples, with
  $\efalse = \emptyset$ and $\etrue = \eset{\etuple{}}$. \todo{TODO: explain
    $(\eisempty e)$}.

\item Pattern matching on tuples and boxes. \todo{TODO: explain $\prim{split}$.}

%% \item In the core language, semilattice operations ($\bot$, $e \vee f$, and
%%   \kw{for}-loops) are permitted only at \emph{first-order} semilattice types
%%   $\eqt L$. This limitation is lifted in the surface language using
%%   $\eta$-expansion (\cref{fig:desugaring-higher-order-semilattice}).

\item In the core language, we express fixed point computations using what
  amounts to a higher-order operator $\prim{fix} : \iso{(\fixt L \to \fixt L)}
  \to \fixt L$. In the surface language there is a more convenient binding form,
  $\efixisd x e$.

\end{enumerate}


\section{Datafun models}
\label{sec:datafun-models}

The \emph{``standard''} model of Datafun interprets types as posets and terms as
monotone maps. Generalizing this, we can interpret Datafun terms into any
bicartesian closed category \catC{} equipped with the following structures:

\begin{enumerate}
\item A \emph{``discreteness''} comonad $\tuple{\disco{}, \varepsilon, \delta}$
  equipped with distributive morphisms:

  \nopagebreak[2]
  \begin{align*}
    \discox &: \prod_i \discof{A_i} \to \disco \prod_i A_i
    &
    \discosum &: \disco \sum_i A_i \to \sum_i \discof{A_i}
  \end{align*}

\item A \emph{``finite powerset''} functor \(\pfin : \catC \to \catC\) equipped
  with morphism families:

  \nopagebreak[2]
  \begin{align*}
    \morph{single} &: \iso A \to \pfinof{A} &
    \morph{isEmpty} &: \discof{\pfinof{\termO}} \to \termO + \termO
  \end{align*}

  %% \noindent
  %% and, moreover, for any objects $\G,A$, semilattice object $L$, and morphism
  %% $f : \G \x \iso A \to L$, a ``collecting'' morphism \( \pcollect{f} : \G \x
  %% \pfinof A \to L \).

\item \emph{Equality}, \emph{semilattice}, and \emph{fixed point objects}, which
  must interpret first-order, semilattice, and fixed point types respectively.
  Letting $\eqt A$ be an equality object, $L$ be a semilattice object, and
  $\fixt L$ be a fixed point object, these must have morphisms:

  \nopagebreak[2]
  \begin{align*}
    \morph{eq} &: \iso{\eqt A} \x \iso{\eqt A} \to \pfinof\termO
    &
    \morph{join}_n &: L^n \to L
    &
    \morph{fix} &: \iso{(\expO{\fixt L}{\fixt L})} \to \fixt L
  \end{align*}

\item To interpret \kw{for}-loops, for any objects $\G, A$ and any
  semilattice object $L$ we require a family of morphisms $\pcollect{f}$:

  \nopagebreak[2]
  \[
    \infer{f : \G \x \iso A \to L}
          {\pcollect{f} : \G \x \pfinof{A} \to L}
  \]

%% , Given objects $\G, A$, a semilattice equality type $\eqt L$, and
%%   a morphism $f : \G \x \iso{A} \to \den{\eqt L}$, we require a morphism
%%   $\pcollect{f} : \G \x \pfinof{A} \to \den{\eqt L}$. Or, abusing inference rule
%%   notation:
%%   %
%%   \[
%%     \infer{f : \G \x \iso A \to \den{\eqt L}}
%%           {\pcollect{f} : \G \x \pfinof{A} \to \den{\eqt L}}
%%   \]

\end{enumerate}

%% ---- Semantics in a Datafun Model ----
\begin{figure*}
  \textsc{Type and Context Denotations}

  %% TODO: revert to this more readable version if space allows.
  %% \begin{align*}
  %%   \den{\tunit} &= \termO & \den{A \to B} &= \expO{\den{A}}{\den{B}}
  %%   \\
  %%   \den{\tseteq A} &= \pfinof{\den{\eqt A}}
  %%   & \den{A \x B} &= \den{A} \x \den{B}
  %%   \\
  %%   \den{\iso A} &= \iso{\den{A}} & \den{A + B} &= \den{A} + \den{B}
  %% \end{align*}

  \begin{align*}
    \den{\tunit} &= \termO
    & \den{\tseteq A} &= \pfinof{\den{\eqt A}}
    & \den{\iso A} &= \iso{\den{A}}
    \\
    \den{A \to B} &= \expO{\den{A}}{\den{B}}
    & \den{A \x B} &= \den{A} \x \den{B}
    & \den{A + B} &= \den{A} + \den{B}
  \end{align*}

  \begin{align*}
    \den{\G} &= \prod_{H \in \G} \den{H} &
    \den{\h x A} &= \den{A} & \den{\hd x A} &= \iso{\den{A}} &
    \den{\G \vdash A} &= \Poset(\den\G, \den A)
  \end{align*}
%  \vspace{0pt} % yes, this matters.

  \textsc{Term Denotations}

  \begin{displaymath}
    \def\arraystretch{1.15}
    \begin{array}{lcl}
      \den{\J {\dvar x} \G A} &=& \pi_{\dvar x} \then \varepsilon \qquad \text{(for $\hd x A \in \G$)} \\
      \den{\J x \G A} &=& \pi_x \qquad\quad\, \text{(for $\h x A \in \G$)} \\
      \den{\J {\fnof x e} \G {A \to B}} &=& \lambda\den{\J e {\G, \h x A} B} \\
      \den{\J {f\<e} \G B} &=& \fork{\den{\J f \G {A \to B}}, \den{\J e \G A}} \then \eval \\
      \den{\J {\etuple{e_1, e_2}} \G {A_1 \times A_2}} &=&
           \fork{\den{\J {e_1} \G {A_1}}, \den{\J {e_2} \G {A_2}}} \\
      \den{\J {\pi_i\<e} \G {A_i}} &=& \den{\J e \G {A_1 \times A_2}} \then \pi_i \\
      \den{\J {\ebox e} \G {\iso A}} &=& \mkbox_\Gamma(\den {\J e {\stripcx \G} A}) \\
      \den{\J {\elet{\ebox x = e} f} \G B} &=&  \fork{\id_\Gamma, \den{\J e \G {\iso A}}} \then \den{\J f {\G, \hd x A} B}  \\
      \den{\J \bot \G L} &=& \termI \then !_\Gamma \then \morph{join}^L_0 \\
      \den{\J {e \vee f} \G L} &=& \fork{\den{\J e \G L}, \den{\J f \G L}} \then \morph{join}^L_2 \\
      \den{\J {\eisempty e} \G {1+1}}&=& \mkbox_\Gamma(\den{\J e {\stripcx \G} {\tset A}}) \then \morph{isEmpty} \\
      \den{\J {\esplit e} \G {\iso A + \iso B}} &=& \den{\J e \G {\iso(A + B)}}\then \isosum \\

      \den{\eeq{e_1}{e_2}} &=&
          \fork{\mkbox_\Gamma(\den{\J {e_1} {\stripcx \G} A}),
                \mkbox_\Gamma(\den{\J {e_2} {\stripcx \G} A})}
          \then \morph{eq} \\
      \den{\J {\efix e} \G L} &=& \den{\J e \G {\iso(L \to L)}} \then \morph{fix} \\

      \den{\eset{e_i}_i} &=& \fork{\mkbox_\Gamma(\den{\J {e_1} {\stripcx \G} A}) \then \morph{singleton}}_i \then \morph{join}^L \\

      \den{\J {\efor x e f} \G L} &=&    \fork{\id,\den{\J e \G {\tset A}}} \then \pcollect{\den{\J f {\G, \hd x A} L}} \\
    \den{\J {\inj i e} \G {A_1 + A_2}} &=& \den{\J e \G {A_i}} \then \injc_i \\
    \den{\J {\ecase{e} (\inj i{x_i} \caseto f_i)_i} \G B} &=&
    \fork{\id, \den{\J e \G {A_1 + A_2}}} \then \morph{dist}^\x_+ \then
           \bigkrof{\den{\J {f_i} {\G, \h {x_i} {A_i}} B}}_i \\
    \end{array}
  \end{displaymath}
  \vspace{0pt} % yes, this matters

  \textsc{Auxilliary Operations}

  \begin{align*}
    \morph{dist}^\x_+ &~:~ A \x (B_1 + B_2) \to (A \x B_1) + (A \x B_2)
    &
    \mkbox_\Gamma &~:~ \Poset(\den{\stripcx \G}, A) \to \Poset(\den{\G}, \iso A) \\
    % this could be simpler if it distributed in the opposite direction.
    \morph{dist}^\x_+ &= \fork{\pi_2 \then \krof{\lambda (\fork{\pi_2,\pi_1} \then \injc_i)}_i, \pi_1}
    \then \eval
    &
    \mkbox_\Gamma(f) &= \fork{\pi_{\dvar x} \then \delta}_{\hd x A \in \G} \then \isox \then \iso(f)
  \end{align*}

  \caption{Semantics of Datafun}
  \label{fig:semantics}\label{def:strip}
\end{figure*}


\noindent
\Cref{fig:semantics} shows how to interpret Datafun into any Datafun
model. The standard semantics is an instance of this, letting \catC{} be
\Poset{}, \iso{} be \iso{}, and \pfin{} be the finite powerset functor.
\todo{TODO: explain in more detail?}

Since we have not imposed any laws, there is no useful general equational theory
of Datafun models. Instead, our approach is to relate the models we construct to
the standard model and use \emph{its} equational theory.


\section{The category \CP}
\label{sec:changeposet}

Objects $A$ of \CP{} are tuples $\tuple{\vals A,\, \chgs A,\, {\validto_A},\,
  \dummy_A}$ where
%
\begin{enumerate}
\item $\vals A$ and $\chgs A$ are posets of ``values'' and ``changes'' respectively.

\item ${\validto_A} \subseteq \chgs A \x \vals A \x \vals A$ is a relation. We
  gloss $(\dx,x,y) \in {\validto_A}$ as ``$\dx$ changes $x$ into $y$'', and
  usually write it as $\valid \dx x y$ or more briefly $\vld \dx x y$.

\item $\dummy_A$ is a map $\iso \vals A \to \D A$. Classically, the existence of
  $\dummy$ is equivalent to saying that if $\vals A$ is inhabited then so is $\D
  A$. As we will see later, this is a crude hack to ensure the existence of
  (weak) sums.
\end{enumerate}

\noindent
Morphisms $A \to B$ consist of a map $f : \vals A \to \vals B$ equipped with a
derivative $\deriv f : \iso \vals A \to \chgs A \to \chgs B$ yielding the change
in $f$'s output given a change to its input:
%
\begin{equation}\label{eqn:deriv}
    \vld \dx x y \implies \vld {\deriv f \<x\<\dx} {f\<x}{f\<y}
\end{equation}

\noindent
Identity and composition follow the chain rule:
%
\begin{align*}
  \deriv \id \<x \<\dx &= \dx &
  \deriv{(f \then g)} \<x \<\dx &= g \<(f\<x) \<(\deriv f \<x \<\dx)
\end{align*}


\subsection{\boldmath The forgetful functor \valfn}

The functor $\valfn : \CP \to \Poset$ takes objects $A$ to $\vals A$ and
morphisms $(f,\deriv f)$ to the underlying map $f : \vals A \to \vals B$. This
can be seen as a process of ``forgetting''. On objects, it forgets a value
poset's associated change-structure; on morphisms, it forgets their derivatives.
%
Our constructions in \CP{} will commute with \valfn, inheriting from the
corresponding constructions in \Poset. For example, products have $\vals{(A \x
  B)} = \vals A \x \vals B$ and $\vals{\fork{f,g}} = \fork{\vals f, \vals g}$.
%
Where applicable, we omit for brevity this inherited structure.


\subsection{Cartesian structure}
\label{sec:CP-cartesian}
\renewcommand\tuple[1]{(#1)}
\begin{figure*}
  \begin{align*}
    \vals{\prod_i A_i} &= \prod_i \vals A_i &
    \chgs{\prod_i A_i} &= \prod_i \chgs A_i &
    \infer{
      \fa{i} \valid{\dx_i}{x_i}{y_i}
    }{\textstyle
      \valid{\tuple{\dx_i}_i}{\tuple{x_i}_i}{\tuple{y_i}_i}
    }
    \\[.5\baselineskip]
    \vals{\sum_i A_i} &= \sum_i \vals A_i &
    \chgs{\sum_i A_i} &= \sum_i \chgs A_i &
    \infer{
      \valid{\dx}{x}{y}
    }{\textstyle
      \valid{\inj i \dx}{\inj i x}{\inj i y}
    }
  \end{align*}

  \begin{align*}
    \dummy \<\tuple{x_i}_i &= \tuple{\dummy\<x_i}_i &
    \deriv\pi_i \<x \<\dx &= \prj i \dx &
    \deriv{\fork{f_i}_i} \<x \<\dx &= \tuple{\deriv f_i \<x \<\dx}_i
    \\
    \dummy\<(\inj i x) &= \inj i (\dummy\<x) &
    \deriv\injc_i \<x \<\dx &= \inj i \dx
  \end{align*}

  \begin{equation*}
    \deriv{\krof{f_i}_i} \<(\inj i x) \<(\inj j \dx)
    =
    \begin{cases}
      \deriv f_i \<x \<\dx & \text{if}~i = j\\
      \deriv f_i \<x \<(\dummy \<x) & \text{otherwise}
    \end{cases}
  \end{equation*}

  \caption{Products and sums in \CP}
  \label{fig:CP-cartesian}
\end{figure*}


\CP\ has all finite products and (weak) sums. Their value-level structure is
inherited from \Poset; \valfn\ distributes across products, sums, $\pi_i$,
$\injc_i$, $\fork{f_i}_i$, and $\krof{f_i}_i$. We give $\chgfn$, $\dummy$, and
the relevant derivatives in \cref{fig:CP-cartesian}. \todo{Give the proof that
  $\deriv{\fork{f_i}_i}$ is unique.}

Observe that we need the definition of \dummy\ to construct the derivative of
$\krof{f_i}_i$. \todo{Discuss dead code. Show how this makes
  $\deriv{\krof{f_i}_i}$ not unique and sums weak. Discuss how sums become
  strong if we forget derivatives into differentiability.}



\section{The semi\naive\ Datafun model in \CP}

\subsection{\boldmath The ``tangent bundle'' comonad \tangfn}
\label{sec:tangent-comonad}

\begin{align*}
  \vals{\tang A} &= \iso \vals A \x \chgs A &
  \chgs{\tang A} &= \termO &
  \infer{\validat A \dx x x}{
    \validat {\tang A} {\tuple{}} {\tuple{x,\dx}} {\tuple{x,\dx}}
  }
  \span
\end{align*}

\[ \dummy_{\tang A} \<\pwild = \tuple{} \]




\section{Semi\naive{} evaluation}
\label{sec:seminaive}

\begin{figure}\centering
  \figsectionname{On Types}

  %% \begin{align*}
  %%   \D\tunit &= \tunit & \Phi\tunit &= \tunit
  %%   \\
  %%   \D \iso A &= \tunit & \Phi \iso A &= \iso{(\Phi A \x \DP A)}
  %%   \\
  %%   \D\tseteq A &= \tseteq A &
  %%   \Phi\tseteq A &= \tset{\Phi{\eqt A}}
  %%   \qquad\text{(see \cref{thm:phi-eqtype})}
  %%   \\
  %%   \D(A \x B) &= \D A \x \D B & \Phi(A \x B) &= \Phi A \x \Phi B\\
  %%   \D(A + B) &= \D A + \D B & \Phi(A + B) &= \Phi A + \Phi B\\
  %%   \D(A \to B) &= \iso A \to \D A \to \D B &
  %%   \Phi(A \to B) &= \Phi A \to \Phi B
  %% \end{align*}

  \begin{align*}
    \D\tunit &= \tunit &
    \D(A \x B) &= \D A \x \D B
    \\
    \D \iso A &= \tunit &
    \D(A + B) &= \D A + \D B
    \\
    \D\tseteq A &= \tseteq A &
    \D(A \to B) &= \iso A \to \D A \to \D B
  %\\[1em]
  \end{align*}\par\begin{align*}
    \Phi\tunit &= \tunit &
    \Phi(A \x B) &= \Phi A \x \Phi B
    \\
    \Phi \iso A &= \iso{(\Phi A \x \DP A)}
    &
    \Phi(A + B) &= \Phi A + \Phi B
    \\
    \Phi\tseteq A &= \tset{\Phi{\eqt A}}
    \quad\text{(see \cref{thm:phi-eqtype})}
    &
    \Phi(A \to B) &= \Phi A \to \Phi B
  \end{align*}
  \vspace{0pt} % yes, this matters

  \figsectionname{On Contexts}

  \begin{align*}
    \D(\h x A) &= \h \dx {\D A} & \D(\hd x A) &= \emptycx\quad\text{(the empty context)}
    \\
    \Phi(\h x A) &= \h x {\Phi A} & \Phi(\hd x A) &= \hd x {\Phi A}, \hd \dx {\DP A}
    \\
    \iso{(\h x A)} &= \hd x A & \iso{(\hd x A)} &= \hd x A
  \end{align*}

  \caption{The $\D$, $\Phi$, and $\iso$ type and context translations}
  \label{fig:PhiDelta}
\end{figure}

\input{seminaive/fig-seminaive-phi}
%% ----- Figure: Seminaive δ derivative -----
\begin{figure*}
  \[ \delta\bot = \delta\esetd{e_i}_i = \delta(\eeqd e f) = \delta(\efix e) = \bot \]

  \begin{align*}
    \delta x &= \dx &
    \delta \dvar x &= \dvar\dx\\
    \delta(\fnof{x} e) &= \fnof{\eboxd x} \fnof\dx \delta e
    & \delta(e\<f) &= \delta e \<\eboxd{\color{Rhodamine}\phi e} \<\delta f\\
    \delta\etuple{e_i}_i &= \etuple{\delta e_i}_i
    & \delta(\pi_i\<e) &= \pi_i\<\delta e\\
    \delta(\inj i e) &= \inj i {\delta e} &
    \delta(e \vee f) &= \delta e \vee \delta f\\
    \delta\eboxd{e} &= \etuple{} &
    \delta(\elet{\eboxd x = e} f)
    &= \elet{\eboxd{\etuple{x,\dx}} = \phi e} \delta f
  \end{align*}

  \begin{align*}
    %% TODO: double-check this!
    \delta(\ecase e (\inj i x \caseto f_i)_i)
    &= \ecase{\esplit{\eboxd{\color{Rhodamine} \phi e}},\, \delta e}\\
    &\qquad ({\inj i {\eboxd x},\, \inj i \dx} \caseto \delta f_i)_i\\
    &\qquad ({\inj i {\eboxd x},\, \pwild}
      \caseto \subst{\delta f_i}{\dx \substo \dummy\<x})_i
    %% &= \ecase{\esplit{\eboxd{\phi e}}}\\
    %% &\qquad (\inj i \eboxd x
    %% \caseto \subst{\delta f_i}{\dx \substo \extract_i\<x\<\delta e})_i\\
    %% \phi(\eif e f g) &= \eif{\phi e}{\phi f}{\phi g}\\
    %% \delta(\eif e f g) &= \eif{\phi e}{\delta f}{\delta g}
    \\
    %% \delta(\ebigvee x e f)
    %% &= (\ebigvee x {\delta e} {\color{Rhodamine}\phi f})
    %% \vee
    %% (\ebigvee x {{\color{Rhodamine}\phi e} \vee \delta e} {\delta f})
    %% TODO: is weakening being used here?
    \delta(\eford x e f)
    &= (\eford x {\delta e} \substd{\phi f}{\dx \substo \zero\<x}) \\
    &\vee (\eford x {{\phi e} \vee \delta e} %\elet{\eboxd\dx = \eboxd{\zero\<x}}
    \substd{\delta f}{\dx \substo \zero\<x})
    \\
    \delta(\eisEmpty e) &= \eisEmpty {\color{Rhodamine} \phi e}
    \\
    \delta(\esplit e) &= \ecase{\phi e}
    (\eboxd{\eboxd{\etuple{\inj i \pwild, \pwild}}}
    \caseto \inj i {\etuple{}} )_i
  \end{align*}

  \caption{Semi\naive{} derivative translation, $\delta$}
  \label{fig:seminaive-delta}
\end{figure*}

\begin{figure}
  \begin{align*}
    \dummy_{\tseteq A} \<\pwild &= \eset{}
    &
    \dummy_{A \x B} \<\etuple{x,y} &= \etuple{\dummy\<x, \dummy\<y}
    \\
    \dummy_\tunit \<\etuple{} &= \etuple{}
    &
    \dummy_{A + B} \<(\inj i x) &= \inj i (\dummy\<x)
    \\
    \dummy_{\iso A} \<\ebox{x} &= \ebox{\dummy\<x}
    &
    \dummy_{A \to B} \<f &= \fnof{x} \dummy\<(f\<x)
  \end{align*}
  \caption{The function $\dummy_A : A \to \D A$}
  \label{fig:dummy}
\end{figure}


\todo{TODO: explain \fastfix. explain $\Phi$ type transformation as
  ``decorating'' $\iso A$ values with their zero changes so that \prim{fix}
  gets its argument's derivative and can be implemented via \fastfix.
  Give semantics for \fastfix. Explain that we need $\Phi{\eqt A}$ to be
  an eqtype for $\Phi{\tseteq{A}} = \tset{\Phi{\eqt A}}$ to be well-formed, and
  in fact we can be more specific, as in \cref{thm:phi-eqtype}.}

\begin{restatable}[$\Phi$ leaves equality types alone]{theorem}{thmPhiEqtype}
  \label{thm:phi-eqtype}
  $\Phi{\eqt A} = \eqt A$.
\end{restatable}

\begin{proof}
  \XXX
\end{proof}

\Cref{fig:seminaive-phi,fig:seminaive-delta} define two mutually recursive
static transformations: $\phi e$, which computes $e$ \emph{semi\naive{}ly},
speeding up fixed point computation by using derivatives; and $\delta e$, which
computes the \emph{derivative} of $\phi e$.
%
To describe the types of $\phi e$ and $\delta e$, we'll need three new operators
on contexts: $\iso\G$, $\D\G$, and $\Phi\G$. \Cref{fig:PhiDelta} defines these
pointwise by their action on hypotheses. For example,
%\par\nopagebreak[2]\vspace{-.5\baselineskip}
\begin{mathpar}
  \iso{(\h x A, \hd y B)} = \hd x A, \hd y B

  \D(\h x A, \hd y B) = \h \dx {\D A}

  \Phi(\h x A, \hd y B) = \h x {\Phi A}, \hd y {\Phi B}, \hd \dy {\DP B}
\end{mathpar}

\noindent
The idea is that $\iso\G$, $\D\G$, and $\Phi\G$ mirror the effect of $\iso$,
$\D$, and $\Phi$ on the semantics of $\G$. In particular, $\den{\iso\G} =
\iso{\den\G}$, and:

\nopagebreak[2]
\begin{align*}
  \den{\D(\h x A)} &\cong \den{\D A}
  &
  \den{\Phi(\h x A)} &\cong \den{\Phi A}
  \\
  \den{\D(\hd x A)} &\cong \den{\D \iso A}
  &
  \den{\Phi(\hd x A)} &\cong \den{\Phi \iso A}
\end{align*}

\noindent
These defined, we can state the sense in which $\phi$ and $\delta$ are
type-correct:

\begin{restatable}[Type-correctness]{theorem}{thmTypeCorrect}
  \label{thm:type-correct}
  If $\J e \G A$, then
  \[ \J {\phi e} {\Phi\G} {\Phi A}
  \qquad\text{and}\qquad
  \J {\delta e} {\iso{\Phi\G}, \DP\G} {\DP A}
  \]
\end{restatable}
\begin{proof}
  By induction on typing derivations; see \XXX. %\cpageref{proof:typeCorrect}.
\end{proof}

\noindent To get the hang of these context and type transformations, suppose $\J
e {\h x A, \hd y B} C$. Then \cref{thm:type-correct} tells us:

\nopagebreak[2]
\begin{align*}
  \Jalign {\phi e} {\h x {\Phi A}, \hd y {\Phi B}, \hd \dy {\DP B}} {\Phi C}
  \\
  \Jalign {\delta e} {\hd x {\Phi A}, \h\dx{\DP A}, \hd y {\Phi B}, \hd \dy {\DP B}} {\DP C}
\end{align*}


\section{Correctness via logical relations}
\label{sec:logical-relations}

\begin{definition}\label{def:weird}
  Inductively on types $A$, given $a,b : \den{A}$, $x,y : \den{\Phi A}$, and
  $\dx : \den{\D\Phi A}$, we define $\weirdat{A}{\dx}{x}{a}{y}{b}$ as the
  relation generated by these inference rules:

  %% \nopagebreak[2]
  %% \begin{align*}
  %%   \weirdat{\tunit}{\tuple{}}{\tuple{}}{\tuple{}}{\tuple{}}{\tuple{}}
  %%   \\
  %%   \weirdat{\tseteq A}{\dx}{x}{x}{x \cup \dx}{x \cup \dx}
  %%   \\
  %%   \weirdat{\iso A}{\tuple{}}{(x,\dx)}{a}{(x,\dx)}{a}
  %%   &\impliedby \weirdat{A}{\dx}{x}{a}{x}{a}
  %%   \\
  %%   \weirdat{A \to B}{\df}{f}{f_s}{g}{g_s}
  %%   &\impliedby
  %%   \fa{\weirdat{A}{\dx}{x}{a}{y}{b}}
  %%   \\
  %%   &\phantom{{}\impliedby{}}
  %%   \weirdat{B}{\df\<x\<\dx}{f\<x}{f_s\<a}{g\<y}{g_s\<b}
  %%   \\
  %%   \weirdat{A_1 \x A_2}{\vec{\dx}}{\vec x}{\vec a}{\vec y}{\vec b}
  %%   &\impliedby
  %%   \fa{i} \weirdat{A_i}{\dx_i}{x_i}{a_i}{y_i}{b_i}
  %%   \\
  %%   \weirdat{A_1 + A_2}{\inj i \dx}{\inj i x}{\inj i a}{\inj i y}{\inj i b}
  %%   &\impliedby
  %%   \weirdat{A_i}{\dx}{x}{a}{y}{b}
  %% \end{align*}

  \nopagebreak[2]\vspace{-.5\baselineskip}
  \begin{mathpar}
    \weirdat{\tunit}{\tuple{}}{\tuple{}}{\tuple{}}{\tuple{}}{\tuple{}}

    \weirdat{\tseteq A}{\dx}{x}{x}{x \cup \dx}{x \cup \dx}

    \infer{\weirdat{A}{\dx}{x}{a}{x}{a}}{
      \weirdat{\iso A}{\tuple{}}{(x,\dx)}{a}{(x,\dx)}{a}}

    \infer{\fa{\weirdat{A}{\dx}{x}{a}{y}{b}}
      \weirdat{B}{\df\<x\<\dx}{f\<x}{f_s\<a}{g\<y}{g_s\<b}
    }{
      \weirdat{A \to B}{\df}{f}{f_s}{g}{g_s}}

    \infer{\fa{i} \weirdat{A_i}{\dx_i}{x_i}{a_i}{y_i}{b_i}}{
      \weirdat{A_1 \x A_2}{\vec{\dx}}{\vec x}{\vec a}{\vec y}{\vec b}}

    \infer{\weirdat{A_i}{\dx}{x}{a}{y}{b}}{
      \weirdat{A_1 + A_2}{\inj i \dx}{\inj i x}{\inj i a}{\inj i y}{\inj i b}}
  \end{mathpar}
\end{definition}


%%% ---------- Definitions of logical relations for correctness proof ----------

%% TODO: put these in files so we can include them in figures.tex!
\begin{definition} \label{def:changes}
  Inductively on types $A$, we define $(\changesat{A}{\dx}{x}{y})$ for $\dx :
  \den{\D A}$ and $x,y : \den{A}$, glossed as ``$\dx$ changes $x$ into $y$'', as
  the least relation such that:

  \begin{align*}
    \changes{(\dx_i)_i}{(x_i)_i}{(y_i)_i}
    &\impliedby \fa{i} \changes{\dx_i}{x_i}{y_i}
    \\
    \changes{\inj i \dx}{\inj i x}{\inj i y}
    &\impliedby \changes{\dx}{x}{y}
    \\
    \changesat{A \to B}{\df}{f}{g}
    &\impliedby \fa{\changes\dx x y} \changes{\df\<x\<\dx}{f\<x}{g\<y}
    \\
    \changesat{\iso A}{\tuple{}}{x}{x}\\
    \changesat{\tseteq A}{\dx}{x}{(x \cup \dx)}
  \end{align*}

  \noindent This extends to contexts $\G$, letting $\dgamma : \den{\D\G}$ and
  $\g,\g' : \den{\G}$, as follows:

  \begin{align*}
    \changesat{\G}{\dgamma}{\g}{\g'}
    &\iff \fa{\h x A \in \G} \changesat{A}{\dgamma_x}{\g_x}{\g'_x}
    \\
    &\hspace*{1.8em}\wedge
    %\wedge
    \fa{\hd x A \in \G}
    \changesat{\iso A}{\tuple{}}{\g_{\dvar x}}{\g'_{\dvar x}}
  \end{align*}

  \noindent Observe that the condition $\changesat{\iso A}{\tuple{}}{\g_{\dvar
      x}}{\g'_{\dvar x}}$ is equivalent to $\g_{\dvar x} = \g'_{\dvar x}$.

  %% Finally, this extends to \Poset-morphisms $f,g : \den{\G} \to \den{A}$ and
  %% $\df : \den{\iso \G \x \D\G} \to \den{\D A}$:
  %% \begin{align*}
  %%   \changesat{\G \vdash A}{\df}{f}{g} \impliedby&
  %%   \fa{\changesat{\G}{\dgamma}{\g}{\g'}}
  %%   \changesat{A}{\df\<\tuple{\g,\dgamma}}{f\<\g}{g\<\g'}
  %% \end{align*}
\end{definition}

\begin{definition}\label{def:impls}
  Inductively on types $A$, we define ${\impls_A} \subseteq \den{\Phi A} \x
  \den{A}$ as the least relation such that:
  %\par\nopagebreak[2]\vspace{-1ex}
  \begin{mathpar}
    %% \impls
    \infer{}{a \impls_{\tseteq A} a}

    \infer{x \impls_A a \\ \changesat{\Phi A}\dx x x}{
      (x,\dx) \impls_{\iso A} a}

    \tuple{} \impls_\tunit \tuple{}

    \infer{x \impls_A a \\ y \impls_B b}{
      (x,y) \impls_{A \x B} (a,b)}

    \infer{x \impls_{A_i} a}{\inj i x \impls_{A_1 + A_2} \inj i a}

    \infer{\fa{x \impls_A a} f\<x \impls_B g\<a}{f \impls_{A \to B} g}
  \end{mathpar}

  \noindent
  This lifts to contexts, ${\impls_\G} \subseteq \den{\Phi\G} \x \den\G$, as
  follows:

  \nopagebreak[2]
  \begin{align*}
    \gamma \impls_\G \rho
    &\iff \fa{\h x A \in \G} \gamma_x \impls \rho_x
    %\wedge
    \\&\hspace*{1.8em}\wedge
    \fa{\hd x A \in \G}
    (\gamma_{\dvar x}, \gamma_{\dvar \dx}) \impls_{\iso A} \rho_{\dvar x}
  \end{align*}

  \noindent Observe that the last condition is equivalent to \( (\gamma_{\dvar
    x} \impls_A \rho_{\dvar x}) \wedge (\changesat{A}{\gamma_{\dvar
      \dx}}{\gamma_{\dvar x}}{\gamma_{\dvar x}}) \).
\end{definition}


%% \begin{restatable}[Correctness]{theorem}{thmCorrect}\label{thm:correct}
%%   If $\J e \G A$ then

%%   \begin{enumerate}
%%   \item If $\gamma \impls_\G \rho$ then \(\den{\phi e} \<\gamma \impls_A \den{e}
%%     \<\rho\).
%%   \item If $\changesat{\G}{\dgamma}{\g}{\g'}$ then \(\changesat{A}{\den{\delta
%%       e} \<\tuple{\gamma,\dgamma}}{\den{\phi e} \<\gamma}{\den{\phi e}
%%     \<\gamma'}\).
%%   \end{enumerate}

%%   %% \nopagebreak[2]
%%   %% \begin{gather*}
%%   %% \text{(1)}~~ \den{\phi e}\<\gamma \impls_A \den{e}\<\rho
%%   %% \qquad\text{and}\qquad
%%   %% \text{(2)}~~
%%   %% \vld{\den{\delta e} \tuple{\gamma,\dgamma}}{\den{\phi e} \<\gamma}{\den{\phi e} \<\gamma'}
%%   %% : \incdens{\Phi{A}}
%%   %% \end{gather*}

%%   \noindent
%%   Note that $\tuple{\gamma,\dgamma}$ is a slight abuse of notation here;
%%   \todo{TODO explain that \emph{monotone} variables in $\g$ when we use it as
%%     $\den{\phi e}\<\g$ turn into \emph{discrete} variables in $\extend\g\dgamma$
%%     when we use it in $\den{\delta e} \<\extend\g\dgamma$}.
%% \end{restatable}
%% \begin{proof}
%%   By induction on the derivation of $\J e \G A$. See \XXX.
%%   %\cpageref{proof:correct}.
%% \end{proof}


%%% ---------- Change posets ----------
%% We're not really using this section at present, and we don't need the full
%% generality of the section on ``Datafun models'' either.
\input{seminaive/oldchangeposet}

\maketitle

\begin{abstract}
  We generalize semi\naive{} evaluation from Datalog to Datafun and prove it
  correct.
\end{abstract}

\section{Strategy}
We first generalize Datafun's \Poset{} semantics to any category with
appropriate structures (\cref{sec:datafun-models}). We construct such a
category, \CP{}, whose objects are posets equipped with a notion of increasing
changes, and whose morphisms are monotone maps possessing \emph{derivatives}
relating input and output changes (\cref{sec:changeposets}). We show this
``incremental'' semantics agrees with Datafun's standard semantics
(\cref{sec:incremental}). We give a static transformation on Datafun that
reinterprets the discreteness comonad to compute derivatives, and uses these to
speed up fixed point computations. Finally, we prove this semi\naive{}
evaluation transform correct.

% TODO: more section links once I finish them.


\section{Notation and conventions}

\begin{enumerate}
\item Adjacency $f\<x$ denotes the function $f$ applied to an argument $x$.

\item I write $f \then g$ for the composition of morphisms $f : A \to B$ and $g
  : B \to C$. This is the reverse of the usual composition operator, $g \circ
  f$.

\item I write $\subst{e}{x \substo f}$ to mean ``$e$ with all free occurrences of $x$
  replaced by $f$ ($\alpha$-varying as necessary to avoid capture)''.

  \newcommand\iexpr{\phi}
\item I write $(\iexpr_i)_i$ for the tuple or sequence $(\iexpr_1,
  ..., \iexpr_n)$, leaving $n$ unspecified. For example,

  \nopagebreak[2]
  \[ (2^ix + y_i)_i \quad\text{means}\quad
  (2^1x + y_1,~ 2^2x + y_2,~ ...,~ 2^nx + y_n) \]

  \noindent
  I also use this notation with other braces, eg. $\fork{\iexpr_i}_i =
  \fork{\iexpr_0, ..., \iexpr_n}$.

\item $\expO A B$ denotes the exponential $B^A$ in a cartesian closed category.

\item Given a family of morphisms $f_i : A \to B_i$ in a category with finite
  products, their \emph{``fork''} $\fork{f_i}_i : A \to \textstyle\prod_{i} B_i$
  uniquely satisfies $\fork{f_i}_i \then \pi_i = f_i$. Dually, given $f_i : A_i
  \to B$ in a category with finite sums, their \emph{``split''} $\krof{f_i}_i :
  \sum_i A_i \to B$ uniquely satisfies $\injc_i \then \krof{f_i}_i = f_i$.
  %
  So in particular, $\termI : A \to \termO$ is the unique map into the terminal
  object and $\initE : \initO \to B$ the unique map out of the initial object.

\item The comonad $\iso : \Poset \to \Poset$ takes a poset to the
  \emph{discrete} poset on the same elements, defined:

  \begin{eqnarray*}
    x \in \isof A &\iff& x \in A\\
    x \le y : \isof A &\iff& x = y\\
    \isof(f : A \to B) &=& f : \isof A \to \isof B
  \end{eqnarray*}
\end{enumerate}


\begin{figure*}\centering
  \begin{mathpar}
    \setlength\arraycolsep{.33em}\begin{array}{r@{\hskip 1em}ccl}
      \text{types} & A,B &\bnfeq& \isof A \bnfor \tset{\eqt A}
      \bnfor \tunit \bnfor A \x B \bnfor A + B \bnfor A \to B
      \\
      \text{eqtypes} & \eqt A, \eqt B &\bnfeq&
      %% TODO: revamp to make \isof an eqtype. this requires \Phi(\isof{\eqt A})
      %% = \isof{\Phi{\eqt A}} = \isof{\eqt A} and changing \phi & \delta
      %% accordingly.
      %\isof{\eqt A} \bnfor
      \tset{\eqt A} \bnfor
      \tunit \bnfor \eqt A \x \eqt B \bnfor \eqt A + \eqt B
      \\
      \text{semilattices} & L,M &\bnfeq& \tset{\eqt A} \bnfor \tunit \bnfor L \x M
      \\
      \text{fixtypes} & \fixt L, \fixt M &\bnfeq&
      \tset{\color{red}\fint A} \bnfor \tunit \bnfor \fixt L \x \fixt M
      \\[.5em]
      \text{terms} & e,f,g &\bnfeq& x \bnfor \dvar x \bnfor \fnof{x} e
      \bnfor e\<f \bnfor \etuple{} \bnfor \etuple{e,f} \bnfor \pi_i\<e\\
      &&& \inj i e \bnfor \ecase{e} (\inj i x \caseto f_i)_{i\in\{1,2\}}\\
      &&& \eboxd e \bnfor \elet{\eboxd x = e} f\\
      &&& \bot \bnfor e \vee f \bnfor \esetd{e_i}_i \bnfor \eford x e f\\
      &&& \eeqd e f \bnfor \eisEmpty e \bnfor \esplit e \bnfor \efix e
      \\[.5em]
      \text{contexts} & \G &\bnfeq& (H_i)_i\\
      \text{hypotheses} & H &\bnfeq& \h x A \bnfor \hd x A
    \end{array}
    \\
    \stripcxd\G = (\hd x A)_{\hd x A \in \G}
    \\
    \infer{\h x A \in \G}{\J x \G A}

    \infer{\hd x A \in \G}{\J {\dvar x} \G A}

    \infer{\J e {\G,\h x A} B}{\J {\fnof x e} \G {A \to B}}

    \infer{\J e \G {A \to B} \\ \J f \G A}{\J {e\<f} \G B}

    \infer{\quad}{\J {\etuple{}} \G \tunit}

    \infer{(\J{e_i}\G{A_i})_i}{\J{\etuple{e_1,e_2}} \G {A_1 \x A_2}}

    \infer{\J e \G {A_1 \x A_2}}{\J{\pi_i\<e}\G{A_i}}

    \infer{\J e \G A_i}{\J{\inj i e}\G{A_1 + A_2}}

    \infer{\J e \G {A_1 + A_2} \\
      (\J {f_i} {\G,\h {x_i} {A_i}} {B})_i
    }{
      \J {\ecase{e} (\inj i {x_i} \caseto f_i)_i} \G B
    }

    \infer{\J {\isocolor e} {\stripcxd\G} A}{\J{\eboxd e} \G {\isof A}}

    \infer{\J e \G {\isof A} \\ \J f {\G,\hd x A} B}{
      \J {\elet{\eboxd x = e} f} \G B}

    \infer{\quad}{\J\bot\G {\eqt L}}

    \infer{(\J{e_i} \G {\eqt L})_i}{\J{e_1 \vee e_2}\G {\eqt L}}

    %% \infer{\J e \G {\eqt A}}{\J {\edown e} \G {\tdown {\eqt A}}}
    \infer{(\J {\isocolor e_i} {\stripcxd\G} {\eqt A})_i}{
      \J {\esetd{e_i}_i} \G {\tset{\eqt A}}}

    %% \infer{\J e \G {\tdown {\eqt A}} \\
    %%   \J f {\G,\h x {\eqt A}} L
    %% }{\J {\ebigvee x e f} \G L}
    %%
    \infer{
      \J e \G {\tset A} \\
      \J f {\G,\hd x A} {\eqt L}
    }{\J {\eford x e f} \G {\eqt L}}

    %%\infer{\J e \G {\isof{(\eqt A \x \eqt A)}}}{\J{\prim{eq}\<e} \G {\tdown\tunit}}
    \infer{(\J {\isocolor e_i} {\stripcxd\G} {\eqt A})_i}
          {\J {\eeqd{e_1}{e_2}} \G \tbool}

    \infer{\J {\isocolor e} {\stripcxd\G} {\tset\tunit}}{
      \J {\eisEmpty e} \G {\tunit + \tunit}}

    \infer{\J e \G {\isof{(A + B)}}}{\J{\esplit e} \G {\isof A + \isof B}}

    \infer{\J e \G {\isof{(\fixt L \to \fixt L)}}}{\J{\prim{fix}\< e} \G {\fixt L}}
  \end{mathpar}

  \caption{Datafun core syntax and typing rules}
  \label{fig:core-datafun}
\end{figure*}


%% ---- Semantics in a Datafun Model ----
\begin{figure*}
  \figsectionname{Types and Contexts}

  \begin{align*}
    \den{\tunit} &= \termO & \den{A \to B} &= \expO{\den{A}}{\den{B}}
    \\
    \den{\tseteq A} &= \pfinof{\den{\eqt A}}
    & \den{A \x B} &= \den{A} \x \den{B}
    \\
    \den{\iso A} &= \discof{\den{A}} & \den{A + B} &= \den{A} + \den{B}
  \end{align*}

  \begin{align*}
    \den{\G} &= \prod_{H \in \G} \den{H} &
    \den{\h x A} &= \den{A} & \den{\hd x A} &= \discof{\den{A}} &
    \den{\G \vdash A} &= \catC(\den\G, \den A)
  \end{align*}
  \vspace{0pt} % yes, this matters.

  \figsectionname{Terms}

  \begin{align*}
    \den{\dvar x} &= \pi_{\dvar x} \then \varepsilon \quad \text{($\dvar x$ discrete)}
    & \den{x} &= \pi_x \quad \text{($x$ monotone)}
    \\
    \den{\fnof x e} &= \lambda_x\den{e}
    & \den{e\<f} &= \fork{\den{e}, \den{f}} \then \eval
    \\
    \den{\etuple{e_i}_i} &= \fork{\den{e_i}}_i
    & \den{\prj i e} &= \den{e} \then \pi_i
    \\
    \bigden{\eboxd e} &= \boxd e
    & \bigden{\elet{\eboxd x = e} f} &=
    \fork{\id, \den{e}} \then \den{f}
    \\
    \den{\bot} &= \termI \then \morph{join}_0
    &
    \den{e \vee f} &= \fork{\den{e}, \den{f}} \then \morph{join}_2
    \\
    %\den{\color{red}\prim{isEmpty}\<e} &= \den{e} \then \morph{isEmpty}
    \den{\eisempty e} &= \boxd{e} \then \morph{isEmpty}
    & \den{\esplit e} &= \den{e}\then \discosum
    \\
    %\den{\color{red}\prim{eq}\<e} &= \den{e} \then \morph{eq}
    \den{\eeqd{e_1}{e_2}} &= \fork{\boxd{e_i}}_i \then \morph{eq}
    & \den{\efix e} &= \den{e} \then \morph{fix}
    \\
    \den{\esetd{e_i}_i} &= \fork{\boxd{e_i} \then \morph{single}}_i \then \morph{join}
    &
    \den{\eford x e f} &=
    \fork{\id,\den{e}} \then \pcollect{\den{f}}
    %\den{\ebigvee x e f} &=
    %\fork{\id,\den{e}} \then \morph{strong} \then \downof{\den{f}} \then \morph{collect}
  \end{align*}

  \begin{align*}
    \den{\inj i e} &= \den{e} \then \injc_i
    \\
    \den{\ecase{e} (\inj i x_i \caseto f_i)_i} &=
    \fork{\id, \den{e}} \then \morph{dist}^\x_+ \then \bigkrof{\den{f_i}}_i
    %\bigkrof{\den{f_i}}_i \circ \morph{dist}^\x_+ \circ \fork{\id, \den{e}}
  \end{align*}
  \vspace{0pt} % yes, this matters

  \figsectionname{Lemmas}

  \begin{align*}
    \morph{box} &: (\stripcxd\G \vdash A) \to \den{\G \vdash \iso A}
    %\den\G \to_\catC \discof{\den{\stripcxd\G}} % of \strip
    \\
    \boxd e &= \fork{\pi_{\dvar x} \then \delta}_{\hd x A \in \G} \then \discox
    \then \iso \den{\isocolor e}
    \\[1ex]
    \morph{dist}^\x_+ &: A \x (B_1 + B_2) \to_\catC (A \x B_1) + (A \x B_2)
    \\
    % this could be simpler if it distributed in the opposite direction.
    \morph{dist}^\x_+ &= \fork{\pi_2 \then \krof{\lambda (\fork{\pi_2,\pi_1} \then \injc_i)}_i, \pi_1}
    \then \eval
  \end{align*}

  %% \raggedright\footnotesize Strictly speaking, the cases for
  %% $\den{\esetd{e_i}_i}$, $\den{\eisempty e}$, and $\den{\eeqd e f}$ are not
  %% structurally inductive; nonetheless $\den{e}$ is well-defined.

  \caption{Datafun semantics in a Datafun model $\tuple{\catC, \disco, \pfin}$}
  \label{fig:semantics}\label{def:strip}
\end{figure*}

\section{The category \CP}
\label{sec:changeposet}

Objects $A$ of \CP{} are tuples $\tuple{\vals A,\, \chgs A,\, {\validto_A},\,
  \dummy_A}$ where
%
\begin{enumerate}
\item $\vals A$ and $\chgs A$ are posets of ``values'' and ``changes'' respectively.

\item ${\validto_A} \subseteq \chgs A \x \vals A \x \vals A$ is a relation. We
  gloss $(\dx,x,y) \in {\validto_A}$ as ``$\dx$ changes $x$ into $y$'', and
  usually write it as $\valid \dx x y$ or more briefly $\vld \dx x y$.

\item $\dummy_A$ is a map $\iso \vals A \to \D A$. Classically, the existence of
  $\dummy$ is equivalent to saying that if $\vals A$ is inhabited then so is $\D
  A$. As we will see later, this is a crude hack to ensure the existence of
  (weak) sums.
\end{enumerate}

\noindent
Morphisms $A \to B$ consist of a map $f : \vals A \to \vals B$ equipped with a
derivative $\deriv f : \iso \vals A \to \chgs A \to \chgs B$ yielding the change
in $f$'s output given a change to its input:
%
\begin{equation}\label{eqn:deriv}
    \vld \dx x y \implies \vld {\deriv f \<x\<\dx} {f\<x}{f\<y}
\end{equation}

\noindent
Identity and composition follow the chain rule:
%
\begin{align*}
  \deriv \id \<x \<\dx &= \dx &
  \deriv{(f \then g)} \<x \<\dx &= g \<(f\<x) \<(\deriv f \<x \<\dx)
\end{align*}


\subsection{\boldmath The forgetful functor \valfn}

The functor $\valfn : \CP \to \Poset$ takes objects $A$ to $\vals A$ and
morphisms $(f,\deriv f)$ to the underlying map $f : \vals A \to \vals B$. This
can be seen as a process of ``forgetting''. On objects, it forgets a value
poset's associated change-structure; on morphisms, it forgets their derivatives.
%
Our constructions in \CP{} will commute with \valfn, inheriting from the
corresponding constructions in \Poset. For example, products have $\vals{(A \x
  B)} = \vals A \x \vals B$ and $\vals{\fork{f,g}} = \fork{\vals f, \vals g}$.
%
Where applicable, we omit for brevity this inherited structure.


\subsection{Cartesian structure}
\label{sec:CP-cartesian}
\renewcommand\tuple[1]{(#1)}
\begin{figure*}
  \begin{align*}
    \vals{\prod_i A_i} &= \prod_i \vals A_i &
    \chgs{\prod_i A_i} &= \prod_i \chgs A_i &
    \infer{
      \fa{i} \valid{\dx_i}{x_i}{y_i}
    }{\textstyle
      \valid{\tuple{\dx_i}_i}{\tuple{x_i}_i}{\tuple{y_i}_i}
    }
    \\[.5\baselineskip]
    \vals{\sum_i A_i} &= \sum_i \vals A_i &
    \chgs{\sum_i A_i} &= \sum_i \chgs A_i &
    \infer{
      \valid{\dx}{x}{y}
    }{\textstyle
      \valid{\inj i \dx}{\inj i x}{\inj i y}
    }
  \end{align*}

  \begin{align*}
    \dummy \<\tuple{x_i}_i &= \tuple{\dummy\<x_i}_i &
    \deriv\pi_i \<x \<\dx &= \prj i \dx &
    \deriv{\fork{f_i}_i} \<x \<\dx &= \tuple{\deriv f_i \<x \<\dx}_i
    \\
    \dummy\<(\inj i x) &= \inj i (\dummy\<x) &
    \deriv\injc_i \<x \<\dx &= \inj i \dx
  \end{align*}

  \begin{equation*}
    \deriv{\krof{f_i}_i} \<(\inj i x) \<(\inj j \dx)
    =
    \begin{cases}
      \deriv f_i \<x \<\dx & \text{if}~i = j\\
      \deriv f_i \<x \<(\dummy \<x) & \text{otherwise}
    \end{cases}
  \end{equation*}

  \caption{Products and sums in \CP}
  \label{fig:CP-cartesian}
\end{figure*}


\CP\ has all finite products and (weak) sums. Their value-level structure is
inherited from \Poset; \valfn\ distributes across products, sums, $\pi_i$,
$\injc_i$, $\fork{f_i}_i$, and $\krof{f_i}_i$. We give $\chgfn$, $\dummy$, and
the relevant derivatives in \cref{fig:CP-cartesian}. \todo{Give the proof that
  $\deriv{\fork{f_i}_i}$ is unique.}

Observe that we need the definition of \dummy\ to construct the derivative of
$\krof{f_i}_i$. \todo{Discuss dead code. Show how this makes
  $\deriv{\krof{f_i}_i}$ not unique and sums weak. Discuss how sums become
  strong if we forget derivatives into differentiability.}



\section{The semi\naive\ Datafun model in \CP}

\subsection{\boldmath The ``tangent bundle'' comonad \tangfn}
\label{sec:tangent-comonad}

\begin{align*}
  \vals{\tang A} &= \iso \vals A \x \chgs A &
  \chgs{\tang A} &= \termO &
  \infer{\validat A \dx x x}{
    \validat {\tang A} {\tuple{}} {\tuple{x,\dx}} {\tuple{x,\dx}}
  }
  \span
\end{align*}

\[ \dummy_{\tang A} \<\pwild = \tuple{} \]


%% ---- Seminaive ("go faster") term translation ----
\begin{figure}\centering
  %% TODO: think about this translation and syntax sugar. for example, the
  %% case-split I use in \phi(\esplit e) is only valid *because* \esplit exists.
  %%
  %% Would be nice to give the syntax sugar explicitly, but maybe more effor
  %% than its worth? Might be clearer to mostly avoid syntax sugar? Not sure.
  \figsectionname{Interesting cases}

  \begin{align*}
    \phi(\prim{fix}\<e) &= \fastfix\<\phi e\\
    \phi\eboxd{e} &= \eboxd{\etuple{\phi e, \delta e}}
    \\
    \phi(\elet{\eboxd x = e} f) &= \elet{\eboxd{\etuple{x,\dx}} = \phi e} \phi f
    \\
    \phi(\eford x e f)
    &= \eford x {\phi e} {\phi \substd{f}{dx \substo \zero\<x}}
    \\
    %% TODO: check this lines up with \delta(if then else), because that's where
    %% it comes up.
    \phi(\esplit e) &= \ecase{\phi e}\\
    &\phantom{{}={}}\quad
    \left(\eboxd{\etuple{\inj i x, \inj i \dx}}
    \caseto \inj i {\eboxd {\etuple{x,\dx}}}\right)_{i\in\{1,2\}}
    \\
    &\phantom{{}={}}\quad
    \left(\eboxd{\etuple{\inj i x, \inj j \pwild}}
    \caseto \inj i {\eboxd {\etuple{x, \dummy\<x}}} \right)_{i\ne j\in\{1,2\}}
    %% \phi(\esplit e) &= \color{red} \elet{\eboxd{x} = \phi e}\\
    %% &\phantom{{}={}} \esplit [\isocolor \ecase x\\
    %% &\phantom{= \esplit}\quad\isocolor
    %%   (\etuple{\inj i y, \inj i \dy} \caseto \inj i \etuple{y,\dy})_i\\
    %% &\phantom{= \esplit}\quad
    %%   {\isocolor(\etuple{\inj i y, \pwild}
    %%     \caseto \inj i \etuple{y, \dummy\<y})_i}]
  \end{align*}
  \vspace{0pt} % yes, this matters

  \figsectionname{Distributive cases}

  \begin{align*}
    %% TODO: I should mark discrete vs. monotone variables in some
    %% (colorblind/greyscale-printing)-safe way.
    \phi x &= x & \phi \dvar x &= \dvar x\\
    \phi(\fnof x e) &= \fnof x \phi e & \phi(e\<f) &= \phi e\<\phi f\\
    \phi\etuple{e_i}_i &= \etuple{\phi e_i}_i &
    \phi(\pi_i\<e) &= \pi_i\<\phi e\\
    \phi(\inj i e) &= \inj i \phi e\\
    \phi\bot &= \bot &
    \phi(e \vee f) &= \phi e \vee \phi f\\
    \phi(\esetd{e_i}_i) &= \esetd{\phi e_i}_i &
    \phi(\eeqd e f) &= \eeqd {\phi e} {\phi f}
  \end{align*}

  \begin{align*}
    \phi(\ecase e (\inj i x \caseto f_i)_i)
    &= \ecase{\phi e} (\inj i x \caseto \phi f_i)_i
    \\
    \phi(\eisEmpty e) &= \eisEmpty {\phi e}
  \end{align*}

  \noindent
  \todo{TODO: Explain the use of weakening in semi\naive{} $\delta(e\<f)$,
  hilighted in {\color{Rhodamine}pink}.}

  \todo{TODO: explain implementation of \zero{} via \dummy{}.}

  \caption{Semi\naive{} speed-up translation, $\phi$}
  \label{fig:seminaive-phi}
\end{figure}

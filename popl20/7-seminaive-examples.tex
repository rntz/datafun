%% TODO: rename section if we add more examples.
\section{Applying the Semi\naive\ Transformation to Transitive Closure}
\label{sec:seminaive-examples}

%% TODO: update name if the name of tc changes.
Let's try applying the semi\naive\ transform to a simple Datafun program: the
transitive closure function \name{tc} from
\cref{sec:generic-transitive-closure}:

%\newcommand\var\mathrm
%\renewcommand\isocolor\relax
%\renewcommand\dvar\mathvar
\newcommand\yone{\dvar y_{\isocolor 1}}
\newcommand\ytwo{\dvar y_{\isocolor 2}}

\[\begin{array}{l}
\name{tc} \< \eboxvar{e}
= \efixis{p} \dvar e \cup (\dvar e \bullet p)
\\
s \bullet t =
\eforraw{\etuple{\dvar x, \yone} \in s}
\eforraw{\etuple{\ytwo , \dvar z} \in t}
\ewhen{\eeqraw{\yone}{\ytwo}} \esetraw{\etuple{\dvar x, \dvar z}}
\end{array}
\]

In the process we'll discover that besides $\phi$ itself we need a few simple
optimisations to actually speed up our program: most importantly, we need to
propagate $\bot$ expressions.
%
In our experience, performing $\phi$ and $\delta$ by hand is easiest when you
work inside-out. At the core of transitive closure is a relation composition,
$(\dvar e \bullet p)$, and at the core of relation composition is a
\kw{when}-expression. Let's take a look at its $\phi$ and $\delta$ translations:

\begin{align*}
  \phi(\ewhen {\eeqraw \yone \ytwo} \esetraw{\etuple{\dvar x, \dvar z}})
  &= \phi(\eforraw {\etuple{} \in \eeqraw \yone \ytwo} \esetraw{\etuple{\dvar x,
      \dvar z}})
  && \text{desugaring}
  \\
  &= \eforraw{\etuple{} \in \eeqraw \yone \ytwo}
  \phi{\esetraw{\etuple{\dvar x, \dvar z}}}
  && \text{omitting a needless \kw{let}-binding}\\
  &= \ewhen{\eeqraw \yone \ytwo} \esetraw{\etuple{\dvar x, \dvar z}}
  && \text{resugaring}
\end{align*}

\noindent
Frequently, as in this case, $\phi$ does nothing interesting. For brevity we'll
skip such no-op translations.

\begin{align*}
  &\mathrel{\hphantom{=}}
  \delta(\ewhen {\eeqraw \yone \ytwo} \esetraw{\etuple{\dvar x, \dvar z}})
  \\
  &= \delta(\eforraw{\etuple{} \in \yone = \ytwo} \esetraw{\etuple{\dvar x, \dvar z}})
  && \text{desugaring \kw{when}}
  \\
  &= \eforraw{\etuple{} \in \delta(\yone = \ytwo)}
  \phi\esetraw{\etuple{\dvar x, \dvar z}}
  && \text{omitting needless \kw{let}-bindings}
  \\
  &\cup
  \eforraw{\etuple{} \in \phi(\yone = \ytwo) \cup \delta(\yone = \ytwo)}
  \delta\esetraw{\etuple{\dvar x, \dvar z}}
  \\
  &= \eforraw{\etuple{} \in \bot} \esetraw{\etuple{\dvar x, \dvar z}}
  \cup \eforraw{\etuple{} \in \phi(\yone = \ytwo) \cup \bot} \bot
  && \text{rules for $\phi(\eeq e f)$ and $\delta\eset{e_i}_i$}
  \\
  &= \bot && \text{propagating }\bot
\end{align*}

The core insight here is that $\yone = \ytwo$ can't change, and neither can
$\esetraw{\etuple{\dvar x,\dvar z}}$. By propagating this information --- for
example, rewriting $(\eforraw{x \in \bot} e)$ to $\bot$ --- we can simplify our
derivative down to nothing.
%
Now let's pull out and examine $\eforraw{\etuple{\ytwo, \dvar z} \in t}
\ewhen{\yone = \ytwo} \esetraw{\etuple{\dvar x, \dvar z}}$. The $\phi$
translation is again a no-op.

\begin{align*}
  &\mathrel{\hphantom{=}}
  \delta(\eforraw{\etuple{\ytwo, \dvar z} \in t}
  \ewhen{\yone = \ytwo} \esetraw{\etuple{\dvar x, \dvar z}})
  \\
  &= \eforraw{\etuple{\ytwo, \dvar z} \in \dt}
  \phi(\ewhen{\yone = \ytwo} \esetraw{\etuple{\dvar x, \dvar z}})
  && \text{omitting needless \kw{let}-bindings}
  \\
  &\cup \eforraw{\etuple{\ytwo, \dvar z} \in t \cup \dt}
  \delta(\ewhen{\yone = \ytwo} \esetraw{\etuple{\dvar x, \dvar z}})
  \\
  &= \eforraw{\etuple{\ytwo, \dvar z} \in \dt}
  \ewhen{\yone = \ytwo} \esetraw{\etuple{\dvar x, \dvar z}}
  && \text{propagating }\bot
\end{align*}

\noindent Tackling the outermost \kw{for} loop:

\begin{align*}
  &\mathrel{\hphantom{=}}
  \delta(
  \eforraw{\etuple{\dvar x, \yone} \in s}
  \eforraw{\etuple{\ytwo, \dvar z} \in t}
  \ewhen{\yone = \ytwo} \esetraw{\etuple{\dvar x, \dvar z}})
  \\
  &= \eforraw{\etuple{\dvar x, \yone} \in \ds}
  \phi(\eforraw{\etuple{\ytwo, \dvar z} \in t}
  \ewhen{\yone = \ytwo} \esetraw{\etuple{\dvar x, \dvar z}})
  && \text{definition of $\delta(\kw{for} \dots)$}
  \\
  &\cup \eforraw{\etuple{\dvar x, \yone} \in s \cup \ds}
  \delta(\eforraw{\etuple{\ytwo, \dvar z} \in t}
  \ewhen{\yone = \ytwo} \esetraw{\etuple{\dvar x, \dvar z}})
  \\
  &= \eforraw{\etuple{\dvar x, \yone} \in \ds}
  \eforraw{\etuple{\ytwo, \dvar z} \in t}
  \ewhen{\yone = \ytwo} \esetraw{\etuple{\dvar x, \dvar z}}
  && \text{applying previous work}
  \\
  &\cup
  \eforraw{\etuple{\dvar x, \yone} \in s \cup \ds}
  \eforraw{\etuple{\ytwo, \dvar z} \in \dt}
  \ewhen{\yone = \ytwo} \esetraw{\etuple{\dvar x, \dvar z}}
  \\
  &= (\ds \bullet t) \cup ((s \cup \ds) \bullet \dt)
  && \text{rewriting in terms of ${\bullet}$}
\end{align*}

\noindent
This, then, is the derivative $\delta(s \bullet t)$ of relation composition.
With a bit of rewriting, this is equivalent to $(\ds \bullet t) \cup (s \bullet
\dt) \cup (\ds \bullet \dt)$, which is perhaps the derivative a human would
give.

Let's use this to figure out $\phi(\name{tc}\<\eboxvar{e})$. Working from the
inside out, we start with the derivative of the loop body, $\delta(\dvar e \cup
(\dvar e \bullet p))$:

\begin{align*}
  \delta({\dvar e \cup (\dvar e \bullet p)})
  &= \delta\dvar e \cup \delta(\dvar e \bullet p)\\
  &= \delta\dvar e
  \cup (\delta\dvar e \bullet p)
  \cup ((\dvar e \cup \delta\dvar e) \bullet \deep)
  \\
  &= \bot \cup (\bot \bullet p) \cup ((\dvar e \cup \bot) \bullet \deep)
  && \dvar e ~\text{is discrete, and cannot change}
  \\
  &= \dvar e \bullet \deep
  && \text{propagate}~\bot
\end{align*}

\noindent
Observe that this requires a new optimization: by definition, $\delta\dvar e =
\dvar{de}$. However, since $\dvar e$ is discrete we know it's not changing, and
since it's of set type, $\dvar{de}$ may as well be the empty set. So we replace
$\delta e$ with $\bot$ instead.
%
Finally, putting everything together:

\begin{align*}
  \phi(\efixis p \dvar e \cup (\dvar e \bullet p)
  &= \phi(\efix \eboxraw{\fnof{p} \dvar e \cup (\dvar e \bullet p)})
  && \text{desugaring}
  \\
  &= \fastfix\< \phi\eboxraw{\fnof{p} \dvar e \cup (\dvar e \bullet p)}
  \\
  &= \fastfix\<\eboxraw{\etuple{\phi({\fnof{p} \dvar e \cup (\dvar e \bullet p)}),\
  \delta({\fnof{p} \dvar e \cup (\dvar e \bullet p)})}}
  \\
  &= \fastfix\<\eboxraw{\etuple{
      ({\fnof{p} \dvar e \cup (\dvar e \bullet p)}),\
      ({\fnof{\eboxvar p} \fnof{\deep} \dvar e \bullet \deep})}}
  && \text{previous work}
\end{align*}

Examining the recurrence produced by this use of \fastfix, we recover exactly
the semi\naive\ transitive closure algorithm we gave in
\cref{sec:seminaive-tc-in-datafun}:

\begin{align*}
  x_0 &= \bot & x_{i+1} &= x_i \cup \dx_i\\
  \dx_0 &= ({\fnof{p} \dvar e \cup (\dvar e \bullet p)}) \<\bot
  = \dvar e
  &
  \dx_{i+1} &=
  ({\fnof{\eboxvar p} \fnof{\dx} \dvar e \bullet \deep})
  \<\eboxraw{x_i} \<\dx_i
  = \dvar e \bullet \dx_i
\end{align*}

\begin{figure}
  \begin{tikzpicture}[baseline=(current bounding box.center)]
    \begin{axis}[
        xlabel={Number of nodes},
        ylabel={Seconds},
        height=140pt, width=220pt,
        % If we include the explicit table, we can make it this large without
        % enlarging the figure.
        height=195pt, width=280pt,
        legend entries={\naive,semi\naive},
        legend cell align=left,
        legend pos = north west,
        legend style={
          draw=none,
          nodes={inner sep=3pt,}
        },
      ]
      \addplot table {naive.dat};
      \addplot table {seminaive.dat};
    \end{axis}
  \end{tikzpicture}
  \hfil
  {\small
  \begin{tabular}{@{}rrr@{}}
    \parbox[b]{24pt}{\raggedleft Graph size} & \Naive &
    \parbox[b]{24pt}{\raggedleft Semi-\naive}
    \\\midrule
60 & 0.23s & 0.01s\\
70 & 0.43s & 0.01s\\
80 & 0.74s & 0.01s\\
90 & 1.14s & 0.02s\\
100 & 1.72s & 0.03s\\
110 & 2.50s & 0.04s\\
120 & 4.49s & 0.05s\\
130 & 7.13s & 0.07s\\
140 & 8.38s & 0.08s\\
150 & 10.05s & 0.09s\\
160 & 15.98s & 0.11s\\
170 & 21.42s & 0.14s\\
180 & 28.98s & 0.16s
  \end{tabular}
  }

  \caption{\Naive\ vs semi\naive\ transitive closure on a linear graph}
  \label{fig:seminaive-vs-naive-graph}
\end{figure}

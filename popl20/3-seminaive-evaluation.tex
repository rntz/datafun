\section{From Semi\naive{} Evaluation to the Incremental \boldfn-Calculus}
\label{sec:seminaive-and-ilc}

\todo{explain semi\naive{} evaluation on transitive closure in Datalog. Borrow from \emph{Fixing Incremental Computation} (with citation).}

\subsection{Finding Fixed Points Faster via Derivatives}


%% Now let's move from Datalog to Datafun. The transitive closure of \name{edge} is
%% the fixed point of the monotone function $f$ defined by:

%% \nopagebreak[2]
%% \[\begin{array}{l}
%%   f \<\name{path} = \name{edge} \cup
%%   \setfor{(x,z)}{(x,y) \in \name{edge}, (y,z) \in \name{path}}
%% \end{array}\]

%% \noindent
%% The \naive{} way to find a fixed point is to iterate $f$, computing the
%% recurrence

%% \nopagebreak[2]
%% \begin{align*}
%%   x_0 &= \emptyset & x_{i+1} &= f\<x
%% \end{align*}

%% \noindent
%% However, since inductively $x_i \subseteq f\<x_i$, each iteration re-computes
%% every value found in the previous iteration.
%% %
%% Following Datalog, we'd prefer to compute only the \emph{change} between
%% iterations.
%% %
%% Since $x_i \subseteq x_{i+1}$, iterations change only by gaining elements. So
%% given a set $x$, and a set $\dx$ of added elements, let's imagine a function
%% $f'\<x\<\dx$ that tells us \emph{how $f\<x$ changes}, such that:

%% \begin{equation} f \<(x \cup \dx) = f\<x \cup f'\<x\<\dx
%%   \label{eqn:f-derivative}
%% \end{equation}

%% \noindent
%% Given such an $f'$ we can set up an equivalent recurrence:

%% \begin{align*}
%%   x_0 &= \emptyset & x_{i+1} &= x_i \cup \dx_i
%%   \\
%%   \dx_0 &= f\<\emptyset & \dx_{i+1} &= f' \<x_i \<\dx_i
%% \end{align*}

%% \noindent
%% From \cref{eqn:f-derivative} it follows inductively that $x_i = f^i\<\emptyset$.

%% So, how do we find such an $f'$? One trivial answer is to let $f'\<x\<\dx =
%% f\<(x \cup \dx)$. Unfortunately, this solution is worse than the problem:
%% re-computing $f$ is exactly what we wish to avoid!
%% %
%% For our \emph{specific} $f$, there is a better answer:

%% \nopagebreak[2]
%% \[\begin{array}{l}
%%   f' \<\name{path} \<\name{dpath} =
%%   \setfor{(x,z)}{(x,y) \in \name{edge}, (y,z) \in \name{dpath}}
%% \end{array}\]


Now let's move from Datalog to Datafun. The transitive closure of \name{edge} is
the fixed point of the monotone function \name{step} defined by:

\nopagebreak[2]
\[\begin{array}{l}
  \name{step} \<\name{path} = \name{edge} \cup
  \setfor{(x,z)}{(x,y) \in \name{edge}, (y,z) \in \name{path}}
\end{array}\]

\noindent
The \naive{} way to compute $\efix \name{step}$ is to iterate it: start with
\(\name{path}_0 = \emptyset\) and compute \(\name{path}_{i+1} =
\name{step}\<\name{path}_i\) until \(\name{path}_i = \name{path}_{i+1}\).
%
But since $\name{path}_i \subseteq \name{step}\<\name{path}_i$, each iteration
re-computes every path found by the previous iteration.
%
Following Datalog, we'd prefer to compute only the \emph{change} between
iterations.
%
So consider $\name{step}'$ defined by:

\nopagebreak[2]
\[\begin{array}{l}
  \name{step}' \<\name{dpath} =
  \setfor{(x,z)}{(x,y) \in \name{edge}, (y,z) \in \name{dpath}}
\end{array}\]

\noindent
Observe that $\name{step} \<(\name{path} \cup \name{dpath}) =
\name{step}\<\name{path} \cup \name{step}'\<\name{dpath}$.
%
That is, $\name{step}'$ tells us how \name{step} changes as its input grows.
%
Using this property, we can directly compute the changes $\name{dpath}_i$
between our iterations $\name{path}_i$:

\begin{align*}
  \name{dpath}_0 &= \name{step}\<\emptyset
  & \name{dpath}_{i+1} &= \name{step}'\<\name{dpath}_i
  & \name{path}_{i+1} &= \name{path}_i \cup \name{dpath}_i
\end{align*}
%
\todo{Explain why computing $\name{step}'\<\name{dpath}_i$ isn't
  \naive/ever-increasing.}
%
This is the analogue of semi\naive{} evaluation in a functional setting.

How do we find functions like $\name{step}'$?
%
\todo{explain how $\name{step}'$ looks like a derivative of $\name{step}$ in the
  sense of \citet{incremental}.}

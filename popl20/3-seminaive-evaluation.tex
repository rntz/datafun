\section{From Semi\naive{} Evaluation to the Incremental \boldfn-Calculus}
\label{sec:seminaive-and-ilc}

\todo{explain semi\naive{} evaluation on transitive closure in Datalog. Borrow from \emph{Fixing Incremental Computation} (with citation).}

\subsection{Finding Fixed Points Faster via Derivatives}

Now let's move from Datalog to Datafun. The transitive closure of \name{edge} is the fixed point of the function $f$ defined by:

\newcommand\hilitebox[1]{{%
  \setlength\fboxsep{1pt}%
  \colorbox{Goldenrod}{\hspace{-1pt}{#1}\hspace{-1pt}}%
}}
%\renewcommand\hilitebox\relax

\nopagebreak[2]
\[\begin{array}{l}
  f \<\name{path} = \name{edge} \cup
  \setfor{(x,z)}{(x,y) \in \name{edge}, (y,z) \in \name{path}}
\end{array}\]

\noindent
The \naive{} way to compute $\efix f$ is to iterate $f$:
%computing the sequence
start with \(\name{path}_0 = \emptyset\) and compute \(\name{path}_{i+1} =
f\<\name{path}_i\) until \(\name{path}_i = \name{path}_{i+1}\).
%
%% \nopagebreak[2]
%% \begin{align*}
%%   \name{path}_0 &= \emptyset & \name{path}_{i+1} &= f\<\name{path}_i
%% \end{align*}
%
\noindent
This is \naive\ because $\name{path}_i \subseteq f\<\name{path}_i$, so each
iteration re-computes every path found by the previous iteration.
%
Following Datalog, we'd prefer to compute only the \emph{change} between
iterations.
%
So consider $f'$ defined by:

\nopagebreak[2]
\[\begin{array}{l}
  f' \<\name{dpath} =
  \setfor{(x,z)}{(x,y) \in \name{edge}, (y,z) \in \name{dpath}}
\end{array}\]

\noindent
Observe that $f \<(\name{path} \cup \name{dpath}) = f\<\name{path} \cup
f'\<\name{dpath}$. That is, $f'$ tells us how $f$ \emph{changes} when its input
grows! Using this property, we can directly compute the changes $\name{dpath}_i$
between our iterations $\name{path}_i$:

\begin{align*}
  \name{dpath}_0 &= f\<\emptyset
  & \name{dpath}_{i+1} &= f'\<\name{dpath}_i
  & \name{path}_{i+1} &= \name{path}_i \cup \name{dpath}_i
\end{align*}

\todo{Explain why computing $f'\<\name{dpath}_i$ isn't \naive/ever-increasing.}
%
This is the analogue of semi\naive{} evaluation in a functional setting: \todo{foo}. But how can we find functions like $f'$ in general?
%
\todo{explain how $f'$ looks like a derivative of $f$ in the sense of
  \citet{incremental}.}

\section{Proving the Semi\naive\ Transformation Correct}
\label{sec:logical-relation}

We have given two program transformations: $\phi e$, which optimizes $e$ by
computing fixed points semi\naive{}ly; and $\delta e$, which finds the change in
$\phi e$ under a change in its free variables.
%
To state the correctness of $\phi$ and $\delta$, we need to show that $\phi e$
preserves the meaning of $e$ and that $\delta e$ correctly updates $\phi e$ with
respect to changes in its variable bindings.
%
Since our transformations modify the types of higher-order expressions to
include the extra information needed for semi\naive\ evaluation, we cannot
directly prove that the semantics is preserved.
%
Instead, we formalize the relationship between $e$, $\phi e$, and $\delta e$
using a logical relation, and use this relation to prove an \emph{adequacy
  theorem} saying that the semantics is preserved for closed, first-order
programs.

\begin{figure}
  \begin{align*}
    \weirdat{\tunit}{\tuple{}}{\tuple{}}{\tuple{}}{\tuple{}}{\tuple{}}
    &\iff \top
    \\
    \weirdat{A_1 \x A_2}{\vec{\dx}}{\vec x}{\vec a}{\vec y}{\vec b}
    &\iff \fa{i} \weirdat{A_i}{\dx_i}{x_i}{a_i}{y_i}{b_i}
    \\
    \weirdat{A_1 + A_2}{\inj i \dx}{\inj j x}{\inj k a}{\inj l y}{\inj m b}
    &\iff i = j = k = l = m \wedge \weirdat{A_i}{\dx}{x}{a}{y}{b}
    \\
    \weirdat{A \to B}{\df}{f_\phi}{f}{g_\phi}{g}
    &\iff
    \fa{\weirdat{A}{\dx}{x}{a}{y}{b}} \\
    &\hphantom{{}\iff{}}
    \weirdat{B}{\df\<x\<\dx}{f_\phi\<x}{f\<a}{g_\phi\<y}{g_\phi\<b}
    \\
    \weirdat{\tseteq A}{\dx}{x}{a}{y}{b}
    &\iff (x,y,x \cup \dx) = (a,b,y)
    \\
    \weirdat{\iso A}{\tuple{}}{(x,\dx)}{a}{(y,\dy)}{b}
    &\iff (a,x,\dx) = (b,y,\dy) \wedge \weirdat{A}{\dx}{x}{a}{y}{b}
  \end{align*}
  \caption{Definition of the logical relation}
  \label{fig:logical-relation}
\end{figure}


So, inductively on types $A$, letting $a,b \in \den{A}$, $x,y \in \den{\Phi A}$,
and $\dx \in \den{\DP A}$, we define a five place relation
$\weirdat{A}{\dx}{x}{a}{y}{b}$, meaning roughly ``$x,y$ speed up $a,b$
respectively, and $\dx$ changes $x$ into $y$''. The full definition is in
\cref{fig:logical-relation}.

At product, sum, and function types this is essentially a more elaborate version
of the change structures given in \cref{sec:change-structures}.
%
At set types, changes are still a set of values added to the initial value, but
we additionally insist that the ``slow'' $a,b$ and ``speedy'' $x,y$ are equal.
%
This is because we have engineered the definitions of $\Phi$ and $\phi$ to
preserve behavior on equality types.
%
Finally, since $\iso A$ represents values which cannot change, $\dx$ is an
uninformative empty tuple and the original and updated values are identical.
%
However, the ``speedy'' values are now \emph{pairs} of a value and its zero
change.
%
This ensures that at a boxed function type, we will always have a derivative (a
zero change) available.

The logical relation is defined on simple values, and so before we can state the
fundamental theorem, we have to extend it to contexts $\G$ and substitutions,
letting $\rho,\rho' \in \den{\G}$, $\g,\g' \in \den{\Phi\G}$, and $\dgamma \in
\den{\DP\G}$:

\nopagebreak[1]
\begin{align*}
  \weirdat{\G}{\dgamma}{\g}{\rho}{\g'}{\rho'}
  &\iff \fa{\h x A \in \G} \weirdat{A}{\dgamma_{\dx}}{\g_x}{\rho_x}{\g'_x}{\rho'_x}
  \\
  &\hphantom{{}\iff{}} \hspace*{-13.2pt} \wedge \fa{\hd x A \in \G}
  \weirdat{\iso A}
          {\etuple{}}
          {(\g_{\dvar \dx}, \g_{\dvar x})}
          {\rho_{\dvar x}}
          {(\g'_{\dvar\dx}, \g'_{\dvar x})}
          {\rho'_{\dvar x}}
\end{align*}

\noindent
With that in place, we can state the fundamental theorem, showing that
$\phi$ and $\delta$ generate expressions which satisfy this logical
relation:

\begin{theorem}[fundamental property]
  If\/ $\J e \G A$ and\/ $\weirdat{\G}{\dgamma}{\g}{\rho}{\g'}{\rho'}$ then
  \[\weirdat{A}{\den{\delta e} \<\tuple{\g,\dgamma}}{\den{\phi
      e}\<\g}{\den{e}\<\rho}{\den{\phi e}\<\g'}{\den{e}\<\rho'}\]
\end{theorem}

\noindent
This theorem follows by a structural induction on typing derivations as usual,
but a number of lemmas need to be proved in order to establish the fundamental
theorem.

By and large, these lemmas generalize or build on results stated
earlier in this paper regarding the simpler change structures from
\cref{sec:change-structures}.
%
For example, we build on \cref{thm:phi-eqt,lem:dummy-change} to characterize the
logical relation at equality types $\eqt A$ and the behavior of \dummy:

\begin{lemma}[equality changes]\label{lem:equality-changes}
  If\/ $\weirdat{\eqt A}{\dx}{x}{a}{y}{b}$ then\/ $x = a$ and\/ $y = b$.
\end{lemma}

\begin{lemma}[dummy is zero at eqtypes]\label{lem:lr-dummy}
  If\/ $x \in \den{\eqt A}$ then $\weirdat{\eqt A}{\dummy\<x}{x}{x}{x}{x}$.
\end{lemma}

\noindent
\Cref{lem:equality-changes} tells us that at equality types, the sped-up version
of a value is the value itself. This is used later to prove our adequacy
theorem.
%
\Cref{lem:lr-dummy} is an analogue of \cref{lem:dummy-change}, showing that
\dummy\ function computes zero changes at equality types.
%
This is used in the proof of the fundamental theorem for \kw{for}-loops,
in whose $\phi$ and $\delta$ translations $\zero$ is implemented by \dummy.

Next, we generalize \cref{lem:DeltaL} to characterize changes at semilattice
type:

%% The lemma involves the lattice types, showing that a change for a lattice type
%% $L$ is something that can be joined on to it:

\begin{lemma}[semilattice changes]
  At any semilattice type\/ $L$, we have\/ $\Delta L = L$, and moreover\/
  $\weirdat{L}{\dx}{x}{a}{y}{b}$ iff\/ $x = a$ and\/ $y = b = x \vee_L \dx$
\end{lemma}

\noindent
This follows by induction on semilattice types $L$, and from
\cref{lem:equality-changes} (noting that every semilattice type is an equality
type). We require this lemma in the proofs of the fundamental theorem in all the
cases involving semilattice types -- namely $\bot$, ${\vee}$, \kw{for}-loops,
and \prim{fix}.

Since typing rules that involve discreteness (such as the $\iso$ rules)
manipulate the context, we need some lemmas regarding these manipulations.
First, we show that all valid changes for a context with only discrete variables
send substitutions to themselves, recalling that $\stripcx{\G}$ contains only the
discrete variables from $\G$.

\begin{lemma}[discrete contexts don't change]
  If\/ $\weirdat{\stripcx{\G}}{\tuple{}}{\gamma}{\rho}{\gamma'}{\rho'}$ then\/
  $\gamma = \gamma'$ and\/ $\rho = \rho'$.
\end{lemma}

\noindent
We use this lemma in combination with the next, which says that any valid
context change gives rise to a valid change on a stripped context:

\begin{lemma}[context stripping]
  If\/ $\weirdat{\G}{\dgamma}{\gamma}{\rho}{\gamma'}{\rho'}$
  then

  \[
  \weirdat
      {\stripcx{\G}}
      {\tuple{}}
      {\strip_{\Phi\G}(\g)}
      {\strip_{\G}(\rho)}
      {\strip_{\Phi\G}(\g')}
      {\strip_{\G}(\rho')}
  \]

  where $\strip_\G = \fork{\pi_{\dvar x}}_{\hd x A \in \G}$ keeps only the
  discrete variables from a substitution.
\end{lemma}

\noindent
Jointly, these two lemmas ensure that a valid change to any context is an
identity on the discrete part. We use these in all the cases of the fundamental
theorem involving discrete expressions -- equality $\eeq{e_1}{e_2}$, set
literals $\esetsub{e_i}{i}$, emptiness tests $\eisempty e$, and box
introduction $\ebox e$.

Once the fundamental theorem has been established, we can specialize it to
closed terms and equality types, Then, the equality changes lemma implies
adequacy -- that first-order closed programs compute the same result when
$\phi$-translated:

\begin{theorem}[adequacy]
  If\/ $\J e {\emptycx} {\eqt A}$ then\/ $\den{e} = \den{\phi e}$.
\end{theorem}

%% \nopagebreak[2]
%% \vspace{-\baselineskip}
%% \begin{mathpar}
%%   \weirdat{\tunit}{\tuple{}}{\tuple{}}{\tuple{}}{\tuple{}}{\tuple{}}

%%   \weirdat{\tseteq A}{\dx}{x}{x}{x \cup \dx}{x \cup \dx}

%%   \infer{\weirdat{A}{\dx}{x}{a}{x}{a}}{
%%     \weirdat{\iso A}{\tuple{}}{(x,\dx)}{a}{(x,\dx)}{a}}

%%   \infer{\fa{i} \weirdat{A_i}{\dx_i}{x_i}{a_i}{y_i}{b_i}}{
%%     \weirdat{A_1 \x A_2}{\vec{\dx}}{\vec x}{\vec a}{\vec y}{\vec b}}

%%   \infer{\weirdat{A_i}{\dx}{x}{a}{y}{b}}{
%%     \weirdat{A_1 + A_2}{\inj i \dx}{\inj i x}{\inj i a}{\inj i y}{\inj i b}}

%%   \infer{\fa{\weirdat{A}{\dx}{x}{a}{y}{b}}
%%     \weirdat{B}{\df\<x\<\dx}{f\<x}{f_s\<a}{g\<y}{g_s\<b}
%%   }{
%%     \weirdat{A \to B}{\df}{f}{f_s}{g}{g_s}}
%% \end{mathpar}

\section{Discussion and Related Work}
\label{sec:related-work}

\newcommand\mvar[1]{\mathvar{\color{black}#1}}

\paragraph{Nested fixed points} The typing rule for $\efix e$ requires $e : \iso(\fixt L \to \fixt L)$.
%
The $\phi$ translation takes advantage of this $\iso$, decorating expressions of
type $\iso A$ with their zero-changes.
%
However, it also prevents an otherwise valid idiom: in a nested fixed-point
expression $\efixis x \dots (\efixis y e) \dots$, the inner fixed point body $e$
cannot use the monotone variable $x$!
%
This restriction is not present in \citet{datafun}; its addition brings Datafun
closer to Datalog, whose syntax cannot express this sort of nested fixed point.

We suspect it is possible to lift this restriction without losing
sem\naive\ evaluation, by decorating \emph{all} expressions and variables (not
just discrete ones) with zero-changes.
%
However, this also invalidates $\delta(\efix f) = \bot$: now that $f$ can
change, so can $\efix f$.
%
Luckily, there is a simple and correct solution: $\Deriv(\efix f) = \efix \ebox{\Deriv f \<\ebox{\efix f}}$~\cite{delta-fix}.
%
However, to compute this new fixed point semi\naive{}ly, we need a \emph{second
  derivative}: the zero-change to $\Deriv f \<\ebox{\efix f}$. Indeed, for a
program with fixed points nested $n$ deep, we need $n$\textsuperscript{th}
derivatives. We leave this to future work.

%% Can't we just have \delta produce two expressions: the derivative, and the
%% zero-change to the derivative?

%% \todo{Discuss possibility of allowing nested fixed points by removing the $\iso$
%%   from $\prim{fix}$'s argument. The problem here is $\delta(\efix f)$. I've
%%   previously worked out what this should be (cite that 3-page proof on my
%%   website): $\delta(\efix f) = \efix(\delta f \<(\efix f))$. But if we want to
%%   compute \emph{this} fixed point faster, we need a \emph{second derivative}. So
%%   if your program has $n$ nested fixed points, you probably need
%%   $n$\textsuperscript{th} derivatives. Future work!}

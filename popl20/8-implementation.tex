\section{Implementation and Optimization}
\label{sec:implementation}

\newcommand\rewrites\leadsto

To test whether the $\phi$ translation can produce the asymptotic performance
gains we claim, we have implemented a compiler from a fragment of Datafun
(omitting sum types) to Haskell.
%
We use Haskell's \texttt{Data.Set} to represent Datafun sets, and
typeclasses to implement Datafun's notions of equality and semilattice types.
%
We do no query planning; relational joins, written in Datafun as nested
\kw{for}-loops, are compiled into nested loops.
%
Consequently our performance is worse than any real Datalog engine.
%
However, we do implement the $\phi$ translation, along with the following
optimizations:

\begin{itemize}
\item Propagating $\bot$; for example, rewriting $(e \vee \bot) \rewrites e$ and
  $(\efor{x}{e} \bot) \rewrites \bot$.

\item Inserting $\bot$ in place of semilattice-valued zero changes (for example,
  changes to discrete variables $\delta \dvar x$). This makes $\bot$-propagation
  more effective.

\item Recognising complex zero change expressions; for example, $\delta e
  \<\ebox{\phi f} \<\delta f$ is a zero change if $\delta e$ and $\delta f$ are.
  This allows more zero changes to be replaced by $\bot$, especially in
  higher-order code such as our regular expression example.
  %% TODO: does this actually make a difference?
\end{itemize}

\providecommand\AlegreyaSansTOsF{}
\begin{figure}
  \centering\small\sffamily\AlegreyaSansTOsF
  \begin{tikzpicture}[baseline=(current bounding box.center)]
    \begin{axis}[
        title={{\scshape transitive closure on a linear graph}\vphantom{\texttt{/a*/}$\texttt{a}^n$\textsuperscript{N}}},
        xlabel={Number of nodes},
        ylabel={Time (seconds)},
        height=123.6pt, width=200pt,
        height=130pt, width=210.34pt,
        legend entries={\naive,semi\naive\ raw,semi\naive\ full},
        legend cell align=left,
        legend pos = north west,
        legend style={
          draw=none,
          nodes={inner sep=3pt,}
        },
        xtick = {120, 160, ..., 320}, ytick = {0, 60, ..., 150},
      ]
      \addplot [color=red,mark=square*] table [x=n,y=naive] {trans.dat};
      \addplot [color=black,mark=triangle*] table [x=n,y=seminaive_raw] {trans.dat};
      \addplot [color=blue,mark=*] table  [x=n,y=seminaive] {trans.dat};
    \end{axis}
  \end{tikzpicture}
  \hfil
  \begin{tikzpicture}[baseline=(current bounding box.center)]
    \begin{axis}[
        title={{\scshape matching \texttt{/a*/} against} $\texttt{a}^{n}$},
        xlabel={Number of characters},
        height=123.6pt, width=200pt,
        height=130pt, width=210.34pt,
        legend entries={\naive,semi\naive\ raw,semi\naive\ full},
        legend cell align=left,
        legend pos = north west,
        legend style={
          draw=none,
          nodes={inner sep=3pt,}
        },
        xtick = {120, 160, ..., 320}, ytick = {0, 60, ..., 150},
      ]
      \addplot [color=red,mark=square*] table [x=n,y=naive] {astar.dat};
      \addplot [color=black,mark=triangle*] table [x=n,y=seminaive_raw] {astar.dat};
      \addplot [color=blue,mark=*] table  [x=n,y=seminaive] {astar.dat};
    \end{axis}
  \end{tikzpicture}
  \scriptsize

  \vspace{\baselineskip}

  %% We color rows according to corresponding lines in the graphs.
  \colorlet{darkred}{red!65!black}
  \colorlet{darkblue}{blue!65!black}

  \setlength\tabcolsep{4pt}
  \begin{tabu}{@{}l*{11}{r}@{}}
    & \multicolumn{11}{c}{\scshape graph size / string length}\\
    & 120 & 140 & 160 & 180 & 200 & 220 & 240 & 260 & 280 & 300 & 320
    \\\midrule
    \rowfont{\color{darkred}} \scshape regex search, \naive & 1.683 & 2.786 & 4.549 & 7.324 & 11.357 & 17.304 & 25.788 & 45.634 & 65.174 & 90.934 & 123.023
    \\
    \rowfont{\color{darkred}} \scshape transitive closure, \naive & 1.446 & 2.599 & 4.356 & 6.933 & 10.840 & 16.803 & 27.159 & 44.136 & 64.953 & 88.154 & 119.604
    \\
    \scshape regex search, semi\naive\ raw & 1.687 & 2.454 & 4.134 & 6.573 & 9.854 & 14.611 & 21.661 & 39.171 & 56.345 & 79.687 & 108.236
    \\
    \scshape transitive closure, semi\naive\ raw & 1.187 & 2.163 & 4.154 & 5.835 & 8.982 & 13.350 & 21.069 & 36.512 & 53.197 & 75.209 & 101.933
    \\
    \rowfont{\color{darkblue}} \scshape regex search, semi\naive\ full& 0.028 & 0.037 & 0.054 & 0.075 & 0.102 & 0.133 & 0.171 & 0.220 & 0.269 & 0.331 & 0.401
    \\
    \rowfont{\color{darkblue}} \scshape transitive closure, semi\naive\ full & 0.026 & 0.037 & 0.056 & 0.072 & 0.099 & 0.130 & 0.170 & 0.204 & 0.259 & 0.312 & 0.377
  \end{tabu}

  \caption{\Naive\ vs semi\naive\ evaluation of transitive closure and regex matching in Datafun}
  \label{fig:seminaive-vs-naive-graph}
\end{figure}


We benchmarked the transitive closure function \name{tc} from
\cref{sec:generic-transitive-closure}, compiled both \naive{}ly and
semi\naive{}ly (i.e. omitting or using $\phi$), against the linear graph
$\setfor{(i,i+1)}{0 \le i \le n}$. We chose transitive closure on a linear graph
because it is a best-case for semi\naive\ evaluation, with a well-understood
asymptotic speedup (discussed in \cref{sec:seminaive-and-ilc}). The results
(\cref{fig:seminaive-vs-naive-graph}) are consistent with our expectation that
$\phi$-translated code, with appropriate optimisation, can perform
asymptotically better than \naive\ evaluation. As in Datalog, we do not expect
this to generalize to all recursive programs; as a rule of thumb,
semi\naive\ evaluation speeds up programs that take many iterations
to reach a fixed point.
%
%% We expect and hope the same pattern holds true of Datafun.
%
\todo{discuss what kinds of problems semi\naive\ evaluation produces
  speedups on? (generally, ones which require many iterations, since
  \naive\ evaluation grows monotonically more expensive with more iterations.)
  can we cite somebody here -- maybe ``Fixing incremental computation''?}

%% TODO: Discuss when \& why inlining might be helpful.
%% We speculate that \emph{function inlining} would also be a helpful optimisation.

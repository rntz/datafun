\section{Implementation and Optimization}
\label{sec:implementation}

\newcommand\rewrites\leadsto

We have implemented a compiler from a fragment of Datafun (omitting sum types) to Haskell, available at \url{https://github.com/rntz/datafun/tree/popl20/v4-fastfix}.
%\url{http://www.rntz.net/datafun/popl20/}. % TODO
%
We use Haskell's \texttt{Data.Set} to represent Datafun sets, and typeclasses to implement Datafun's notions of equality and semilattice types.
%
We do no query planning and implement no join algorithms; relational
joins, written in Datafun as nested \kw{for}-loops, are compiled into nested
loops.
%
Consequently our performance is worse than any real Datalog engine.
%
However, we do implement the $\phi$ translation, along with the following
optimizations:

\begin{enumerate}
\item Propagating $\bot$; for example, rewriting $(e \vee \bot) \rewrites e$ and
  $(\efor{x}{e} \bot) \rewrites \bot$.

\item Inserting $\bot$ in place of semilattice-valued zero changes (for example,
  changes to discrete variables $\delta \dvar x$). This makes $\bot$-propagation
  more effective.

\item Recognising complex zero change expressions; for example, $\delta e
  \<\ebox{\phi f} \<\delta f$ is a zero change if $\delta e$ and $\delta f$ are.
  This allows more zero changes to be replaced by $\bot$, especially in
  higher-order code such as our regular expression example.
\end{enumerate}

\begin{figure}
  \begin{tikzpicture}[baseline=(current bounding box.center)]
    \begin{axis}[
        xlabel={Number of nodes},
        ylabel={Seconds},
        height=140pt, width=220pt,
        % If we include the explicit table, we can make it this large without
        % enlarging the figure.
        height=195pt, width=280pt,
        legend entries={\naive,semi\naive},
        legend cell align=left,
        legend pos = north west,
        legend style={
          draw=none,
          nodes={inner sep=3pt,}
        },
      ]
      \addplot table {naive.dat};
      \addplot table {seminaive.dat};
    \end{axis}
  \end{tikzpicture}
  \hfil
  {\small
  \begin{tabular}{@{}rrr@{}}
    \parbox[b]{24pt}{\raggedleft Graph size} & \Naive &
    \parbox[b]{24pt}{\raggedleft Semi-\naive}
    \\\midrule
60 & 0.23s & 0.01s\\
70 & 0.43s & 0.01s\\
80 & 0.74s & 0.01s\\
90 & 1.14s & 0.02s\\
100 & 1.72s & 0.03s\\
110 & 2.50s & 0.04s\\
120 & 4.49s & 0.05s\\
130 & 7.13s & 0.07s\\
140 & 8.38s & 0.08s\\
150 & 10.05s & 0.09s\\
160 & 15.98s & 0.11s\\
170 & 21.42s & 0.14s\\
180 & 28.98s & 0.16s
  \end{tabular}
  }

  \caption{\Naive\ vs semi\naive\ transitive closure on a linear graph}
  \label{fig:seminaive-vs-naive-graph}
\end{figure}


\noindent
To test whether the $\phi$ translation can produce the asymptotic performance
gains we claim, we benchmark two example Datafun programs:

\begin{enumerate}
\item Finding the transitive closure of a linear graph using the \name{trans}
  function from \cref{sec:generic-transitive-closure}. We chose this example
  because, as discussed in \cref{sec:seminaive-and-ilc}, it has a well
  understood asymptotic speed-up under semi\naive\ evaluation. \todo{explain
    further? ``maximize visibility of failure''?}

\item Finding all matches of the regular expression \texttt{/a*/} in the string
  $\texttt{a}^n$, using the regex combinators from \cref{sec:regex-combinators}.
  Finding all matches for \texttt{/a*/} amounts to finding the reflexive,
  transitive closure of the matches of \texttt{/a/}, and on $\texttt{a}^n$ these
  form a linear graph. Thus it is a close computational analogue of our first
  example, written in a higher-order style. We chose this example to test
  whether our extension of semi\naive\ evaluation properly handles Datafun's
  distinctive feature: higher-order programming.
\end{enumerate}

We compiled each program in three distinct ways: \emph{\naive{}}, without the
$\phi$ transform (but with $\bot$-propagation); \emph{semi\naive{} unoptimized},
with the $\phi$ transform but without further optimization; and
\emph{semi\naive{} optimized}, with the $\phi$ transform followed by all three
optimizations listed previously. The results are shown in
\cref{fig:seminaive-vs-naive-graph}. The $\phi$ transform alone
(\emph{semi\naive\ unoptimized}) provides only a small speed-up, roughly
\todo{15--20\%}. Only when followed by other optimizations
(\emph{semi\naive\ optimized}) does it provide the expected asymptotic speedup.
We believe this is because, as we observed in \cref{sec:seminaive-examples}, the

\todo{why are these optimizations necessary?}

%% \noindent
%% We benchmarked the transitive closure function \name{trans} from
%% \cref{sec:generic-transitive-closure}, compiled both \naive{}ly and
%% semi\naive{}ly (i.e. omitting or using $\phi$), against the linear graph
%% $\setfor{(i,i+1)}{0 \le i \le n}$. We chose transitive closure on a linear graph
%% because it is a best-case for semi\naive\ evaluation, with a well-understood
%% asymptotic speedup (discussed in \cref{sec:seminaive-and-ilc}).

%% The results
%% (\cref{fig:seminaive-vs-naive-graph}) are consistent with our expectation that
%% $\phi$-translated code, with appropriate optimization, can perform
%% asymptotically better than \naive\ evaluation.

As in Datalog, we do not expect semi\naive\ evaluation to be so useful on
\emph{all} recursive programs; as a rule of thumb, it speeds up programs that
take many iterations to reach a fixed point.
%
%% We expect and hope the same pattern holds true of Datafun.
%
\todo{discuss what kinds of problems semi\naive\ evaluation produces
  speedups on? (generally, ones which require many iterations, since
  \naive\ evaluation grows monotonically more expensive with more iterations.)
  can we cite somebody here -- maybe ``Fixing incremental computation''?}

%% TODO: Discuss when \& why inlining might be helpful.
%% We speculate that \emph{function inlining} would also be a helpful optimization.

\section{The \boldphi\ and \bolddelta\ Transformations}
\label{sec:transformations}

We use two static transformations, $\phi$ and $\delta$. Their definitions are
given in \cref{fig:phi,fig:delta} respectively, but rather than diving straight
in, we \XXX.

The speed-up transform $\phi e$ computes fixed points semi\naive{}ly by
replacing $\efix f$ by $\fastfix\<({f,f'})$.
%
But to find the derivative $f'$ of $f$ we'll need a second transform, called
$\delta e$.
%
Since a derivative is a zero change, can $\delta e$ simply find a zero change to
$e$?
%
Unfortunately, this is not strong enough.
%
For example, the derivative of $\fnof x e$ depends on how $e$ changes as its
free variable $x$ changes --- which is not necessarily a zero change.
%
To compute derivatives, we need to solve the general problem of computing
\emph{changes}.
%
So, modelled on the incremental \fn-calculus' $\Deriv$ \citep{incremental},
$\delta e$ will compute how $\phi e$ changes as its free variables
change.%
%% \footnote{In order to make their mutual recursion work, $\delta e$
%%   incrementalizes $\phi e$, not $e$. We'll see why this is necessary later.
%%   \todo{insert forward ref.}}

However, to speed up $\efix e$ we don't want the change to $e$; we want its
derivative.
%
Since derivatives are zero-changes, function changes and derivatives coincide if
\emph{the function cannot change}.
%
This is why the typing rule for $\efix e$ (\todo{xref}) requires that $e : \iso(\fixt L
\to \fixt L)$: the use of $\iso$ prevents $e$ from changing!
%
So the key strategy of our speed-up transformation is to {\bfseries\boldmath
  decorate expressions of type ${\iso A}$ with their zero-changes.}
%
This makes derivatives available exactly where we need them: at \prim{fix}
expressions.


\subsection{Typing \boldphi\ and \bolddelta}

\begin{figure}\centering
  \begin{align*}
    \D\tseteq A &= \tseteq A
    &
    \Phi\tseteq A &= \tset{\Phi{\eqt A}}
    \qquad\text{\todo{(see \cref{thm:phi-eqtype})}}
    \\
    \D \iso A &= \tunit
    &
    \Phi \iso A &= \iso{(\Phi A \x \DP A)}
    \\
    \D\tunit &= \tunit
    &
    \Phi\tunit &= \tunit
    \\
    \D(A \x B) &= \D A \x \D B
    &
    \Phi(A \x B) &= \Phi A \x \Phi B
    \\
    \D(A + B) &= \D A + \D B
    &
    \Phi(A + B) &= \Phi A + \Phi B
    \\
    \D(A \to B) &= \iso A \to \D A \to \D B
    &
    \Phi(A \to B) &= \Phi A \to \Phi B
  \end{align*}

  \caption{$\D$ and $\Phi$ type transformations}
  \label{fig:DeltaPhi}
\end{figure}


In order to decorate expressions with extra information, $\phi$ also needs to
decorate their types. In \cref{fig:DeltaPhi} we give a type translation $\Phi A$
capturing this.
%
In particular, if $e : \iso A$ then $\phi e$ will have type $\Phi(\iso A) =
\iso(\Phi A \x \DP A)$.
%
The idea is that evaluating $\phi e$ will produce a pair
$\eboxraw{\etuple{x,\dx}}$ where $x : \Phi A$ is the sped-up result and $\dx :
\DP A$ is a zero-change to $x$.
%
Thus, if $e : \iso(\fixt L \to \fixt L)$, then $\phi e$ will compute
$\eboxraw{\etuple{f,f'}}$, where $f'$ is the derivative of $f$.

On types other than $\iso A$, there is no information we need to add, so $\Phi$
simply distributes.
%
In particular, source programs and sped-up programs agree on the shape of
first-order data:

\begin{lemma}\label{thm:phi-eqt}
  $\Phi\eqt A = \eqt A$.
\end{lemma}
\begin{proof}
  Induct on $\eqt A$.
\end{proof}

For reasons that we'll discuss later \todo{fwdref}, $\phi$ and $\delta$ are
mutually recursive; to make this work, $\delta e$ must find the change to $\phi
e$ rather than $e$.
%
So if $e : A$ then $\phi e : \Phi A$ and $\delta e : \DP A$.
%
However, so far we have neglected to say what $\phi$ and $\delta$ do to typing
contexts.
%
To understand this, it's helpful to look at what $\Phi$ and $\DP$ do to
functions and to $\iso$.
%
This is because expressions denote functions of their free variables.
%
Moreover, in Datafun free variables come in two flavors, monotone and discrete, and discrete variables are semantically $\iso$-ed.

If we view expressions as functions of their free variables, $\delta e$ will
denote the \emph{derivative} of the function $\phi e$ denotes.
%
And just as the derivative of a unary function $f\<x$ has \emph{two} arguments,
$\df\<x\<\dx$, the derivative of an expression $e$ with $n$ variables $x_1,
\dots, x_n$ will have $2n$ variables: the original $x_1, \dots, x_n$ and their
changes $\dx_1, \dots, \dx_n$.%
%
\footnote{We assume throughout the paper as a matter of notational convenience
  that source programs contain no variables starting with the letter \emph{d}.}
%
However, this says nothing yet about monotonicity or discreteness.
%
To make this precise, we'll use three context transformations, named according
to the analogous type operators $\iso$, $\Phi$, and $\Delta$:
%% %
%% We define these pointwise by their action on singleton contexts (they all
%% preserve empty contexts and distribute across context union):

\begin{align*}
  \iso{(\h x A)} &= \hd x A & \iso{(\hd x A)} &= \hd x A
  \\
  \Phi(\h x A) &= \h x {\Phi A} & \Phi(\hd x A) &= \hd x {\Phi A}, \hd \dx {\DP A}
  \\
  \D(\h x A) &= \h \dx {\D A}
  & \D(\hd x A) &= \emptycx \quad\text{\small(the empty context)}
\end{align*}

%% \begin{align*}
%%   \iso{(\h x A)} &= \hd x A & \iso{(\hd x A)} &= \hd x A
%%   \\
%%   \Phi(\h x A) &= \h x {\Phi A} & \Phi(\hd x A) &= \hd x {\Phi A}, \hd \dx {\DP A}
%%   \\
%%   \D(\h x A) &= \h \dx {\D A} & \D(\hd x A) &= \emptycx\quad\text{(the empty context)}
%% \end{align*}

\noindent
%% (All three preserve empty contexts and distribute over union; e.g. $\Phi\emptycx
%% = \emptycx$ and $\Phi(\G_1,\G_2) = \Phi\G_1, \Phi\G_2$.)
(Otherwise all three operators distribute; e.g.\ $\iso\emptycx = \emptycx$ and
$\iso(\G_1,\G_2) = \iso\G_1, \iso\G_2$.)

Intuitively, $\iso\G$, $\Phi\G$, and $\D\G$ mirror the effect of
$\iso$, $\Phi$, and $\D$ on the semantics of $\G$:

\begin{align*}
  \den{\iso\G} &\cong \iso\den\G
  &
  \begin{aligned}
    \den{\Phi(\h x A)} &\cong \den{\Phi A}
    \\
    \den{\Phi(\hd x A)} &\cong \den{\Phi \iso A}
  \end{aligned}
  &&
  \begin{aligned}
    \den{\D(\h x A)} &\cong \den{\D A}
    \\
    \den{\D(\hd x A)} &\cong \den{\D \iso A}
  \end{aligned}
\end{align*}

%% \begin{align*}
%%   \multirow{2}{*}{\den{\iso \G} = \iso\den\G}
%%   &&
%%   \den{\D(\h x A)} &\cong \den{\D A}
%%   &
%%   \den{\Phi(\h x A)} &\cong \den{\Phi A}
%%   \\
%%   &&
%%   \den{\D(\hd x A)} &\cong \den{\D \iso A}
%%   &
%%   \den{\Phi(\hd x A)} &\cong \den{\Phi \iso A}
%% \end{align*}

\noindent
These defined, we can state the types of $\phi e$ and $\delta e$:

\begin{theorem}[Well-typedness]
  \label{thm:type-correct}
  If $\J e \G A$, then
  \begin{align*}
    \Jalign {\phi e} {\Phi\G} {\Phi A}\\
    \Jalign {\delta e} {\iso{\Phi\G}, \DP\G} {\DP A}
  \end{align*}
\end{theorem}

\begin{proof}
  By induction on typing derivations; see appendix. \XXX
\end{proof}

\noindent As expected if we view expressions as functions of their free
variables, if we pretend $\G$ is a type, these correspond to $\Phi(\G \to A)$
and $\DP(\G \to A)$ respectively:

\begin{align*}
  \Phi(\G \to A) &= \Phi\G \to \Phi A
  & \DP(\G \to A) &= \iso\Phi\G \to \DP\G \to \DP A
\end{align*}

\noindent
To get the hang of these context and type transformations, suppose $\J
e {\hd x A, \h y B} C$. Then \cref{thm:type-correct} tells us:

\begin{align*}
  \Jalign{\phi e} {\hd x{\Phi A},\, \hd \dx {\DP A},\, \h y {\Phi B}} {\Phi C}
  \\
  \Jalign{\delta e} {\hd x{\Phi A},\, \hd\dx{\DP A},\, \hd y{\Phi B}, \h\dy{\DP B}} {\DP C}
\end{align*}

We now have enough information to tackle the definitions of $\phi$ and $\delta$
given in \cref{fig:phi,fig:delta}. In the remainder of this section, we'll
examine the most interesting and important parts of these definitions in detail.

%% ---- "Go faster" term translation, phi ----
\begin{figure}\centering
  \begin{align*}
    \phi x &= x & \phi \dvar x &= \dvar x\\
    \phi(\fnof x e) &= \fnof x \phi e & \phi(e\<f) &= \phi e\<\phi f\\
    \phi\etuple{e_i}_i &= \etuple{\phi e_i}_i &
    \phi(\pi_i\<e) &= \pi_i\<\phi e\\
    \phi(\inj i e) &= \inj i \phi e
    &
    \phi(\ecase e (\inj i x \caseto f_i)_i)
    &= \ecase{\phi e} (\inj i x \caseto \phi f_i)_i
    \\
    \phi\bot &= \bot &
    \phi(e \vee f) &= \phi e \vee \phi f\\
    \phi(\eset{e_i}_i) &= \eset{\phi e_i}_i
    &
    %% replaced substitution by let-binding
    \phi(\efor x e f) &= \efor x {\phi e}
        %{\substd{\phi f}{\dvar\dx \substo \zero\<\dvar x}}
        {\eletbox{\dx}{\ebox{\zero\<\dvar x}} \phi f}
    \\
    \phi\ebox{e} &= \ebox{\etuple{\phi e, \color{Rhodamine} \delta e}}
    &
    \phi(\eletbox x e f)
    &= \elet{\ebox{\etuple{\dvar x,\dvar\dx}} = \phi e} \phi f
    \\
    \phi(\eeq e f) &= (\eeq {\phi e} {\phi f})
    &
    \phi(\eisempty e) &= \eisempty {\phi e}
    \\
    \phi(\efix e) &= \fastfix\<\phi e
    &
    %% split
    \phi(\esplit e) &= \ecase{\phi e}
    \\
    &&&\phantom{{}={}}\
    \left(\ebox{\etuple{\inj i \dvar x, \inj i \dvar \dx}}
    \caseto \inj i {\ebox {\etuple{\dvar x,\dvar\dx}}}\right)_{i}
    \\
    &&&\phantom{{}={}}\
    \left(\ebox{\etuple{\inj i \dvar x, \inj j \pwild}}
    \caseto \inj i {\ebox {\etuple{\dvar x, \dummy\<\dvar x}}} \right)_{i\ne j}
  \end{align*}

  \caption{Semi\naive{} speed-up translation, $\phi$}
  \label{fig:phi}
\end{figure}


%% ---- "Derivative" term translation, delta
\begin{figure}\centering
  \[ \delta\bot = \delta\eset{e_i}_i = \delta(\eeq e f) = \delta(\efix e) = \bot \]
  %
  \begin{align*}
    \delta x &= \dx &
    \delta \dvar x &= \dvar\dx\\
    \delta(\fnof{x} e) &= \fnof{\ebox x} \fnof\dx \delta e
    & \delta(e\<f) &= \delta e \<\ebox{\color{Rhodamine}\phi e} \<\delta f\\
    \delta\etuple{e_i}_i &= \etuple{\delta e_i}_i
    & \delta(\pi_i\<e) &= \pi_i\<\delta e\\
    \delta(\inj i e) &= \inj i {\delta e} &
    \delta(e \vee f) &= \delta e \vee \delta f\\
    \delta\ebox{e} &= \etuple{} &
    \delta(\eletbox x e f)
    &= \elet{\ebox{\etuple{\dvar x,\dvar\dx}} = \phi e} \delta f
    \\
    \delta(\eisempty e) &= \eisempty {\color{Rhodamine} \phi e}
    &
    \delta(\esplit e) &= \ecase{\phi e}
    (\ebox{\etuple{\inj i \pwild, \pwild}}
    \caseto \inj i {\etuple{}} )_i
  \end{align*}
  %
  \begin{align*}
    \delta(\ecase e (\inj i x \caseto f_i)_i)
    &= \ecase{\esplit{\ebox{\color{Rhodamine} \phi e}},\, \delta e}\\
    &\qquad ({\inj i {\eboxvar x},\, \inj i \dx} \caseto \delta f_i)_{i}\\
    &\qquad ({\inj i {\eboxvar x},\, \inj j \pwild}
    %\caseto \subst{\delta f_i}{\dx \substo \dummy\<\dvar x})_{i\ne j}
    \caseto \elet{\dx = \dummy\<\dvar x} \delta f_i)_{i\ne j}
    \\
    \delta(\efor x e f)
    &= (\efor x {\delta e}
    %\substd{\phi f}{\dvar\dx \substo \zero\<\dvar x}) \\
    \eletbox \dx {\zero\<\dvar x} \phi f) \\
    &\vee (\efor x {{\phi e} \vee \delta e}
    %\substd{\delta f}{\dvar\dx \substo \zero\<\dvar x})
    \eletbox{\dx}{\zero\<\dvar x} \delta f)
  \end{align*}

  \caption{Semi\naive{} derivative translation, $\delta$}
  \label{fig:delta}
\end{figure}



\subsection{Variables, \boldfn, and application}

At the core of a functional language are variables, \fn, and application. The
$\phi$ translation leaves these alone, simply distributing over subexpressions.
On variables, $\delta$ yields the corresponding change variables. On functions
and application, $\delta$ is more interesting:

\begin{align*}
  \DP(A \to B) &= \iso\Phi A \to \DP A \to \DP B
  &
  \delta(\fnof x e) &= \fnof{\eboxvar x} \fnof\dx \delta e
  &
  \delta(e\<f) &= \delta e \<\ebox{\color{Rhodamine}\phi f} \<\delta f
\end{align*}

The intuition behind $\delta(\fnof x e) = \fnof{\eboxvar x} \fnof\dx \delta e$
is that a function change takes two arguments, a base point $\dvar x$ and a
change $\dx$, and yields the change in the result of the function, $\delta e$.
However, we are given an argument of type $\iso \Phi A$, but consulting
\cref{thm:type-correct} for the type of $\delta e$, we need a discrete variable
$\hd x {\Phi A}$, so we use pattern-matching to unbox our argument.

The intuition behind $\delta(e\<f) = \delta e \<\ebox{\phi f} \<\delta f$ is
much the same: $\delta e$ needs two arguments, the original input $\phi f$ and
its change $\delta f$, to return the change in the function's output. Moreover,
it's discrete in its first argument, so we need to box it, $\ebox{\phi f}$.

One might wonder whether this type-checks, since $\phi e$ and $\delta e$ don't
use the same typing context.
%
We're even boxing $\phi f$, hiding all monotone variables;
%
consequently, it gets the context $\stripcxraw{\iso\Phi\G, \DP\G}$.
%
However, $\iso$ makes every variable discrete, and $\stripcxraw{-}$ leaves
discrete variables alone, so this includes \emph{at least} $\iso\Phi\G$. The
context $\phi f$ needs is $\Phi\G$. Since $\iso$ only makes a context stronger,
we're safe.
%
To emphasize this, we've marked all \emph{discrete} uses of $\phi e$ inside
$\delta e$ in \hilite{pink}.
%
The same argument applies (all the more easily) when $\phi e$ is used in a
monotone rather than a discrete position.

\todo{Discuss self-maintainable functions.}

\todo{Explain how uses of $\phi$ in $\delta e$ involve weakening, and how this
  justifies their use in discrete contexts (hilighted in
  {\color{Rhodamine}pink}), eg. in $\delta(e\<f) = \delta e \<\eboxraw{\phi f}
  \<\delta f$.}

%% Finally, it's worth noting that the context $\Phi\G$ that $\phi e$ requires is a
%% \emph{weakening} of the context $\iso\Phi\G, \DP\G$ of $\delta e$. Indeed, it's
%% even a weakening of $\stripcx{\iso\Phi\G,\DP\G}$.


\subsection{Fixed points}

\todo{Give typing rules and semantics for \fastfix. Semantics does not depend on
  \fastfix\ being given a derivative; it's well-defined regardless.}


\subsection{The discreteness comonad, \iso}

\todo{Discuss $\ebox{e}$ and $\eletbox x e f$. Explain why $\delta e$ can be
  used inside $\phi e$ in discrete position, highlighted in
  {\color{Rhodamine}pink}.}


\subsection{Case analysis, \prim{split}, and \name{dummy}}

\todo{explain \name{dummy}}


\subsection{Semilattices and comprehensions}

\todo{explain use of \name{dummy} to implement \zero.}

%% %% TODO: should we include this? I think the point is pretty well-made already.
%% \noindent
%% These correspond respectively to $\Phi$ and $\DP$ applied to $\iso A \to B \to
%% C$:
%%
%% \begin{align*}
%%   \Phi(\iso A \to B \to C)
%%   &= \iso (\Phi A \x \DP A) \to \Phi B \to \Phi C
%%   \\
%%   \DP(\iso A \to B \to C)
%%   &= \D(\iso (\Phi A \x \DP A) \to \Phi B \to \Phi C)\\
%%   &= \iso (\iso (\Phi A \x \DP A))
%%   \to \D(\iso (\Phi A \x \DP A))
%%   \to \D(\Phi B \to \Phi C)\\
%%   &\cong \iso (\Phi A \x \DP A)
%%   \to \tunit
%%   \to \iso\Phi B \to \DP B \to \DP C\\
%%   &\cong \iso \Phi A \to \iso \DP A \to \iso\Phi B \to \DP B \to \DP C
%% \end{align*}


\subsection{Leftovers}

\todo{Ideas not yet covered:
  \begin{itemize}
  \item things that can't change because they're discrete, like $\delta(\efix
    e)$, $\delta(\eeq e f)$, $\delta{\eset{e_i}_i}$ --- or maybe leave those for
    later?
\end{itemize}}

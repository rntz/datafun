\section{The Incremental Transformation}
\label{sec:incremental-transformation}

\begin{figure}\centering
  \begin{align*}
    \D\tseteq A &= \tseteq A
    &
    \Phi\tseteq A &= \tset{\Phi{\eqt A}}
    \qquad\text{\todo{(see \cref{thm:phi-eqtype})}}
    \\
    \D \iso A &= \tunit
    &
    \Phi \iso A &= \iso{(\Phi A \x \DP A)}
    \\
    \D\tunit &= \tunit
    &
    \Phi\tunit &= \tunit
    \\
    \D(A \x B) &= \D A \x \D B
    &
    \Phi(A \x B) &= \Phi A \x \Phi B
    \\
    \D(A + B) &= \D A + \D B
    &
    \Phi(A + B) &= \Phi A + \Phi B
    \\
    \D(A \to B) &= \iso A \to \D A \to \D B
    &
    \Phi(A \to B) &= \Phi A \to \Phi B
  \end{align*}

  \caption{$\D$ and $\Phi$ type transformations}
  \label{fig:DeltaPhi}
\end{figure}



\todo{Explain \cref{fig:DeltaPhi}. Then explain corresponding context transformations:}
\begin{align*}
  \D(\h x A) &= \h \dx {\D A} & \D(\hd x A) &= \emptycx\quad\text{(the empty context)}
  \\
  \Phi(\h x A) &= \h x {\Phi A} & \Phi(\hd x A) &= \hd x {\Phi A}, \hd \dx {\DP A}
  \\
  \iso{(\h x A)} &= \hd x A & \iso{(\hd x A)} &= \hd x A
\end{align*}

\todo{Explain how uses of $\phi$ in $\delta e$ involve weakening, and how this
  justifies their use in discrete contexts (hilighted in {\color{Rhodamine}pink}), eg. in $\delta(e\<f) = \delta e \<\eboxraw{\phi f} \<\delta f$.}

\todo{explain implementation of \zero{} via \dummy{}.}

%% ---- "Go faster" term translation, phi ----
\begin{figure}\centering
  \begin{align*}
    \phi x &= x & \phi \dvar x &= \dvar x\\
    \phi(\fnof x e) &= \fnof x \phi e & \phi(e\<f) &= \phi e\<\phi f\\
    \phi\etuple{e_i}_i &= \etuple{\phi e_i}_i &
    \phi(\pi_i\<e) &= \pi_i\<\phi e\\
    \phi(\inj i e) &= \inj i \phi e
    &
    \phi(\ecase e (\inj i x \caseto f_i)_i)
    &= \ecase{\phi e} (\inj i x \caseto \phi f_i)_i
    \\
    \phi\bot &= \bot &
    \phi(e \vee f) &= \phi e \vee \phi f\\
    \phi(\eset{e_i}_i) &= \eset{\phi e_i}_i
    &
    %% replaced substitution by let-binding
    \phi(\efor x e f) &= \efor x {\phi e}
        %{\substd{\phi f}{\dvar\dx \substo \zero\<\dvar x}}
        {\eletbox{\dx}{\ebox{\zero\<\dvar x}} \phi f}
    \\
    \phi\ebox{e} &= \ebox{\etuple{\phi e, \color{Rhodamine} \delta e}}
    &
    \phi(\eletbox x e f)
    &= \elet{\ebox{\etuple{\dvar x,\dvar\dx}} = \phi e} \phi f
    \\
    \phi(\eeq e f) &= (\eeq {\phi e} {\phi f})
    &
    \phi(\eisempty e) &= \eisempty {\phi e}
    \\
    \phi(\efix e) &= \fastfix\<\phi e
    &
    %% split
    \phi(\esplit e) &= \ecase{\phi e}
    \\
    &&&\phantom{{}={}}\
    \left(\ebox{\etuple{\inj i \dvar x, \inj i \dvar \dx}}
    \caseto \inj i {\ebox {\etuple{\dvar x,\dvar\dx}}}\right)_{i}
    \\
    &&&\phantom{{}={}}\
    \left(\ebox{\etuple{\inj i \dvar x, \inj j \pwild}}
    \caseto \inj i {\ebox {\etuple{\dvar x, \dummy\<\dvar x}}} \right)_{i\ne j}
  \end{align*}

  \caption{Semi\naive{} speed-up translation, $\phi$}
  \label{fig:phi}
\end{figure}


%% ---- "Derivative" term translation, delta
\begin{figure}\centering
  \[ \delta\bot = \delta\eset{e_i}_i = \delta(\eeq e f) = \delta(\efix e) = \bot \]
  %
  \begin{align*}
    \delta x &= \dx &
    \delta \dvar x &= \dvar\dx\\
    \delta(\fnof{x} e) &= \fnof{\ebox x} \fnof\dx \delta e
    & \delta(e\<f) &= \delta e \<\ebox{\color{Rhodamine}\phi e} \<\delta f\\
    \delta\etuple{e_i}_i &= \etuple{\delta e_i}_i
    & \delta(\pi_i\<e) &= \pi_i\<\delta e\\
    \delta(\inj i e) &= \inj i {\delta e} &
    \delta(e \vee f) &= \delta e \vee \delta f\\
    \delta\ebox{e} &= \etuple{} &
    \delta(\eletbox x e f)
    &= \elet{\ebox{\etuple{\dvar x,\dvar\dx}} = \phi e} \delta f
    \\
    \delta(\eisempty e) &= \eisempty {\color{Rhodamine} \phi e}
    &
    \delta(\esplit e) &= \ecase{\phi e}
    (\ebox{\etuple{\inj i \pwild, \pwild}}
    \caseto \inj i {\etuple{}} )_i
  \end{align*}
  %
  \begin{align*}
    \delta(\ecase e (\inj i x \caseto f_i)_i)
    &= \ecase{\esplit{\ebox{\color{Rhodamine} \phi e}},\, \delta e}\\
    &\qquad ({\inj i {\eboxvar x},\, \inj i \dx} \caseto \delta f_i)_{i}\\
    &\qquad ({\inj i {\eboxvar x},\, \inj j \pwild}
    %\caseto \subst{\delta f_i}{\dx \substo \dummy\<\dvar x})_{i\ne j}
    \caseto \elet{\dx = \dummy\<\dvar x} \delta f_i)_{i\ne j}
    \\
    \delta(\efor x e f)
    &= (\efor x {\delta e}
    %\substd{\phi f}{\dvar\dx \substo \zero\<\dvar x}) \\
    \eletbox \dx {\zero\<\dvar x} \phi f) \\
    &\vee (\efor x {{\phi e} \vee \delta e}
    %\substd{\delta f}{\dvar\dx \substo \zero\<\dvar x})
    \eletbox{\dx}{\zero\<\dvar x} \delta f)
  \end{align*}

  \caption{Semi\naive{} derivative translation, $\delta$}
  \label{fig:delta}
\end{figure}


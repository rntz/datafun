\section{Datalog and Datafun, informally}
\label{sec:datalog-and-datafun}

\subsection{Datalog}
\label{sec:datalog}

Datalog's syntax is a subset of Prolog's. Programs are collections of predicate
declarations:

\newcommand\datum[1]{\textsf{#1}}

\begin{code}
  \name{parent}(\datum{aerys}, \datum{rhaegar})\\
  \name{parent}(\datum{rhaegar}, \datum{jon})\\
  \name{parent}(\datum{lyanna}, \datum{jon})
  \\[6pt]
  \name{ancestor}(X,Z) \gets \name{parent}(X,Z)\\
  \name{ancestor}(X,Z) \gets \name{parent}(X,Y) \wedge \name{ancestor}(Y,Z)
\end{code}

\noindent
This defines two binary relations, \name{parent} and \name{ancestor}. Lowercase
sans-serif words like \datum{aerys} and \datum{rhaegar} are symbols \`a la
Lisp, and uppercase characters like $X,Y,Z$ are variables.
%
The \name{parent} relation is defined as a set of ground facts: we assert that
\datum{aerys} is \datum{rhaegar}'s parent, that \datum{rhaegar} is \datum{jon}'s
parent, and so on. The \name{ancestor} relation is defined by a pair of rules:
first, that $X$ is $Z$'s ancestor if $X$ is $Z$'s parent; second, that $X$ is
$Z$'s ancestor if $X$ has a child $Y$ who is an ancestor of $Z$.

Semantically, a predicate denotes the set of tuples that satisfy it. Compared to
Prolog, one of the key restrictions Datalog imposes is that these sets are
always \emph{finite}. This helps keep proof search decidable, allowing for a
variety of implementation strategies. In practice, most Datalog engines use
bottom-up evaluation instead of Prolog's top-down backtracking search.

Recursive definitions like \name{ancestor} give rise to the set of facts
deducible from the rules defining them. More formally, we can view these rules
as defining a relation transformer and producing its least fixed point. For this
to make sense, these rules must be \emph{stratified}: a recursive definition
cannot refer to itself beneath a negation. For example, the liar paradox is
prohibited:

\begin{code}
  \quad\name{liar}() \gets \neg \name{liar}()
  \qquad\color{Red}\text{\scshape\ding{55}\: not valid datalog}
\end{code}

\noindent
Stratification ensures the transformer the rules define is monotone,
guaranteeing a unique least fixed point.


\subsection{Datafun}

\begin{figure}
  \[\setlength\arraycolsep{.4em}
  \begin{array}{r@{\hskip 1em}ccl}
    \text{types} & A,B &\bnfeq& \iso A \bnfor \tset{\eqt A}
    \bnfor \tunit \bnfor A \x B \bnfor A + B \bnfor A \to B
    \\[.5ex]
    \text{eqtypes} & \eqt A, \eqt B &\bnfeq&
    \tset{\eqt A} \bnfor
    \tunit \bnfor \eqt A \x \eqt B \bnfor \eqt A + \eqt B
    \\[.5ex]
    \text{semilattices} & L,M &\bnfeq& \tset{\eqt A} \bnfor \tunit \bnfor L \x M
    \\[.5ex]
    \text{fixtypes} & \fixt L, \fixt M &\bnfeq&
    \tset{\color{red}\fint A} \bnfor \tunit \bnfor \fixt L \x \fixt M
    \\[.5em]
    \text{terms} & e,f,g &\bnfeq& x \bnfor \dvar x \bnfor \fnof{x} e
    \bnfor e\<f \bnfor \etuple{} \bnfor \etuple{e,f} \bnfor \pi_i\<e
    \\&&&
    \inj i e \bnfor \ecase{e} (\inj i x_i \caseto f_i)_{i\in\{1,2\}}
    \\&&&
    \ebox e \bnfor \eletbox x e f
    \\&&&
    \bot \bnfor e \vee f \bnfor \eset{e_i}_i \bnfor \efor x e f\\
    &&& \eeq e f \bnfor \eisempty e \bnfor \esplit e \bnfor \efix e
    \\[1ex]
    %% \multicolumn{2}{r}{\textsf{\scshape Surface syntax}}
    %% \\[1ex]
    \text{surface types} & A,B &\bnfeq& ... \bnfor \tbool
    \\[.5ex]
    \text{surface terms} & e,f,g &\bnfeq&
    ... \bnfor \elet{x = e} f \bnfor \efixis x e\\
    &&& \efalse \bnfor \etrue \bnfor \eif e f g \bnfor \ewhen e f
  \end{array}
  \]

  \todo{What about pattern matching?}

  \caption{Datafun syntax}
  \label{fig:syntax}
\end{figure}


The idea behind Datafun is to capture the essence of Datalog in a typed,
higher-order, functional setting.
%
Since the key restriction that makes Datalog tractable --- stratification ---
requires tracking \emph{monotonicity}, we locate Datafun's semantics in the
category \Poset\ of partial orders and monotone maps.
%
Since \Poset\ is bicartesian closed, it can interpret the simply typed
\fn-calculus, giving us a notation for writing monotone and higher-order
functions.
%
This lets us \emph{abstract} over Datalog rules, something not
possible in Datalog itself!
%
In the remainder of this section we reconstruct Datafun hewing closely
to this semantic intuition.

Datafun begins as the simply-typed \fn-calculus with functions ($\fnof x e$ and
$e\<f$), sums ($\inj i e$ and $\ecase{e}{\dots}$), and products ($\etuple{e,f}$
and $\pi_i\<e$).
%
To represent relations, we add a type of finite sets\footnote{To implement set
  types, their elements must support decidable equality. In our core calculus,
  we use a subgrammar of ``eqtypes'', and in our implementation (which compiles
  to Haskell) we use typeclass constraints to pick out such types.} $\tset{\eqt
  A}$, introduced with set literals $\eset{e_0, \ldots e_n}$, and eliminated
using Moggi's monadic bind syntax, $\efor{x}{e_1}{e_2}$, signifying the union
over all $\dvar x \in e_1$ of $e_2$.
%
Since we are working in \Poset, each type comes with a partial order on it; sets
are ordered by inclusion, $x \le y : \tseteq{A} \iff x \subseteq y$.

As long as all primitives are monotone, every definable function is also
monotone. This is necessary for defining fixed points, but may seem too
restrictive. There are many useful non-monotone operations, such as equality
tests $\eeq e f$. For example, $\esetraw{} = \esetraw{}$ is true, but if the
first argument increases to $\esetraw{1}$ it becomes false, a \emph{decrease}
(as we'll see later, in Datafun, $\efalse < \etrue$).

How can we express non-monotone operations if all functions are monotone?
%
We square this circle by introducing the \emph{discreteness} type constructor,
$\iso A$.
%
The elements of $\iso A$ are the same as those of $A$, but the partial order on
$\iso A$ is discrete, $x \le y : \iso A \iff x = y$.
%
Monotonicity of a function $\iso A \to B$ is vacuous: $x = y$ implies $f(x) \le
f(y)$ by reflexivity!
%
In this way we represent ordinary, possibly non-monotone, functions $A \to B$ as
monotone functions $\iso A \to B$.

Semantically, $\iso$ is a monoidal comonad or necessity modality, and so we base
our syntax on \citet{jrml}'s syntax for the necessity fragment of constructive
S4 modal logic.
%
This involves distinguishing two kinds of variable: discrete variables $\dvar x$
are italic and {\isocolor\isocolorname}, while monotone variables $x$ are
upright and black.
%
Discrete variables may be used wherever they're in scope, but crucially,
monotone variables are hidden within non-monotone expressions.
%
For example, in an equality test $\eeq e f$, the terms $e$ and $f$ cannot refer
to monotone variables bound outside the equality expression.
%
We highlight such expressions with a
\adjustbox{bgcolor=isobg}{\isobgname\ background}.
%
Putting this all together, we construct the type $\iso A$ with the non-monotone
introduction form $\ebox{e}$ and eliminate it by pattern-matching, $\eletbox x e
f$, giving access to a discrete variable $\dvar{x}$.

%% TODO: should we note that we really definitely mean □(L → L) and
%% not (□L → L)? Reader may be confused.
Finally, Datafun includes fixed points, $\efix{e}$. The $\prim{fix}$ combinator
takes a function $\iso (\fixt L \to \fixt L)$ and returns its least fixed point.
Besides monotonicity of the function, we impose two restrictions on the fixed
point operator to ensure well-definedness and termination. First, we require
that recursion occur at \emph{semilattice types with no infinite ascending
  chains}, $\fixt L$. A join-semilattice is a partial order with a least element
$\bot$ and a least upper bound operation $\vee$ (``join''). Finite sets (with
the empty set as least element, and union as join) are an example, as are tuples
of semilattices. As long as the semilattice has no infinite ascending chains,
recursion from the bottom element is guaranteed to find the least fixed point.

Second, we require that the recursive function be boxed, $\iso(\fixt L \to \fixt
L)$. Since boxed expressions can only refer to discrete values, and fixed point
functions themselves must be monotone, this has the effect of preventing
semantically nested fixed points. We discuss this in more detail in
\cref{sec:nested-fixed-points}. Note that this does not prevent mutual
recursion, which can be expressed by taking a fixed point at product type, nor
stratified fixed points \`a la Datalog.


\section{Datafun by example}

\begin{figure}
  \begin{displaymath}
    \begin{array}{cc}
    \begin{array}{ccc}
      \kw{bool} & \triangleq & \tset 1 \\
      \efalse   & \triangleq & \eset{} \\
      \etrue    & \triangleq & \eset{\etuple{}} \\
      \ewhen{b}{e} & \triangleq & \efor{ \etuple }{b}{e} \\
    \end{array}
    \qquad &
    \begin{array}{ccc}
      \setfor{e}{\cdot} & \triangleq & \eset{e} \\
      \setfor{e}{p \in e', \ldots} & \triangleq & \efor{p}{e'} {\setfor{e}{\ldots}} \\
      \setfor{e}{b, \ldots} & \triangleq & \ewhen{b} {\setfor{e}{\ldots}} \\
    \end{array}
    \end{array}
\end{displaymath}
  \caption{Syntactic Sugar}
  \label{fig:sugar}
\end{figure}

For the sake of brevity and clarity, the examples that follow make use of some
syntax sugar:

\begin{enumerate}

\item We mentioned earlier that Datafun's boolean type $\tbool$ is ordered
  $\efalse < \etrue$. This is because we encode booleans as sets of empty
  tuples, $\tset 1$, with $\efalse$ being the empty set $\esetraw{}$ and
  $\etrue$ being the singleton $\esetraw{\etuple{}}$. At semilattice type we
  also permit a ``one-sided'' conditional test, $\ewhen{b}{e}$, which yields $e$
  if $b$ is \etrue\ and $\bot$ otherwise. Encoding booleans as sets has the
  advantage that $\ewhen{b}{e}$ is monotone in the condition $b$.

\item We make use of set comprehensions, which can be desugared into the monadic
  operators \kw{for} and \kw{when} in the usual
  way~\cite{wadler-monad-comprehensions}.

\item It is convenient to treat $\prim{fix}$ as a binding form, $\efixis x e$,
  rather than supplying a function, $\efix\ebox{\fnof x e}$.

\item Finally, we make free use of curried functions and pattern matching.
  Desugaring these is relatively standard, and so we will say little about it,
  with one exception: the box-elimination form $\eletbox x e e'$ is a pattern
  matching form, and so we allow it to occur inside of patterns. The effect of a
  box pattern $\pbox{\isocolor p}$ is to ensure that all of the variables bound
  in the pattern $\isocolor p$ are treated as discrete variables.

\end{enumerate}

\noindent
We summarize (except for pattern matching) the desugaring rules we use in
\cref{fig:sugar}.


\subsection{Set operations}

%\newcommand\isin{\mathbin{\in?}}
\newcommand\mem{\name{member}}
\newcommand\isa{~:~}
\newcommand\relcomp\bullet
\newcommand\bone{\dvar b_{\isocolor 1}}
\newcommand\btwo{\dvar b_{\isocolor 2}}
\newcommand\yone{\dvar y_{\isocolor 1}}
\newcommand\ytwo{\dvar y_{\isocolor 2}}

Even before higher-order functions, one of the main benefits of Datafun over
Datalog is that it permits manipulating relations as first class values.
%
In this subsection we will show how a variety of standard operations on sets can
be represented in Datafun.
%
The first operation we consider is testing membership:

\begin{code}
  \mem \isa \eqt A \to \tseteq A \to \tbool\\
  \mem \<\pboxvar x \<s = \efor{y}{s} \eeq {\dvar x} {\dvar y}
\end{code}

\noindent
This checks if $\dvar x$ is equal to any element $\dvar y \in s$. The argument
$\dvar x$ is discrete because increasing $\dvar x$ might send it from being
\emph{in} the set to being \emph{outside} the set (e.g. $1 \in \esetraw{1}$ but
$2 \notin \esetraw{1}$). Notice that here we're taking advantage of encoding
booleans as sets of empty tuples --- unioning these sets implements logical
\emph{or}.

Using $\mem$ we can define set intersection by taking the union of every
singleton set $\esetraw{\dvar x}$ where $\dvar x$ is an element of both
$s$ and $t$:

\begin{code}
  \pwild \cap \pwild \isa \tseteq A \to \tseteq A \to \tseteq A\\
  s \cap t = \efor x {s} \ewhen{\mem \<\eboxvar x \<t} \eset{\dvar x}
\end{code}

\noindent
Using comprehensions, this could alternately be written as:

\begin{code}
  s \cap t = \esetfor{\dvar x}{\dvar x \in s, \mem \<\eboxvar x \<s}
\end{code}

\noindent From now on, we'll use comprehensions whenever possible. For example,
we can also define the composition of two relations in Datafun:

\begin{code}
  \pwild \relcomp \pwild \isa \tset{\eqt A \x \eqt B} \to \tset{\eqt B \x \eqt
    C} \to \tset{\eqt A \x \eqt C}\\
  s \relcomp t = \esetfor{\etuple{\dvar a, \dvar c}}{
    \ptuple{\dvar a, \bone} \in s,
    \ptuple{\btwo, \dvar c} \in t,
    \eeq \bone \btwo}
\end{code}

\noindent
This is basically a transcription of the mathematical definition, where we build
those pairs which agree on their $B$-typed components.

We can also define set difference, although we must first detour into boolean
negation:

\begin{code}
  {\neg} \isa \iso\tbool \to \tbool\\
  \neg \pboxvar t = \ecase{\eisempty{\dvar t}}
  \inj 1 \ptuple{} \caseto \etrue;\ \inj 2 \ptuple{} \caseto \efalse
  \\[8pt]
  \pwild\setminus\pwild \isa \tseteq A \to \iso \tseteq A \to \tseteq A\\
  s \setminus \pboxvar t =
  \setfor{\dvar x}{\neg \ebox{\mem \<\eboxvar x \<\dvar t}}
\end{code}

\noindent
To implement boolean negation, we need the primitive operator $\eisempty e$,
which produces a tag indicating whether its argument $e$ (a set of empty tuples,
i.e. a boolean) is the empty set.
%
This in turn lets us define set difference, the analogue in Datafun of negation
in Datalog.
%
Note that in both boolean negation and set difference the ``negated'' argument
$\dvar t$ is boxed, because the operation is not monotone in $\dvar t$.
%
This enforces stratification.

%\subsubsection{Transitive Closure}
\label{sec:generic-transitive-closure}
Finally, generalizing the \name{ancestor} relation from the Datalog program in
\cref{sec:datalog}, we can define the transitive closure of a relation:

\nopagebreak[2]
\begin{code}
  \name{trans} \isa \iso \tset{\eqt A \x \eqt A} \to \tset{\eqt A \x \eqt A}\\
  \name{trans} \<\pboxvar{edge} = \efixis{s}{\dvar{edge} \vee (\dvar{edge} \relcomp s)}
\end{code}

\noindent
This definition uses a least fixed point, just like the mathematical definition
-- a transitive closure is the least relation $R$ containing the original
relation $\dvar{edge}$ and the composition of $\dvar{edge}$ with $R$.
%
However, one feature of this definition peculiar to Datafun is that the argument
type is $\iso \tset{\eqt A \times \eqt A}$ -- the transitive closure takes a
\emph{discrete} relation.
%
This is because we must use the relation within the fixed point, and so its
parameter needs to be discrete to occur within.
%
This restriction is artificial --- transitive closure is semantically a monotone
operation --- but we'll see why it's useful in \cref{sec:transformations}.


\subsection{Regular expression combinators}
\label{sec:regex-combinators}

\newcommand\tre{\typename{re}}
\newcommand\tchar{\typename{char}}

Datafun permits tightly integrating the higher-order functional and
bottom-up logic programming styles. In this section, we illustrate the
benefits of doing so by showing how to implement a regular expression
matching library in combinator parsing style.  Like combinator parsers
in functional languages, the code is very concise.  However, support
for the relational style ensures we can write \naive\ code
\emph{without} the exponential backtracking cliffs typical of parser
combinators in functional languages.

For these examples we'll assume the existence of eqtypes \tstring, \tchar, and
\tint, an addition operator $+$, and functions \name{length} and \name{chars}
satisfying:

\begin{code}
  \name{length} \isa \iso\tstring \to \tint\\
  \name{length} \<\pboxvar s = \text{the length of }\dvar s
  \\[8pt]
  \name{chars} \isa \iso\tstring \to \tset{\tint \x \tchar}\\
  \name{chars} \<\pboxvar s = \setfor{(i,c)}{\text{the $i$\textsuperscript{th}
      character of $\dvar s$ is $c$}}
\end{code}

\noindent
Note that by always boxing string arguments, we avoid committing ourselves to
any particular partial ordering on \tstring.

These assumed, we define the type of regular expression matchers:

\begin{code}
\kw{type}\ \tre = \iso \tstring \to \tset{\tint \times \tint}
\end{code}

\noindent
A regular expression takes a discrete string $\pboxvar s$ and returns the set of
all pairs $(i,j)$ such that the substring $\dvar s_i,\, \ldots,\, \dvar s_{j-1}$
matches the regular expression. For example, to find all matches for a single
character $c$, we return the range $(i,i+1)$ whenever $(i,c) \in
\name{chars}\<\eboxvar s$:

\begin{code}
  \name{sym} \isa \iso\tchar \to \tre\\
  \name{sym} \<\pboxvar c \<\pboxvar s =
  \esetfor{(\dvar i, \dvar i + 1)}{
    \ptuple{\dvar i, \dvar c'} \in \name{chars} \<\eboxvar s,
    \eeq{\dvar c}{\dvar{c'}}}
\end{code}

\noindent
To find all matches for the empty regex, i.e.\ all empty substrings, including
the one ``beyond the last character'':

\begin{code}
  \name{nil} \isa \tre\\
  \name{nil} \<\pboxvar s =
  \esetfor{\dvar i}{\ptuple{\dvar i, \pwild} \in \name{chars}\<\eboxvar s}
  \vee \eset{\name{length}\<\eboxvar s}
\end{code}

\noindent
Appending regexes $r_1, r_2$ amounts to relation composition, since we wish to
find all substrings consisting of adjacent substrings $s_i \ldots s_{j-1}$ and
$s_j \ldots s_{k-1}$ matching $r_1$ and $r_2$ respectively:

\nopagebreak[2]
\begin{code}
  \name{seq} \isa \tre \to \tre \to \tre\\
  \name{seq} \<r_1 \<r_2 \<s = r_1\<s \relcomp r_2\<s
\end{code}

\noindent
Similarly, regex alternation \texttt{r\textsubscript{1}|r\textsubscript{2}} is
accomplished by unioning all matches of each:

\nopagebreak[2]
\begin{code}
  \name{alt} \isa \tre \to \tre \to \tre\\
  \name{alt} \<r_1 \<r_2 \<s = r_1\<s \vee r_2\<s
\end{code}

\noindent
The most interesting regular expression combinator is Kleene star. Thinking
relationally, if we consider the set of pairs $(i,j)$ matching some regex
\texttt{r}, then \texttt{r*} matches its \emph{reflexive, transitive closure}.
This can be accomplished by combining \emph{nil} and \emph{trans}.

\begin{code}
  \name{star} \isa \iso\tre \to \tre\\
  \name{star} \<\pboxvar r \<\pboxvar s =
  \name{nil}\<\eboxvar s \vee
  \name{trans} \<\ebox{\dvar r \<\eboxvar s}
\end{code}

\noindent
Note that the argument $\dvar r$ is must be discrete because \name{trans} uses
it to compute a fixed point.\footnote{As a technical note, sets of pairs of
  integers do not form a \todo{finite-height} lattice, so by typing this program
  is not an acceptable fixed point expression. However, since the positions in a
  string form a finite set, semantically it is fine. The original Datafun paper
  shows how one can define bounded fixed points to handle cases like this, so we
  will not be scrupulous.}


\subsection{Regular expression combinators, take 2}

The combinators in the previous section found \emph{all} matches
within a given substring, but often we are not interested in all
matches: we only want to know if a string can match starting at a
particular location. We can easily refactor the combinators above to
work in this style, which illustrates the benefits of tightly
integrating functional and relational styles of programming -- we can
use functions to manage strict input/output divisions, and relations
to manage nondeterminism and search.

\begin{code}
  \kw{type}\ \tre = \iso (\tstring \x \tint) \to \tset{\tint}
\end{code}

\noindent
Our new type of combinators takes a string and a starting position, and returns
a set of ending positions. For example, $\name{sym} \<\eboxvar c$ checks if
$\dvar c$ occurs at the start position $\dvar i$, yielding $\esetraw{\dvar i+1}$
if it does and the empty set otherwise, while $\name{nil}$ simply returns the
start position $\dvar i$.

\begin{code}
  \name{sym} \isa \iso\tchar \to \tre\\
  \name{sym} \<\pboxvar c \<\pboxtuple{\dvar s, \dvar i}
  = \esetfor{\dvar i+1}{
    \ptuple{\dvar j, \dvar d} \in \name{chars} \<\eboxvar s,
    \eeq {\dvar i} {\dvar j},
    \eeq {\dvar c} {\dvar d}}
  \\[8pt]
  \name{nil} \isa \tre \to \tre\\
  \name{nil} \<\pboxtuple{\dvar s, \dvar i} = \eset{\dvar i}
\end{code}

\noindent
Appending regexes $\name{seq}\<r_1\<r_2$ simply applies $r_2$ starting from
every ending position that $r_1$ can find:

\begin{code}
  \name{seq} \isa \tre \to \tre \to \tre\\
  \name{seq} \<r_1 \<r_2 \<\pboxtuple{\dvar s, \dvar i} =
  \efor{j}{r_1\<\eboxtuple{\dvar s, \dvar i}} r_2 \<\eboxtuple{\dvar s, \dvar j}
\end{code}

\noindent
Regex alternation \name{alt} is effectively unchanged:

\begin{code}
  \name{alt} \isa \tre \to \tre \to \tre\\
  \name{alt} \<r_1 \<r_2 \<x = r_1\<x \vee r_2\<x
\end{code}

\noindent
Finally, Kleene star is implemented by recursively appending $\dvar r$ to a
set $x$ of matches found so far:

\begin{code}
  \name{star} \isa \iso\tre \to \tre\\
  \name{star} \<\pboxvar r \<\pboxtuple{\dvar s, \dvar i}
  = \efixis{x}{\bigl(\eset{\dvar i} \vee
    \efor j {x} \dvar r\<\eboxtuple{\dvar s, \dvar j}
    \bigr)}
\end{code}

\noindent
It's worth noting that this definition is effectively \emph{left-recursive} ---
it takes the endpoints from the fixed point $x$, and then continues matching
using the argument $\dvar r$. This should make clear that this is not just plain
old functional programming --- we are genuinely relying upon the fixed point
semantics of Datafun.

%% TODO: aspect ratio
\documentclass[fleqn,aspectratio=169,dvipsnames]{beamer}

\usepackage{amssymb,amsmath,amsthm,latexsym}
\usepackage{url,hyperref}
\usepackage{mathpartir}         % \mathpar, \infer
\usepackage{booktabs}           % \midrule
\usepackage{colonequals}
\usepackage{tikz,tikz-cd}       % Hasse & commutative diagrams.

%% font fiddling
\usefonttheme{professionalfonts}
\renewcommand{\familydefault}{\rmdefault}
%\usepackage[scaled=0.9543]{XCharter}
\usepackage[semibold,scaled=0.9663]{sourceserifpro}
\usepackage[semibold,scaled=0.9663]{sourcesanspro}
%\usepackage[osf,scaled=1.0155]{AlegreyaSans}
\usepackage[scaled=1.00438,varqu,var0]{inconsolata}
\usepackage{eulervm}
\usepackage[bb=boondox]{mathalfa} % or bb=ams
\usepackage[T1]{fontenc}

%% TODO: more tracking for charter.
\usepackage[tracking,letterspace=20]{microtype}
\frenchspacing

\setlength\mathindent{12pt}
\def\arraystretch{1.1}


%% ===== COMMANDS =====
\newcommand\strong\textbf

\newcommand\N{\mathbb{N}}
\newcommand\x\times
\newcommand\G\Gamma
\newcommand\D\Delta
\newcommand\fn{\ensuremath{\lambda}}
\newcommand\isa{\mathrel{\ratio}}

\newcommand{\setfor}[2]{\{{#1} \mathrel{|} {#2}\}}

\newcommand\kw[1]{\ensuremath{\textbf{#1}}}
\newcommand\n\mathit
\newcommand\tpname\mathrm
\newcommand\tset{\tpname{Set}~}
\newcommand\zero{\ensuremath{\mathbf{0}}}

\newcommand\naive{na\"ive}
\newcommand\Naive{Na\"ive}

\let\oldcup\cup\renewcommand\cup{\mathrel{\oldcup}}

\newcommand\hilite{\color{Rhodamine}}
\newcommand\hi[1]{{\hilite#1}}
%\newcommand\hilitetime{\color{Orange}}\newcommand\hiti[1]{{\hilitetime#1}}


\title{Semi\naive\ Evaluation for a Higher-Order Language}
\author{Michael Arntzenius}
\institute{University of Birmingham}
\date{POPL 2020}
\begin{document}
  \LARGE

  \begin{frame}
    \[\begin{array}{l}
      \n{edge}, \n{path} \isa \tset (\N \x \N)
      \\
      \alt<1>{
        \hi{\n{path}} = \n{edge} \cup
        \setfor{(x,z)}{(x,y) \in \n{edge}, (y,z) \in \hi{\n{path}}}
      }{
        \n{step} \; {\only<2>{\hilite}s} = \n{edge} \cup
        \setfor{(x,z)}{(x,y) \in \n{edge}, (y,z) \in {\only<2>{\hilite}s}}
      }
      \\\pause
      \alt<3->{
        \n{path}_{\hi 0} = \emptyset
      }{\n{path} = \n{step} \; \n{path}}
      \\\pause
      \n{path}_{\hi{i+1}} = \n{step} \, \n{path}_{\hi i}
    \end{array}\]

    wait until $\n{path}_i = \n{path}_{i+1}$
  \end{frame}


  \begin{frame}
    \begin{mathpar}
      \begin{array}{l}
        \n{step} \; s = \n{edge} \cup
        \setfor{(x,z)}{(x,y) \in \n{edge}, (y,z) \in s}
        \\
        \n{path}_i = \n{step}^i \;\emptyset =
        \text{paths of length $\le i$}
      \end{array}
      \\\pause
      \begin{tikzpicture}[thick,baseline=(current bounding box.center), scale=1.5]
        \node (a) at (0, 0) {$\n{path}_0$};
        \node (ab) at (1, 0) {$\subseteq$};
        \node (b) at (2, 0) {$\n{path}_1$};
        \node (bc) at (3, 0) {$\subseteq$};
        \node (c) at (4, 0) {$\n{path}_2$};
        \node (cd) at (5, 0) {$\subseteq$};
        \node (d) at (6, 0) {$\cdots$};
        \path[-Latex, bend right=50,below]
          (a) edge node {$\n{dpath}_0$} (b)
          (b) edge node {$\n{dpath}_1$} (c)
          (c) edge node {$\n{dpath}_2$} (d);
      \end{tikzpicture}
    \end{mathpar}
  \end{frame}


  \begin{frame}
    \strong{Bottom-up fixed points} of \strong{monotone maps}:
    \vspace{1ex}

    \begin{itemize}\setlength\itemsep{1ex}
    \item Graph algorithms (eg.~shortest paths)
    \item Static analysis (abstract interpretation)
    \item Parsing (regexes, CFGs)
    \item \emph{Datalog!} {\normalsize\sffamily(if it's a map on finite sets)}
    \end{itemize}
  \end{frame}

  \begin{frame}
    \begin{center}
      \scshape\huge semi\naive\ evaluation!
    \end{center}

    \pause
    Datalog is a \emph{first-order} logic language.
    \vspace{1em}

    \pause
    \strong{Datafun}\textsuperscript{\sffamily\scshape[icfp 2016]}
    is a simply-typed \fn-calculus with:
    \begin{itemize}
    \item a finite set datatype \& set comprehensions
    \item bottom-up monotone fixed points
    \item a type system that tracks monotonicity
    %% \item where types are posets, \\
    %%   and monotonicity is tracked in the type system
    \end{itemize}
    \vspace{1em}
  \end{frame}
\end{document}



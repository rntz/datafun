%% TODO: aspect ratio
\documentclass[aspectratio=169,dvipsnames]{beamer}

\usepackage{amssymb,amsmath,amsthm,latexsym}
\usepackage{url,hyperref}
\usepackage{nccmath}            % fleqn environment
\usepackage{mathpartir}         % \mathpar, \infer
\usepackage{booktabs}           % \midrule
\usepackage{colonequals}
\usepackage{tikz,tikz-cd}       % Hasse & commutative diagrams.
\usepackage{accents}            % \underaccent
\usepackage{pdfpages}           % \includepdf
\usepackage{mathtools}          % \dblcolon
\usepackage{pgfplots}\pgfplotsset{compat=1.5} % the perf graph

%% TODO: more tracking for charter.
\usepackage{iftex}
\ifPDFTeX\PassOptionsToPackage{spacing=true,tracking=true,letterspace=40,}{microtype}\else\fi
\usepackage{microtype}
\frenchspacing


%% Format fiddling.
\linespread{1.0608}
\def\arraystretch{1.141}

\let\olditemize\itemize
\renewcommand\itemize{\olditemize\setlength\itemsep{1ex}}

\setbeamertemplate{navigation symbols}{}
\setbeamercolor{background canvas}{bg=} % necessary for \includepdf

\setbeamertemplate{footline}[frame number]
\setbeamerfont{footline}{size=\scriptsize,family=\sffamily,shape=\scshape,}
%\setbeamertemplate{footline}{\hfil\insertpagenumber\vspace{2ex}}
%% \setbeamerfont{headline}{size=\scriptsize}
%% \setbeamertemplate{headline}{\vspace{2ex}\hfill\insertpagenumber\hspace*{2ex}}

%\title{Semi\naive\ Evaluation for a Higher-Order Language}
%\author{Michael Arntzenius \& Neel Krishnaswami}
%\institute{University of Birmingham}
%\date{POPL 2020}
\date{}


%% font fiddling
\usepackage[LY1,OT1,T1]{fontenc}
\ifPDFTeX\else\usepackage[no-math]{fontspec}\fi
\usefonttheme{professionalfonts}
\renewcommand{\familydefault}{\rmdefault}

%\usepackage[scaled=0.9543]{XCharter}
%\usepackage[llscale=1.0699300699300702,scale=1.0699300699300702,p,mono=false]{libertine}
\usepackage[semibold,scaled=0.9663]{sourceserifpro}
\usepackage[osf,semibold,scaled=0.9663]{sourcesanspro}
%\usepackage[osf,scaled=1.0155]{Alegreya}
%\usepackage[sfdefault,osf,scaled=1.0155]{AlegreyaSans}
\usepackage[scaled=1.00438,varqu,var0]{inconsolata}

\usepackage{fontchoice}

\usepackage[euler-digits]{eulervm}
\usepackage[bb=boondox]{mathalfa} % or bb=ams

\usepackage[euler-digits]{eulervm}


%% ===== COMMANDS =====
\providecommand\strong[1]{{\bfseries#1}}
\newcommand\textsfit[1]{\textsf{\itshape#1}}

\newcommand\N{\mathbb{N}}
\newcommand\x\times
\providecommand\G{}\renewcommand\G\Gamma
\newcommand\D\Delta
\newcommand\fn{\ensuremath{\lambda}}
\newcommand\isa{\mathrel{\,\text{:}\,}}
\newcommand\fnof[1]{\fn{#1}.\;}
\newcommand\fa[1]{(\forall {#1})\ }
\newcommand\iso{{\texorpdfstring{\ensuremath{\square}}{box}}}
\newcommand\ebox[1]{[{#1}]}
\newcommand\pbox[1]{[{#1}]}

\newcommand{\setfor}[2]{\{{#1} \mathrel{|} {#2}\}}

\newcommand\kw\textbf\newcommand\n\textit\newcommand\tpname\text
%\newcommand\kw[1]{\textsfit{#1}}\newcommand\n\textsfit\newcommand\tpname\textsf
\renewcommand\c\textsf
\newcommand\tset{\tpname{Set}~}
\newcommand\tunit{\ensuremath{1}}
\newcommand\tnode{\tpname{Node}}
\newcommand\zero{\ensuremath{\mathbf{0}}}

\newcommand\fix{\n{fix}}
\newcommand\semifix{\n{semifix}}

\newcommand\naive{na\"ive}
\newcommand\Naive{Na\"ive}

\let\oldcup\cup\renewcommand\cup{\mathrel{\oldcup}}

\newcommand\hilite{\color{Rhodamine}}
\newcommand\hi[1]{{\hilite#1}}
%\newcommand\hilitetime{\color{Orange}}\newcommand\hiti[1]{{\hilitetime#1}}


\begin{document}
  \Large


  %\includepdf[fitpaper]{title.pdf}
  \includepdf[fitpaper]{uncropped.pdf}

  \begin{frame}
    \textsc{pop quiz}\\ How can you compute each of the following?
    \vspace{1ex}
    \begin{itemize}
    \item Graph reachability
    \item Regular expression matching
    \item Abstract interpretation
    \end{itemize}

    \pause\vspace{1ex}

    \textsc{answer}\\ \itshape Iterate a monotone map to its fixed point.

    \vspace{\baselineskip}

%% They're all expressible as, and computable by, iterating a
%%     monotone map until it reaches a fixed point.
  \end{frame}


  \begin{frame}
    \begin{fleqn}[1em]
    \begin{minipage}[t][.38\paperheight]{\paperwidth}
      \[\begin{array}{l}
      \n{reach}(\c{start}).\\
      \n{reach}(Y) \gets \n{reach}(X) \wedge \n{edge}(X,Y).\\[1ex]\pause
      \n{reach2}(\c{start}).\\
      \n{reach2}(Y) \gets \n{reach2}(X) \wedge \n{edge2}(X,Y).\\[1ex]
      \n{reach3}(\c{start}).\\
      \n{reach3}(Y) \gets \n{reach3}(X) \wedge \n{isnt-this-tedious}(X,Y).
      \\[1ex]
      \n{reach4}(\c{start}).\\
      \n{reach4}(Y) \gets \n{reach4}(X) \wedge \n{yes-it-is-rather}(X,Y).
      \\[1ex]
      \n{reach5}(\c{start}).\\
      \n{reach5}(Y) \gets \n{reach5}(X) \wedge \n{yes-it-is-rather}(X,Y).\\
      \end{array}\]
    \end{minipage}
    \end{fleqn}
  \end{frame}


  %% TODO: connect this up somehow, either arrows or hilighting in different
  %% colors.
  \newcommand\setcolor[1]{{\alt<3->{\color{ForestGreen}}{}#1}}
  \newcommand\compcolor[1]{{\alt<3->{\color{RoyalBlue}}{}#1}}
  \newcommand\fixptcolor[1]{{\alt<4->{\color{Red}}{}#1}}

  \begin{frame}
    \[\begin{array}{l}
    \n{reach} \isa \setcolor{\tset (\tnode \x \tnode)}
    \to \tnode
    \to \setcolor{\tset \tnode}\\
    \n{reach} \;\n{edge} \;\n{start} =
    \fixptcolor{\kw{fix}}\; (\fnof R \compcolor{\{\n{start}\}
    \cup \setfor{y}{x \in R, (x,y) \in \n{edge}}})
    \end{array}\]

    %% Datalog is a \emph{first-order} logic language.
    %% \vspace{\baselineskip}

    %% \strong{Datafun}\textsuperscript{\sffamily\scshape[icfp 2016]}
    %% is a simply-typed \fn-calculus with:\vspace{1ex}
    %% \begin{itemize}
    %% \item a finite set datatype \& set comprehensions
    %% \item bottom-up monotone fixed points
    %% \item a type system that tracks monotonicity
    %% \end{itemize}

    \pause
    \uncover<5->{\strong{Datafun}\textsuperscript{\sffamily\scshape[icfp 2016]}
    is:}
    \begin{itemize}\setlength\itemsep{.5ex}
    \item a simply-typed \fn-calculus with \pause
    \item a \setcolor{finite set datatype}
      \& \compcolor{set comprehensions}, \pause
    \item a \fixptcolor{monotone fixed point operator}, \pause
    \item where \emph{types are posets} and \emph{all functions are monotone},
      \pause
    \item and non-monotonicity is handled via a comonad type $\iso A$.
    \end{itemize}
    \vspace{\baselineskip}

    %% \strong{Datafun}\textsuperscript{\sffamily\scshape[icfp 2016]} is a
    %% simply-typed \fn-calculus where \emph{types are posets} and \emph{all functions are
    %% monotone},
    %% %
    %% extended with\vspace{1ex}
    %% \begin{itemize}
    %% \item a finite set datatype \& set comprehensions;
    %% \item a bottom-up monotone fixed point operator;
    %% \item and a \emph{discreteness} comonad, $\iso$.
    %% \end{itemize}

  \end{frame}

  \begin{frame}
    The type $\iso A$ is defined:
    \begin{align*}
      x \in \iso A &\iff x \in A\\
      x \le y \isa \iso A &\iff x = y
    \end{align*}

    Thus $f : \iso A \to B$ is monotone iff:

    \[ x = y \implies f(x) \le f(y) \]

    i.e., \strong{always!}
  \end{frame}


  %% \begin{frame}
  %%   \begin{fleqn}[1em]
  %%     \textsc{math}
  %%
  %%     \[\begin{array}{l}
  %%     \n{reach} \isa \tset (\tnode \x \tnode) \to \tnode \to \tset \tnode\\
  %%     \n{reach} \;\n{edge} \;\n{start} =
  %%     \kw{fix}\; (\fnof R \{\n{start}\}
  %%     \cup \setfor{y}{x \in R, (x,y) \in \n{edge}})
  %%     \end{array}\]
  %%
  %%     \vspace{\baselineskip}
  %%     \textsc{datafun}
  %%
  %%     \[
  %%     \begin{array}{l}
  %%       \n{reach} \isa \iso(\tset(\tnode \x \tnode)) \to \iso\tnode
  %%       \to \tset \tnode\\
  %%       \n{reach} \;\pbox{\n{edge}} \;\pbox{\n{start}} =\\
  %%       \qquad
  %%       \kw{fix}\; \ebox{\fnof R \{\n{start}\}
  %%         \cup \setfor{y}{x \in R, (x',y) \in \n{edge}, x = x'}}
  %%     \end{array}
  %%     \]
  %%   \end{fleqn}
  %% \end{frame}

  \begin{frame}
    \begin{fleqn}[2em]
      \centering
      \textsc{\alt<2->{datafun}{math}}
      \vspace{1ex}

      \[\begin{array}{l}
      \n{reach} \isa
      \uncover<2>{\iso(}\tset (\tnode \x \tnode)\uncover<2>{)}
      \to \uncover<2>{\iso}\tnode
      \to \tset \tnode
      \\
      \n{reach} \;\uncover<2>{[}\n{edge}\uncover<2>{]}
      \;\uncover<2>{[}\n{start}\uncover<2>{]} =\\
      \qquad
      \kw{fix}\; \alt<2>{\kern.8pt [\kern .8pt}{(}\fnof R \{\n{start}\}
      \cup \setfor{y}{x \in R, (x\uncover<2>{'},y) \in \n{edge}%
        \uncover<2>{, x = x'}}%
      \alt<2>{\kern .8pt]}{)}
      \end{array}\]
    \end{fleqn}
  \end{frame}


  \begin{frame}
    \[\n{step}\;R =
    {\color<-2>{RoyalBlue}\{\n{start}\}}
    \cup
    {\color<-2>{Red}\setfor{y}{x \in R, (x,y) \in \n{edge}}}
    \]

    \pause\begin{align*}
      R_0 &= \emptyset
      & \uncover<3->{dR_0} &\uncover<3->{= }\uncover<4->{\color<-5>{RoyalBlue}\{\n{start}\}}
      \\
      R_{i+1} &= \makebox[4.3em][l]{$\alt<6->{\hi{R_i \cup dR_i}}{\n{step}\;R_i}$}
      & \uncover<3->{dR_{i+1}} &\uncover<3->{= }\uncover<5->{\color<-5>{Red} \setfor{y}{x \in dR_i, (x,y) \in \n{edge}}}
    \end{align*}

    \centering\alt<3->{
      \includegraphics[height=45mm]{imgs/focus-1240.jpg}
    }{
      \includegraphics[height=45mm]{imgs/focus-0946.jpg}
    }
    \vspace{1ex}
  \end{frame}

  %% \includepdf[fitpaper]{imgs/naive-0946.pdf}
  %% \includepdf[fitpaper]{imgs/naive-1240.pdf}
  %% \includepdf[fitpaper]{imgs/naive-1429.pdf}
  %% \includepdf[fitpaper]{imgs/naive-1642.pdf}

  %% \begin{frame}
  %%     \[\begin{array}{l}
  %%       \n{step} \;R = \{\n{start}\} \cup \setfor{y}{x \in R, (x,y) \in
  %%         \n{edge}}\\
  %%       R_0 = \emptyset\\
  %%       R_{i+1} = \n{step}\;R_i
  %%     \end{array}\]
  %%     \vspace{\baselineskip}

  %%     \color{red} TODO: pictures here explaining
  %%     \begin{itemize}
  %%     \item \naive\ evaluation
  %%     \item semi\naive\ evaluation as ``keeping a frontier'' ie. ``computing the changes''
  %%     \end{itemize}

  %%     %% \begin{align*}
  %%     %%   R_0 &= \emptyset & R_{i+1} &= \n{step}\;R_i
  %%     %% \end{align*}
  %% \end{frame}


  %% \begin{frame}
  %%   \begin{mathpar}
  %%     \begin{array}{l}
  %%       \n{step} \; s = \n{edge} \cup
  %%       \setfor{(x,z)}{(x,y) \in \n{edge}, (y,z) \in s}
  %%       \\
  %%       \n{path}_i = \n{step}^i \;\emptyset =
  %%       \text{paths of length $\le i$}
  %%     \end{array}
  %%     %% TODO: explain why no work is shared!
  %%     %% (step path_i) appends an edge to _every_ element of path_i
  %%   \end{mathpar}

  %%   \vspace{1ex} $\n{path}_{i+1} = \n{step}\;\n{path}_i$ appends an edge to
  %%   \emph{every} previously discovered path!
  %%   \vspace{1ex}\pause

  %%   \begin{center}
  %%     \begin{tikzpicture}[thick,baseline=(current bounding box.center), scale=1.5]
  %%       \node (a) at (0, 0) {$\n{path}_0$};
  %%       \node (ab) at (1, 0) {$\subseteq$};
  %%       \node (b) at (2, 0) {$\n{path}_1$};
  %%       \node (bc) at (3, 0) {$\subseteq$};
  %%       \node (c) at (4, 0) {$\n{path}_2$};
  %%       \node (cd) at (5, 0) {$\subseteq$};
  %%       \node (d) at (6, 0) {$\cdots$};
  %%       \path[-Latex, bend right=50,below]
  %%       (a) edge node {$\n{dpath}_0$} (b)
  %%       (b) edge node {$\n{dpath}_1$} (c)
  %%       (c) edge node {$\n{dpath}_2$} (d);
  %%     \end{tikzpicture}
  %%   \end{center}

  %% \end{frame}


  \begin{frame}{}{}\huge\centering%\setlength\baselineskip{2em}

    {\scshape semi\naive\ evaluation}
    \\[2ex]
    {means}
    \\[2ex]
    {\bf computing the changes between iterations}

    %% {by}

    %% {\bfseries incrementalizing the step function}
  \end{frame}


  \newcommand\chgto\rightsquigarrow
  \newcommand\chgat[4]{{\ensuremath{{#2} \dblcolon_{#1} {#3} \chgto {#4}}}}
  \newcommand\chg[3]{{\ensuremath{{#1} \dblcolon {#2} \chgto {#3}}}}

  \begin{frame}{Incremental \fn-calculus
      \textsuperscript{\sffamily[%
          Cai et al \textsc{pldi 2014},
          Giarrusso's thesis \textsc{2017},
          Giarrusso et al \textsc{esop 2019}]}}
    Every type $A$ has a \emph{change type} $\D A$ and a relation
    $\chgat{A}{dx}{x}{y}$, where $x,y : A$ and $dx : \D A$.

    \vspace{1ex}
    \setlength\arraycolsep{.2em}
    \[
    \begin{array}{rcl@{\qquad}c}
      \D(\tset A) &=& \tset A & \chgat{\tset A}{dx}{x}{x \cup dx}
      \\[1.5ex]\pause
      \D(A \x B) &=& \D A \x \D B &
      \infer{\chgat A{da} a {a'} \\ \chgat B{db} b {b'}
      }{\chgat{A \x B}{(da,db)}{(a,b)}{(a',b')}}
      \\[3.5ex]\pause
      \D(\iso A) &=& \tunit & \chgat{\iso A}{()}{x}{x}
    \end{array}
    \]
  \end{frame}

  \newcommand\underexplain[4]{%
    \underaccent{%
      \parbox[t][#1]{#2}{\sffamily\centering\vspace{4pt}\small#3}
    }{#4}}

  \newcommand\chgatcov[4]{{\uncover<2->{{#2} \dblcolon_{#1}} {#3} \chgto {#4}}}
  %\newcommand\chgatcovb[4]{{\uncover<5->{{#2} \dblcolon_{#1}} {#3} \to {#4}}}
  \begin{frame}\LARGE
    %\setbeamercovered{transparent}

    \begin{mathpar}
      \D(A \to B)
      = \underexplain{}{1.5cm}{original input}{\iso A}
      \to \underexplain{}{1.2cm}{change to input}{\D A}
      \to \underexplain{}{1.4cm}{change in output}{\D B}
      \\\pause
      \chgat{A \to B}{df}{f}{\alt<5->{\hi f}{g}} \iff \pause
      \infer{\chgat{A}{dx}{x}{y}
      }{\chgat{B}{\uncover<4->{df\;x\;dx}}{f\;x}{\alt<5->{\hi f}{g}\;y}}
      %% \infer{\chgatcov{A \to B}{df}{f}{\alt<3>{\hi{f}}{g}}\\
      %%   \chgatcov{A}{dx}{x}{y}
      %% }{
      %%   \chgatcov{B}{df\;x\;dx}{f\;x}{\alt<3>{\hi f}{g}\;y}
      %% }
    \end{mathpar}

    \setbeamercovered{invisible}
    \vspace{2ex}\centering\pause\pause
    In this case, we call $df$ the
    \strong{derivative} of $f$.

    \vspace{\baselineskip}

    %% \[
    %% \infer{
    %%   \fa{\chgat A {dx} x y}\, \chgat B {df\;x\;dx} {f\;x} {g\;y}
    %% }{\chgat{A \to B}{df} f g}
    %% \]
  \end{frame}


  %% %% TODO?
  %% \begin{frame}
  %%   \begin{mathpar}
  %%     \infer{
  %%       \chgatcov{A \to B}{df}{f}{g} \\
  %%       \chgat{A}{dx}{x}{y}
  %%     }{\chgatcov{B}{df\;x\;dx}{f\;x}{g\;y}}
  %%   \end{mathpar}
  %% \end{frame}


  \begin{frame}
      \begin{align*}
        \n{step} &\isa \tset A \to \tset A
        %% & \uncover<2->{R_i} &\uncover<2->{= \n{step}^i \;\emptyset}
        \\
        \n{step}' &\isa
        \underexplain{}{4cm}{\vspace{-3.43pt}known world}{\iso(\tset A)}
        \to \underexplain{}{1.8cm}{frontier}{\tset A}
        \to \underexplain{}{2cm}{new frontier}{\tset A}
        %% & \uncover<3->{dR_i} &\uncover<3->{\dblcolon R_i \chgto \n{step}\;R_i}
      \end{align*}

      \textsc{strategy}
      \begin{align*}
        \underexplain{}{2cm}{known world}{{R_i}} &= \n{step}^i\;\emptyset
        &\uncover<2->{\chg{\underexplain{}{1.5cm}{frontier}{{dR_i}}}{R_i}{\n{step}\;R_i}}
      \end{align*}
      \pause\pause

      %\setbeamercovered{transparent}
      \textsc{implementation}
      \begin{align*}
        R_0 &= \emptyset %\pause
        & dR_0 &= \uncover<4->{\n{step}\;\emptyset} %\pause
        %\dblcolon \emptyset \chgto \n{step}\;\emptyset
        \\
        R_{i+1} &= \uncover<5->{R_i \cup dR_i} %\pause
        & dR_{i+1} &= \uncover<6->{\n{step}' \;R_i \;dR_i}
        %\dblcolon \n{step}\;R_i \chgto \n{step}\;(R_i \cup dR_i)
      \end{align*}
      \vspace{\baselineskip}
  \end{frame}


  %% \begin{frame}
  %%   %% TODO: this slide should give an intuition for what dR_i is. Perhaps
  %%   %% graphically, or at least as "the set of nodes at distance exactly i".
  %%   Suppose $\n{step}'$ is the derivative of $\n{step}$. Then:

  %%   \[
  %%   \n{step} \;(R_i \cup dR_i) = \n{step}\;R_i \cup \n{step}'\;R_i\;dR_i
  %%   \]

  %%   So consider the recurrences:
  %%   \begin{align*}
  %%     R_0 &= \emptyset & R_{i+1} &= R_i \cup dR_i\\
  %%     dR_0 &= \n{step}\;\emptyset & dR_{i+1} &= \n{step}'\;R_i\;dR_i
  %%   \end{align*}

  %%   \strong{Theorem:} $R_i = \n{step}^i\;\emptyset$.
  %%   \\[\baselineskip]
  %%   So let $\semifix\;(\n{step}, \n{step}') \triangleq R_i$ such that $R_i =
  %%   R_{i+1}$.
  %%   \\
  %%   Then $\semifix\;(\n{step},\n{step}') = \fix\;\n{step}$.
  %%   \vspace{\baselineskip}
  %% \end{frame}


  \begin{frame}
    %\begin{center}\scshape the core idea\end{center}

    %% On any expression $e$, we define two
    %% mutually recursive static transformations:
    Two static transformations on expressions $e$:
    \vspace{1ex}
    \begin{itemize}
    \item \hi{$\phi e$} annotates functions used by a fixed point with their
      derivatives.\\[1ex]

      {\normalsize\sffamily\textsc{complication:} \it We propagate
        derivatives by hijacking the $\iso$ comonad.\par}

    \item \hi{$\delta e$} computes how $\phi e$ changes as
      its free variables change.
      \\[1ex]
      {\normalsize\sffamily\textsc{complication:}
        \it Computing how set comprehensions change.
        \par}
    \end{itemize}
  \end{frame}

  \begin{frame}{}{}
    \centering
    \begin{itemize}
    \item Regex combinators!\\
      {\normalsize as a use-case for higher-order functions.\par}

    \item Logical relations!\\
      {\normalsize to prove $\phi/\delta$ correct.\par}

    \item Further optimizations!\\
      {\normalsize Must propagate $\emptyset$ to get asymptotic speedups.\par}

    \item Benchmarks!\\[1ex]
      {\scriptsize
        \begin{tikzpicture}[baseline=(current bounding box.center)]
          \begin{axis}[
              xlabel={Number of nodes},
              ylabel={Seconds},
              height=90pt, width=145.62306pt,
              legend entries={\naive,semi\naive},
              legend cell align=left,
              legend pos = north west,
              legend style={
                draw=none,
                nodes={inner sep=2pt,}
              },
              xtick = {120, 320}, ytick = {0, 120},
            ]
            \addplot [color=red,thick] table [x=n,y=naive] {trans.dat};
            %\addplot [color=black,mark=triangle*] table [x=n,y=seminaive_raw] {trans.dat};
            \addplot [color=blue,thick] table  [x=n,y=seminaive] {trans.dat};
          \end{axis}
        \end{tikzpicture}
        \par}
    \end{itemize}

  \end{frame}


  \begin{frame}
    \strong{Ideas to take away:}\vspace{1ex}
    \begin{itemize}
    \item Bottom-up monotone fixed points are cool.
    \item To compute them efficiently, incrementalize the step function.
    \item Types let us control what things we need to incrementalize!
    \end{itemize}
    \vspace{\baselineskip}
  \end{frame}

  \appendix
  \begin{frame}
    \centering\Huge
    \color{gray!10}\itshape%
    un{\upshape\scshape\color{black}fin}%
    \color{gray!10}ished
  \end{frame}


  %% \begin{frame}{}{}
  %%   \begin{fleqn}[1em]
  %%   {\bfseries\boldmath Incremental \fn-calculus} \textsuperscript{\sffamily\scshape[pldi 2014]}
  
  %%   \begin{align*}
  %%     \makebox[1.2cm][r]{$\alt<2->{\n{step}}{f}$} &\isa \alt<2->{\tset A \to \tset A}{A \to B}\\
  %%     \alt<2->{\n{step}}{f}' &\isa
  %%     \alt<2->{
  %%       \underaccent{\parbox[t][1.4cm]{1.5cm}{\uncover<3->{\centering\vspace{4pt}\small\sffamily paths of length $< i$}}
  %%       }{\tset A}
  %%       \to \only<3->{\underaccent{\parbox{2cm}{\centering\vspace{4pt}\small\sffamily paths of length $i$}%
  %%       }}{\hi{\alt<3->{\tset A}{\D(\tset A)}}}
  %%       \to \only<3->{\underaccent{\parbox{2cm}{\centering\vspace{4pt}\small\sffamily paths of length $i+1$}%
  %%       }}{\hi{\alt<3->{\tset A}{\D(\tset A)}}}
  %%     }{
  %%       \underaccent{\parbox{1.3cm}{\centering\vspace{4pt}\small\textsf{input}}}{A}
  %%       \to \underaccent{\parbox{1.3cm}{\centering\small\sffamily\vspace{4pt} change to input}}{\D A}
  %%       \to \underaccent{\parbox[t][1.4cm]{1.3cm}{\centering\small\sffamily\vspace{4pt} resulting change in output}}{\D B}
  %%     }
  %%   \end{align*}
  
  %%   \pause\pause\pause
  %%   \begin{align*}
  %%     \n{path}_0 &= \emptyset
  %%     & \n{path}_{i+1} &= \n{path}_i \cup \n{dpath}_{i}\\
  %%     \n{dpath}_0 &= \n{step} \;\emptyset
  %%     & \n{dpath}_{i+1} &= \n{step}' \;\n{path}_i \;\n{dpath}_i
  %%   \end{align*}
  
  %%   \vspace{1ex} Let $\semifix\;(\n{step}, \n{step}') = \n{path}_i$ st.
  %%   $\n{path}_i = \n{path}_{i+1}$.
  %%   \end{fleqn}
  %%   %% \begin{align*}
  %%   %%   f &\isa A \to B\\
  %%   %%   f' &\isa \underaccent{\parbox{1.3cm}{\centering\vspace{4pt}\small\textsf{input}}}{A}
  %%   %%     \to \underaccent{\parbox{1.3cm}{\centering\small\sffamily\vspace{4pt} change to input}}{\hi{\D A}}
  %%   %%     \to \underaccent{\parbox{1.3cm}{\centering\small\sffamily\vspace{4pt} resulting change in output}}{\hi{\D B}}
  %%   %% \end{align*}
  %% \end{frame}

  %% %% \begin{frame}
  %% %%   \begin{fleqn}
  %% %%   \begin{align*}
  %% %%     \n{step} &\isa \tset A \to \tset A\\
  %% %%     \n{step}' &\isa
  %% %%     \underaccent{\parbox{1.3cm}{\centering\vspace{4pt}\small\sffamily previous paths}}{\tset A}
  %% %%     \to \underaccent{wub
  %% %%     }{\hi{\alt<2->{\tset A}{\D(\tset A)}}}
  %% %%     \to \hi{\alt<2->{\tset A}{\D(\tset A)}}
  %% %%   \end{align*}
  %% %%   \end{fleqn}
  %% %% \end{frame}

\end{document}

% From https://tex.stackexchange.com/questions/39415/unload-a-latex-package
\makeatletter
\newcommand{\dontusepackage}[2][]{%
  \@namedef{ver@#2.sty}{9999/12/31}%
  \@namedef{opt@#2.sty}{#1}}
\makeatother

\providecommand\mathvar\mathit
\PassOptionsToPackage{dvipsnames}{xcolor}

%% Avoid using acmart's fonts.
\dontusepackage[tt=false,type1=true,]{libertine}
\dontusepackage[libertine]{newtxmath}
%% Scale inconsolata to match XCharter[scaled=.94].
%% Might want to scale it down further.
\PassOptionsToPackage{scaled=0.9893675916139859}{zi4}


%% For review. Maximizes space available.
\documentclass[acmsmall,]{acmart}\settopmatter{printfolios=true,printccs=false,printacmref=false}
%% For final camera-ready submission, w/ required CCS and ACM Reference
%\documentclass[acmsmall]{acmart}\settopmatter{}

%% Journal information
%% Supplied to authors by publisher for camera-ready submission;
%% use defaults for review submission.
\acmJournal{PACMPL}
\acmVolume{1}
\acmNumber{CONF} % CONF = POPL or ICFP or OOPSLA
\acmArticle{1}
\acmYear{2018}
\acmMonth{1}
\acmDOI{} % \acmDOI{10.1145/nnnnnnn.nnnnnnn}
\startPage{1}

%% Copyright information
%% Supplied to authors (based on authors' rights management selection;
%% see authors.acm.org) by publisher for camera-ready submission;
%% use 'none' for review submission.
\setcopyright{none}
%\setcopyright{acmcopyright}
%\setcopyright{acmlicensed}
%\setcopyright{rightsretained}
%\copyrightyear{2018}           %% If different from \acmYear

%% Bibliography style
\bibliographystyle{ACM-Reference-Format}
%% Citation style
%% Note: author/year citations are required for papers published as an
%% issue of PACMPL.
\citestyle{acmauthoryear}   %% For author/year citations


\usepackage[utf8]{inputenc}
\usepackage{style}
\usepackage{notation}

%% Uncomment this to show all the bibliographic keys
% \usepackage{showkeys}

\begin{document}

%% Title information: \title[Short Title]{Full Title}
%% Short Title is optional; when present, will be used in header instead of Full Title.
\title{Semi\naive{} Evaluation for a Higher-Order Functional Language}
%\title{Semi\naive{} Evaluation for Datafun}
% Optional: \subtitle{}, \titlenote{}, \subtitlenote{}

%% Author information
%% Contents and number of authors suppressed with 'anonymous'.
%% Each author should be introduced by \author, followed by
%% \authornote (optional), \orcid (optional), \affiliation, and
%% \email.
%% An author may have multiple affiliations and/or emails; repeat the
%% appropriate command.
%% Many elements are not rendered, but should be provided for metadata
%% extraction tools.

\author{Michael Arntzenius}
\affiliation{
  %\position{Position1}
  \department{School of Computer Science}
  \institution{University of Birmingham}
  %\streetaddress{Street1 Address1}
  \city{Birmingham}
  %\state{State1}
  \postcode{B15 2TT}
  \country{United Kingdom}
}
\email{daekharel@gmail.com}

\author{Neelakantan R. Krishnaswami}
\affiliation{
  %\position{Position2a}
  \department{Department of Computer Science and Technology} %% \department is recommended
  \institution{University of Cambridge}           %% \institution is required
  %\streetaddress{Street2a Address2a}
  \city{Cambridge}
  %\state{State2a}
  \postcode{CB2 1TN}
  \country{United Kingdom}
}
\email{first2.last2@inst2a.com FIXME FIXME}         %% \email is recommended


%% ---------- Abstract ----------
%% Note: \begin{abstract}...\end{abstract} environment must come
%% before \maketitle command
\begin{abstract}
One of the workhorse techniques for implementing bottom-up Datalog engines is
semi\naive\ evaluation~\cite{seminaive}. This optimization improves the
performance of Datalog's most distinctive feature: recursively defined
predicates. These are computed iteratively, and under a \naive\ evaluation
strategy, each iteration recomputes all previous values. Semi\naive\ evaluation
computes a safe approximation of the \emph{difference} between iterations. This
can \emph{asymptotically} improve the performance of Datalog queries.

Semi\naive\ evaluation is defined partly as a program transformation and partly
as a modified iteration strategy, and takes advantage of the first-order nature
of Datalog code.
%
This paper extends the semi\naive\ transformation to higher-order programs
written in the Datafun language, which extends Datalog with features like
first-class relations, higher-order functions, and datatypes like sum types.
\end{abstract}


%% ---------- Classification and keywords nonsense ----------
%% 2012 ACM Computing Classification System (CSS) concepts
%% XXX: Generate at 'http://dl.acm.org/ccs/ccs.cfm'.
 \begin{CCSXML}
<ccs2012>
<concept>
<concept_id>10003752.10010070.10010111.10010113</concept_id>
<concept_desc>Theory of computation~Database query languages (principles)</concept_desc>
<concept_significance>500</concept_significance>
</concept>
<concept>
<concept_id>10003752.10010070.10010111.10011734</concept_id>
<concept_desc>Theory of computation~Logic and databases</concept_desc>
<concept_significance>500</concept_significance>
</concept>
<concept>
<concept_id>10003752.10003790.10003795</concept_id>
<concept_desc>Theory of computation~Constraint and logic programming</concept_desc>
<concept_significance>300</concept_significance>
</concept>
<concept>
<concept_id>10003752.10010124.10010131.10010133</concept_id>
<concept_desc>Theory of computation~Denotational semantics</concept_desc>
<concept_significance>300</concept_significance>
</concept>
<concept>
<concept_id>10011007.10011006.10011008.10011009.10011012</concept_id>
<concept_desc>Software and its engineering~Functional languages</concept_desc>
<concept_significance>500</concept_significance>
</concept>
</ccs2012>
\end{CCSXML}

\ccsdesc[500]{Theory of computation~Database query languages (principles)}
\ccsdesc[500]{Theory of computation~Logic and databases}
\ccsdesc[300]{Theory of computation~Constraint and logic programming}
\ccsdesc[300]{Theory of computation~Denotational semantics}
\ccsdesc[500]{Software and its engineering~Functional languages}
%% End of generated code

% TODO: comma separated list of keywords.
% \keywords are mandatory in final camera-ready submission
\keywords{Datafun, Datalog, functional languages, semi\naive\ evaluation, incremental computation}


%% ---------- Sections ----------
%% Note: \maketitle command must come after title commands, author
%% commands, abstract environment, Computing Classification System
%% environment and commands, and keywords command.
\maketitle

\section{Introduction}
\label{sec:intro}

Datalog, along with the $\pi$-calculus and \fn-calculus, is one of the jewel
languages of theoretical computer science, connecting programming language
theory, database theory, and complexity theory. In terms of programming
languages, Datalog can be understood as a fully declarative subset of
Prolog~\cite{datalog-from-prolog} which is guaranteed to terminate and so can be
evaluated in both top-down and bottom-up fashion. In terms of database
theory~\cite{datalog-relalg}, it is equivalent to the extension of relational
algebra with a fixed point operator, and in terms of complexity theory, it does
not just terminate, but also (when full Datalog with stratified negation is
interpreted over ordered structures) characterizes polytime
computation~\cite{datalog-polytime}.

In addition to its theoretical elegance, over the past twenty years
Datalog has seen a surprisingly wide surge of use across a variety of
practical domains, both in research and in industry. Whaley and Lam
\cite{whaley-lam,whaley-phd} implemented pointer analysis algorithms
in Datalog, and found that they could reduce their analyses from
thousands of lines of C code to \emph{tens} of lines of Datalog code,
while retaining competitive performance. Semmle has developed the .QL
language~\cite{semmlecode,ql-inference} based on Datalog for analysing
source code (which was used to analyze the code for NASA's Curiosity
Mars rover), and LogicBlox has developed the LogiQL~\cite{logicblox}
language for business analytics and retail prediction. The Boom
project at Berkeley has developed the Bloom language for distributed
programming~\cite{bloom}, and the Datomic cloud
database~\cite{datomic} uses Datalog (embedded in Clojure) as its
query language. Microsoft's SecPAL language~\cite{secpal} uses Datalog
as the foundation of its decentralised authorization specification
language. In each case, when the problem formulated in Datalog, the
specification became directly implementable, while remaining
dramatically shorter and clearer than alternatives implemented in more
conventional languages.

\todo{We should cite DOOP (which used to use LogicBlox and now uses Souffle)}

\todo{We should cite Yannis's
  recent SNAPL paper ``Next-Paradigm Programming Languages: What Will They
  Look Like and What Changes Will They Bring?''}

However, there are two flies in the ointment. First, even though each
of these applications is supported by the skeleton of Datalog, they
all had to extend it significantly beyond the theoretical core
calculus.  For example, core Datalog does not even support arithmetic,
since its semantics only speaks of finite sets supporting equality of
their elements. Moreover, arithmetic is not a trivial extension, since
it can greatly complicates the semantics (for example, proving that
termination continues to hold). So despite the fact that kernel
Datalog has a very clean semantics, its metatheory seemingly needs to
be laboriously re-established once again for each extension.

A natural approach to solving this problem is to find a language in
which to write the extensions, which preserves the semantic guarantees
that Datalog offers. Two such proposals are the Flix
language~\cite{flix} and the Datafun language~\cite{datafun}.  Very
conveniently for our exposition, these two languages embody two
alternative design strategies.

Flix adopts the route of treating Datalog as an embedded
domain-specific language. That is, Flix is a fairly conventional
albeit well-designed functional programming language roughly
comparable to ML or Haskell, extended with types representing Datalog
predicates and program. The evaluation of a Flix program builds a
Datalog program, which is then handed off to a custom Datalog engine.
This stratification means that (a) standard Datalog implementation
techniques can be used off-the-shelf, but that (b) its functional
programming side is essentially just a very powerful macro system for
Datalog.

Datafun takes a somewhat different approach. Like Flix, it is a
functional programming language, but unlike Flix, its type discipline
has been extended to support tracking whether functions are monotone
or not. Then, Datalog-style recursively defined relations are just
ordinary recursive defininitions on set-valued functions, with
monotoncity typing ensuring that every definable function continues to
make sense in the model-theoretic semantics of Datalog. Tracking
monotonicity information in types permits a much tighter integration
between the functional and logic programming styles, but it comes at
the cost that many of Datalog's standard implementation techniques,
which were originally developed in the context of a first-order logic
language, are no longer obviously applicable in the higher-order
functional setting.

Second, even though the semantics of Datalog is simple, actually
making it perform well enough to use in practice calls for very
sophisticated program analysis and compiler engineering. (This is a
familiar issue from databases, where query planners embody a startling
amount of optimization knowledge to optimize relatively simple SQL
queries.) A wide variety of techniques for optimizing Datalog programs
have been studied in the literature, such as using binary decision
diagrams to represent relations~\cite{whaley-phd}, exploiting the
structure of well-behaved subsets (e.g., CFL-reachability problems
correspond to the ``chain program'' fragment of
Datalog~\cite{chain-programs}), and combining top-down and bottom-up
evaluation via the ``magic sets'' algorithm~\cite{magic-sets}.

Today, one of the workhorse techniques for implementing bottom-up
Datalog engines is \emph{semi\naive\
  evaluation}~\cite{seminaive}. This optimization improves the
performance of Datalog's most distinctive feature: recursively defined
predicates. Such a predicate can be understood as the fixed point of a
set-valued function $f$; a \naive\ fixed point computation will
directly compute the limit of the sequence
$\emptyset, f(\emptyset), f^2(\emptyset), \dots$. At each iteration,
many values are unnecessarily re-computed; semi\naive\ evaluation
computes a safe approximation of the \emph{new} values generated at
each step. This optimization is critical, as it can
\emph{asymptotically} improve the performance of Datalog queries.


\paragraph{Contributions} The semi\naive\ evaluation algorithm is
defined partly as a program transformation on sets of Datalog rules,
and partly as a modification of the fixed point computation algorithm.
The central contribution of this paper is to give an extended version
of this transformation which works on higher-order programs written
in the Datafun language. 

\begin{itemize}
\item We reformulate Datafun in terms of a kernel calculus based on
  the modal logic S4. Instead of giving a calculus with distinct
  monotonic and discrete function types, as in the original Datafun
  paper, we make discreteness into a comonad. In addition to
  regularizing the calculus and slightly improving its expressiveness,
  the explicit comonadic structure lets us impose a modal constraint
  on recursion reminiscent of Hoffman's work on safe
  recursion~\cite{hoffman-safe-recursion}. This brings the semantics
  of Datafun more closely in line with Datalog's, and substantially
  simplifies the program transformation we present.
  
\item We define a program transformation to statically convert
  well-typed Datafun programs into an incrementalized version. The
  translation is a compositional type-and-syntax-directed
  transformation, which works uniformly at all types including
  function types. Our approach builds upon the static program
  incrementalization introduced by \citet{ostermann}, extending it to
  support for sum types, set types, comonads, and (well-founded)
  recursion.

\item We establish the correctness of our transformation using a novel
  logical relation which simultaneously defines the changes connecting
  old and updated programs, as well as the optimized version of both.
  The fundamental lemma then lets us show that our transformation
  is semantics-preserving, in the sense that any closed program term
  of predicate type has the same meaning after optimization. 
  
\item We then discuss our implementation of the  
\end{itemize}


%% \begin{itemize}

%% \item We define a program transformation to statically convert well-typed
%%   Datafun programs (including ones that use higher-order functions) into an
%%   incrementalized version. We then use a logical relation to show that our
%%   incrementalization is correct, and so can be used to optimize Datafun
%%   programs.

%% \item We also implement our program transformation in a small compiler, and use
%%   this to show that we can automatically compile Datafun terms into optimized
%%   Haskell programs with the expected operational behaviour.

%% \end{itemize}


%%% Local Variables:
%%% TeX-master: "seminaive-datafun"
%%% End:

\section{Datalog and Datafun}
\label{sec:datalog-and-datafun}
\section{From Semi\naive{} Evaluation to the Incremental \boldfn-Calculus}
\label{sec:seminaive-and-ilc}

Consider the following Datalog fragment:\footnote{This example and explanation
  of semi\naive\ evaluation is borrowed almost entirely from
  \citet{DBLP:conf/esop/Alvarez-Picallo19}.}

\begin{align*}
  \name{path}(x,y) &\gets \name{edge}(x,y)
  &
  \name{path}(x,z) &\gets \name{edge}(x,y) \wedge \name{path}(y,z)
\end{align*}


\noindent
The denotation of \name{path} is the least fixed point of the rules defining
it.
% mention Kleene fixed point theorem?
We can compute this by repeatedly applying these rules until the collection
of known paths (initially empty) stops growing.
%
For example, if the \name{edge} relation is \{(1,\,2), (2,\,3), (3,\,4)\}, we
get the following evaluation trace:

%% TODO: make sure there's enough vertical space around this.
\begin{center}
  \setlength\tabcolsep{1em}
  \begin{tabular}{@{}rll@{}}
    Step
    & Previously known paths
    & Paths deduced at this step
    \\\midrule
    0
    & none
    & (1,\,2) (2,\,3) (3,\,4)
    \\
    1
    & (1,\,2) (2,\,3) (3,\,4)
    & (1,\,2) (2,\,3) (3,\,4) (1,\,3) (2,\,4)
    \\
    2
    & (1,\,2) (2,\,3) (3,\,4) (1,\,3) (2,\,4)
    & (1,\,2) (2,\,3) (3,\,4) (1,\,3) (2,\,4) (1,\,4)
    \\
    3
    & (1,\,2) (2,\,3) (3,\,4) (1,\,3) (2,\,4) (1,\,4)
    & as above
  \end{tabular}
\end{center}

\noindent We have now reached the desired fixed point. However, this process is
quite wasteful: we deduced the path (1,\,2) at \emph{every} iteration; ideally
we'd only deduce it once. On a chain of $n$ edges, we deduce $\Theta(n^3)$
facts, even though there are only $\Theta(n^2)$ paths!

The standard improvement to this strategy is
\emph{semi\naive\ evaluation}~\cite{seminaive}, which transforms recursive rules
into explicitly iterative time-indexed rules of two kinds: \emph{derivative}
rules, to compute the new facts at each iteration; and \emph{accumulator} rules,
to collect these facts into a final result.
%
In this case, the derivative rule is simple: we discover new paths at iteration
$i+1$ by appending edges to paths which were new at iteration $i$:

\begin{align*}
  \name{dpath}_0(x,y) &\gets \name{edge}(x,y)
  \\
  \name{dpath}_{i+1}(x,z) &\gets \name{edge}(x,y) \wedge \name{dpath}_i(y,z)
  \\
  \name{path}_{i+1}(x,y) &\gets \name{path}_i(x,y) \vee \name{dpath}_i(x,y)
\end{align*}

\noindent This yields the execution trace:

\begin{center}
  \setlength\tabcolsep{1em}
  \begin{tabular}{@{}rll@{}}
    Step & $\name{path}_i$ & $\name{dpath}_i$
    \\\midrule
    0 & empty & (1,\,2) (2,\,3) (3,\,4)
    \\
    1 & (1,\,2) (2,\,3) (3,\,4) & (1,\,3) (2,\,4)
    \\
    2 & (1,\,2) (2,\,3) (3,\,4) (1,\,3) (2,\,4) & (1,\,4)
    \\
    3 & (1,\,2) (2,\,3) (3,\,4) (1,\,3) (2,\,4) (1,\,4) & empty
  \end{tabular}
\end{center}

%% TODO: email Alvarez-Picallo, Michael P-J, etc; they say "a quadratic
%% computation into a linear one" but AFAICT this is not so! Unless the claim is
%% that the naive version is quadratic in the _output_?!
%%
%% Perhaps we should say, "--- our computation is now linear in the size of its
%% output". But is this really justified?
\noindent
This is much better --- we have turned a cubic computation into a quadratic one!


\subsection{Semi\naive\ evaluation as incremental computation}

Now let's move from Datalog to Datafun. The transitive closure of \name{edge} is
the fixed point of the monotone function \name{step} defined by:

\[
\name{step} \<\name{path} = \name{edge} \cup
\setfor{(x,z)}{(x,y) \in \name{edge}, (y,z) \in \name{path}}
\]

\noindent
The \naive\ way to compute \name{step}'s fixed point is to iterate it: start
with \(\name{path}_0 = \emptyset\) and compute \(\name{path}_{i+1} =
\name{step}\<\name{path}_i\) for increasing $i$ until \(\name{path}_i =
\name{path}_{i+1}\).
%
But since $\name{path}_i \subseteq \name{step}\<\name{path}_i$, each iteration
re-computes every path found by the previous iteration.
%
Following Datalog, we'd prefer to compute only the \emph{change} between
iterations.
%
So consider $\name{step}'$ defined by:

\[
\name{step}' \<\name{dpath} =
\setfor{(x,z)}{(x,y) \in \name{edge}, (y,z) \in \name{dpath}}
\]

\noindent
Observe that $\name{step} \<(\name{path} \cup \name{dpath}) =
\name{step}\<\name{path} \cup \name{step}'\<\name{dpath}$.
%
That is, $\name{step}'$ tells us how \name{step} changes as its input grows.
%
Using this property, we can directly compute the changes $\name{dpath}_i$
between our iterations $\name{path}_i$:

%% \begin{align*}
%%   \name{dpath}_0 &= \name{step}\<\emptyset
%%   & \name{dpath}_{i+1} &= \name{step}'\<\name{dpath}_i
%%   & \name{path}_{i+1} &= \name{path}_i \cup \name{dpath}_i
%% \end{align*}

%% \[\def\arraystretch{1.2}
%% \begin{array}{rclcl}
%%   \name{dpath}_0 &=& \name{step}\<\emptyset
%%   &=& \name{edge}
%%   \\
%%   \name{dpath}_{i+1} &=& \name{step}'\<\name{dpath}_i
%%   &=& \setfor{(x,z)}{(x,y) \in \name{edge}, (y,z) \in \name{dpath}_i}
%%   \\
%%   \name{path}_{i+1} &=& \name{path}_i \cup \name{dpath}_i
%% \end{array}\]

\begin{align*}
  \name{dpath}_0
  &= \name{step}\<\emptyset
  = \name{edge}
  \\
  \name{dpath}_{i+1}
  &= \name{step}'\<\name{dpath}_i
  = \setfor{(x,z)}{(x,y) \in \name{edge}, (y,z) \in \name{dpath}_i}
  \\
  \name{path}_{i+1}
  &= \name{path}_i \cup \name{dpath}_i
\end{align*}

\noindent These exactly mirror the derivative and accumulator rules for
\(\name{path}_i\) and \(\name{dpath}_i\) we gave earlier.
%
\todo{Explain how this lets us compute $\name{path}_i$ more
  efficiently and wait until it quiesces as before.}

The problem of semi\naive\ evaluation for Datafun, then, reduces to the problem
of finding functions like $\name{step}'$, which compute the change in a
function's output given a change to its input.
%
This is a problem of \emph{incremental computation}, and since Datafun is a
functional language, we turn to the
\emph{incremental \fn-calculus}~\citep{incremental}.


\subsection{Change structures and the incremental \boldfn-calculus}
\label{sec:seminaive-via-incremental}

To formalize the notion of change, an incremental \fn-calculus associates every
type $A$ with a \emph{change structure}, consisting of:\footnote{Our notion of
  change structure differs significantly from that of \citeauthor{incremental};
  we discuss the divergence in \cref{sec:differences-from-incremental}. As we do
  not use change structures \emph{per se} in our proofs, we treat them
  informally, as a source of intuition rather than rigor.}

\begin{enumerate}
\item A type $\D A$ of possible changes to values of type $A$.
\item A relation $\changesat{A}{\dx}{x}{y}$ for $\dx : \D A$ and $x,y : A$,
  glossed as ``$\dx$ changes $x$ into $y$''.
\end{enumerate}

\noindent
Since the iterations of a fixed point grow monotonically, in Datafun we only
need \emph{increasing} changes.
%
For example, sets change by gaining new elements:

\begin{align*}
  \D\tseteq{A} &= \tseteq{A}
  &
  \changesat{\tseteq{A}}{\dx}{x}{x \cup \dx}
\end{align*}

\noindent Set changes may be the most significant for fixed point purposes, but
to handle all of Datafun we need a change structure for every type. For products
and sums, for example, the change structure is pointwise:

\begin{align*}
  \D\tunit &= \tunit
  &
  \D(A \x B) &= \D A \x \D B
  &
  \D(A + B) &= \D A + \D B
\end{align*}

\begin{align*}
  \changesat{\tunit}{\tuple{}}{\tuple{}}{\tuple{}}
  &&
  \infer{
    \fa{i} \changesat{A_i}{\dx_i}{x_i}{y_i}
  }{\textstyle\changesat{A_1 \x A_2}
    {\tuple{\vec\dx}}
    {\tuple{\vec x}}
    {\tuple{\vec y}}
  }
  %
  %% \infer{
  %%   \fa{i} \changesat{A_i}{\dx_i}{x_i}{y_i}
  %% }{\textstyle\changesat{A_1 \x A_2}
  %%   {\tuple{\dx_1,\dx_2}}
  %%   {\tuple{x_1,x_2}}
  %%   {\tuple{y_1,y_2}}
  %% }
  %
  %% \infer{
  %%   \changesat{A}{\da}{a_1}{a_2}
  %%   \\
  %%   \changesat{B}{\db}{b_1}{b_2}
  %% }{\textstyle\changesat{A \x B}
  %%   {\tuple{\da,\db}}
  %%   {\tuple{a_1,b_1}}
  %%   {\tuple{a_2,b_2}}
  %% }
  &&
  \infer{
    \changesat{A_i}{\dx_i}{x_i}{y_i}
  }{
    \textstyle\changesat{A_1 + A_2}{\inj i \dx}{\inj i x}{\inj i y}
  }
\end{align*}
\vspace{0pt} % FIXME. yes, this makes a difference, believe it or not

%% \begin{align*}
%%   \D\tunit &= \tunit
%%   &
%%   \D(A \x B) &= \D A \x \D B
%%   &
%%   %% \infer{
%%   %%   %\fa{i} \changesat{A_i}{\dx_i}{x_i}{y_i}
%%   %%   \changesat{A_1}{\dx_1}{x_1}{y_1}
%%   %%   \\
%%   %%   \changesat{A_2}{\dx_2}{x_2}{y_2}
%%   %% }{\changesat{A_1 \x A_2}
%%   %%   {\tuple{\dx_1,\dx_2}}
%%   %%   {\tuple{x_1,x_2}}{\tuple{y_1,y_2}}
%%   %% }
%%   %
%%   %% \infer{
%%   %%   \fa{i} \changesat{A_i}{\dx_i}{x_i}{y_i}
%%   %% }{\changesat{A_1 \x A_2}
%%   %%   {\tuple{\dx_1,\dx_2}}
%%   %%   {\tuple{x_1,x_2}}{\tuple{y_1,y_2}}
%%   %% }
%%   %
%%   \infer{
%%     \fa{i} \changesat{A_i}{\dx_i}{x_i}{y_i}
%%   }{\textstyle\changesat{\prod_i A_i}
%%     {\tuple{\vec\dx}}
%%     {\tuple{\vec x}}{\tuple{\vec y}}
%%   }
%% \end{align*}

%% %\noindent For sums, the change structure is disjoint:

%% \begin{align*}
%%   \D(A + B) &= \D A + \D B
%%   &
%%   \infer{
%%     \changesat{A_i}{\dx_i}{x_i}{y_i}
%%   }{
%%     \textstyle\changesat{A_1 + A_2}{\inj i \dx}{\inj i x}{\inj i y}
%%   }
%% \end{align*}

\noindent
Since we only consider increasing changes, and $\iso A$ is ordered discretely,
the only ``change'' permitted is to stay the same. Consequently, no information
is necessary to indicate what changed:

\begin{align*}
  \D(\iso A) &= \tunit
  &&
  \changesat{\iso A}{\tuple{}}{x}{x}
\end{align*}

\noindent
Finally we come to the most interesting case: functions.

\begin{align*}
  \D(A \to B) &= \iso A \to \D A \to \D B
  &
  \infer[FnChange]{
    \fa{\changesat A \dx x y}
    \changesat B {\df\<x\<\dx} {f\<x} {g\<y}
  }{
    \changesat{A \to B}{\df}{f}{g}
  }
\end{align*}

\noindent
Observe that a function change $\df : \iso A \to \D A \to \D B$ takes two
arguments: a base point $x$ and a change $\dx$.
%
To understand why we need both, consider incrementalizing function application:
we wish to know how $(f\<x)$ changes as both $f$ and $x$ change. So fix
$\changes{\df}{f}{g}$ and $\changes{\dx}{x}{y}$. How do we find the change $f\<x
\changesto g\<y$ that updates both function and argument? If changes were given
pointwise, taking only a base point, we'd stipulate that $\changes{\df}{f} g$
iff $\fa{x} \changes{\df\<x}{f\<x}{g\<x}$. But this only gets us to $g\<x$, not
$g\<y$: we've accounted for the change in the function, but not the argument.
%
We can account for both by giving $\df$ an additional parameter: not just the
base point $x$ but also the change $\dx$ to it. Then by inverting
\textsc{FnChange} we have $\changes{\df\<x\<\dx}{f\<x}{g\<y}$ as desired.

%% This makes it easy to incrementalize function application, $f\<x$; given
%% changes $\changes \df f g$ and $\changes \dx x y$ to the function and its
%% argument, we want to compute the change that takes us to the updated
%% application $g\<y$. By inverting \textsc{FnChange} we know that
%% $\changes{\df\<x\<\dx}{f\<x}{g\<y}$, so $\df\<x\<\dx$ gives us the desired
%% change.

%% If instead changes were given pointwise, letting $\D(A \to B)= \iso A \to \D B$,
%% then it'd be natural to let $\changes{\df}{f}{g} \iff \fa{x}
%% \changes{\df\<x}{f\<x}{g\<x}$.

Second, note the mixture of monotonicity and non-monotonicity in $\df : \iso A
\to \D A \to \D B$. Since our functions are monotone --- increasing inputs yield
increasing outputs --- function \emph{changes} are also monotone on input
changes $\D A$ --- a larger increase in the input yields a larger increase in
the output. However, there's no particular reason to expect the change in the
output to increase as the \emph{base point} increases --- hence the $\iso A$.

\todo{Explain:
  \begin{itemize}
  \item Function derivatives are zero-changes.

  \item Since we're tracking changes during fixed point iteration, and these
    increase monotonically, we only allow increasing changes. So $\iso A$
    (discussed in \cref{sec:datalog-and-datafun}) prevents things from changing!

  \item How $\name{step}'$ looks like a derivative of $\name{step}$ in the
  sense of \citet{incremental}.
  \end{itemize}

  Do \textbf{not} explain:
  \begin{itemize}
  \item what it does to contexts or variables
  \item derivatives of expressions
  \end{itemize}
}

%% Revisiting our running example, the crucial property of $\name{step}'$ now
%% becomes:

%% \[ \changesat{\tseteq{A}}{\name{step}' \<\dx}{\name{step}\<x}{\name{step}\<(x
%%   \cup \dx)} \]

%% or, more suggestively:

%% \[ \changesat{\tseteq A}{\dx}{x}{y} \implies
%% \changesat{\tseteq A}{\name{step}' \<\dx}{\name{step}\<x}{\name{step}\<y}
%% \]


\subsection{Examples of semi\naive\ Datafun programs}
\XXX

In order to explain in detail how our static transformation operates and why it
is correct, however, we first need a fuller account of Datafun's type system and
semantics.

\section{Types and Semantics}
\label{sec:typing-and-semantics}

%In this section, we give the typing and semantics of core Datafun.
\begin{figure}
  \begin{mathpar}
    \setlength\arraycolsep{.4em}
    \begin{array}{r@{\hskip 1em}ccl}
      \text{contexts} & \G &\bnfeq& \emptycx \bnfor \G, H \\
      \text{hypotheses} & H &\bnfeq& \h x A \bnfor \hd x A
    \end{array}

    \begin{array}{lcl}
      \stripcxraw{\emptycx} & = & \emptycx\\
      \stripcxraw{\G, \h x A} & = & \stripcxraw\G\\
      \stripcxraw{\G, \hd x A} & = & \stripcxraw\G, \hd x A
    \end{array}
    \\
    \infer[var]{\h x A \in \G}{\J x \G A}

    \infer[dvar]{\hd x A \in \G}{\J {\dvar x} \G A}

    \infer[lam]{\J e {\G,\h x A} B}{\J {\fnof x e} \G {A \to B}}

    \infer[app]{\J e \G {A \to B} \\ \J f \G A}{\J {e\<f} \G B}

    \infer[unit]{\quad}{\J {\etuple{}} \G \tunit}

    \infer[pair]{(\J{e_i}\G{A_i})_i}{\J{\etuple{e_1,e_2}} \G {A_1 \x A_2}}

    \infer[prj]{\J e \G {A_1 \x A_2}}{\J{\pi_i\<e}\G{A_i}}

    \infer[inj]{\J e \G A_i}{\J{\inj i e}\G{A_1 + A_2}}

    \infer[case]{\J e \G {A_1 + A_2} \\
      (\J {f_i} {\G,\h {x_i} {A_i}} {B})_i
    }{
      \J {\ecase{e} (\inj i {x_i} \caseto f_i)_i} \G B
    }

    \infer[box]{\J {\eisiso e} {\stripcx\G} A}{\J{\ebox e} \G {\iso A}}

    \infer[letbox]{\J e \G {\iso A} \\ \J f {\G,\hd x A} B}{
      \J {\elet{\ebox x = e} f} \G B}

    \infer[bot]{\quad}{\J\bot\G {\eqt L}}

    \infer[join]{(\J{e_i} \G {\eqt L})_i}{\J{e_1 \vee e_2}\G {\eqt L}}

    %% \infer{\J e \G {\eqt A}}{\J {\edown e} \G {\tdown {\eqt A}}}
    \infer[set]{(\J {\eisiso e_i} {\stripcx\G} {\eqt A})_i}{
      \J {\esetsub{e_i}{i}} \G {\tset{\eqt A}}}

    %% \infer{\J e \G {\tdown {\eqt A}} \\
    %%   \J f {\G,\h x {\eqt A}} L
    %% }{\J {\ebigvee x e f} \G L}
    %%
    \infer[for]{
      \J e \G {\tset A} \\
      \J f {\G,\hd x A} {\eqt L}
    }{\J {\efor x e f} \G {\eqt L}}

    %%\infer{\J e \G {\iso{(\eqt A \x \eqt A)}}}{\J{\prim{eq}\<e} \G {\tdown\tunit}}
    \infer[eq]{(\J {\eisiso e_i} {\stripcx\G} {\eqt A})_i}
          {\J {\eeq{e_1}{e_2}} \G \tbool}

    \infer[empty?]{\J {\eisiso e} {\stripcx\G} {\tset\tunit}}{
      \J {\eisempty e} \G {\tunit + \tunit}}

    \infer[split]{\J e \G {\iso{(A + B)}}}{\J{\esplit e} \G {\iso A + \iso B}}

    \infer[fix]{\J e \G {\iso{(\fixt L \to \fixt L)}}}{\J{\prim{fix}\< e} \G {\fixt L}}
  \end{mathpar}

  \caption{Datafun core syntax and typing rules}
  \label{fig:core-datafun}
\end{figure}


The syntax of core Datafun is given in \cref{fig:syntax} and its typing rules in
\cref{fig:core-datafun}. Contexts are lists of hypotheses $H$; a hypothesis
gives the type of either a monotone variable $\h x A$ or a discrete variable
$\hd x A$. The stripping operation $\stripcx\G$ drops all monotone hypotheses
from the context $\G$, leaving only the discrete ones.
%
The typing judgement $\J{e}{\G}{A}$ may be read as ``under hypotheses $\G$,
the term $e$ has type $A$''.

The \rn{var} and \rn{dvar} rules say that both monotone hypotheses $\h x A$ and
discrete hypotheses $\hd x A$ justify ascribing the variable $x$ the type $A$.
%
The \rn{lam} rule is the familiar rule for $\fn$-abstraction. However, note that
we introduce the argument variable $\h x A$ as a \emph{monotone} hypothesis, not
a discrete one. (This is the ``right'' choice because in \Poset\ the exponential
object is the poset of monotone functions.)
%
The application rule \rn{app} is standard, as are the rules \rn{unit},
\rn{pair}, \rn{prj}, \rn{inj}, and \rn{case}. As with \rn{lam}, the variables
$\h {x_i}{A_i}$ bound in the case branches $f_i$ are monotone.

\rn{box} says that $\ebox{e}$ has type $\iso A$ when $e$ has type $A$ in the
stripped context $\stripcx\G$. This restricts $e$ to refer only to discrete
variables, ensuring we don't smuggle any information we must treat monotonically
into a discretely-ordered $\iso$ expression. The elimination rule \rn{letbox}
for $(\eletbox x e f)$ allows us to ``cash in'' a boxed expression $e : \iso A$
by binding its result to a discrete variable $\hd x A$ in the body $f$.

At this point, our typing rules correspond to standard constructive S4 modal
logic~\cite{jrml}. We get to Datafun by adding a handful of domain-specific
types and operations.
%
First, \rn{split} provides an operator $\prim{split} : \iso(A + B) \to \iso A +
\iso B$ to distribute box across sum types.\footnote{An alternative syntax,
  pursued in \citet{datafun}, would be to give two rules for $\kw{case}$,
  depending on whether or not the scrutinee could be typechecked in a stripped
  context.}
%
The other direction, $\iso A + \iso B \to \iso (A + B)$, is already derivable,
as is the isomorphism $\iso A \times \iso B \cong \iso (A \times B)$.
%
This is used implicitly by box pattern-matching -- e.g., in the pattern $\pboxtuple{\inj 1 \dvar x, \inj 2 \dvar y}$, the variables $\dvar x$ and $\dvar y$ are both discrete, which is information we propagate via these conversions.
%
%% These conversions are used implicitly in pattern-matching to propagate
%% discreteness onto variables bound inside box patterns -- e.g, in the pattern
%% $\pboxtuple{\inj 1 \dvar x, \inj 2 \dvar y}$.
%
%% TODO: an example of desugaring pattern-matching here?
%
Semantically, all of these operations are the identity, as we
shall see shortly.


This leaves only the rules for manipulating sets and other semilattices.
\rn{bot} and \rn{join} tell us that $\bot$ and $\vee$ are valid at any
semilattice type $L$, that is, at sets and products of semilattice types.
%
The rule for set-elimination, \rn{for}, is \emph{almost} monadic bind.
%
However, we generalize it by allowing $\efor{x}{e}{f}$ to eliminate into any
semilattice type, not just sets, denoting a ``big semilattice join'' rather than
a ``big union''.
%
Finally, the introduction rule \rn{set} is says that $\esetsub{e_i}{i\in I}$ has
type $\tseteq A$ when each of the $\eisiso{e_i}$ has type $\eqt A$.
%
Just as in \rn{box}, each $\eisiso{e_i}$ has to typecheck in a stripped context;
constructing a set is a discrete operation, since $1 \le 2$ but $\esetraw{1}
\not\subseteq \esetraw{2}$.

Likewise discrete is equality comparison $\eeq{e_1}{e_2}$, whose rule \rn{eq} is
otherwise straightforward; and \rn{empty?}, which requires more explanation. The
idea is that $\eisempty e$ determines whether $e : \tset{\tunit}$ is empty,
returning $\inj 1 \etuple{}$ if it is, and $\inj 2 \etuple{}$ if it isn't. This
lets us turn ``booleans'' (sets of units) into values we can \kw{case}-analyse.
This is, however, not monotone, because while booleans are ordered $\efalse <
\etrue$, sum types are ordered disjointly; $\inj 1 ()$ and $\inj 2 ()$ are
simply incomparable.

Finally, the rule \rn{fix} for fixed points $\efix e$ takes a function $e : \isofixLtoL$ and yields an expression of type $\fixtLkern$.
%
The restriction to ``fixtypes'' ensures $\fixt L$ has no infinite ascending chains, guaranteeing the recursion will terminate.


\subsection{Semantics}\label{sec:semantics}

The syntax of core Datafun can be interpreted in $\Poset$, the category of
partially ordered sets and monotone maps. That is, an object of $\Poset$ is a
pair $(A, \leq_A)$ consisting of a set $A$ and a reflexive, transitive,
antisymmetric relation $\leq_A \subseteq A \times A$, while a morphism $f : A
\to B$ is a function such that $x \leq_A y \implies f(a) \leq_B f(b)$.

\subsubsection{Bicartesian Structure}

The bicartesian closed structure of $\Poset$ is largely the same as in $\Set$.
%
The product and sum sets are constructed the same way, and ordered pointwise:

\begin{align*}
  (a,b) \le_{A \x B} (a',b') &\iff a \le_A a \wedge b \le_B b'\\
  \inj i x \le_{A_1 + A_2} \inj j y &\iff i = j \wedge x \le_{A_i} y
\end{align*}

\noindent Projections $\pi_i$, injections $\injc_i$, tupling $\fork{f,g}$ and
case-analysis $\krof{f,g}$ are all the same as in \Set, pausing only to note
that all these operations preserve monotonicity, as we need.

The exponential $\expO A B$ consists of only the \emph{monotone} maps $f : A \to
B$, again ordered pointwise:

\[ f \le_{\expO A B} g \iff \fa{x \le_A y} f\<x \le_B g\<y \]

\noindent
Currying $\fn$ and evaluation are the same as in \Set. Supposing $f : A \x B \to
C$, then:

\begin{align*}
  \fn(f) &\isa A \to (\expO B C) &
  \eval_{A,B} &\isa (\expO A B) \x A \to B
  \\
  \fn(f) &= x \mapsto y \mapsto f(x,y) &
  \eval_{A,B} &= (g,x) \mapsto g(x)
\end{align*}

\noindent
Monotonicity here follows from the monotonicity of $f$ and $g$ and the pointwise ordering of $\expO A B$.


\subsubsection{The Discreteness Comonad}

Given a poset $(A, \leq_A)$ we define the discreteness comonad $\iso(A, \leq_A)$
as $(A, \leq_{\iso A})$, where \( a \leq_{\iso A} a' \iff a = a' \).
%
That is, the discrete order preserves the underlying elements, but reduces the
partial order to mere equality.
%
This forms a rather boring comonad whose functorial action $\iso(f)$, extraction $\varepsilon_A : \iso A \to A$, and duplication $\delta_A : \iso A \to \iso \iso A$ are all identities on the underlying sets:

\nopagebreak[2]
\begin{align*}
  \iso(f) &= f & \varepsilon_A &= a \mapsto a & \delta_A &= a \mapsto a
\end{align*}

\noindent
This makes the functor and comonad laws trivial. Monotonicity holds in each case because \emph{all} functions are monotone with respect to $\le_{\iso A}$.
%
It is also immediate that $\iso$ is monoidal with respect to \emph{both}
products and coproducts. That is, $\iso (A \times B) \cong \iso A \times \iso B$
and $\iso (A + B) \cong \iso A + \iso B$.
%
In both cases the isomorphism is witnessed by identity on the underlying
elements.
%
These lift to $n$-ary products and sums as well, which we write as $\isox : \prod_i \iso A_i \to \iso\prod_i A_i$ and $\isosum : \iso \sum_i
A_i \to \sum_i \iso A_i$.
%% %
%% We will write $\isox : \prod_i \iso A_i \to \iso\prod_i A_i$ to name the map
%% witnessing distributivity of products over $\iso$, and $\isosum : \iso \sum_i
%% A_i \to \sum_i \iso A_i$ to name the map witnessing distributivity of $\iso$
%% over coproducts.


\subsubsection{Sets and Semilattices}

Given a poset $(A, \leq_A)$ we define the finite powerset poset $\pfinof(A,
\leq_A)$ as $(P_{\mathrm{fin}}\, A, \subseteq)$, with finite subsets of $A$ as
elements, ordered by subset inclusion.
%
Note that the subset ordering completely ignores the element ordering $\leq_A$.
%
Finite sets admit a pair of useful morphisms:

\begin{align*}
  \morph{singleton} &\isa \iso A \to \pfinof A
  &
  \morph{isEmpty} &\isa \iso \pfinof A \to \termO + \termO
  \\
  \morph{singleton} &= a \mapsto \{a\}
  &
  \morph{isEmpty} &= X \mapsto 
  \begin{cases}
    \inj 1 () &\text{when }X = \emptyset\\
    \inj 2 () &\text{otherwise}
  \end{cases}
\end{align*}

\noindent
The \morph{singleton} function takes a value and makes a singleton set out of
it. The domain must be discrete, as otherwise the map will not be monotone (sets
are ordered by inclusion, and set membership relies on equality, not the partial
order). Similarly, the emptiness test \morph{isEmpty} also takes a discrete
set-valued argument, because otherwise the boolean test would not be monotone.

Sets also form a semilattice, with the least element given by the empty set, and
join given by union.
%
For this and other semilattices $L \in \Poset$, in particular products of
semilattices, we will write $\morph{join}^L_n : L^n \to L$ to denote the $n$-ary
semilattice join (least upper bound).
%
Also, if $f : A \times \iso B \to L$, we can define a morphism
$\pcollect{f} : A \times \pfinof{B} \to L$ as follows:

\begin{displaymath}
 \pcollect{f}  = (a, X) \mapsto \bigvee_{b \in X} f(a, b)
\end{displaymath}

\noindent
We will use this to interpret \kw{for}-loops. However, it is worth noting that
the discreteness restrictions on \morph{singleton} mean that finite sets do
not quite form a monad in $\Poset$.


\subsubsection{Equality} Every object $A \in \Poset$ admits an equality-test morphism \morph{eq}:

\begin{align*}
  \morph{eq} &\isa \iso A \x \iso A \to \pfinof{\termO}\\
  \morph{eq} &= (x,y) \mapsto 
  \begin{cases}
    \{()\} &\text{if }x = y\\
    \emptyset &\text{otherwise}
  \end{cases}
\end{align*}

\noindent
The domain must be discrete, since $x = y$ and $y \le z$ certainly doesn't imply $x = z$.


\subsubsection{Fixed Points}

Given a semilattice $L \in \Poset$ without infinite ascending chains, we can
define a fixed point operation $\prim{fix} : (L \to L) \to L$ as follows:

\begin{displaymath}
  \efix{} = f \mapsto \bigvee_{n \in \mathbb{N}} f^n(\bot)
\end{displaymath}

\noindent
A routine inductive argument shows this must yield a least fixed point.


%% ---- Semantics in a Datafun Model ----
\begin{figure*}
  \textsc{Type and Context Denotations}

  %% TODO: revert to this more readable version if space allows.
  %% \begin{align*}
  %%   \den{\tunit} &= \termO & \den{A \to B} &= \expO{\den{A}}{\den{B}}
  %%   \\
  %%   \den{\tseteq A} &= \pfinof{\den{\eqt A}}
  %%   & \den{A \x B} &= \den{A} \x \den{B}
  %%   \\
  %%   \den{\iso A} &= \iso{\den{A}} & \den{A + B} &= \den{A} + \den{B}
  %% \end{align*}

  \begin{align*}
    \den{\tunit} &= \termO
    & \den{\tseteq A} &= \pfinof{\den{\eqt A}}
    & \den{\iso A} &= \iso{\den{A}}
    \\
    \den{A \to B} &= \expO{\den{A}}{\den{B}}
    & \den{A \x B} &= \den{A} \x \den{B}
    & \den{A + B} &= \den{A} + \den{B}
  \end{align*}

  \begin{align*}
    \den{\G} &= \prod_{H \in \G} \den{H} &
    \den{\h x A} &= \den{A} & \den{\hd x A} &= \iso{\den{A}} &
    \den{\G \vdash A} &= \Poset(\den\G, \den A)
  \end{align*}
%  \vspace{0pt} % yes, this matters.

  \textsc{Term Denotations}

  \begin{displaymath}
    \def\arraystretch{1.15}
    \begin{array}{lcl}
      \den{\J {\dvar x} \G A} &=& \pi_{\dvar x} \then \varepsilon \qquad \text{(for $\hd x A \in \G$)} \\
      \den{\J x \G A} &=& \pi_x \qquad\quad\, \text{(for $\h x A \in \G$)} \\
      \den{\J {\fnof x e} \G {A \to B}} &=& \lambda\den{\J e {\G, \h x A} B} \\
      \den{\J {f\<e} \G B} &=& \fork{\den{\J f \G {A \to B}}, \den{\J e \G A}} \then \eval \\
      \den{\J {\etuple{e_1, e_2}} \G {A_1 \times A_2}} &=&
           \fork{\den{\J {e_1} \G {A_1}}, \den{\J {e_2} \G {A_2}}} \\
      \den{\J {\pi_i\<e} \G {A_i}} &=& \den{\J e \G {A_1 \times A_2}} \then \pi_i \\
      \den{\J {\ebox e} \G {\iso A}} &=& \mkbox_\Gamma(\den {\J e {\stripcx \G} A}) \\
      \den{\J {\elet{\ebox x = e} f} \G B} &=&  \fork{\id_\Gamma, \den{\J e \G {\iso A}}} \then \den{\J f {\G, \hd x A} B}  \\
      \den{\J \bot \G L} &=& \termI \then !_\Gamma \then \morph{join}^L_0 \\
      \den{\J {e \vee f} \G L} &=& \fork{\den{\J e \G L}, \den{\J f \G L}} \then \morph{join}^L_2 \\
      \den{\J {\eisempty e} \G {1+1}}&=& \mkbox_\Gamma(\den{\J e {\stripcx \G} {\tset A}}) \then \morph{isEmpty} \\
      \den{\J {\esplit e} \G {\iso A + \iso B}} &=& \den{\J e \G {\iso(A + B)}}\then \isosum \\

      \den{\eeq{e_1}{e_2}} &=&
          \fork{\mkbox_\Gamma(\den{\J {e_1} {\stripcx \G} A}),
                \mkbox_\Gamma(\den{\J {e_2} {\stripcx \G} A})}
          \then \morph{eq} \\
      \den{\J {\efix e} \G L} &=& \den{\J e \G {\iso(L \to L)}} \then \morph{fix} \\

      \den{\eset{e_i}_i} &=& \fork{\mkbox_\Gamma(\den{\J {e_1} {\stripcx \G} A}) \then \morph{singleton}}_i \then \morph{join}^L \\

      \den{\J {\efor x e f} \G L} &=&    \fork{\id,\den{\J e \G {\tset A}}} \then \pcollect{\den{\J f {\G, \hd x A} L}} \\
    \den{\J {\inj i e} \G {A_1 + A_2}} &=& \den{\J e \G {A_i}} \then \injc_i \\
    \den{\J {\ecase{e} (\inj i{x_i} \caseto f_i)_i} \G B} &=&
    \fork{\id, \den{\J e \G {A_1 + A_2}}} \then \morph{dist}^\x_+ \then
           \bigkrof{\den{\J {f_i} {\G, \h {x_i} {A_i}} B}}_i \\
    \end{array}
  \end{displaymath}
  \vspace{0pt} % yes, this matters

  \textsc{Auxilliary Operations}

  \begin{align*}
    \morph{dist}^\x_+ &~:~ A \x (B_1 + B_2) \to (A \x B_1) + (A \x B_2)
    &
    \mkbox_\Gamma &~:~ \Poset(\den{\stripcx \G}, A) \to \Poset(\den{\G}, \iso A) \\
    % this could be simpler if it distributed in the opposite direction.
    \morph{dist}^\x_+ &= \fork{\pi_2 \then \krof{\lambda (\fork{\pi_2,\pi_1} \then \injc_i)}_i, \pi_1}
    \then \eval
    &
    \mkbox_\Gamma(f) &= \fork{\pi_{\dvar x} \then \delta}_{\hd x A \in \G} \then \isox \then \iso(f)
  \end{align*}

  \caption{Semantics of Datafun}
  \label{fig:semantics}\label{def:strip}
\end{figure*}


\subsubsection{Interpretation} The semantic interpetation (defined
over typing derivations) is given in \cref{fig:semantics}.
%
The interpretation itself mostly follows the usual interpretation for
constructive S4~\cite{depaiva-s4}, with what novelty there is occuring in the
interpretation of sets and fixed points.
%
Even there, the semantics is straightforward, making fairly direct use of the
combinators defined above.
%
We give the interpretation in combinatory style, and to increase readability, we
freely use $n$-ary products to elide the book-keeping associated with
reassociating binary products.

Regarding notation, we write composition in diagrammatic or ``pipeline'' order with a simple centered dot, letting $f \cdot g : A \to C$ mean $f : A \to B$ followed by $g : B \to C$.
%
If $f_i : A \to B_i$ then we write $\fork{f_i}_i : A \to \prod_i B_i$ for the
``tupling map'' such that $\fork{f_i}_i \then \pi_j = f_j$.
%
In particular, $\fork{}$ is the map into the terminal object.
%
Dually, if $g_i : A_i \to B$ then we write $\krof{g_i}_i : \sum_i A_i
\to B$ for the ``case-analysis map'' such that $\injc_j \then \krof{g_i}_i =
g_j$.


\subsection{Metatheory}
\label{sec:weakening}

If we were presenting core Datafun in isolation, the usual thing to do would be
to prove the soundness of syntactic substitution, show that syntactic and
semantic substitution agree, and then establish the equational theory. However,
that is not our goal in this paper. We want to prove the correctness of the
semi\naive\ translation, which we will do with a logical relations argument.
Since we can harvest almost all the properties we need from the logical
relation, only a small residue of metatheory needs to be established manually --
indeed, the only thing we need to prove at this stage is the type-correctness of
weakening, which we will need to show the type-correctness of the
semi\naive\ transformation.

We define the weakening relation $\Gamma \sqsubseteq \Delta$ in
\cref{fig:weakening}. This says that $\Delta$ is a weakening of $\Gamma$, either
because it has extra hypotheses (\rn{drop}), or because a hypothesis in $\Gamma$
becomes discrete in $\Delta$ (\rn{disc}). The idea is that making a hypothesis
discrete only increases the number of places it can be used.

\begin{lemma}\label{thm:weaken}
  If\/ $\J e \G A$ and $\G \sqsubseteq \D$ then $\J e \D A$.
\end{lemma}

\noindent This follows by the usual induction on typing derivations.

%% ---- Semantics in a Datafun Model ----
\begin{figure*}
  \begin{mathpar}
    \infer[Empty]
        { }{ \emptycx \sqsubseteq \emptycx }
    \and
    \infer[Cons]
        { \Gamma \sqsubseteq \Delta}{ \Gamma, H \sqsubseteq \Delta, H}
    \and
    \infer[Drop]
        { \Gamma \sqsubseteq \Delta}{ \Gamma \sqsubseteq \Delta, H}
    \and
    \infer[Disc]
        { \Gamma \sqsubseteq \Delta}{ \Gamma, \h x A \sqsubseteq \Delta, \hd x A}
  \end{mathpar}
  \caption{Weakening relation}
  \label{fig:weakening}
\end{figure*}


\newcommand\boldphi{\texorpdfstring{$\boldsymbol\phi$}{\textphi}}
\newcommand\bolddelta{\texorpdfstring{$\boldsymbol\delta$}{\textdelta}}

\section{The \boldphi\ and \bolddelta\ Transformations}
\label{sec:transformations}

We use two static transformations, $\phi$ and $\delta$. Their definitions are
given in \cref{fig:phi,fig:delta} respectively, but rather than diving straight
in, we \XXX.

The speed-up transform $\phi e$ computes fixed points semi\naive{}ly by
replacing $\efix f$ by $\fastfix\<({f,f'})$.
%
But to find the derivative $f'$ of $f$ we'll need a second transform, called
$\delta e$.
%
Since a derivative is a zero-change, can $\delta e$ simply find a zero-change to
$e$?
%
Unfortunately, this is not strong enough.
%
For example, the derivative of $\fnof x e$ depends on how $e$ changes as its
free variable $x$ changes --- which is not necessarily a zero-change.
%
To compute derivatives (zero-changes), we need to solve the general problem of
computing \emph{changes}.
%
So, modelled on the incremental \fn-calculus' $\Deriv$ \citep{incremental},
$\delta e$ will compute how $\phi e$ changes as its free variables
change.%
%% \footnote{In order to make their mutual recursion work, $\delta e$
%%   incrementalizes $\phi e$, not $e$. We'll see why this is necessary later.
%%   \todo{insert forward ref.}}

However, to speed up $\efix e$ we don't want the change to $e$; we want its
derivative.
%
Since derivatives are zero-changes, function changes and derivatives coincide if
\emph{the function cannot change}.
%
This is why the typing rule for $\efix e$ (\todo{xref}) requires that $e : \iso(\fixt L
\to \fixt L)$: the use of $\iso$ prevents $e$ from changing!
%
So the key strategy of our speed-up transformation is to {\bfseries\boldmath
  decorate expressions of type ${\iso A}$ with their zero-changes.}
%
This makes derivatives available exactly where we need them: at \prim{fix}
expressions.


\subsection{Typing \boldphi\ and \bolddelta}

\begin{figure}\centering
  \begin{align*}
    \Phi\tunit &= \tunit
    &
    \D\tunit &= \tunit
    \\
    \Phi\tseteq A &= \tset{\Phi{\eqt A}}
    \quad\text{\small(see \cref{thm:phi-eqt})}
    &
    \D\tseteq A &= \tseteq A
    \\
    \Phi(\iso A) &= \iso{(\Phi A \x \DP A)}
    &
    \D(\iso A) &= \tunit
    \\
    \Phi(A \x B) &= \Phi A \x \Phi B
    &
    \D(A \x B) &= \D A \x \D B
    \\
    \Phi(A + B) &= \Phi A + \Phi B
    &
    \D(A + B) &= \D A + \D B
    \\
    \Phi(A \to B) &= \Phi A \to \Phi B
    &
    \D(A \to B) &= \iso A \to \D A \to \D B
  \end{align*}

  \caption{$\D$ and $\Phi$ type transformations}
  \label{fig:DeltaPhi}
\end{figure}


In order to decorate expressions with extra information, $\phi$ also needs to
decorate their types. In \cref{fig:DeltaPhi} we give a type translation $\Phi A$
capturing this.
%
In particular, if $e : \iso A$ then $\phi e$ will have type $\Phi(\iso A) =
\iso(\Phi A \x \DP A)$.
%
The idea is that evaluating $\phi e$ will produce a pair
$\eboxraw{\etuple{x,\dx}}$ where $x : \Phi A$ is the sped-up result and $\dx :
\DP A$ is a zero-change to $x$.
%
Thus, if $e : \iso(\fixt L \to \fixt L)$, then $\phi e$ will compute
$\eboxraw{\etuple{f,f'}}$, where $f'$ is the derivative of $f$.

On types other than $\iso A$, there is no information we need to add, so $\Phi$
simply distributes.
%
In particular, source programs and sped-up programs agree on the shape of
first-order data:

\begin{lemma}\label{thm:phi-eqt}
  $\Phi\eqt A = \eqt A$.
\end{lemma}
\begin{proof}
  Induct on $\eqt A$.
\end{proof}

For reasons that we will discuss in \todo{insert fwd ref}, $\phi$ and $\delta$
are mutually recursive. Thus $\delta e$ finds the change to $\phi e$ rather than
$e$. So if $e : A$ then $\phi e : \Phi A$ and $\delta e : \DP A$.
%
However, so far we have neglected to say what $\phi$ and $\delta$ do to typing
contexts.
%
To understand this, it's helpful to look at what $\Phi$ and $\DP$ do to
functions and to $\iso$.
%
This is because expressions denote functions of their free variables.
%
Moreover, in Datafun free variables come in two flavors, monotone and discrete, and discrete variables are semantically $\iso$-ed.

If we view expressions as functions of their free variables, $\delta e$ will
denote the \emph{derivative} of the function $\phi e$ denotes.
%
And just as the derivative of a unary function $f\<x$ has \emph{two} arguments,
$\df\<x\<\dx$, the derivative of an expression $e$ with $n$ variables $x_1,
\dots, x_n$ will have $2n$ variables: the original $x_1, \dots, x_n$ and their
changes $\dx_1, \dots, \dx_n$.%
%
\footnote{We assume throughout the paper as a matter of notational convenience
  that source programs contain no variables starting with the letter \emph{d}.}
%
However, this says nothing yet about monotonicity or discreteness.
%
To make this precise, we'll use three context transformations, named according
to the analogous type operators, $\iso$, $\Phi$, $\Delta$:

\begin{align*}
  \iso{(\h x A)} &= \hd x A & \iso{(\hd x A)} &= \hd x A
  \\
  \Phi(\h x A) &= \h x {\Phi A} & \Phi(\hd x A) &= \hd x {\Phi A}, \hd \dx {\DP A}
  \\
  \D(\h x A) &= \h \dx {\D A} & \D(\hd x A) &= \emptycx\quad\text{(the empty context)}
\end{align*}

\noindent
Intuitively, $\iso\G$, $\Phi\G$, and $\D\G$ mirror the effect of
$\iso$, $\Phi$, and $\D$ on the semantics of $\G$:

\nopagebreak[2]
%% This is a hack to make up for not having a \multirow that works inside align*
%% & co without producing warnings.
\begin{center}
  \(\displaystyle \den{\iso\G} = \iso\den\G \)
  %
  \hfil
  %
  \(\displaystyle\begin{aligned}
    \den{\Phi(\h x A)} &\cong \den{\Phi A}
    \\
    \den{\Phi(\hd x A)} &\cong \den{\Phi \iso A}
  \end{aligned}\)
  %
  \hfil
  %
  %% TODO: needs more vertical space?
  \(\displaystyle\begin{aligned}
    \den{\D(\h x A)} &\cong \den{\D A}
    \\
    \den{\D(\hd x A)} &\cong \den{\D \iso A}
  \end{aligned}\)
\end{center}

%% \nopagebreak[2]
%% \begin{align*}
%%   \multirow{2}{*}{\den{\iso \G} = \iso\den\G}
%%   &&
%%   \den{\D(\h x A)} &\cong \den{\D A}
%%   &
%%   \den{\Phi(\h x A)} &\cong \den{\Phi A}
%%   \\
%%   &&
%%   \den{\D(\hd x A)} &\cong \den{\D \iso A}
%%   &
%%   \den{\Phi(\hd x A)} &\cong \den{\Phi \iso A}
%% \end{align*}

\noindent
These defined, we can state the sense in which $\phi$ and $\delta$ are
type-correct:

\begin{theorem}[Type-correctness]
  \label{thm:type-correct}
  If $\J e \G A$, then
  %% \[ \J {\phi e} {\Phi\G} {\Phi A}
  %% \quad\text{and}\quad
  %% \J {\delta e} {\iso{\Phi\G}, \DP\G} {\DP A}
  %% \]
  \begin{align*}
    \Jalign {\phi e} {\Phi\G} {\Phi A}\\
    \Jalign {\delta e} {\iso{\Phi\G}, \DP\G} {\DP A}
  \end{align*}
\end{theorem}

\begin{proof}
  By induction on typing derivations; see appendix. \XXX
\end{proof}

\noindent To get the hang of these context and type transformations, suppose $\J
e {\h x A, \hd y B} C$. Then \cref{thm:type-correct} tells us:

\nopagebreak[2]
\begin{align*}
  \Jalign {\phi e} {\h x {\Phi A}, \hd y {\Phi B}, \hd \dy {\DP B}} {\Phi C}
  \\
  \Jalign {\delta e} {\hd x {\Phi A}, \h\dx{\DP A}, \hd y {\Phi B}, \hd \dy {\DP B}} {\DP C}
\end{align*}

\todo{Ideas not yet covered:
  \begin{itemize}
  \item Give typing rules \& semantics for \fastfix. Semantics does not depend on
    \fastfix\ being given a derivative; it's well-defined regardless.


  \item Explain how uses of $\phi$ in $\delta e$ involve weakening, and how this
    justifies their use in discrete contexts (hilighted in
    {\color{Rhodamine}pink}), eg. in $\delta(e\<f) = \delta e \<\eboxraw{\phi f}
    \<\delta f$.

  \item explain implementation of \zero{} via \dummy{}.
\end{itemize}}


%% ---- "Go faster" term translation, phi ----
\begin{figure}\centering
  \begin{align*}
    \phi x &= x & \phi \dvar x &= \dvar x\\
    \phi(\fnof x e) &= \fnof x \phi e & \phi(e\<f) &= \phi e\<\phi f\\
    \phi\etuple{e_i}_i &= \etuple{\phi e_i}_i &
    \phi(\pi_i\<e) &= \pi_i\<\phi e\\
    \phi(\inj i e) &= \inj i \phi e
    &
    \phi(\ecase e (\inj i x \caseto f_i)_i)
    &= \ecase{\phi e} (\inj i x \caseto \phi f_i)_i
    \\
    \phi\bot &= \bot &
    \phi(e \vee f) &= \phi e \vee \phi f\\
    \phi(\esetsub{e_i}{i}) &= \esetsub{\phi e_i}{i}
    &
    %% replaced substitution by let-binding
    \phi(\efor x e f) &= \efor x {\phi e}
        %{\substd{\phi f}{\dvar\dx \substo \zero\<\dvar x}}
        {\eletbox{\dx}{\ebox{\zero\<\dvar x}} \phi f}
    \\
    \phi\ebox{e} &= \eboxtuple{\phi e, \delta e}
    &
    \phi(\eletbox x e f)
    &= \elet{\eboxtuple{\dvar x,\dvar\dx} = \phi e} \phi f
    \\
    \phi(\eeq e f) &= (\eeq {\phi e} {\phi f})
    &
    \phi(\eisempty e) &= \eisempty {\phi e}
    \\
    \phi(\efix e) &= \fastfix\<\phi e
    &
    %% split
    \phi(\esplit e) &= \ecase{\phi e}
    \\
    &&&\phantom{{}={}}\
    \left(\pboxtuple{\inj i \dvar x, \inj i \dvar \dx}
    \caseto \inj i {\eboxtuple{\dvar x,\dvar\dx}}\right)_{i}
    \\
    &&&\phantom{{}={}}\
    \left(\pboxtuple{\inj i \dvar x, \inj j \pwild}
    \caseto \inj i {\eboxtuple{\dvar x, \dummy\<\dvar x}} \right)_{i\ne j}
  \end{align*}

  \caption{Semi\naive{} speed-up translation, $\phi$}
  \label{fig:phi}
\end{figure}


%% ---- "Derivative" term translation, delta
\begin{figure}\centering
  \[ \delta\bot = \delta\esetsub{e_i}{i} = \delta(\eeq e f) = \delta(\efix e) = \bot \]
  %
  \begin{align*}
    \delta x &= \dx &
    \delta \dvar x &= \dvar\dx\\
    \delta(\fnof{x} e) &= \fnof{\pboxvar x} \fnof\dx \delta e
    & \delta(e\<f) &= \delta e \<\ebox{\phi e} \<\delta f\\
    \delta\etuple{e_i}_i &= \etuple{\delta e_i}_i
    & \delta(\pi_i\<e) &= \pi_i\<\delta e\\
    \delta(\inj i e) &= \inj i {\delta e} &
    \delta(e \vee f) &= \delta e \vee \delta f\\
    \delta\ebox{e} &= \etuple{} &
    \delta(\eletbox x e f)
    &= \elet{\pboxtuple{\dvar x,\dvar\dx} = \phi e} \delta f
    \\
    \delta(\eisempty e) &= \eisempty {\phi e}
    &
    \delta(\esplit e) &= \ecase{\phi e}
    (\pboxtuple{\inj i \pwild, \pwild}
    \caseto \inj i {\etuple{}} )_i
  \end{align*}
  %
  \begin{align*}
    \delta(\ecase e (\inj i x \caseto f_i)_i)
    &= \ecase{\esplit{\ebox{\phi e}},\, \delta e}\\
    &\qquad ({\inj i {\pboxvar x},\, \inj i \dx} \caseto \delta f_i)_{i}\\
    &\qquad ({\inj i {\pboxvar x},\, \inj j \pwild}
    %\caseto \subst{\delta f_i}{\dx \substo \dummy\<\dvar x})_{i\ne j}
    \caseto \elet{\dx = \dummy\<\dvar x} \delta f_i)_{i\ne j}
    \\
    \delta(\efor x e f)
    &= (\efor x {\delta e}
    %\substd{\phi f}{\dvar\dx \substo \zero\<\dvar x}) \\
    \eletbox \dx {\zero\<\dvar x} \phi f) \\
    &\vee (\efor x {\phi e \vee \delta e}
    %\substd{\delta f}{\dvar\dx \substo \zero\<\dvar x})
    \eletbox{\dx}{\zero\<\dvar x} \delta f)
  \end{align*}

  \caption{Semi\naive{} derivative translation, $\delta$}
  \label{fig:delta}
\end{figure}


\section{Proving the Semi\naive\ Transformation Correct}

We have given a program transformation which computes a change on a
term $e$ given a change in its free variables, by giving a function
$\delta(e)$ computing the change in $e$, and another function
$\phi(e)$ using the change to optimize $e$. To state the correctness
of $\delta()$ and $\phi()$, we need to show that $\phi(e)$ preserves
the meaning of $e$, and that
$\delta(e)$ correctly sends $e$ with one set of bindings for its free
variables to $e$ with the changed bindings.
%
Since Datafun supports higher-order functions, we cannot directly
prove that the semantics is preserved. Instead, we will need to
formalize the intended relationship of $\phi e$ and $\delta e$ using a
logical relation, and then use that relation to prove an
\emph{adequacy theorem} saying that the semantics is preserved for
first-order types.

So inductively on types $A$, letting $a,b \in \den{A}$,
$x,y \in \den{\Phi A}$, and $\dx \in \den{\DP A}$, we define a five
place relation $\weirdat{A}{\dx}{x}{a}{y}{b}$, which may be glossed as
``$x,y$ speed up $a,b$ respectively, and $\dx$ changes $x$ into $y$''.
The definition is given in \cref{fig:logical-relation}. Units,
products and sums essentially mimic the definition of change
structures we give in \cref{sec:change-structures}. At set types
$\tset{A}$, a change is a set of values to add to the
starting set. It is also the case that values and their
sped-up versions are equal, because we engineered the definition of
$\Phi(A)$ to be the identity on equality types.

Since the box type $\Box A$ represents values which do not change, the
only valid change is the unit value, which takes a value and a sped-up
value to the very same value and sped-up value. However, the sped-up
version of a value is taken to be a \emph{pair} of a value and a
change, and the change is a zero-change for that value at type $A$.
This ensures that at a boxed function type, we will always have a
derivative (i.e. a zero-change) available.
%
Speaking of functions, the relation at the type $A \to B$ works very
much the way one would expect from the incremental lambda calculus.
$\df$ is a valid change from $f_\phi$ to $g_\phi$ (with $f$ and $g$ as the
respective slow versions), when, given any
change $dx$ sending $x$ to $y$ (with $a$ and $b$ as respective sped-up
versions), we have that $\df\,x\dx$ is a valid change sending
$f_\phi\,x$ to $g_\phi\,y$ (with $f\,a$ and $g\,b$ as the respective
slow versions).

The logical relation is defined on closed terms, and so
before we can state the fundamental theorem, we have to extend
the relation to contexts $\G$ and substitutions, letting
$\rho,\rho' \in \den{\G}$, $\g,\g' \in \den{\Phi\G}$, and
$\dgamma \in \den{\DP\G}$:

\nopagebreak[1]
\begin{align*}
  \weirdat{\G}{\dgamma}{\g}{\rho}{\g'}{\rho'}
  &\iff \fa{\h x A \in \G} \weirdat{A}{\dgamma_{\dx}}{\g_x}{\rho_x}{\g'_x}{\rho'_x}
  \\
  &\hphantom{{}\iff{}} \hspace*{-13.2pt} \wedge \fa{\hd x A \in \G}
  \weirdat{\iso A}
          {\etuple{}}
          {(\g_{\dvar \dx}, \g_{\dvar x})}
          {\rho_{\dvar x}}
          {(\g'_{\dvar\dx}, \g'_{\dvar x})}
          {\rho'_{\dvar x}}
\end{align*}
\noindent
With that in place, we can state the fundamental theorem, showing that
$\phi$ and $\delta$ generate expressions which satisfy this logical
relation:

\begin{theorem}[Fundamental Property]
  If $\J e \G A$ and $\weirdat{\G}{\dgamma}{\g}{\rho}{\g'}{\rho'}$ then
  \[\weirdat{A}{\den{\delta e} \<\tuple{\g,\dgamma}}{\den{\phi
      e}\<\g}{\den{e}\<\rho}{\den{\phi e}\<\g'}{\den{e}\<\rho'}\]
\end{theorem}


\begin{figure}
\begin{align*}
  \weirdat{\tunit}{\tuple{}}{\tuple{}}{\tuple{}}{\tuple{}}{\tuple{}}
  &\iff \top
  \\
  \weirdat{\tseteq A}{\dx}{x}{a}{y}{b}
  &\iff (x,y,x \cup \dx) = (a,b,y)
  \\
  \weirdat{\iso A}{\tuple{}}{(x,\dx)}{a}{(y,\dy)}{b}
  &\iff (a,x,\dx) = (b,y,\dy) \wedge \weirdat{A}{\dx}{x}{a}{y}{b}
  \\
  \weirdat{A_1 \x A_2}{\vec{\dx}}{\vec x}{\vec a}{\vec y}{\vec b}
  &\iff \fa{i} \weirdat{A_i}{\dx_i}{x_i}{a_i}{y_i}{b_i}
  \\
  \weirdat{A_1 + A_2}{\inj i \dx}{\inj j x}{\inj k a}{\inj l y}{\inj m b}
  &\iff i = j = k = l = m \wedge \weirdat{A_i}{\dx}{x}{a}{y}{b}
  \\
  \weirdat{A \to B}{\df}{f_\phi}{f}{g_\phi}{g}
  &\iff
  \fa{\weirdat{A}{\dx}{x}{a}{y}{b}}\\
  &\hphantom{{}\iff{}}
  \weirdat{B}{\df\<x\<\dx}{f_\phi\<x}{f\<a}{g_\phi\<y}{g_\phi\<b}
\end{align*}
  \caption{Definition of the logical relation}
  \label{fig:logical-relation}
\end{figure}

This theorem follows by a structural induction on typing derivations
as usual, but a number of lemmas need to be proved in order to
establish the fundamental theorem.

First, we prove a pair of lemmas about the behaviour of the logical
relation at equality types. Both of these follow by induction on the
structure of equality types.

\begin{lemma}[Equality Changes]
For all equality types $\eqt A$, we have that $\weirdat{\eqt A}{\dx}{x}{a}{y}{b}$ implies $x = a$ and $y = b$.
\end{lemma}

\begin{lemma}[Dummy Changes]
  For all equality types $\eqt A$, we have that $\weirdat{\eqt A}{\dummy(x)}{x}{x}{x}{x}$.
\end{lemma}

%% FIXME: broken reference
The equality changes lemma shows that at equality types, the sped-up
version of a value is the value itself. We then show a relational
analogue of Lemma~\ref{lem:dummy-change}, that the $\dummy$ function
calculates a zero change for all equality types. This justifies the
use of $\zero$ to notate uses of $\dummy$ in the $\phi$ and $\delta$
translations.

The lemma involves the lattice types, showing that a change for a
lattice type $L$ is something that can be joined on to it:

\begin{lemma}[Semilattice Changes]
  At lattice types $L$, we have that $\Delta L = L$ and $\weirdat{L}{\dx}{x}{a}{y}{b}$ if and only if $x = a$ and $y = b = x \vee_L \dx$
\end{lemma}
\begin{proof}
  This follows by induction on the structure of $L$. The first
  property is a consequence of the fact that all lattice types are
  equality types. The first two properties of the bi-implication are a
  corollary of the equality changes lemma, and the third property
  follows from the fact that it is true by definition for set types,
  and that products work pointwise.
\end{proof}

This lemma is used in the proofs of the fundamental theorem in all the
cases involving lattice types -- we use it in the cases for the least
element $\bot$, the join $e \vee e'$, for-comprehensions
$\efor x e f$, and the fixed point operator $\prim{fix}\< e$.

Similarly, we also need to prove theorems about context manipulations
to establish the soundness of the rules for the box type. First, we
show that all valid changes for a discrete context (i.e., a context
with only discrete variables) send terms to themselves.

\begin{lemma}[Discrete Contexts Don't Change]
  If $\weirdat{\stripcx{\G}}{\tuple{}}{\gamma}{\rho}{\gamma'}{\rho'}$ then $\gamma = \gamma'$ and $\rho = \rho'$.
\end{lemma}

We state this lemma by quantifying over all contexts and then
stripping because we use it in the proof of the next lemma, which
says that for any valid context change, its effect on the discrete
part is a no-op:

\todo{Define strip}

\begin{lemma}[Context Stripping]
  If $\weirdat{\G}{\dg}{\gamma}{\rho}{\gamma'}{\rho'}$
  then $\weirdat{\stripcx{\G}}{\tuple{}}{\strip(\g)}{\strip(\rho)}{\strip(\g')}{\strip(\rho')}$.
\end{lemma}

This lemma is used in all the cases of the fundamental theorem
involving discrete expressions -- equality $\eeq{e_1}{e_2}$, set
constructors like $\esetsub{e_i}{i}$, emptiness tests $\eisempty e$,
and box introduction $\ebox e$.



Once the fundamental theorem has been established, we can specialize
it to closed terms and equality types, Then, the equality changes
lemma implies adequacy -- that first-order closed programs compute the
same result when $\phi$-translated:

\begin{theorem}[Adequacy of the Translation]
  If $\J e {\emptycx} {\eqt A}$ then $\den{e} = \den{\phi e}$.
\end{theorem}

%% \nopagebreak[2]
%% \vspace{-\baselineskip}
%% \begin{mathpar}
%%   \weirdat{\tunit}{\tuple{}}{\tuple{}}{\tuple{}}{\tuple{}}{\tuple{}}

%%   \weirdat{\tseteq A}{\dx}{x}{x}{x \cup \dx}{x \cup \dx}

%%   \infer{\weirdat{A}{\dx}{x}{a}{x}{a}}{
%%     \weirdat{\iso A}{\tuple{}}{(x,\dx)}{a}{(x,\dx)}{a}}

%%   \infer{\fa{i} \weirdat{A_i}{\dx_i}{x_i}{a_i}{y_i}{b_i}}{
%%     \weirdat{A_1 \x A_2}{\vec{\dx}}{\vec x}{\vec a}{\vec y}{\vec b}}

%%   \infer{\weirdat{A_i}{\dx}{x}{a}{y}{b}}{
%%     \weirdat{A_1 + A_2}{\inj i \dx}{\inj i x}{\inj i a}{\inj i y}{\inj i b}}

%%   \infer{\fa{\weirdat{A}{\dx}{x}{a}{y}{b}}
%%     \weirdat{B}{\df\<x\<\dx}{f\<x}{f_s\<a}{g\<y}{g_s\<b}
%%   }{
%%     \weirdat{A \to B}{\df}{f}{f_s}{g}{g_s}}
%% \end{mathpar}

%% TODO: rename section if we add more examples.
\section{Applying the Semi\naive\ Transformation to Transitive Closure}
\label{sec:seminaive-examples}

%% TODO: update name if the name of tc changes.
Let's try applying the semi\naive\ transform to a simple Datafun program: the
transitive closure function \name{tc} from
\cref{sec:generic-transitive-closure}:

%\newcommand\var\mathrm
%\renewcommand\isocolor\relax
%\renewcommand\dvar\mathvar
\newcommand\yone{\dvar y_{\isocolor 1}}
\newcommand\ytwo{\dvar y_{\isocolor 2}}

\[\begin{array}{l}
\name{tc} \< \eboxvar{e}
= \efixis{p} \dvar e \cup (\dvar e \bullet p)
\\
s \bullet t =
\eforraw{\etuple{\dvar x, \yone} \in s}
\eforraw{\etuple{\ytwo , \dvar z} \in t}
\ewhen{\eeqraw{\yone}{\ytwo}} \esetraw{\etuple{\dvar x, \dvar z}}
\end{array}
\]

In the process we'll discover that besides $\phi$ itself we need a few simple
optimisations to actually speed up our program: most importantly, we need to
propagate $\bot$ expressions.
%
In our experience, performing $\phi$ and $\delta$ by hand is easiest when you
work inside-out. At the core of transitive closure is a relation composition,
$(\dvar e \bullet p)$, and at the core of relation composition is a
\kw{when}-expression. Let's take a look at its $\phi$ and $\delta$ translations:

\begin{align*}
  \phi(\ewhen {\eeqraw \yone \ytwo} \esetraw{\etuple{\dvar x, \dvar z}})
  &= \phi(\eforraw {\etuple{} \in \eeqraw \yone \ytwo} \esetraw{\etuple{\dvar x,
      \dvar z}})
  && \text{desugaring}
  \\
  &= \eforraw{\etuple{} \in \eeqraw \yone \ytwo}
  \phi{\esetraw{\etuple{\dvar x, \dvar z}}}
  && \text{omitting a needless \kw{let}-binding}\\
  &= \ewhen{\eeqraw \yone \ytwo} \esetraw{\etuple{\dvar x, \dvar z}}
  && \text{resugaring}
\end{align*}

\noindent
Frequently, as in this case, $\phi$ does nothing interesting. For brevity we'll
skip such no-op translations.

\begin{align*}
  &\mathrel{\hphantom{=}}
  \delta(\ewhen {\eeqraw \yone \ytwo} \esetraw{\etuple{\dvar x, \dvar z}})
  \\
  &= \delta(\eforraw{\etuple{} \in \yone = \ytwo} \esetraw{\etuple{\dvar x, \dvar z}})
  && \text{desugaring \kw{when}}
  \\
  &= \eforraw{\etuple{} \in \delta(\yone = \ytwo)}
  \phi\esetraw{\etuple{\dvar x, \dvar z}}
  && \text{omitting needless \kw{let}-bindings}
  \\
  &\cup
  \eforraw{\etuple{} \in \phi(\yone = \ytwo) \cup \delta(\yone = \ytwo)}
  \delta\esetraw{\etuple{\dvar x, \dvar z}}
  \\
  &= \eforraw{\etuple{} \in \bot} \esetraw{\etuple{\dvar x, \dvar z}}
  \cup \eforraw{\etuple{} \in \phi(\yone = \ytwo) \cup \bot} \bot
  && \text{rules for $\phi(\eeq e f)$ and $\delta\eset{e_i}_i$}
  \\
  &= \bot && \text{propagating }\bot
\end{align*}

The core insight here is that $\yone = \ytwo$ can't change, and neither can
$\esetraw{\etuple{\dvar x,\dvar z}}$. By propagating this information --- for
example, rewriting $(\eforraw{x \in \bot} e)$ to $\bot$ --- we can simplify our
derivative down to nothing.
%
Now let's pull out and examine $\eforraw{\etuple{\ytwo, \dvar z} \in t}
\ewhen{\yone = \ytwo} \esetraw{\etuple{\dvar x, \dvar z}}$. The $\phi$
translation is again a no-op.

\begin{align*}
  &\mathrel{\hphantom{=}}
  \delta(\eforraw{\etuple{\ytwo, \dvar z} \in t}
  \ewhen{\yone = \ytwo} \esetraw{\etuple{\dvar x, \dvar z}})
  \\
  &= \eforraw{\etuple{\ytwo, \dvar z} \in \dt}
  \phi(\ewhen{\yone = \ytwo} \esetraw{\etuple{\dvar x, \dvar z}})
  && \text{omitting needless \kw{let}-bindings}
  \\
  &\cup \eforraw{\etuple{\ytwo, \dvar z} \in t \cup \dt}
  \delta(\ewhen{\yone = \ytwo} \esetraw{\etuple{\dvar x, \dvar z}})
  \\
  &= \eforraw{\etuple{\ytwo, \dvar z} \in \dt}
  \ewhen{\yone = \ytwo} \esetraw{\etuple{\dvar x, \dvar z}}
  && \text{propagating }\bot
\end{align*}

\noindent Tackling the outermost \kw{for} loop:

\begin{align*}
  &\mathrel{\hphantom{=}}
  \delta(
  \eforraw{\etuple{\dvar x, \yone} \in s}
  \eforraw{\etuple{\ytwo, \dvar z} \in t}
  \ewhen{\yone = \ytwo} \esetraw{\etuple{\dvar x, \dvar z}})
  \\
  &= \eforraw{\etuple{\dvar x, \yone} \in \ds}
  \phi(\eforraw{\etuple{\ytwo, \dvar z} \in t}
  \ewhen{\yone = \ytwo} \esetraw{\etuple{\dvar x, \dvar z}})
  && \text{definition of $\delta(\kw{for} \dots)$}
  \\
  &\cup \eforraw{\etuple{\dvar x, \yone} \in s \cup \ds}
  \delta(\eforraw{\etuple{\ytwo, \dvar z} \in t}
  \ewhen{\yone = \ytwo} \esetraw{\etuple{\dvar x, \dvar z}})
  \\
  &= \eforraw{\etuple{\dvar x, \yone} \in \ds}
  \eforraw{\etuple{\ytwo, \dvar z} \in t}
  \ewhen{\yone = \ytwo} \esetraw{\etuple{\dvar x, \dvar z}}
  && \text{applying previous work}
  \\
  &\cup
  \eforraw{\etuple{\dvar x, \yone} \in s \cup \ds}
  \eforraw{\etuple{\ytwo, \dvar z} \in \dt}
  \ewhen{\yone = \ytwo} \esetraw{\etuple{\dvar x, \dvar z}}
  \\
  &= (\ds \bullet t) \cup ((s \cup \ds) \bullet \dt)
  && \text{rewriting in terms of ${\bullet}$}
\end{align*}

\noindent
This, then, is the derivative $\delta(s \bullet t)$ of relation composition.
With a bit of rewriting, this is equivalent to $(\ds \bullet t) \cup (s \bullet
\dt) \cup (\ds \bullet \dt)$, which is perhaps the derivative a human would
give.

Let's use this to figure out $\phi(\name{tc}\<\eboxvar{e})$. Working from the
inside out, we start with the derivative of the loop body, $\delta(\dvar e \cup
(\dvar e \bullet p))$:

\begin{align*}
  \delta({\dvar e \cup (\dvar e \bullet p)})
  &= \delta\dvar e \cup \delta(\dvar e \bullet p)\\
  &= \delta\dvar e
  \cup (\delta\dvar e \bullet p)
  \cup ((\dvar e \cup \delta\dvar e) \bullet \deep)
  \\
  &= \bot \cup (\bot \bullet p) \cup ((\dvar e \cup \bot) \bullet \deep)
  && \dvar e ~\text{is discrete, and cannot change}
  \\
  &= \dvar e \bullet \deep
  && \text{propagate}~\bot
\end{align*}

\noindent
Observe that this requires a new optimization: by definition, $\delta\dvar e =
\dvar{de}$. However, since $\dvar e$ is discrete we know it's not changing, and
since it's of set type, $\dvar{de}$ may as well be the empty set. So we replace
$\delta e$ with $\bot$ instead.
%
Finally, putting everything together:

\begin{align*}
  \phi(\efixis p \dvar e \cup (\dvar e \bullet p)
  &= \phi(\efix \eboxraw{\fnof{p} \dvar e \cup (\dvar e \bullet p)})
  && \text{desugaring}
  \\
  &= \fastfix\< \phi\eboxraw{\fnof{p} \dvar e \cup (\dvar e \bullet p)}
  \\
  &= \fastfix\<\eboxraw{\etuple{\phi({\fnof{p} \dvar e \cup (\dvar e \bullet p)}),\
  \delta({\fnof{p} \dvar e \cup (\dvar e \bullet p)})}}
  \\
  &= \fastfix\<\eboxraw{\etuple{
      ({\fnof{p} \dvar e \cup (\dvar e \bullet p)}),\
      ({\fnof{\eboxvar p} \fnof{\deep} \dvar e \bullet \deep})}}
  && \text{previous work}
\end{align*}

Examining the recurrence produced by this use of \fastfix, we recover exactly
the semi\naive\ transitive closure algorithm we gave in
\cref{sec:seminaive-tc-in-datafun}:

\begin{align*}
  x_0 &= \bot & x_{i+1} &= x_i \cup \dx_i\\
  \dx_0 &= ({\fnof{p} \dvar e \cup (\dvar e \bullet p)}) \<\bot
  = \dvar e
  &
  \dx_{i+1} &=
  ({\fnof{\eboxvar p} \fnof{\dx} \dvar e \bullet \deep})
  \<\eboxraw{x_i} \<\dx_i
  = \dvar e \bullet \dx_i
\end{align*}

\section{Implementation and Optimization}
\label{sec:implementation}

\todo{Things we need to make derivative of regex go fast, probably:
  \begin{enumerate}
  \item Inlining function definitions.
  \item Applying a zero-change to a zero-change produces a zero-change. This was
    mentioned in \citet{incremental}.
  \end{enumerate}}

\section{Discussion and related work}
\label{sec:related-work}

\label{sec:differences-from-incremental}

\paragraph{Nested fixed points}\label{sec:nested-fixed-points}
%
The typing rule for $\efix e$ requires $e : \iso(\fixt L \to \fixt L)$.
%
The $\phi$ translation takes advantage of this $\iso$, decorating expressions of
type $\iso A$ with their zero changes.
%
However, it also prevents an otherwise valid idiom: in a nested fixed-point
expression $\efixis{x}{\dots (\efixis{y}{e}) \dots}$, the inner fixed point body $e$
cannot use the monotone variable $x$!
%
This restriction is not present in \citet{datafun}; its addition brings Datafun
closer to Datalog, whose syntax cannot express this sort of nested fixed point.

We suspect it is possible to lift this restriction without losing
semi\naive\ evaluation, by decorating \emph{all} expressions and variables (not
just discrete ones) with zero changes.
%
However, this also invalidates $\delta(\efix f) = \bot$: now that $f$ can
change, so can $\efix f$.
% Argh, using \delta here is technically wrong. because \delta should be
% incrementalizing \phi. this is more like \Deriv. But if we use \Deriv people
% will be confused, and this is a very technical point.
Luckily, there is a simple and correct solution: $\delta(\efix f) =
\efix \ebox{\delta f \<\ebox{\efix f}}$~\cite{delta-fix}.
%
However, to compute this new fixed point semi\naive{}ly, we need a \emph{second
  derivative}: the zero change to $\delta f \<\ebox{\efix f}$. Indeed, for a
program with fixed points nested $n$ deep, we need $n$\textsuperscript{th}
derivatives. We leave this to future work.

%% Can't we just have \delta produce two expressions: the derivative, and the
%% zero change to the derivative?


\paragraph{Self-maintainability}

In the incremental $\fn$-calculus, a function $f$ is called
\emph{self-maintainable} if its derivative $f'$ depends only upon the change
$\dx$ to the argument and not upon the base point $x$. This is a crucial
property, because it means we can compute the change in the function's result
without re-computing the original input, which might be expensive. So it's
reasonable to ask whether self-maintainability (or rather, its absence) is ever
an issue for semi\naive\ Datafun. We suspect (but have no proof) that because of
the limited way semi\naive\ evaluation uses incremental computation, it usually
isn't. For example, consider a variant definition of transitive closure, as the
fixed point of \(f = \fnof{\name{path}} \name{edge} \cup (\name{path} \relcomp
\name{path})\). This is not self-maintainable; its derivative is:

\begin{code}
  f' \<\name{path} \<\name{dpath} =
  (\name{path} \relcomp \name{dpath})
  \vee (\name{dpath} \relcomp \name{path})
  \vee (\name{dpath} \relcomp \name{dpath})
\end{code}

\noindent
However, this is not a problem when computing its fixed point semi\naive{}ly,
because both \name{path} and \name{dpath} are available from the previous
iteration. Thus non-self-maintainable fixed points do not appear to be forced
into doing extensive recomputation.



\paragraph{Related work}

The incremental lambda calculus was introduced by \citet{incremental},
as a static program transformation which associated a type of
\emph{changes} to each base type, along with operations to update a
value based on a change.  Then, a program transformation on the
simply-typed lambda calculus with base types and functions was
defined, which rewrote lambda terms into incremental functions which
propagated changes as needed to reduce recomputation. The fundamental
idea of the incremental function type taking two arguments (a base
point and a change) is one we have built on, though we have extended
the transformation to support many more types like sums, sets,
modalities, and fixed points.
%
Subsequently, \citet{DBLP:conf/esop/GiarrussoRS19} extended this work
to support the \emph{untyped} lambda calculus, additionally also
extending the incremental transform to support additional
\emph{caching}. In this work, the overall correctness of change
propagation was proven using a step-indexed logical relation, which
defined which changes were valid in a fashion very similar to our own.

The motivating example of this line of work was to optimize bulk collection
operations. However, all of the intuitions were phrased in terms of calculus ---
a change structure can be thought of as a space paired with its tangent space, a
zero change on functions is a derivative, and so on. However, the idea of a
derivative as a linear approximation is taken most seriously in the work on the
differential lambda calculus~\cite{dlc}. These calculi have the beautiful
property that the \emph{syntactic} linearity in the lambda calculus corresponds
to the \emph{semantic} notion of linear transformation.

Unfortunately, the intuition of a derivative has its limits. A function's
derivative is \emph{unique}, a property which models of differential lambda
calculi have gone to considerable length to
enforce~\cite{differential-categories}. This is problematic from the point of
view of semi\naive\ evaluation, since we make use of the freedom to
overapproximate.
%
In \cref{sec:semilattice-delta-phi}, we followed common practice from Datalog
and took the derivative $\delta(e \vee f)$ to be $\delta(e) \vee \delta(f)$,
which may overapproximate the change to $e \vee f$.
%
This spares us from having to do certain recomputations to construct set
differences; it is not clear to what extent semi\naive\ evaluation's practical
utility depends on this approximation.

\citet{DBLP:conf/esop/Alvarez-Picallo19} offer an alternative
formulation of change structures, by requiring changes to form a
monoid, and representing the change itself with a monoid action. They
use change actions to prove the correctness of semi\naive\ evaluation
for Datalog, and express the hope that it could apply to Datafun.
Unfortunately, it does not seem to -- the natural notion of function
change in their setting is pointwise, which does not seem to lead to
the derivatives we want in the examples we considered.

Overall, there seems to be a lot of freedom in the design space for
incremental calculi, and the tradeoffs different choices are making
remain unclear. Much further investigation is warranted!


% \paragraph{Semantics of Seminaive Evaluation}

% \paragraph{Flix and Higher-Order Datalog}




  

%% ---------- Acknowledgments ----------
%% \begin{acks}                            %% acks environment is optional
%% % TODO: acknowledge
%% % - Paolo Giarrusso, for discussion on incremental lambda calc.

%% % - Michael Peyton-Jones and possibly other Semmle folk, for discussion of
%% %   seminaive evaluation.

%% % - Max S New for help with the question "how do I prove my semantics are
%% %   naturally isomorphic without inducting on typing derivations"?

%% % - Achim Jung, for pointing out that to prevent nesting fixed points we'd need
%% %   a modal type system, which eventually led to the realization about
%% %   reinterpreting \iso.

%%   %%                                       %% contents suppressed with 'anonymous'
%%   %% %% Commands \grantsponsor{<sponsorID>}{<name>}{<url>} and
%%   %% %% \grantnum[<url>]{<sponsorID>}{<number>} should be used to
%%   %% %% acknowledge financial support and will be used by metadata
%%   %% %% extraction tools.
%%   %% This material is based upon work supported by the
%%   %% \grantsponsor{GS100000001}{National Science
%%   %%   Foundation}{http://dx.doi.org/10.13039/100000001} under Grant
%%   %% No.~\grantnum{GS100000001}{nnnnnnn} and Grant
%%   %% No.~\grantnum{GS100000001}{mmmmmmm}.  Any opinions, findings, and
%%   %% conclusions or recommendations expressed in this material are those
%%   %% of the author and do not necessarily reflect the views of the
%%   %% National Science Foundation.
%% \end{acks}


%% ---------- Bibliography ----------
\FloatBarrier
\bibliography{seminaive-datafun}


%% %% ---------- Appendix ----------
%% \appendix
%% \section{Appendix}

%% Text of appendix \ldots

\end{document}

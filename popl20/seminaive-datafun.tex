% From https://tex.stackexchange.com/questions/39415/unload-a-latex-package
\makeatletter
\newcommand{\dontusepackage}[2][]{%
  \@namedef{ver@#2.sty}{9999/12/31}%
  \@namedef{opt@#2.sty}{#1}}
\makeatother

\providecommand\mathvar\mathit
\PassOptionsToPackage{dvipsnames}{xcolor}

%% Avoid using acmart's fonts.
\dontusepackage[tt=false,type1=true,]{libertine}
\dontusepackage[libertine]{newtxmath}

%% Scale inconsolata to match XCharter[scaled=.94].
%% Might want to scale it down further.
\PassOptionsToPackage{scaled=0.9893675916139859}{zi4}


%% For review. Maximizes space available.
\documentclass[acmsmall,review,anonymous,]{acmart}\settopmatter{printfolios=true,printccs=false,printacmref=false}
%% For final camera-ready submission, w/ required CCS and ACM Reference
%\documentclass[acmsmall]{acmart}\settopmatter{}

%% Journal information
%% Supplied to authors by publisher for camera-ready submission;
%% use defaults for review submission.
\acmJournal{PACMPL}
\acmVolume{1}
\acmNumber{CONF} % CONF = POPL or ICFP or OOPSLA
\acmArticle{1}
\acmYear{2018}
\acmMonth{1}
\acmDOI{} % \acmDOI{10.1145/nnnnnnn.nnnnnnn}
\startPage{1}

%% Copyright information
%% Supplied to authors (based on authors' rights management selection;
%% see authors.acm.org) by publisher for camera-ready submission;
%% use 'none' for review submission.
\setcopyright{none}
%\setcopyright{acmcopyright}
%\setcopyright{acmlicensed}
%\setcopyright{rightsretained}
%\copyrightyear{2018}           %% If different from \acmYear

%% Bibliography style
\bibliographystyle{ACM-Reference-Format}
%% Citation style
%% Note: author/year citations are required for papers published as an
%% issue of PACMPL.
\citestyle{acmauthoryear}   %% For author/year citations


\usepackage{style}
\usepackage{notation}

\begin{document}

%% Title information: \title[Short Title]{Full Title}
%% Short Title is optional; when present, will be used in header instead of Full Title.
\title{Semi\naive{} Evaluation for a Higher-Order Functional Language}
%\title{Semi\naive{} Evaluation for Datafun}
% Optional: \subtitle{}, \titlenote{}, \subtitlenote{}

%% Author information
%% Contents and number of authors suppressed with 'anonymous'.
%% Each author should be introduced by \author, followed by
%% \authornote (optional), \orcid (optional), \affiliation, and
%% \email.
%% An author may have multiple affiliations and/or emails; repeat the
%% appropriate command.
%% Many elements are not rendered, but should be provided for metadata
%% extraction tools.

\author{Michael Arntzenius}
\affiliation{
  %\position{Position1}
  \department{School of Computer Science}
  \institution{University of Birmingham}
  %\streetaddress{Street1 Address1}
  \city{Birmingham}
  %\state{State1}
  \postcode{B15 2TT}
  \country{United Kingdom}
}
\email{daekharel@gmail.com}

\author{Neelakantan R. Krishnaswami}
\affiliation{
  %\position{Position2a}
  \department{Department of Computer Science and Technology} %% \department is recommended
  \institution{University of Cambridge}           %% \institution is required
  %\streetaddress{Street2a Address2a}
  \city{Cambridge}
  %\state{State2a}
  \postcode{CB2 1TN}
  \country{United Kingdom}
}
\email{first2.last2@inst2a.com FIXME FIXME}         %% \email is recommended


%% ---------- Abstract ----------
%% Note: \begin{abstract}...\end{abstract} environment must come
%% before \maketitle command
\begin{abstract}
\todo{write abstract.}
\end{abstract}


%% ---------- Classification and keywords nonsense ----------
%% 2012 ACM Computing Classification System (CSS) concepts
%% TODO
%% Generate at 'http://dl.acm.org/ccs/ccs.cfm'.
\begin{CCSXML}
<ccs2012>
<concept>
<concept_id>10011007.10011006.10011008</concept_id>
<concept_desc>Software and its engineering~General programming languages</concept_desc>
<concept_significance>500</concept_significance>
</concept>
<concept>
<concept_id>10003456.10003457.10003521.10003525</concept_id>
<concept_desc>Social and professional topics~History of programming languages</concept_desc>
<concept_significance>300</concept_significance>
</concept>
</ccs2012>
\end{CCSXML}

\ccsdesc[500]{Software and its engineering~General programming languages}
\ccsdesc[300]{Social and professional topics~History of programming languages}
%% End of generated code

% comma separated list of keywords. TODO.
\keywords{keyword1, keyword2, keyword3}  %% \keywords are mandatory in final camera-ready submission


%% ---------- Sections ----------
%% Note: \maketitle command must come after title commands, author
%% commands, abstract environment, Computing Classification System
%% environment and commands, and keywords command.
\maketitle

\section{Introduction}
\label{sec:intro}

Datalog~\cite{datalog}, along with the $\pi$-calculus and \fn-calculus, is one
of the jewel languages of theoretical computer science, connecting programming
language theory, database theory, and complexity theory. In terms of programming
languages, Datalog can be understood as a fully declarative subset of Prolog
which is guaranteed to terminate and so can be evaluated in both top-down and
bottom-up fashion. In terms of database theory, it is equivalent to the
extension of relational algebra with a fixed point operator. In terms of
complexity theory, stratified Datalog over ordered databases characterizes
polytime computation~\cite{datalog-polytime}.

In addition to its theoretical elegance, over the past twenty years
Datalog has seen a surprisingly wide array of uses across a variety of
practical domains, both in research and in industry.
%
Whaley and Lam \cite{whaley-lam,whaley-phd} implemented pointer analysis
algorithms in Datalog, and found that they could reduce their analyses from
thousands of lines of C code to \emph{tens} of lines of Datalog code, while
retaining competitive performance. The DOOP pointer analysis
framework~\cite{doop}, built using the Souffl\'{e} Datalog
engine~\cite{souffle}, shows that this approach can handle almost all of
industrial languages like Java, even analysing features like
reflection~\cite{doop-java-reflection}. Semmle has developed the Datalog-based
.QL language~\cite{semmlecode,ql-inference} for analysing source code (which was
used to analyze the code for NASA's Curiosity Mars rover), and LogicBlox has
developed the LogiQL~\cite{logicblox} language for business analytics and retail
prediction. The Boom project at Berkeley has developed the Bloom language for
distributed programming~\cite{bloom}, and the Datomic cloud
database~\cite{datomic} uses Datalog (embedded in Clojure) as its query
language. Microsoft's SecPAL language~\cite{secpal} uses Datalog as the
foundation of its decentralised authorization specification language. In each
case, when the problem formulated in Datalog, the specification became directly
implementable, while remaining dramatically shorter and clearer than
alternatives implemented in more conventional languages.

% TODO: cite Yannis's recent SNAPL paper ``Next-Paradigm Programming
% Languages: What Will They Look Like and What Changes Will They
% Bring?''

However, there are two flies in the ointment. First, even though each
of these applications is supported by the skeleton of Datalog, they
all had to extend it significantly beyond the theoretical core
calculus.  For example, core Datalog does not even support arithmetic,
since its semantics only speaks of finite sets supporting equality of
their elements. Moreover, arithmetic is not a trivial extension, since
it can greatly complicates the semantics (for example, proving that
termination continues to hold). So despite the fact that kernel
Datalog has a very clean semantics, its metatheory seemingly needs to
be laboriously re-established once again for each extension.

A natural approach to solving this problem is to find a language in
which to write the extensions, which preserves the semantic guarantees
that Datalog offers. Two such proposals are the Flix
language~\cite{flix} and the Datafun language~\cite{datafun}. 
Conveniently for our exposition, these two languages embody two
alternative design strategies.

%% FIXME: this is not up-to-date with Flix's current design. I think
%% it's even not up-to-date with Flix's old design. Old Flix wasn't a
%% functional program building a Datalog program, it's the reverse: a
%% Datalog program that can run functional code in certain places to
%% eg. compute lattice aggregations.
%
%% New Flix has both functional code generating Datalog code, and
%% Datalog code being able to run functional code in certain places.
%% So it seems like it should be _very_ close to Datafun in
%% expressivity. I think Madsen has either a draft or a
%% soon-to-be-published paper on this new design.
%
%% Differences between Flix & Datafun as I see them:
%%
%% - Flix doesn't have monotonicity types.
%
%% - Instead Flix integrates with SMT solvers for lightweight
%%   verification of properties including monotonicity,
%%   distributivity, soundness, completeness, etc. However, mostly
%%   only works for 1st-order stuff.
%
%% - Flix makes it easy to extend language with new semilattice types:
%%   you just write the code for the join operation.
%
%% - In Datafun, Datalog is just a mode of use of functional language.
%%   Flix separates the relational/Datalog and functional
%%   sublanguages.
%
%% - In Flix, the semantics of passing around Datalog code in the
%%   functional language are unclear; talking with Madsen, he seemed
%%   unsure what design choice was best.
%
%% But a summary this^ long belongs in related work, not here.

Flix adopts the route of treating Datalog as an embedded
domain-specific language~\cite{edsl}. That is, Flix is a fairly
conventional (albeit well-designed) functional programming language
roughly comparable to ML or Haskell, extended with types representing
Datalog predicates and programs. The evaluation of a Flix program
builds a Datalog program, which is then handed off to a custom Datalog
engine (albeit extended to support arbitrary semilattices).  This
stratification means that (a) standard Datalog implementation
techniques can be used mostly off-the-shelf, but that (b) its
functional programming side is mostly a very powerful macro system for
Datalog.  The only way Flix code runs at Datalog execution time is via
the definition of new semilattices (which is functional Flix code
implementing a semilattice interface), and for this Flix offers
program-verification style correctness checking (including SMT-based
tooling).

Like Flix, Datafun is a functional programming language, but unlike
Flix, its type discipline supports tracking \emph{monotonicity} of
functions. Datalog-style recursively defined relations are given via
an explicit fixed point operator; monotonicity ensures uniqueness of
this fixed point, playing a role similar to Datalog's stratification
condition.
%
Tracking monotonicity permits a much tighter integration between the
functional and logic programming styles, but it comes at a cost: many
of Datalog's standard implementation techniques, developed in the
context of a first-order logic language, are not obviously applicable
in a higher-order functional setting.

Second, actually making Datalog perform well enough to use in practice
calls for very sophisticated program analysis and compiler
engineering. (This is a familiar experience from the database
community, where query planners must encode a startling amount of
knowledge to optimize relatively simple SQL queries.) A wide variety
of techniques for optimizing Datalog programs have been studied in the
literature, such as using binary decision diagrams to represent
relations~\cite{whaley-phd}, exploiting the structure of well-behaved
subsets (e.g., CFL-reachability problems correspond to the ``chain
program'' fragment of Datalog~\cite{chain-programs}), and combining
top-down and bottom-up evaluation via the ``magic sets''
algorithm~\cite{magic-sets}.

Today, one of the workhorse techniques for implementing bottom-up
Datalog engines is \emph{semi\naive\ evaluation}~\cite{seminaive}.
This optimization improves the performance of Datalog's most
distinctive feature: recursively defined predicates. These can be
understood as the fixed point of a set-valued function $f$. The
\naive\ way to compute this is to iterate the sequence $\emptyset,
f(\emptyset), f^2(\emptyset), \dots$ until $f^i(\emptyset) =
f^{i+1}(\emptyset)$. However, each iteration will recompute all
previous values. Semi\naive\ evaluation instead computes a safe
approximation of the \emph{difference} between iterations. This
optimization is critical, as it can asymptotically improve the
performance of Datalog queries.

\paragraph{Contributions} The semi\naive\ evaluation algorithm is
defined partly as a program transformation on sets of Datalog rules,
and partly as a modification of the fixed point computation algorithm.
The central contribution of this paper is to give an extended version
of this transformation which works on higher-order programs written
in the Datafun language. 

\begin{itemize}
\item We reformulate Datafun in terms of a kernel calculus based on
  the modal logic S4. Instead of giving a calculus with distinct
  monotonic and discrete function types, as in the original Datafun
  paper, we make discreteness into a comonad. In addition to
  regularizing the calculus and slightly improving its expressiveness,
  the explicit comonadic structure lets us impose a modal constraint
  on recursion reminiscent of Hoffman's work on safe
  recursion~\cite{hofmann-safe-recursion}. This brings the semantics
  of Datafun more closely in line with Datalog's, and substantially
  simplifies the program transformation we present.
  
\item We define a program transformation to \emph{incrementalize}
  well-typed Datafun programs. The translation is a compositional
  type-and-syntax-directed transformation, and works uniformly at all
  types including function types. We build on the \emph{change
    structure} approach to static program incrementalization
  introduced by \citet{incremental}, extending it to support sum
  types, set types, comonads, and (well-founded) fixed points.

\item We establish the correctness of our transformation using a novel
  logical relation which simultaneously defines the changes connecting
  old and updated programs, as well as the optimized version of both.
  The fundamental lemma shows that our transformation is
  semantics-preserving: any closed program of first-order type has the
  same meaning after optimization.
  
\item We then discuss our implementation of a compiler from Datafun to
  Haskell, in both \naive\ and semi\naive\ form. This lets us
  empirically demonstrate the asymptotic speedups predicted by the
  theory. We additionally discuss the (surprisingly modest) set of
  program optimizations we found helpful for putting the optimization
  into practice.
\end{itemize}

\section{Datalog and Datafun, informally}
\label{sec:datalog-and-datafun}

\subsection{Datalog}
\label{sec:datalog}

Datalog's syntax is a subset of Prolog's. Programs are collections of predicate
declarations:

\newcommand\datum[1]{\textsf{#1}}

\begin{code}
  \name{parent}(\datum{aerys}, \datum{rhaegar})\\
  \name{parent}(\datum{rhaegar}, \datum{jon})\\
  \name{parent}(\datum{lyanna}, \datum{jon})
  \\[6pt]
  \name{ancestor}(X,Z) \gets \name{parent}(X,Z)\\
  \name{ancestor}(X,Z) \gets \name{parent}(X,Y) \wedge \name{ancestor}(Y,Z)
\end{code}

\noindent
This defines two binary relations, \name{parent} and \name{ancestor}. Lowercase
sans-serif words like \datum{aerys} and \datum{rhaegar} are symbols \`a la
Lisp, and uppercase characters like $X,Y,Z$ are variables.
%
The \name{parent} relation is defined as a set of ground facts: we assert that
\datum{aerys} is \datum{rhaegar}'s parent, that \datum{rhaegar} is \datum{jon}'s
parent, and so on. The \name{ancestor} relation is defined by a pair of rules:
first, that $X$ is $Z$'s ancestor if $X$ is $Z$'s parent; second, that $X$ is
$Z$'s ancestor if $X$ has a child $Y$ who is an ancestor of $Z$.

Semantically, a predicate denotes the set of tuples that satisfy it. Compared to
Prolog, one of the key restrictions Datalog imposes is that these sets are
always \emph{finite}. This helps keep proof search decidable, allowing for a
variety of implementation strategies. In practice, most Datalog engines use
bottom-up evaluation instead of Prolog's top-down backtracking search.

Recursive definitions like \name{ancestor} give rise to the set of facts
deducible from the rules defining them. More formally, we can view these rules
as defining a relation transformer and producing its least fixed point. For this
to make sense, these rules must be \emph{stratified}: a recursive definition
cannot refer to itself beneath a negation. For example, the liar paradox is
prohibited:

\begin{code}
  \quad\name{liar}() \gets \neg \name{liar}()
  \qquad\color{Red}\text{\scshape\ding{55}\: not valid datalog}
\end{code}

\noindent
Stratification ensures the transformer the rules define is monotone,
guaranteeing a unique least fixed point.


\subsection{Datafun}

\begin{figure}
  \[\setlength\arraycolsep{.4em}
  \begin{array}{r@{\hskip 1em}ccl}
    \text{types} & A,B &\bnfeq& \iso A \bnfor \tset{\eqt A}
    \bnfor \tunit \bnfor A \x B \bnfor A + B \bnfor A \to B
    \\[.5ex]
    \text{eqtypes} & \eqt A, \eqt B &\bnfeq&
    \tset{\eqt A} \bnfor
    \tunit \bnfor \eqt A \x \eqt B \bnfor \eqt A + \eqt B
    \\[.5ex]
    \text{semilattices} & L,M &\bnfeq& \tset{\eqt A} \bnfor \tunit \bnfor L \x M
    \\[.5ex]
    \text{fixtypes} & \fixt L, \fixt M &\bnfeq&
    \tset{\color{red}\fint A} \bnfor \tunit \bnfor \fixt L \x \fixt M
    \\[.5em]
    \text{terms} & e,f,g &\bnfeq& x \bnfor \dvar x \bnfor \fnof{x} e
    \bnfor e\<f \bnfor \etuple{} \bnfor \etuple{e,f} \bnfor \pi_i\<e
    \\&&&
    \inj i e \bnfor \ecase{e} (\inj i x_i \caseto f_i)_{i\in\{1,2\}}
    \\&&&
    \ebox e \bnfor \eletbox x e f
    \\&&&
    \bot \bnfor e \vee f \bnfor \eset{e_i}_i \bnfor \efor x e f\\
    &&& \eeq e f \bnfor \eisempty e \bnfor \esplit e \bnfor \efix e
    \\[1ex]
    %% \multicolumn{2}{r}{\textsf{\scshape Surface syntax}}
    %% \\[1ex]
    \text{surface types} & A,B &\bnfeq& ... \bnfor \tbool
    \\[.5ex]
    \text{surface terms} & e,f,g &\bnfeq&
    ... \bnfor \elet{x = e} f \bnfor \efixis x e\\
    &&& \efalse \bnfor \etrue \bnfor \eif e f g \bnfor \ewhen e f
  \end{array}
  \]

  \todo{What about pattern matching?}

  \caption{Datafun syntax}
  \label{fig:syntax}
\end{figure}


The idea behind Datafun is to capture the essence of Datalog in a typed,
higher-order, functional setting.
%
Since the key restriction that makes Datalog tractable --- stratification ---
requires tracking \emph{monotonicity}, we locate Datafun's semantics in the
category \Poset\ of partial orders and monotone maps.
%
Since \Poset\ is bicartesian closed, it can interpret the simply typed
\fn-calculus, giving us a notation for writing monotone and higher-order
functions.
%
This lets us \emph{abstract} over Datalog rules, something not
possible in Datalog itself!
%
In the remainder of this section we reconstruct Datafun hewing closely
to this semantic intuition.

Datafun begins as the simply-typed \fn-calculus with functions ($\fnof x e$ and
$e\<f$), sums ($\inj i e$ and $\ecase{e}{\dots}$), and products ($\etuple{e,f}$
and $\pi_i\<e$).
%
To represent relations, we add a type of finite sets\footnote{To implement set
  types, their elements must support decidable equality. In our core calculus,
  we use a subgrammar of ``eqtypes'', and in our implementation (which compiles
  to Haskell) we use typeclass constraints to pick out such types.} $\tset{\eqt
  A}$, introduced with set literals $\eset{e_0, \ldots e_n}$, and eliminated
using Moggi's monadic bind syntax, $\efor{x}{e_1}{e_2}$, signifying the union
over all $\dvar x \in e_1$ of $e_2$.
%
Since we are working in \Poset, each type comes with a partial order on it; sets
are ordered by inclusion, $x \le y : \tseteq{A} \iff x \subseteq y$.

As long as all primitives are monotone, every definable function is also
monotone. This is necessary for defining fixed points, but may seem too
restrictive. There are many useful non-monotone operations, such as equality
tests $\eeq e f$. For example, $\esetraw{} = \esetraw{}$ is true, but if the
first argument increases to $\esetraw{1}$ it becomes false, a \emph{decrease}
(as we'll see later, in Datafun, $\efalse < \etrue$).

How can we express non-monotone operations if all functions are monotone?
%
We square this circle by introducing the \emph{discreteness} type constructor,
$\iso A$.
%
The elements of $\iso A$ are the same as those of $A$, but the partial order on
$\iso A$ is discrete, $x \le y : \iso A \iff x = y$.
%
Monotonicity of a function $\iso A \to B$ is vacuous: $x = y$ implies $f(x) \le
f(y)$ by reflexivity!
%
In this way we represent ordinary, possibly non-monotone, functions $A \to B$ as
monotone functions $\iso A \to B$.

Semantically, $\iso$ is a monoidal comonad or necessity modality, and so we base
our syntax on \citet{jrml}'s syntax for the necessity fragment of constructive
S4 modal logic.
%
This involves distinguishing two kinds of variable: discrete variables $\dvar x$
are italic and {\isocolor\isocolorname}, while monotone variables $x$ are
upright and black.
%
Discrete variables may be used wherever they're in scope, but crucially,
monotone variables are hidden within non-monotone expressions.
%
For example, in an equality test $\eeq e f$, the terms $e$ and $f$ cannot refer
to monotone variables bound outside the equality expression.
%
We highlight such expressions with a
\adjustbox{bgcolor=isobg}{\isobgname\ background}.
%
Putting this all together, we construct the type $\iso A$ with the non-monotone
introduction form $\ebox{e}$ and eliminate it by pattern-matching, $\eletbox x e
f$, giving access to a discrete variable $\dvar{x}$.

%% TODO: should we note that we really definitely mean □(L → L) and
%% not (□L → L)? Reader may be confused.
Finally, Datafun includes fixed points, $\efix{e}$. The $\prim{fix}$ combinator
takes a function $\iso (\fixt L \to \fixt L)$ and returns its least fixed point.
Besides monotonicity of the function, we impose two restrictions on the fixed
point operator to ensure well-definedness and termination. First, we require
that recursion occur at \emph{semilattice types with no infinite ascending
  chains}, $\fixt L$. A join-semilattice is a partial order with a least element
$\bot$ and a least upper bound operation $\vee$ (``join''). Finite sets (with
the empty set as least element, and union as join) are an example, as are tuples
of semilattices. As long as the semilattice has no infinite ascending chains,
recursion from the bottom element is guaranteed to find the least fixed point.

Second, we require that the recursive function be boxed, $\iso(\fixt L \to \fixt
L)$. Since boxed expressions can only refer to discrete values, and fixed point
functions themselves must be monotone, this has the effect of preventing
semantically nested fixed points. We discuss this in more detail in
\cref{sec:nested-fixed-points}. Note that this does not prevent mutual
recursion, which can be expressed by taking a fixed point at product type, nor
stratified fixed points \`a la Datalog.


\section{Datafun by example}

\begin{figure}
  \begin{displaymath}
    \begin{array}{cc}
    \begin{array}{ccc}
      \kw{bool} & \triangleq & \tset 1 \\
      \efalse   & \triangleq & \eset{} \\
      \etrue    & \triangleq & \eset{\etuple{}} \\
      \ewhen{b}{e} & \triangleq & \efor{ \etuple }{b}{e} \\
    \end{array}
    \qquad &
    \begin{array}{ccc}
      \setfor{e}{\cdot} & \triangleq & \eset{e} \\
      \setfor{e}{p \in e', \ldots} & \triangleq & \efor{p}{e'} {\setfor{e}{\ldots}} \\
      \setfor{e}{b, \ldots} & \triangleq & \ewhen{b} {\setfor{e}{\ldots}} \\
    \end{array}
    \end{array}
\end{displaymath}
  \caption{Syntactic Sugar}
  \label{fig:sugar}
\end{figure}

For the sake of brevity and clarity, the examples that follow make use of some
syntax sugar:

\begin{enumerate}

\item We mentioned earlier that Datafun's boolean type $\tbool$ is ordered
  $\efalse < \etrue$. This is because we encode booleans as sets of empty
  tuples, $\tset 1$, with $\efalse$ being the empty set $\esetraw{}$ and
  $\etrue$ being the singleton $\esetraw{\etuple{}}$. At semilattice type we
  also permit a ``one-sided'' conditional test, $\ewhen{b}{e}$, which yields $e$
  if $b$ is \etrue\ and $\bot$ otherwise. Encoding booleans as sets has the
  advantage that $\ewhen{b}{e}$ is monotone in the condition $b$.

\item We make use of set comprehensions, which can be desugared into the monadic
  operators \kw{for} and \kw{when} in the usual
  way~\cite{wadler-monad-comprehensions}.

\item It is convenient to treat $\prim{fix}$ as a binding form, $\efixis x e$,
  rather than supplying a function, $\efix\ebox{\fnof x e}$.

\item Finally, we make free use of curried functions and pattern matching.
  Desugaring these is relatively standard, and so we will say little about it,
  with one exception: the box-elimination form $\eletbox x e e'$ is a pattern
  matching form, and so we allow it to occur inside of patterns. The effect of a
  box pattern $\pbox{\isocolor p}$ is to ensure that all of the variables bound
  in the pattern $\isocolor p$ are treated as discrete variables.

\end{enumerate}

\noindent
We summarize (except for pattern matching) the desugaring rules we use in
\cref{fig:sugar}.


\subsection{Set operations}

%\newcommand\isin{\mathbin{\in?}}
\newcommand\mem{\name{member}}
\newcommand\isa{~:~}
\newcommand\relcomp\bullet
\newcommand\bone{\dvar b_{\isocolor 1}}
\newcommand\btwo{\dvar b_{\isocolor 2}}
\newcommand\yone{\dvar y_{\isocolor 1}}
\newcommand\ytwo{\dvar y_{\isocolor 2}}

Even before higher-order functions, one of the main benefits of Datafun over
Datalog is that it permits manipulating relations as first class values.
%
In this subsection we will show how a variety of standard operations on sets can
be represented in Datafun.
%
The first operation we consider is testing membership:

\begin{code}
  \mem \isa \eqt A \to \tseteq A \to \tbool\\
  \mem \<\pboxvar x \<s = \efor{y}{s} \eeq {\dvar x} {\dvar y}
\end{code}

\noindent
This checks if $\dvar x$ is equal to any element $\dvar y \in s$. The argument
$\dvar x$ is discrete because increasing $\dvar x$ might send it from being
\emph{in} the set to being \emph{outside} the set (e.g. $1 \in \esetraw{1}$ but
$2 \notin \esetraw{1}$). Notice that here we're taking advantage of encoding
booleans as sets of empty tuples --- unioning these sets implements logical
\emph{or}.

Using $\mem$ we can define set intersection by taking the union of every
singleton set $\esetraw{\dvar x}$ where $\dvar x$ is an element of both
$s$ and $t$:

\begin{code}
  \pwild \cap \pwild \isa \tseteq A \to \tseteq A \to \tseteq A\\
  s \cap t = \efor x {s} \ewhen{\mem \<\eboxvar x \<t} \eset{\dvar x}
\end{code}

\noindent
Using comprehensions, this could alternately be written as:

\begin{code}
  s \cap t = \esetfor{\dvar x}{\dvar x \in s, \mem \<\eboxvar x \<s}
\end{code}

\noindent From now on, we'll use comprehensions whenever possible. For example,
we can also define the composition of two relations in Datafun:

\begin{code}
  \pwild \relcomp \pwild \isa \tset{\eqt A \x \eqt B} \to \tset{\eqt B \x \eqt
    C} \to \tset{\eqt A \x \eqt C}\\
  s \relcomp t = \esetfor{\etuple{\dvar a, \dvar c}}{
    \ptuple{\dvar a, \bone} \in s,
    \ptuple{\btwo, \dvar c} \in t,
    \eeq \bone \btwo}
\end{code}

\noindent
This is basically a transcription of the mathematical definition, where we build
those pairs which agree on their $B$-typed components.

We can also define set difference, although we must first detour into boolean
negation:

\begin{code}
  {\neg} \isa \iso\tbool \to \tbool\\
  \neg \pboxvar t = \ecase{\eisempty{\dvar t}}
  \inj 1 \ptuple{} \caseto \etrue;\ \inj 2 \ptuple{} \caseto \efalse
  \\[8pt]
  \pwild\setminus\pwild \isa \tseteq A \to \iso \tseteq A \to \tseteq A\\
  s \setminus \pboxvar t =
  \setfor{\dvar x}{\neg \ebox{\mem \<\eboxvar x \<\dvar t}}
\end{code}

\noindent
To implement boolean negation, we need the primitive operator $\eisempty e$,
which produces a tag indicating whether its argument $e$ (a set of empty tuples,
i.e. a boolean) is the empty set.
%
This in turn lets us define set difference, the analogue in Datafun of negation
in Datalog.
%
Note that in both boolean negation and set difference the ``negated'' argument
$\dvar t$ is boxed, because the operation is not monotone in $\dvar t$.
%
This enforces stratification.

%\subsubsection{Transitive Closure}
\label{sec:generic-transitive-closure}
Finally, generalizing the \name{ancestor} relation from the Datalog program in
\cref{sec:datalog}, we can define the transitive closure of a relation:

\nopagebreak[2]
\begin{code}
  \name{trans} \isa \iso \tset{\eqt A \x \eqt A} \to \tset{\eqt A \x \eqt A}\\
  \name{trans} \<\pboxvar{edge} = \efixis{s}{\dvar{edge} \vee (\dvar{edge} \relcomp s)}
\end{code}

\noindent
This definition uses a least fixed point, just like the mathematical definition
-- a transitive closure is the least relation $R$ containing the original
relation $\dvar{edge}$ and the composition of $\dvar{edge}$ with $R$.
%
However, one feature of this definition peculiar to Datafun is that the argument
type is $\iso \tset{\eqt A \times \eqt A}$ -- the transitive closure takes a
\emph{discrete} relation.
%
This is because we must use the relation within the fixed point, and so its
parameter needs to be discrete to occur within.
%
This restriction is artificial --- transitive closure is semantically a monotone
operation --- but we'll see why it's necessary in \cref{sec:transformations}.


\subsection{Regular expression combinators}
\label{sec:regex-combinators}

\newcommand\tre{\typename{re}}

Datafun permits tightly integrating the higher-order functional and
bottom-up logic programming styles. In this section, we illustrate the
benefits of doing so by showing how to implement a regular expression
matching library in combinator parsing style.  Like combinator parsers
in functional languages, the code is very concise.  However, support
for the relational style ensures we can write \naive\ code
\emph{without} the exponential backtracking cliffs typical of parser
combinators in functional languages.

We assume the existence of a function $\name{pos} : \tstring \to \tset{\tint}$
which takes a string and returns the set of valid indices in that string, and
assume that string indexing is written $s[n]$, as in Java or C. Having done
this, we define the type of regular expression matchers.

\begin{tabbing}
\qquad  $\kw{type}\;\tre\;=\iso \tstring \to \tset{\tint \times \tint}$
\end{tabbing}

A regular expression takes a string (boxed so that it can be used inside fixed
point expressions), and returns a set of pairs of integers. The idea is that if
the regular expression matcher is passed the string argument $s$, then if
$\etuple{i,n}$ is one of the returned values, the substring $s_i, s_{i+1},
\ldots, s_{n-1}$ is matched by the regular expression. That is, it is inclusive
on the left and exclusive on the right.

\begin{tabbing}
\qquad \= $\name{sym} : \iso\typename{char} \to \tre$ \+\\
$\name{sym} \;\pboxvar{c} \;\pboxvar{s} =\; $\=
$\efor{n}{\name{pos}\;\dvar s}{\ewhen{\eeq{\dvar{s}[\dvar{n}]}{\dvar c}}{\eset{\etuple{\dvar n, \dvar n+1}}}}$
\end{tabbing}

\noindent
The $\name{sym}$ combinator takes a character and returns a set of substrings by
returning the set $(\dvar n, \dvar n+1)$ where the $\dvar n$\textsuperscript{th}
element of the string $\dvar s$ is the character $\dvar c$.

\begin{tabbing}
\qquad \=   $\name{nil} : \tre$ \+\\
  $\name{nil}\;\pboxvar{s} = \efor{n}{\name{pos}\;\dvar s}{\eset{\etuple{\dvar n,\dvar n}}}$
\end{tabbing}

\noindent
Then $(\name{nil} \;s)$ yields $\etuple{n,n}$ for each position $n$ in $s$,
since an empty substring can start anywhere.

\begin{tabbing}
\qquad \=   $\name{seq} : \tre \to \tre \to \tre$ \+\\
  $\name{seq}\;r_1\;r_2\;s = r_1\;s \bullet r_2\;s$
\end{tabbing}

\noindent
The $\name{seq}$ combinator takes two regular expressions $r_1$ and $r_2$
as arguments, applies its argument to both, and takes the relational
composition of the results. Therefore, if the range $(i,j)$ was in the
result of $r_1$, and $(j,k)$ was in the result of $r_2$, then
we will return $(i,k)$, just as desired.

\begin{tabbing}
\qquad \=   $\name{bot} : \tre$ \+\\
  $\name{bot}\;\_ = \bot$ \\[1em]

  $\name{alt} : \tre \to \tre \to \tre$ \\
  $\name{alt}\;r_1\;r_2\;s = r_1\;s \vee r_2\;s$
\end{tabbing}

\noindent
We get the empty set of matches from $\name{bot}$, and
$\name{alt}$ unions the matches of its two arguments.

\begin{tabbing}
\qquad \=   $\name{star} : \iso\tre \to \tre$ \+\\
  $\name{star}\;\pboxvar{r}\;\pboxvar{s} = \name{nil}\;\eboxvar{s} \vee \name{trans}\;\ebox{\dvar r\;\eboxvar{s}}$
\end{tabbing}

\noindent
The most interesting regular expression combinator is the Kleene star operation
$\name{star}$. It uses $\name{nil}$ to get the reflexive relation on positions,
and then takes the transitive closure of the regular expression it received as
an argument using the $\name{trans}$ operation. This forces its argument to be
boxed, since $\name{trans}$ calculates a fixed point, and only discrete
variables can occur inside of fixed point expressions.\footnote{As a technical
  note, sets of pairs of integers do not form a \todo{finite-height} lattice, so
  by typing this program is not an acceptable fixed point expression. However,
  since the positions in a string form a finite set, semantically it is fine.
  The original Datafun paper shows how one can define bounded fixed points to
  handle cases like this, so we will not be scrupulous.}


\subsection{Regular expression combinators, take 2}

The combinators in the previous section found \emph{all} matches
within a given substring, but often we are not interested in all
matches: we only want to know if a string can match starting at a
particular location. We can easily refactor the combinators above to
work in this style, which illustrates the benefits of tightly
integrating functional and relational styles of programming -- we can
use functions to manage strict input/output divisions, and relations
to manage nondeterminism and search.

\begin{tabbing}
\qquad \=  $\kw{type}\;\tre\;=\iso (\tstring \times \tint) \to \tset{\tint}$ \+
\end{tabbing}

\noindent
Our new type of combinators takes a string and a starting
position, and returns a set of ending positions. In contrast, the earlier type
took a string and returned a set of start/end pairs.

\begin{tabbing}
\qquad \=  $\name{sym} : \iso\typename{char} \to \tre$\+ \\
  $\name{sym} \;\pboxvar c \;\pboxtuple{\dvar s, \dvar n} = \ewhen{\eeq{\dvar s[\dvar n]}{\dvar c}}{\eset{\dvar n+1}}$
\end{tabbing}

\noindent
$\name{sym}\;c$ checks if $c$ occurs at position $n$, yielding $\esetraw{n+1}$
if it does, and the empty set otherwise.

\begin{tabbing}
\qquad \=  $\name{nil} : \tre$\+ \\
  $\name{nil}\;\pboxtuple{\dvar s, \dvar n} = \eset{\dvar n}$
\\[.5\baselineskip]
  $\name{seq} : \tre \to \tre \to \tre$ \\
  $\name{seq}\;r_1\;r_2 \;\pboxtuple{\dvar s,\dvar n} = \efor{m}{r_1\;\eboxtuple{\dvar s,\dvar n}}{r_2\;\eboxtuple{\dvar s,\dvar m}}$
\end{tabbing}

\noindent
The $\name{nil}$ function simply returns the same index $n$ it received as an
argument, since an empty string matches starting from any position. Sequencing
via $\name{seq}\;r_1\;r_2$ checks first to see the possible ending positions from
matching $r_1$, and carries on with $r_2$ from there.

\begin{tabbing}
  \qquad \=
  $\name{bot} : \tre$ \+\\
  $\name{bot}\;\_ = \bot$ \\[1em]

  $\name{alt} : \tre \to \tre \to \tre$ \\
  $\name{alt}\;r_1\;r_2\;x =\ r_1\;x \vee r_2\;x$
\end{tabbing}

\noindent
We still get the empty set from $\name{bot}$, and $\name{alt}$
still unions the two sets of end positions.

\begin{tabbing}
\qquad \=  $\name{star} : \iso\tre \to \tre$\+ \\
  $\name{star}\;\pboxvar{r}\;\pboxtuple{\dvar s, \dvar n} = \efixis{s}{\bigl(\eset{\dvar n} \vee \efor{m}{\dvar s}{\dvar r\;\eboxtuple{\dvar s,\dvar m}}\bigr)}$
\end{tabbing}

\noindent
As before, the $\name{star}$ combinator takes a boxed regular expression as an
argument, and for the same reason -- we are implementing sequencing with a fixed
point. One thing worth noting about this definition is that it is
\emph{left-recursive} -- the definition takes the endpoints from the fixed point
$\name{self}$, and then continues matching using the argument $\name{r}$. This
should make clear that this is not just plain old functional programming -- we
are genuinely relying upon the fixed point semantics of Datafun.

\section{From Semi\naive{} Evaluation to the Incremental \boldfn-Calculus}
\label{sec:seminaive-and-ilc}

Let's return to our example Datalog program, modified to consider graphs rather
than ancestry:

\begin{align*}
  \name{path}(X,Z) &\gets \name{edge}(X,Z)
  &
  \name{path}(X,Z) &\gets \name{edge}(X,Y) \wedge \name{path}(Y,Z)
\end{align*}

%% \begin{minted}{prolog}
%%   path(X, Z) ← edge(X, Z).
%%   path(X, Z) ← edge(X,Y), path(Y, Z).
%% \end{minted}

\noindent
Suppose \name{edge} denotes a linear graph, $\{(1, 2),\, (2, 3),\, \dots,\,
({n-1}, n)\}$. Then \name{path} should denote its reachability relation,
$\setfor{(i, j)}{1 \le i < j \le n}$. How can we compute this? The simplest
approach is to begin with nothing in the \name{path} relation and repeatedly
apply its rules until nothing more is deducible. We can make this strategy
explicit by time-indexing our rules:

\begin{align*}
  \name{path}_{i+1}(X,Z) &\gets \name{edge}(X,Z)
  &
  \name{path}_{i+1}(X,Z) &\gets \name{edge}(X,Y) \wedge \name{path}_i(Y,Z)
\end{align*}

\noindent
By omission $\name{path}_0 = \emptyset$. From this we can see inductively that
$\name{path}_i \subseteq \name{path}_{i+1}$.
%
This is because at step $i+1$ we re-deduce every fact known at step $i$.
%
For example, suppose $\name{path}_i(j, k)$ holds. Then at step $i+1$ the second
rule deduces $\name{path}_{i+1}({j-1}, k)$ from $\name{edge}({j-1}, j) \wedge
\name{path}_i(j,k)$.
%
But since $\name{path}_{i+1}(j, k)$ holds, we perform the same deduction at time
$i+2$, and again at $i+3$, $i+4$, \emph{etc}.

Because we append edges one at a time, $\name{path}_i$ contains exactly paths of
$i$ or fewer edges.
%
Therefore it takes $n$ steps until we reach our fixed point $\name{path}_{n-1} =
\name{path}_n$.
%
Since step $i$ involves $|\name{path}_i| \in \Theta(i^2)$ deductions, we make
$\Theta(n^3)$ deductions in total.
%
There being only $\Theta(n^2)$ paths in the final result, this is terribly
wasteful; hence we term this \emph{\naive\ evaluation}.

Semi\naive\ evaluation avoids this waste by transforming the rules for
\rel{path} to find the newly deducible paths, $\name{dpath}_i$, at iteration
$i$, and accumulating these changes to produce a final result:

\begin{align*}
  \name{dpath}_0(X,Y) &\gets \name{edge}(X,Y)
  \\
  \name{dpath}_{i+1}(X,Z) &\gets \name{edge}(X,Y) \wedge \name{dpath}_i(Y,Z)
  \\
  \name{path}_{i+1}(X,Y) &\gets \name{path}_i(X,Y) \vee\name{dpath}_i(X,Y)
\end{align*}

\noindent
It's easy to show inductively that $\name{dpath}_{i}$ contains only paths
\emph{exactly} $i+1$ edges long. Consequently $|\name{dpath}_i| \in \Theta(n-i)$
and we make $\Theta(n^2)$ deductions overall.\footnote{Here we must assume the
  accumulation rule $\name{path}_{i+1}(X,Y) \gets \name{path}_i(X,Y) \vee
  \name{dpath}_i(X,Y)$ is implemented using an union operator that is efficient
  when the sets being unioned are of greatly differing sizes.}


\subsection{Semi\naive\ evaluation as incremental computation}
\label{sec:seminaive-tc-in-datafun}

Now let's move from Datalog to Datafun.\footnote{In this section we do not bother marking discrete variables $\dvar x$ or expressions $\eiso e$, as it muddies our examples to no benefit.} The transitive closure of \name{edge} is
the fixed point of the monotone function \name{step} defined by:

\[
\name{step} \<\name{path} = \name{edge} \cup
\setfor{(x,z)}{(x,y) \in \name{edge}, (y,z) \in \name{path}}
\]

\noindent
The \naive\ way to compute \name{step}'s fixed point is to iterate it: start from \(\name{path}_0 = \emptyset\) and compute \(\name{path}_{i+1} =
\name{step}\<\name{path}_i\) for increasing $i$ until \(\name{path}_i =
\name{path}_{i+1}\).
%
But as before, $\name{path}_i \subseteq \name{step}\<\name{path}_i$; each iteration re-computes the paths found by its predecessor.
%
Following Datalog, we'd prefer to compute only the \emph{change} between
iterations.
%
So consider $\name{step}'$ defined by:

\[
\name{step}' \<\name{dpath} =
\setfor{(x,z)}{(x,y) \in \name{edge}, (y,z) \in \name{dpath}}
\]

\newcommand\colorpath{{\color{BlindVermilion}\name{path}}}
\newcommand\colordpath{{\color{BlindBlue}\name{dpath}}}
\newcommand\colorA{\color{ColorA}}
\newcommand\colorB{\color{ColorB}}
\renewcommand\colorpath{\name{path}}
\renewcommand\colordpath{\name{dpath}}
%% \renewcommand\colorA\relax
%% \renewcommand\colorB\relax

\noindent
Observe that
%
\begin{align*}
  &\mathrel{\hphantom{=}} \name{step} \<(\colorpath \cup \colordpath)
  \\
  &= \name{edge} \cup \setfor{(x,z)}{(x,y) \in \name{edge}, (y,z) \in \colorpath \cup \colordpath}
  \\
  &= {\colorA \name{edge} \cup \setfor{(x,z)}{(x,y) \in \name{edge}, (y,z) \in \colorpath}} \cup {\colorB \setfor{(x,z)}{(x,y) \in \name{edge}, (y,z) \in \colordpath}}
  \\
  &= {\colorA\name{step}\<\colorpath} \cup {\colorB\name{step}'\<\colordpath}
\end{align*}

\noindent
In other words, $\name{step}'$ tells us how \name{step} changes as its input
grows.
%
This lets us directly compute the changes $\name{dpath}_i$ between our
iterations $\name{path}_i$:

\begin{align*}
  \name{dpath}_0
  &= \name{step}\<\emptyset
  = \name{edge}
  \\
  \name{dpath}_{i+1}
  &= \name{step}'\<\name{dpath}_i
  = \setfor{(x,z)}{(x,y) \in \name{edge}, (y,z) \in \name{dpath}_i}
  \\
  \name{path}_{i+1}
  &= \name{path}_i \cup \name{dpath}_i
\end{align*}

\noindent These exactly mirror the derivative and accumulator rules for
\(\name{path}_i\) and \(\name{dpath}_i\) we gave earlier.

%% TODO: Explain how this lets us compute $\name{path}_i$ more
%% efficiently and wait until it quiesces as before.

The problem of semi\naive\ evaluation for Datafun, then, reduces to the problem
of finding functions, like $\name{step}'$, which compute the change in a
function's output given a change to its input.
%
This is a problem of \emph{incremental computation}, and since Datafun is a
functional language, we turn to the \emph{incremental
  \fn-calculus}~\citep{incremental,DBLP:conf/esop/GiarrussoRS19}.


\subsection{Change structures}
\label{sec:change-structures}

To make precise the notion of change, an incremental \fn-calculus associates
every type $A$ with a \emph{change structure}, consisting of:\footnote{Our
  notion of change structure differs significantly from that of
  \citet{incremental}, although it is similar to the logical relation given in
  \citet{DBLP:conf/esop/GiarrussoRS19}; we discuss this in
  \cref{sec:differences-from-incremental}. Although we do not use change
  structures \emph{per se} in our proofs, they are an important source of
  intuition.}

\begin{enumerate}
\item A type $\D A$ of possible changes to values of type $A$.
\item A relation $\changesat{A}{\dx}{x}{y}$ for $\dx : \D A$ and $x,y : A$,
  glossed as ``$\dx$ changes $x$ into $y$''.
\end{enumerate}

\noindent
Since the iterations of a fixed point grow monotonically, in Datafun we only
need \emph{increasing} changes.
%
For example, sets change by gaining new elements:

\begin{align*}
  \D\tseteq{A} &= \tseteq{A}
  &
  \changesat{\tseteq{A}}{\dx}{x}{x \cup \dx}
\end{align*}

Set changes may be the most significant for fixed point purposes, but to handle
all of Datafun we need a change structure for every type. For products and sums,
for example, the change structure is pointwise:

%% \begin{center}
%%   \setlength\tabcolsep{10pt}
%%   \begin{tabular}{@{}ccc@{}}
%%     $\D\tunit = \tunit$
%%     &
%%     \(\D(A \x B) = \D A \x \D B\)
%%     &
%%     \(\D(A + B) = \D A + \D B\)
%%     \\[8pt]
%%     \(\changesat{\tunit}{\tuple{}}{\tuple{}}{\tuple{}}\)
%%     &
%%     \(\infer{
%%       \changesat{A}{\da}{a_1}{a_2}
%%       \\
%%       \changesat{B}{\db}{b_1}{b_2}
%%     }{\changesat{A \x B}
%%       {\tuple{\da,\db}}
%%       {\tuple{a_1,b_1}}
%%       {\tuple{a_2,b_2}}
%%     }\)
%%     &
%%     \(\infer{
%%       \changesat{A_i}{\dx_i}{x_i}{y_i}
%%     }{
%%       \changesat{A_1 + A_2}{\inj i \dx}{\inj i x}{\inj i y}
%%     }\)
%%   \end{tabular}
%% \end{center}

\begin{align*}
  \D\tunit &= \tunit
  &
  \D(A \x B) &= \D A \x \D B
  &
  \D(A + B) &= \D A + \D B
\end{align*}

\begin{align*}
  \changesat{\tunit}{\tuple{}}{\tuple{}}{\tuple{}}
  &&
  %% \infer{
  %%   \fa{i} \changesat{A_i}{\dx_i}{x_i}{y_i}
  %% }{\changesat{A_1 \x A_2}
  %%   {\tuple{\vec\dx}}
  %%   {\tuple{\vec x}}
  %%   {\tuple{\vec y}}
  %% }
  %
  %% \infer{
  %%   \fa{i} \changesat{A_i}{\dx_i}{x_i}{y_i}
  %% }{\changesat{A_1 \x A_2}
  %%   {\tuple{\dx_1,\dx_2}}
  %%   {\tuple{x_1,x_2}}
  %%   {\tuple{y_1,y_2}}
  %% }
  %
  \infer{
    \changesat{A}{\da}{a}{a'}
    \\
    \changesat{B}{\db}{b}{b'}
  }{\changesat{A \x B}
    {\tuple{\da,\db}}
    {\tuple{a,b}}
    {\tuple{a',b'}}
  }
  &&
  \infer{
    \changesat{A_i}{\dx}{x}{y}
  }{
    \changesat{A_1 + A_2}{\inj i \dx}{\inj i x}{\inj i y}
  }
\end{align*}
%\vspace{0pt} % TODO: double-check if this is better.

Since we only consider increasing changes, and $\iso A$ is ordered discretely,
the only ``change'' permitted is to stay the same. Consequently, no information
is necessary to indicate what changed:

\begin{align*}
  \D(\iso A) &= \tunit
  &&
  \changesat{\iso A}{\tuple{}}{x}{x}
\end{align*}

Finally we come to the most interesting case: functions.

\begin{align*}
  \D(A \to B) &= \iso A \to \D A \to \D B
  &
  \infer[fn change]{
    \fa{\changesat A \dx x y}
    \changesat B {\df\<x\<\dx} {f\<x} {g\<y}
  }{
    \changesat{A \to B}{\df}{f}{g}
  }
\end{align*}

\noindent
Observe that a function change $\df$ takes two arguments: a base point $x : \iso A$ and a change $\dx : \D A$.
%
To understand why we need both, consider incrementalizing function application:
we wish to know how $f\<x$ changes as both $f$ and $x$ change.
%
Supposing $\changes{\df}{f}{g}$ and $\changes{\dx}{x}{y}$, how do we find a
change $f\<x \changesto g\<y$ that updates both function and argument?

If changes were given
pointwise, taking only a base point, we'd stipulate that $\changes{\df}{f} g$
iff $\fa{x} \changes{\df\<x}{f\<x}{g\<x}$. But this only gets us to $g\<x$, not
$g\<y$: we've accounted for the change in the function, but not the argument.
%
We can account for both by giving $\df$ an additional parameter: not just the
base point $x$, but also the change $\dx$ to it.
%
Then by inverting \rn{fn change} we have $\changes{\df\<x\<\dx}{f\<x}{g\<y}$ as
desired.

%% This makes it easy to incrementalize function application, $f\<x$; given
%% changes $\changes \df f g$ and $\changes \dx x y$ to the function and its
%% argument, we want to compute the change that takes us to the updated
%% application $g\<y$. By inverting \textsc{FnChange} we know that
%% $\changes{\df\<x\<\dx}{f\<x}{g\<y}$, so $\df\<x\<\dx$ gives us the desired
%% change.

%% If instead changes were given pointwise, letting $\D(A \to B)= \iso A \to \D B$,
%% then it'd be natural to let $\changes{\df}{f}{g} \iff \fa{x}
%% \changes{\df\<x}{f\<x}{g\<x}$.

Note also the mixture of monotonicity and non-monotonicity in the type $\iso A
\to \D A \to \D B$. Since our functions are monotone (increasing inputs yield
increasing outputs), function changes are monotone with respect to input changes
$\D A$: a larger increase in the input yields a larger increase in the output.
However, there's no reason to expect the change in the output to grow as the
base point increases --- hence the use of $\iso$.


\subsection{Zero-changes, derivatives, and faster fixed points}
\label{sec:derivatives}

If $\changesat A \dx x x$, we call $\dx$ a \emph{zero-change} to $x$. Usually
zero-changes are rather boring --- for example, a zero change to a set $x :
\tseteq{A}$ is any $\dx \subseteq x$, and so $\emptyset$ is always a zero
change.
%
However, there is one very interesting exception: function zero changes. Suppose
$\changesat{A \to B}{\df}{f}{f}$. This implies that

\begin{equation}\label{eqn:derivative}
  \changesat A \dx x y \implies \changesat B{\df\<x\<\dx}{f\<x}{f\<y}
\end{equation}

\noindent
In other words, $\df$ yields the change in the output of $f$ given a change to
its input.
%
This is exactly the property of $\name{step}'$ that made it useful for
semi\naive\ evaluation --- indeed, $\name{step}'$ is a zero-change to
\name{step}, modulo not taking the base point $x$ as an argument:

\[ \changesat{\tseteq A} \dx x y \implies
\changesat{\tseteq A}{\name{step}'\<\dx}{\name{step}\<x}{\name{step}\<y}
\]

\noindent
Function zero changes are so important we give them a special name:
\emph{derivatives}. We now have enough machinery to prove correct a
general \emph{semi\naive\ fixed point strategy}. First, observe that:

\begin{lemma}\label{lem:DeltaL}
  At every semilattice type $L$, we have $\D L = L$ and
  $\changesat{L}{\dx}{x}{y} \iff (x \binvee \dx) = y$.
\end{lemma}

%% \begin{proof}
%%   By induction on semilattice types $L$.
%% \end{proof}

\noindent
This holds by a simple induction on semilattice types $L$. Now, given
a monotone map $f : L \to L$ and its derivative $f' : \iso L \to L \to
L$, we can find $f$'s fixed-point as the limit of the sequence $x_i$
defined:

\begin{align*}
  x_0 &= \bot & x_{i+1} &= x_i \vee \dx_i\\
  \dx_0 &= f\<\bot & \dx_{i+1} &= f'\<x_i\<\dx_i
\end{align*}

\noindent Let $\fastfix\<(f,f') = \bigvee_i x_i$. By induction and the
derivative property, we have $\changes{\dx_{i+1}}{x_i}{f\<x_i}$ and so
$x_i = f^i\<x$, and therefore $\fastfix\<(f,f') = \efix f$. Moreover,
if $L$ has no infinite ascending chains, we will reach our fixed point
$x_i = x_{i+1}$ in a finite number of iterations.

\label{sec:seminaive-strategy}

This leads directly to our strategy for semi\naive\ Datafun. \Citet{incremental}
defines a static transformation $\Deriv e$ which computes the change in $e$
given the change in its free variables; it \emph{incrementalizes} $e$. Our goal
is not to incrementalize Datafun \emph{per se}, but to find fixed points faster.
Consequently, we define two mutually recursive transformations: $\phi e$, which
computes $e$ faster by replacing fixed points with calls to \fastfix; and
$\delta e$, which incrementalizes $\phi e$ so that we can compute the derivative
of fixed point functions.
%
In order to define $\phi$ and $\delta$ and show them correct, however, we first
need a fuller account of Datafun's type system and semantics.

\section{Typing and Semantics of Core Datafun}
\label{sec:typing-and-semantics}

In this section, we give the typing and semantics of core Datafun. 
\begin{figure}
  \begin{mathpar}
    \setlength\arraycolsep{.4em}
    \begin{array}{r@{\hskip 1em}ccl}
      \text{contexts} & \G &\bnfeq& \cdot \bnfor \G, H \\
      \text{hypotheses} & H &\bnfeq& \h x A \bnfor \hd x A
    \end{array}

    \begin{array}{lcl}
      \stripcx{\cdot} & = & \cdot \\
      \stripcx{\G, \h x A} & = & \stripcx\G \\
      \stripcx{\G, \hd x A} & = & \stripcx\G, \hd x A
    \end{array}
    \\
    \infer[Var]{\h x A \in \G}{\J x \G A}

    \infer[DVar]{\hd x A \in \G}{\J {\dvar x} \G A}

    \infer[Lam]{\J e {\G,\h x A} B}{\J {\fnof x e} \G {A \to B}}

    \infer[App]{\J e \G {A \to B} \\ \J f \G A}{\J {e\<f} \G B}

    \infer[Unit]{\quad}{\J {\etuple{}} \G \tunit}

    \infer[Pair]{(\J{e_i}\G{A_i})_i}{\J{\etuple{e_1,e_2}} \G {A_1 \x A_2}}

    \infer[Prj]{\J e \G {A_1 \x A_2}}{\J{\pi_i\<e}\G{A_i}}

    \infer[Inj]{\J e \G A_i}{\J{\inj i e}\G{A_1 + A_2}}

    \infer[Case]{\J e \G {A_1 + A_2} \\
      (\J {f_i} {\G,\h {x_i} {A_i}} {B})_i
    }{
      \J {\ecase{e} (\inj i {x_i} \caseto f_i)_i} \G B
    }

    \infer[BoxI]{\J {\isocolor e} {\stripcx\G} A}{\J{\ebox e} \G {\iso A}}

    \infer[BoxE]{\J e \G {\iso A} \\ \J f {\G,\hd x A} B}{
      \J {\elet{\ebox x = e} f} \G B}

    \infer[Bot]{\quad}{\J\bot\G {\eqt L}}

    \infer[Join]{(\J{e_i} \G {\eqt L})_i}{\J{e_1 \vee e_2}\G {\eqt L}}

    %% \infer{\J e \G {\eqt A}}{\J {\edown e} \G {\tdown {\eqt A}}}
    \infer[Set]{(\J {\isocolor e_i} {\stripcx\G} {\eqt A})_i}{
      \J {\eset{e_i}_i} \G {\tset{\eqt A}}}

    %% \infer{\J e \G {\tdown {\eqt A}} \\
    %%   \J f {\G,\h x {\eqt A}} L
    %% }{\J {\ebigvee x e f} \G L}
    %%
    \infer[SetFor]{
      \J e \G {\tset A} \\
      \J f {\G,\hd x A} {\eqt L}
    }{\J {\efor x e f} \G {\eqt L}}

    %%\infer{\J e \G {\iso{(\eqt A \x \eqt A)}}}{\J{\prim{eq}\<e} \G {\tdown\tunit}}
    \infer[Eq]{(\J {\isocolor e_i} {\stripcx\G} {\eqt A})_i}
          {\J {\eeq{e_1}{e_2}} \G \tbool}

    \infer[IsEmpty]{\J {\isocolor e} {\stripcx\G} {\tset\tunit}}{
      \J {\eisempty e} \G {\tunit + \tunit}}

    \infer[Split]{\J e \G {\iso{(A + B)}}}{\J{\esplit e} \G {\iso A + \iso B}}

    \infer[Fix]{\J e \G {\iso{(\fixt L \to \fixt L)}}}{\J{\prim{fix}\< e} \G {\fixt L}}
  \end{mathpar}

  \caption{Datafun core syntax and typing rules}
  \label{fig:core-datafun}
\end{figure}


\subsection{Syntax}

The typing rules of core Datafun are given in Figure~\ref{fig:core-datafun}.
Contexts are lists of hypotheses, and hypotheses are either monotone variables
associated with a type $\h x A$, or they are discrete variables associated
with a type $\hd x A$. Given a context $\G$, we define the \emph{stripping}
operation $\stripcx\G$ which returns a new context which agrees with $\G$
on discrete hypotheses, but drops all the monotone hypotheses.

We now give a judgement $\J{e}{\G}{A}$, which is glossed as ``under
hypotheses $\G$, the term $e$ has the type $A$.'' The \rn{Var} and \rn{DVar}
rules say that both monotone hypotheses $\h x A$ and discrete hypotheses
$\hd x A$ justify ascribing the variable $x$ the type $A$. At first glance,
the \rn{Lam} rule is the familiar rule for lambda-abstraction, which
typechecks a lambda-expression $\fnof x e$ a the function type $A \to B$
when the body $e$ typechecks at $B$ in a context extended with $\h x
A$. However, note the subtlety that we assume $x$ is a \emph{monotone}
variable, not a discrete one. (This is the ``right'' choice since in the
category of posets and monotone maps, the exponential object are posets of
monotone functions.) The application rule \rn{App} is standard, though: a
term $e\>e'$ typechecks at $B$ if the function expression $e$ has type
$A \to B$, and the argument expression $e'$ has the type $B$.

The unit rule \rn{Unit} is standard, with the unit value $\etuple{}$ having
the type $1$ in all contexts. Similarly, the rules for pairs, \rn{Pair} and
\rn{Prj} are also standard. A pair $\etuple{e_1, e_2}$ has the type
$A_1 \times A_2$ when $e_1$ has type $A_1$, and $e_2$ has type $A_2$, and
symmetrically a projection $\pi_i(e)$ has type $A_i$ when $e$ has a pair
type. Similarly, the rules for sums, \rn{Inj} and \rn{Case} are standard.
An injection $\inj i e$ has type $A_1 + A_2$ when $e$ has type $A_i$, and
the typing rule for case expressions \rn{Case} says that a case
expression $\ecase{e} (\inj i {x_i} \caseto f_i)_i$ has type $B$, when
$e$ is of sum type $A_1 + A_2$, and each branch $f_1$ and $f_2$ has type $B$
in a context augmented with $\h {x_1} {A_1}$ and $\h {x_2} {A_2}$ respectively.
Note that both of these are once again monotone variables.

The rule \rn{BoxI} says that $\ebox{e}$ has type $\iso A$ when
$e$ has type $A$ in the stripped context $\stripcx\G$. By deleting
all the monotone variables, the rule ensures that $e$ can only
refer to discrete variables. The elimination rule \rn{BoxE} ascribes
the term $\elet{\ebox x = e}f$ the type $B$, when $e$ has type $\iso A$,
and $f$ typechecks at $B$ in a context extended with a \emph{discrete}
hypothesis $\hd x A$.

At this point, the typing rules correspond to standard constructive S4
modal logic. We can get to Datafun by adding a handful of
domain-specific types and operations. The first specialized rule is
the \rn{Split} rule, which says that $\esplit e$ has the type
$\iso A + \iso B$ when $e$ has the type $\iso (A + B)$. This is
syntax\footnote{An alternative syntax, pursued in \citet{datafun},
  would be to give two rules for $\kw{case}$, depending on whether or
  not the scrutinee could be typechecked in a stripped context.} for
the distributive law of box through sum types. The other direction
(getting $\iso (A + B)$ from $\iso A + \iso B$) is already derivable,
as is the isomorphism between $\iso A \times \iso B$ and
$\iso (A \times B)$. We have used all these operation implicitly, to
propagate discretenes into the variables of a pattern inside a box
pattern -- in the pattern $\ebox{\etuple{\inj 1 x, \inj 2 y}}$, we
expect both the variables $x$ and $y$ to be discrete, which is
information we propagate using these coercions. Semantically, all
of these operations are the identity, as we will see in the following
subsection. 

This leaves only the rules for manipulating sets and other
semilattices. Recall that sets form a semilattice in the inclusion order,
with the empty set as least element and set union as join, and also that
products of semilattices are semilattices (with order and operations given
pointwise). So we take the \emph{semilattice types} $L$ to be the set type
$\tset A$, as well as units $1$ and products of lattice types
$L_1 \times L_2$.  Then, the \rn{Bot} rule says that the term $\bot$ is the
least element of any semilattice type $L$, and the \rn{Join} rule gives the
type of joins, saying that $e_1 \vee e_2$ is of semilattice type $L$ when
each $e_i$ has type $L$. These constructors work for any semilattice $L$,
but there are rules specialized to sets as well. The typing rule for
eliminating sets, \rn{SetFor}, is \emph{almost} the monadic bind. However,
we generalize it by not requiring the term $\efor{x}{e}{e'}$ to eliminate
into a set type. Instead, we permit $e'$ to be \emph{any} semilattice type,
not just sets. Similarly, the introduction form \rn{Set} is specific to set
types, saying that $\eset{e_i}_{i\in I}$ has type $\tset A$ when each of the
$e_i$ has type $A$. Furthermore, each $e_i$ has to typecheck in the discrete
context, since membership relies on the non-monotonic property of equality.
This can be seen in the \rn{Eq} rule, which checks equality of two terms
$\eeq e f$ only when they are discrete, checking in stripped contexts.

Finally, the rule \rn{Fix} for fixed points $\efix e$, takes an endofunction
$e$ of type $\iso (L \to L)$ and yields an expression of type $L$. In addition
to being of semilattice type, we also demand here that $L$ is of finite height,
ensuring that the recursion will terminate. 

\subsection{Semantics }

%% ---- Semantics in a Datafun Model ----
\begin{figure*}
  \textsc{Type and Context Denotations}

  %% TODO: revert to this more readable version if space allows.
  %% \begin{align*}
  %%   \den{\tunit} &= \termO & \den{A \to B} &= \expO{\den{A}}{\den{B}}
  %%   \\
  %%   \den{\tseteq A} &= \pfinof{\den{\eqt A}}
  %%   & \den{A \x B} &= \den{A} \x \den{B}
  %%   \\
  %%   \den{\iso A} &= \iso{\den{A}} & \den{A + B} &= \den{A} + \den{B}
  %% \end{align*}

  \begin{align*}
    \den{\tunit} &= \termO
    & \den{\tseteq A} &= \pfinof{\den{\eqt A}}
    & \den{\iso A} &= \iso{\den{A}}
    \\
    \den{A \to B} &= \expO{\den{A}}{\den{B}}
    & \den{A \x B} &= \den{A} \x \den{B}
    & \den{A + B} &= \den{A} + \den{B}
  \end{align*}

  \begin{align*}
    \den{\G} &= \prod_{H \in \G} \den{H} &
    \den{\h x A} &= \den{A} & \den{\hd x A} &= \iso{\den{A}} &
    \den{\G \vdash A} &= \Poset(\den\G, \den A)
  \end{align*}
%  \vspace{0pt} % yes, this matters.

  \textsc{Term Denotations}

  \begin{displaymath}
    \def\arraystretch{1.15}
    \begin{array}{lcl}
      \den{\J {\dvar x} \G A} &=& \pi_{\dvar x} \then \varepsilon \qquad \text{(for $\hd x A \in \G$)} \\
      \den{\J x \G A} &=& \pi_x \qquad\quad\, \text{(for $\h x A \in \G$)} \\
      \den{\J {\fnof x e} \G {A \to B}} &=& \lambda\den{\J e {\G, \h x A} B} \\
      \den{\J {f\<e} \G B} &=& \fork{\den{\J f \G {A \to B}}, \den{\J e \G A}} \then \eval \\
      \den{\J {\etuple{e_1, e_2}} \G {A_1 \times A_2}} &=&
           \fork{\den{\J {e_1} \G {A_1}}, \den{\J {e_2} \G {A_2}}} \\
      \den{\J {\pi_i\<e} \G {A_i}} &=& \den{\J e \G {A_1 \times A_2}} \then \pi_i \\
      \den{\J {\ebox e} \G {\iso A}} &=& \mkbox_\Gamma(\den {\J e {\stripcx \G} A}) \\
      \den{\J {\elet{\ebox x = e} f} \G B} &=&  \fork{\id_\Gamma, \den{\J e \G {\iso A}}} \then \den{\J f {\G, \hd x A} B}  \\
      \den{\J \bot \G L} &=& \termI \then \morph{join}^L_0 \\
      \den{\J {e \vee f} \G L} &=& \fork{\den{\J e \G L}, \den{\J f \G L}} \then \morph{join}^L_2 \\
      \den{\J {\eisempty e} \G {1+1}}&=& \mkbox_\Gamma(\den{\J e {\stripcx \G} {\tset A}}) \then \morph{isEmpty} \\
      \den{\J {\esplit e} \G {\iso A + \iso B}} &=& \den{\J e \G {\iso(A + B)}}\then \isosum \\

      \den{\eeq{e_1}{e_2}} &=&
          \fork{\mkbox_\Gamma(\den{\J {e_1} {\stripcx \G} A}),
                \mkbox_\Gamma(\den{\J {e_2} {\stripcx \G} A})}
          \then \morph{eq} \\
      \den{\J {\efix e} \G L} &=& \den{\J e \G {\iso(L \to L)}} \then \morph{fix} \\

      \den{\eset{e_i}_i} &=& \fork{\mkbox_\Gamma(\den{\J {e_1} {\stripcx \G} A}) \then \morph{singleton}}_i \then \morph{join}^L \\

      \den{\J {\efor x e f} \G L} &=&    \fork{\id,\den{\J e \G {\tset A}}} \then \pcollect{\den{\J f {\G, \hd x A} L}} \\
    \den{\J {\inj i e} \G {A_1 + A_2}} &=& \den{\J e \G {A_i}} \then \injc_i \\
    \den{\J {\ecase{e} (\inj i{x_i} \caseto f_i)_i} \G B} &=&
    \fork{\id, \den{\J e \G {A_1 + A_2}}} \then \morph{dist}^\x_+ \then
           \bigkrof{\den{\J {f_i} {\G, \h {x_i} {A_i}} B}}_i \\
    \end{array}
  \end{displaymath}
  \vspace{0pt} % yes, this matters

  \textsc{Auxilliary Operations}

  \begin{align*}
    \morph{dist}^\x_+ &~:~ A \x (B_1 + B_2) \to (A \x B_1) + (A \x B_2)
    &
    \mkbox_\Gamma &~:~ \Poset(\den{\stripcx \G}, A) \to \Poset(\den{\G}, \iso A) \\
    % this could be simpler if it distributed in the opposite direction.
    \morph{dist}^\x_+ &= \fork{\pi_2 \then \krof{\lambda (\fork{\pi_2,\pi_1} \then \injc_i)}_i, \pi_1}
    \then \eval
    &
    \mkbox_\Gamma(f) &= \fork{\pi_{\dvar x} \then \delta}_{\hd x A \in \G} \then \isox \then \iso(f)
  \end{align*}

  \caption{Semantics of Datafun}
  \label{fig:semantics}\label{def:strip}
\end{figure*}


\subsection{Metatheory}

If we were presenting core Datafun in isolation, the usual thing to do
would be to prove the soundness of syntactic substitution, show that
syntactic and semantic substitution agree, and then establish the
equational theory. However, that is not our goal in this paper. We
want to prove the correctness of the semi\naive\ translation, which we
will do with a logical relations argument. Since we can harvest almost
all the properties we need from the logical relation, only a small
residue of metatheoretic need to be established at this -- indeed, the
only thing we need to estabish is the type-correctness of weakening.

\section{The Incremental Transformation}
\label{sec:incremental-transformation}

\begin{figure}\centering
  \begin{align*}
    \D\tseteq A &= \tseteq A
    &
    \Phi\tseteq A &= \tset{\Phi{\eqt A}}
    \qquad\text{\todo{(see \cref{thm:phi-eqtype})}}
    \\
    \D \iso A &= \tunit
    &
    \Phi \iso A &= \iso{(\Phi A \x \DP A)}
    \\
    \D\tunit &= \tunit
    &
    \Phi\tunit &= \tunit
    \\
    \D(A \x B) &= \D A \x \D B
    &
    \Phi(A \x B) &= \Phi A \x \Phi B
    \\
    \D(A + B) &= \D A + \D B
    &
    \Phi(A + B) &= \Phi A + \Phi B
    \\
    \D(A \to B) &= \iso A \to \D A \to \D B
    &
    \Phi(A \to B) &= \Phi A \to \Phi B
  \end{align*}

  \caption{$\D$ and $\Phi$ type transformations}
  \label{fig:DeltaPhi}
\end{figure}



\todo{Explain \cref{fig:DeltaPhi}. Then explain corresponding context transformations:}
\begin{align*}
  \D(\h x A) &= \h \dx {\D A} & \D(\hd x A) &= \emptycx\quad\text{(the empty context)}
  \\
  \Phi(\h x A) &= \h x {\Phi A} & \Phi(\hd x A) &= \hd x {\Phi A}, \hd \dx {\DP A}
  \\
  \iso{(\h x A)} &= \hd x A & \iso{(\hd x A)} &= \hd x A
\end{align*}

\todo{Explain how uses of $\phi$ in $\delta e$ involve weakening, and how this
  justifies their use in discrete contexts (hilighted in {\color{Rhodamine}pink}), eg. in $\delta(e\<f) = \delta e \<\eboxraw{\phi f} \<\delta f$.}

\todo{explain implementation of \zero{} via \dummy{}.}

%% ---- "Go faster" term translation, phi ----
\begin{figure}\centering
  \begin{align*}
    \phi x &= x & \phi \dvar x &= \dvar x\\
    \phi(\fnof x e) &= \fnof x \phi e & \phi(e\<f) &= \phi e\<\phi f\\
    \phi\etuple{e_i}_i &= \etuple{\phi e_i}_i &
    \phi(\pi_i\<e) &= \pi_i\<\phi e\\
    \phi(\inj i e) &= \inj i \phi e
    &
    \phi(\ecase e (\inj i x \caseto f_i)_i)
    &= \ecase{\phi e} (\inj i x \caseto \phi f_i)_i
    \\
    \phi\bot &= \bot &
    \phi(e \vee f) &= \phi e \vee \phi f\\
    \phi(\eset{e_i}_i) &= \eset{\phi e_i}_i
    &
    %% replaced substitution by let-binding
    \phi(\efor x e f) &= \efor x {\phi e}
        %{\substd{\phi f}{\dvar\dx \substo \zero\<\dvar x}}
        {\eletbox{\dx}{\ebox{\zero\<\dvar x}} \phi f}
    \\
    \phi\ebox{e} &= \ebox{\etuple{\phi e, \color{Rhodamine} \delta e}}
    &
    \phi(\eletbox x e f)
    &= \elet{\ebox{\etuple{\dvar x,\dvar\dx}} = \phi e} \phi f
    \\
    \phi(\eeq e f) &= (\eeq {\phi e} {\phi f})
    &
    \phi(\eisempty e) &= \eisempty {\phi e}
    \\
    \phi(\efix e) &= \fastfix\<\phi e
    &
    %% split
    \phi(\esplit e) &= \ecase{\phi e}
    \\
    &&&\phantom{{}={}}\
    \left(\ebox{\etuple{\inj i \dvar x, \inj i \dvar \dx}}
    \caseto \inj i {\ebox {\etuple{\dvar x,\dvar\dx}}}\right)_{i}
    \\
    &&&\phantom{{}={}}\
    \left(\ebox{\etuple{\inj i \dvar x, \inj j \pwild}}
    \caseto \inj i {\ebox {\etuple{\dvar x, \dummy\<\dvar x}}} \right)_{i\ne j}
  \end{align*}

  \caption{Semi\naive{} speed-up translation, $\phi$}
  \label{fig:phi}
\end{figure}


%% ---- "Derivative" term translation, delta
\begin{figure}\centering
  \[ \delta\bot = \delta\eset{e_i}_i = \delta(\eeq e f) = \delta(\efix e) = \bot \]
  %
  \begin{align*}
    \delta x &= \dx &
    \delta \dvar x &= \dvar\dx\\
    \delta(\fnof{x} e) &= \fnof{\ebox x} \fnof\dx \delta e
    & \delta(e\<f) &= \delta e \<\ebox{\color{Rhodamine}\phi e} \<\delta f\\
    \delta\etuple{e_i}_i &= \etuple{\delta e_i}_i
    & \delta(\pi_i\<e) &= \pi_i\<\delta e\\
    \delta(\inj i e) &= \inj i {\delta e} &
    \delta(e \vee f) &= \delta e \vee \delta f\\
    \delta\ebox{e} &= \etuple{} &
    \delta(\eletbox x e f)
    &= \elet{\ebox{\etuple{\dvar x,\dvar\dx}} = \phi e} \delta f
    \\
    \delta(\eisempty e) &= \eisempty {\color{Rhodamine} \phi e}
    &
    \delta(\esplit e) &= \ecase{\phi e}
    (\ebox{\etuple{\inj i \pwild, \pwild}}
    \caseto \inj i {\etuple{}} )_i
  \end{align*}
  %
  \begin{align*}
    \delta(\ecase e (\inj i x \caseto f_i)_i)
    &= \ecase{\esplit{\ebox{\color{Rhodamine} \phi e}},\, \delta e}\\
    &\qquad ({\inj i {\eboxvar x},\, \inj i \dx} \caseto \delta f_i)_{i}\\
    &\qquad ({\inj i {\eboxvar x},\, \inj j \pwild}
    %\caseto \subst{\delta f_i}{\dx \substo \dummy\<\dvar x})_{i\ne j}
    \caseto \elet{\dx = \dummy\<\dvar x} \delta f_i)_{i\ne j}
    \\
    \delta(\efor x e f)
    &= (\efor x {\delta e}
    %\substd{\phi f}{\dvar\dx \substo \zero\<\dvar x}) \\
    \eletbox \dx {\zero\<\dvar x} \phi f) \\
    &\vee (\efor x {{\phi e} \vee \delta e}
    %\substd{\delta f}{\dvar\dx \substo \zero\<\dvar x})
    \eletbox{\dx}{\zero\<\dvar x} \delta f)
  \end{align*}

  \caption{Semi\naive{} derivative translation, $\delta$}
  \label{fig:delta}
\end{figure}


\section{Implementation and Optimization}
\label{sec:implementation}

\todo{Things we need to make derivative of regex go fast, probably:
  \begin{enumerate}
  \item Inlining function definitions.
  \item Applying a zero-change to a zero-change produces a zero-change. This was
    mentioned in \citet{incremental}.
  \end{enumerate}}

\section{Discussion and Related Work}
\label{sec:related-work}

\label{sec:differences-from-incremental}

\paragraph{Nested fixed points}
%
The typing rule for $\efix e$ requires $e : \iso(\fixt L \to \fixt L)$.
%
The $\phi$ translation takes advantage of this $\iso$, decorating expressions of
type $\iso A$ with their zero-changes.
%
However, it also prevents an otherwise valid idiom: in a nested fixed-point
expression $\efixis x \dots (\efixis y e) \dots$, the inner fixed point body $e$
cannot use the monotone variable $x$!
%
This restriction is not present in \citet{datafun}; its addition brings Datafun
closer to Datalog, whose syntax cannot express this sort of nested fixed point.

We suspect it is possible to lift this restriction without losing
semi\naive\ evaluation, by decorating \emph{all} expressions and variables (not
just discrete ones) with zero-changes.
%
However, this also invalidates $\delta(\efix f) = \bot$: now that $f$ can
change, so can $\efix f$.
%
Luckily, there is a simple and correct solution: $\Deriv(\efix f) = \efix \ebox{\Deriv f \<\ebox{\efix f}}$~\cite{delta-fix}.
%
However, to compute this new fixed point semi\naive{}ly, we need a \emph{second
  derivative}: the zero-change to $\Deriv f \<\ebox{\efix f}$. Indeed, for a
program with fixed points nested $n$ deep, we need $n$\textsuperscript{th}
derivatives. We leave this to future work.

%% Can't we just have \delta produce two expressions: the derivative, and the
%% zero-change to the derivative?

%% \todo{Discuss possibility of allowing nested fixed points by removing the $\iso$
%%   from $\prim{fix}$'s argument. The problem here is $\delta(\efix f)$. I've
%%   previously worked out what this should be (cite that 3-page proof on my
%%   website): $\delta(\efix f) = \efix(\delta f \<(\efix f))$. But if we want to
%%   compute \emph{this} fixed point faster, we need a \emph{second derivative}. So
%%   if your program has $n$ nested fixed points, you probably need
%%   $n$\textsuperscript{th} derivatives. Future work!}

  

%% ---------- Acknowledgments ----------
\begin{acks}                            %% acks environment is optional
% TODO: acknowledge
% - Paolo Giarrusso, for discussion on incremental lambda calc.

% - Michael Peyton-Jones and possibly other Semmle folk, for discussion of
%   seminaive evaluation.

% - Max S New for help with the question "how do I prove my semantics are
%   naturally isomorphic without inducting on typing derivations"?

% - Achim Jung, for pointing out that to prevent nesting fixed points we'd need
%   a modal type system, which eventually led to the realization about
%   reinterpreting \iso.

  %%                                       %% contents suppressed with 'anonymous'
  %% %% Commands \grantsponsor{<sponsorID>}{<name>}{<url>} and
  %% %% \grantnum[<url>]{<sponsorID>}{<number>} should be used to
  %% %% acknowledge financial support and will be used by metadata
  %% %% extraction tools.
  %% This material is based upon work supported by the
  %% \grantsponsor{GS100000001}{National Science
  %%   Foundation}{http://dx.doi.org/10.13039/100000001} under Grant
  %% No.~\grantnum{GS100000001}{nnnnnnn} and Grant
  %% No.~\grantnum{GS100000001}{mmmmmmm}.  Any opinions, findings, and
  %% conclusions or recommendations expressed in this material are those
  %% of the author and do not necessarily reflect the views of the
  %% National Science Foundation.
\end{acks}


%% ---------- Bibliography ----------
\nocite{incremental} % FIXME remove
\bibliography{seminaive-datafun}


%% %% ---------- Appendix ----------
%% \appendix
%% \section{Appendix}

%% Text of appendix \ldots

\end{document}

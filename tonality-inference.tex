\documentclass{article}
\usepackage[a4paper,margin=0.8in]{geometry}

\usepackage{amssymb,amsmath,amsthm,latexsym}
\usepackage{mathpartir}

\usepackage{url}
\usepackage{hyperref}

% for semantic brackets
\usepackage{stmaryrd}

%% for \dblcolon
\usepackage{mathtools}

%% uses the AMS Euler math font.
%% \usepackage{euler}

%% for \underaccent
\usepackage{accents}

%% for colors
%% must be loaded before tikz in order for options to take effect :/
\usepackage[dvipsnames]{xcolor}

%% for drawing Hasse diagram
\usepackage{tikz}

%% for aligning tikz diagrams vertically with tables
\usepackage{adjustbox}


%% Commands
\newtheorem{theorem}{Theorem}
\newtheorem{corollary}{Corollary}
\newtheorem{lemma}{Lemma}
\newtheorem{principle}{Principle}

\newcommand{\todo}[1]{{\color{red}#1}}

\newcommand{\bnfeq}{\dblcolon=}
\newcommand{\defeq}{\overset{\ms{def}}{=}}

\newcommand{\ms}[1]{\ensuremath{\mathsf{#1}}}
\newcommand{\mb}[1]{\ensuremath{\mathbf{#1}}}
\newcommand{\mi}[1]{\ensuremath{\mathit{#1}}}
\newcommand{\mc}[1]{\ensuremath{\mathcal{#1}}}

\newcommand{\GG}{\Gamma}

\newcommand{\N}{\mathbb{N}}
\newcommand{\fn}{\lambda}
\newcommand{\binder}{.\,}
\newcommand{\bind}[1]{{#1}\binder}
\newcommand{\sub}[1]{\{{#1}\}}
\newcommand{\Disc}[1]{\square{#1}}
\newcommand{\fix}{\ms{fix}}

\newcommand{\den}[1]{\llbracket{#1}\rrbracket}

%% tones: mono, anti, discrete, bivariant
\newcommand{\tm}{\ensuremath{1}}     % monotone
\newcommand{\ta}{\ensuremath{-1}}    % antitone
\newcommand{\ti}{\ensuremath{\star}} % invariant / discrete
\newcommand{\tb}{\ensuremath{0}}     % bivariant
\newcommand{\tc}{\cdot}         % tone composition

\newcommand{\h}[3]{#1 :^{#3}\! {#2}}
\newcommand{\hm}[2]{\h{#1}{#2}{\tm}}
\newcommand{\ha}[2]{\h{#1}{#2}{\ta}}
\newcommand{\hi}[2]{\h{#1}{#2}{\ti}}
\newcommand{\hb}[2]{\h{#1}{#2}{\tb}}

\title{Tonality and inference}
\author{Michael Robert Arntzenius}
\date{16 November 2017}


\begin{document}
\maketitle

\begin{mathpar}
  \begin{array}{cccc}
    \text{tones}
    & s,t,u,v
    & \bnfeq & \tm ~|~ \ta ~|~ \tb ~|~ \ti
    \vspace{0.5em}\\
    \text{contexts}
    & \GG &\bnfeq& \cdot ~|~ \GG, \h{x}{A}{s}
    \vspace{0.5em}\\
    \text{types} & A,B,C \vspace{0.5em}\\
    \text{terms} & M,N,O \vspace{0.5em}\\
    \text{judgment}
    & J &\bnfeq& \GG \vdash \h{M}{A}{s}
  \end{array}
\end{mathpar}

Tones represent ways a function may respect or disrespect the orderings on its
arguments. $\tm$ is monotone; $\ta$ is antitone (monotone in the opposite
ordering); $\tb$ is bivariant (monotone and antitone); and $\ti$ is invariant
(only respects equivalence).

Formally, tones are properties of maps between preorders (sets equipped with
reflexive, transitive relations):\,\footnote{It is probably possible to
  generalize tones to properties of functors and categories, but we do not do
  this here.}

\begin{center}
  \begin{tabular}{lclccc}
    \multicolumn{1}{c}{\textbf{Name}}
    & \multicolumn{1}{c}{\textbf{Symbol}}
    & \multicolumn{1}{c}{\textbf{Respects}}
    & \multicolumn{3}{c}{\textbf{As property of $f$}}
    \\\hline
    \text{Monotone} & \tm
    & \text{ordering}
    & $x \le y$ &$\implies$& $f(x) \le f(y)$
    \\
    \text{Antitone} & \ta
    & \text{opposite ordering}
    & $x \ge y$ &$\implies$& $f(x) \le f(y)$
    \\
    \text{Bivariant} & \tb
    & \text{equivalence closure}
    & $x \le y \vee y \le x$ &$\implies$& $f(x) \le f(y)$
    \\
    \text{Invariant} & \ti
    & \text{induced equivalence}
    & $x \le y \wedge y \le x$ &$\implies$& $f(x) \le f(y)$
  \end{tabular}
\end{center}

In posets, antisymmetry trivializes invariance (because $x \le y \wedge y \le x
\implies x = y \implies f(x) = f(y)$), so we sometimes call $\ti$ the
``discrete'' tone, because it respects only the discrete ordering.

We define two operators of interest on tones: Tone join $s \vee t$ is the join
of the lattice ordered $\tb < \{\tm, \ta\} < \ti$; tone composition $s \tc t$
gives the general tone of a composed function $f \circ g$ when $f$ has tone $t$
and $g$ has tone $s$:

\begin{mathpar}
  %% "baseline=(current bounding box.center)" incantation taken from
  %% https://tex.stackexchange.com/questions/220531/how-to-align-tikzpicture-and-text-in-a-table/220543#220543
  \begin{tikzpicture}[scale=1,baseline=(current bounding box.center)]
    \node (top)  at ( 0, 1) {$\ti$};
    \node (bot)  at ( 0,-1) {$\tb$};
    \node (-1)   at (-1, 0) {$\tm$};
    \node (1)    at ( 1, 0) {$\ta$};
    \draw (top) -- (-1) -- (bot) -- (1) -- (top);
  \end{tikzpicture}

  \begin{array}{c|rrrrr}
    s \vee t & \tm & \tb & \ta & \ti\\\hline
    \tm & \tm & \tm & \ti & \ti\\
    \tb & \tm & \tb & \ta & \ti\\
    \ta & \ti & \ta & \ta & \ti\\
    \ti & \ti & \ti & \ti & \ti
  \end{array}

  \begin{array}{c|rrrr}
    s \tc t & \tm & \ta & \ti & \tb\\\hline
    \tm & \tm & \ta & \ti & \tb\\
    \ta & \ta & \tm & \ti & \tb\\
    \ti & \ti & \ti & \ti & \tb\\
    \tb & \tb & \tb & \tb & \tb
  \end{array}
\end{mathpar}

\emph{N.B.} Monotonicity Types by Clancy, Miller, and Meiklejohn has a diagram
much like this! However, instead of bivariance they have constancy, a stricter
condition. They write $\uparrow$ for our $\tm$, $\downarrow$ for our $\ta$, $?$
for our $\ti$, and $\sim$ for constancy; they also add $=$ for the identity
function.
%%
They also have a contraction operator (``+'') on tones (``qualifiers''). I'm not
sure how or if contraction relates to tone join.

Composition $\tc$ is a commutative monoid with $\tm$ as identity and $\tb$ as
an absorbing (``zero'', ``annihilating'') element. $\ti$ absorbs all elements
other than $\tb$, and $\ta$ is its own inverse. \todo{Does it interact with
  $\vee$ in any interesting ways?}


\section{Substitution as guiding principle}

We wish to justify some variant of the following substitution principle:

\begin{principle}[Substition]
  If $\GG_1 \vdash \h{M}{A}{s}$ and $\GG_2,\, \h{x}{A}{s} \vdash \h{N}{B}{t}$,
  then \(\GG_1 \cup \GG_2 \vdash \h{N\sub{x \mapsto M}}{B}{t}\).
\end{principle}

It's not yet clear what $\GG_1 \cup \GG_2$ should mean, exactly; \todo{we will
  return to this question.}


\section{Equality of judgments at different tones}

\subsection{By example}

We hold the following judgments to be equal:
\begin{equation}\label{eqn:ex1}
 \ha{x}{A} \vdash \hm{M}{B} ~=~ \hm{x}{A} \vdash \ha{M}{B}
\end{equation}

The first judgment says the sub-expression $M$, used monotonically, will use the
variable $x$ anti-tonically. The second says that $M$, used anti-tonically, will
use the variable $x$ monotonically.

Let $f : \den{A} \to \den{B}$ be the denotation of $M$ as a single-argument
function of $x$. The first judgment says that $f$ is a monotone map from
$\den{A}$ to $\den{B}^{\ms{op}}$; the second that $f$ is monotone from
$\den{A}^{\ms{op}}$ to $\den{B}$. These amount to the same thing: $f$ is
antitone.

We also consider these judgments to be equal:
\begin{equation}
  \hm{x}{A} \vdash \hi{M}{B}
  ~=~
  \ha{x}{A} \vdash \hi{M}{B}
  ~=~
  \hb{x}{A} \vdash \hi{M}{B}
\end{equation}

This is because, when the conclusion is invariant, the only tone distinction
that matters on the variables is between invariant and everything else; any two
non-$\ti$ tones are interchangeable.

Finally, we consider these judgments to be equal:
\begin{equation} \label{eqn:ex3}
  \hb{x}{A} \vdash \hb{M}{B}
  ~=~
  \hb{x}{A} \vdash \hm{M}{B}
  ~=~
  \hb{x}{A} \vdash \ha{M}{B}
  ~=~
  \hb{x}{A} \vdash \hi{M}{B}
\end{equation}

This is because, when the hypotheses are all bivariant, the tone of the
conclusion is irrelevant.


\subsection{In general}
The general pattern which unifies equations \ref{eqn:ex1}--\ref{eqn:ex3} is:

\newcommand{\hilited}{\color{blue}}

\begin{equation}
  \overline{\h{x_i}{A_i}{\hilited s_i}}^i \vdash \h{M}{B}{\hilited t \tc u}
  ~=~
  \overline{\h{x_i}{A_i}{\hilited s_i \tc t}}^i
  \vdash \h{M}{B}{\hilited u}
\end{equation}

Or, perhaps more simply:
\begin{equation}
  \overline{\h{x_i}{A_i}{\hilited s_i}}^i \vdash \h{M}{B}{\hilited t}
  ~=~
  \overline{\h{x_i}{A_i}{\hilited s_i \tc t}}^i
  \vdash \h{M}{B}{\hilited\tm}
\end{equation}

\todo{TODO: Justify in terms of semantics of substitution as function
  composition.}

\subsubsection{On overline notation}

An overlined and superscripted meta-expression $\overline{X}^i$ represents a
sequence (of unspecified length) with index variable $i$. The index variables
clarify which bits are repeated \emph{with variation}, and which \emph{without}.
Two examples:

\begin{center}
  \begin{tabular}{lcll}
    $\overline{x_i : A_i}^i$ & stands for
    & $x_1 : A_1,\, x_2 : A_2,\, ...,\, x_n : A_n$
    \\
    $\overline{x_i : A}^i$ & stands for
    & $x_1 : A,\hspace{0.6em} x_2 : A,\hspace{0.6em} ...,\, x_n : A$
    & ($A$ is unsubscripted, so stays the same)
  \end{tabular}
\end{center}

This is inspired by Guy Steele's talk on Computer Science Metanotation.%
%%
\footnote{There are videos of the talk at
  \href{https://www.youtube.com/watch?v=dCuZkaaou0Q}{Clojure/conj 2017} (the one
  I watched), \href{https://www.youtube.com/watch?v=7HKbjYqqPPQ}{PPoPP 2017},
  and
  \href{https://www.youtube.com/watch?v=8fCfkGFF7X8&feature=youtu.be&t=37m46s}{Harvard
    University}. There are also
  \href{http://s3.amazonaws.com/erlang-conferences-production/media/files/000/000/755/original/Guy_L._Steele_-_A_Cobbler's_Child.pdf?1510053539}{slides
    from Code Mesh 2017}.}
%%
In the talk, Guy analyses ``overline notation'' --- the use of $\overline{x}$ to
stand for a sequence $x_1, x_2, ..., x_n$.
%%
Inspired by quasiquotation in Lisp, he proposes using underlines to indicate
parts which are repeated \emph{without variation}. Thus $\overline{x : A}$
stands for $x_1 : A_1, ..., x_n : A_n$, while $\overline{x : \underline{A}}$
stands for $x_1 : A, ..., x_n : A$.
%%
In my Lisp experience, admittedly more limited than Guy's, quasiquotation is
wonderful until you need to nest it. One level of nesting is difficult to read;
two or more are unbearable. Therefore, I prefer explicit indices.


\section{Punchline}

\todo{TODO}

The declarative typing rule for \textbf{let}, which internalizes the
substitution principle, is:
\begin{mathpar}
  \infer{
    \GG \vdash \h{M}{B}{s}
    \quad
    \GG, \h{x}{B}{s} \vdash \h{N}{C}{t}
  }{
    \GG \vdash \textbf{let}~ x = M ~\textbf{in}~ \h{N}{C}{t}
  }
\end{mathpar}

This can be bidirectionalized as follows, recalling that in a context, the
variables' types are inputs, but their tones are outputs:
\begin{mathpar}
  \infer{
    \overline{\h{x_i}{A_i}{\hilited s_i}}^i \vdash M \Rightarrow B
    \quad
    \overline{\h{x_i}{A_i}{\hilited t_i}}^i,\, \h{y}{B}{\hilited u}
    \vdash N : C
  }{
    \overline{\h{x_i}{A_i}{\hilited (s_i \tc u) \vee t_i}}^i
    \vdash \textbf{let}~ y = M ~\textbf{in}~ N : C
  }
\end{mathpar}

\end{document}

%% And extended abstract addressing:
%% - Problem & motivation
%% - Background & related work
%% - Approach & uniqueness
%% - Results and contributions
%%
%% 2 pages, pdf. Reference lists don't count.
%% Options:
%% - acmart[sigplan]
%% - rntz, rntzgeometry[a5], pdfbook
%%   hm, need custom configuration...

\documentclass[sigplan,screen,dvipsnames]{acmart}
\settopmatter{printacmref=false}
\setcopyright{none}
\renewcommand\footnotetextcopyrightpermission[1]{}
\pagestyle{plain}
\usepackage[scale=1.042]{cochineal}


%% %% ---- Formatting ----
%% \documentclass[twocolumn,10pt]{extarticle}
%% \usepackage[dvipsnames]{xcolor}
%% \usepackage[a4,width=175mm]{rntzgeometry}
%% \setlength\columnsep{20pt}
%% \setlength\parskip{0pt}

%% %% TODO: smaller section titles
%% \usepackage{titlesec}    % (sub)section header styling
%% \titleformat*{\section}{\normalfont\large\bfseries}
%% \titleformat*{\subsection}{\normalfont\normalsize\bfseries}
%% \titlespacing{\section}{0pt}{.667em plus .1em minus .1em}{.3em plus .1em}

%% \usepackage[scale=1.0355]{cochineal}
%% \usepackage[semibold,scaled=.911]{sourcesanspro}
%% %\usepackage{biolinum}
%% \usepackage[scaled=.969]{inconsolata}
%% \usepackage[small]{eulervm}

%% \usepackage[spacing=true]{microtype}
%% \frenchspacing


%% ---- Packages ----
\usepackage{amsfonts,amsmath,latexsym,stmaryrd}
\usepackage{anyfontsize}
\usepackage{lipsum}
\usepackage{mathpartir}
\usepackage{mathtools}          % \dblcolon
\usepackage[nameinlink]{cleveref}

%% ---- Commands ----
\newcommand\R{\mathbb{R}}
\newcommand\x\times
\newcommand\todo[1]{{\color{Purple}#1}}

%% ---- Top matter ----
%%
%% Should include:
%% - Student author's name & email
%% - Institutional affiliation
%% - Research advisor's name
%% - ACM student member number
%% - Category (graduate or undergraduate)
%% - Research title
\title{Type inference for monotonicity}
\author{Michael Arntzenius, PhD student}
\email{daekharel@gmail.com}
\affiliation{University of Birmingham}

% arg, should this be Neel or Dan? Neel, probably.
%\author{advised by Dan Ghica}


\begin{document}

\maketitle

\section{Problem \& motivation}

Quick: which of these are monotone in $x$?\footnote{Whenever I say ``monotone''
  I mean ``monotonically nondecreasing'': $f$ is monotone iff $x \le y \implies
  f(x) \le f(y)$.}
%
\begin{mathpar}
  x - \log x

  x + 2^x

  {-2}x

  4
\end{mathpar}

Now, how did you \emph{know}?
%
Did you prove that $x \le y$ implies
%$x + \log x \le y + \log y$?
$x + 2^x \le y + 2^y$?
%
I didn't. I used a shortcut: compositional reasoning. Addition is monotone, and
so is $2^x$; so $x + 2^x$ must be as well.
%
And when I hear ``compositional reasoning'', I think ``type system''.

I aim to give a type system that can statically guarantee a function is monotone.
%
%Moreover, I aim to be able to \emph{infer} whether functions are monotone in their arguments without requiring any annotations other than type annotations.
%
This system is flexible --- capable, for example, of typing a function $f : A \x
B \to C$ which is monotone in $A$ but not $B$.
%
It handles several flavors of function (which I call \emph{tones}): ordinary (or
\emph{invariant}), monotone, antitone, and bivariant (both mono- and anti-tone).
%
And it requires no changes to terms besides type annotations.

What use might a type system guaranteeing monotonicity of certain functions be?
%
Well, monotonicity

\begin{itemize}
\item Monotonicity for fixed points in Datafun
\item Monotonicity for eventual consistency
\item LVars????
\end{itemize}

{\color{MidnightBlue}\lipsum[4]}

\section{Background \& related work}

\newcommand\mto{\overset{+}{\to}}

This is an extension of my prior work with Neel Krishnaswami on
Datafun~\citep{datafun}, a pure and total functional language inspired by the
deductive database language Datalog~\citep{datalog}.
%
\todo{TODO. Datafun implements Datalog's recursive queries using fixed points computed by iteration}. To ensure termination, the function must be monotone.

Datafun's type system as published has two \emph{separate} function types,
ordinary $A \to B$ and monotone $A \mto B$. Thus it cannot type a function $A \x
B \to C$ that is monotone in $A$ but not $B$. This work aims to remove that
restriction.

%% Problems with Datafun as published:
%% - can't have f : A * B -> C monotone in A but not B
%% - doesn't handle antitone functions

\todo{TODO: Pfenning-Davies, Monotonicity Types, variance analysis?, that static analysis someone's pointed out}



\lipsum[2-4]

\section{Approach \& uniqueness}              % \& uniqueness

\lipsum[5]

\section{Results and contributions}

\lipsum[6-8]


%% Bibliography
\bibliographystyle{abbrvnat}
\bibliography{datafun}

\end{document}

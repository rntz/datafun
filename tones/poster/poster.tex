%% TODO:
%% - charter or palatino for body font
%% - understand layout commands
%% - figure out what goes on poster & where

% Based on the Gemini theme, https://github.com/anishathalye/gemini

\documentclass[final,dvipsnames]{beamer}

% ====================
% Packages
% ====================

\usepackage[T1]{fontenc}
%\usepackage{lmodern}
\usepackage{nccmath} %fix spacing issues. needs to load before beamerposter.

%% Paper sizes:
%\usepackage[size=a0, scale=1.25]{beamerposter}
%% 24 x 36" = 60.96 x 91.44cm
%\usepackage[size=custom, width=60.96, height=91.44, scale=1]{beamerposter}
%\usepackage[size=custom, height=60.96, width=91.44, scale=1]{beamerposter}
%% 36 x 48" = 91.44 x 121.92cm
\usepackage[size=custom, width=91.44, height=121.92, scale=1.3]{beamerposter}
%\usepackage[size=custom, height=91.44, width=121.92, scale=1.25]{beamerposter}

\usepackage{graphicx}
\usepackage{booktabs}
\usepackage{tikz}
%\usepackage{pgfplots}
\usepackage{mathpartir}
\usepackage{multirow}

%% Argh, these define commands so I have to include them.
\usetheme{gemini}
\usecolortheme{gemini}          %gemini, mit, labsix

%% \usepackage{PTSans,PTSerif}
%% \usepackage{fbb}
%% \usepackage{cochineal}
%% \usepackage{newpxtext}

%% Load newpxtext last so it gets used by eulervm.
\usepackage{librebaskerville}
\usepackage[scaled=0.95]{sourcesanspro}
\usepackage{eulervm}

\newfontfamily\SSP{SourceSansPro}
\newfontfamily\Baskerville[Scale=0.855]{LibreBaskerville}
%% \newfontfamily\Charter{XCharter}
\newfontfamily\Palatino[Scale=0.98]{TeXGyrePagellaX}
%% % consider bumping scale if I only use this for the title.
%% \newfontfamily{\PTSans}{PTSans}
%% \newfontfamily{\PTSerif}{PTSerif}
%% \newfontfamily{\FBB}{fbb}
%% \newfontfamily{\Cochineal}{Cochineal}
%% \newfontfamily{\Noto}[Scale=0.9]{Noto Sans}

\setbeamerfont{headline}{family=\Baskerville}
\setbeamerfont{block title}{family=\SSP,size=\Large}
\setbeamerfont{heading}{family=\SSP}
\setbeamerfont{block body}{family=\SSP}
%\setbeamerfont{block body}{family=\Baskerville}
%\setbeamerfont{block body}{family=\Palatino}

\definecolor{labsixorange}{RGB}{243, 111, 33}
\definecolor{mitred}{cmyk}{0.24, 1.0, 0.78, 0.17}

\setbeamercolor{headline}{fg=lightgray,bg=darkblue}
\setbeamercolor{headline rule}{bg=gray}
\setbeamercolor{block title}{fg=darkblue,bg=white}
\setbeamercolor{block alerted title}{fg=mitred,bg=lightgray}


% ====================
% Lengths
% ====================

% If you have N columns, choose \sepwidth and \colwidth such that
% (N+1)*\sepwidth + N*\colwidth = \paperwidth
\newlength{\sepwidth}
\newlength{\colwidth}
% 2 columns
\setlength{\sepwidth}{0.03\paperwidth}
\setlength{\colwidth}{0.455\paperwidth}
%% % 3 columns
%% \setlength{\sepwidth}{0.025\paperwidth}
%% \setlength{\colwidth}{0.3\paperwidth}

\newcommand{\separatorcolumn}{\begin{column}{\sepwidth}\end{column}}

% ====================
% Title
% ====================

\title{Type inference for monotonicity}
\author{Michael Arntzenius}
\institute[shortinst]{University of Birmingham}


%% ========== Commands ==========

\newcommand\rulestyle{\sffamily\scshape}
\newcommand\rulename[1]{{\rulestyle#1}}
\newcommand{\catname}[1]{\textbf{#1}}
\newcommand{\Preorder}{\catname{Preorder}}

\newcommand\N{\mathbb{N}}
\newcommand\R{\mathbb{R}}
\newcommand\x\times
\newcommand\todo[1]{{\color{Red}#1}}
\newcommand\G\Gamma
\newcommand\D\Delta
\newcommand\mto{\overset{+}{\to}}

\newcommand{\opcolor}{\color{ForestGreen}}
\newcommand{\isocolor}{\color{NavyBlue}}
\newcommand{\pathcolor}{\color{Bittersweet}}

\newcommand{\id}{\mathrm{id}}
\newcommand{\op}{\mathrm{\opcolor op}}
\newcommand{\iso}{{\texorpdfstring{\ensuremath{\isocolor\Box}}{iso}}}
\renewcommand{\path}{{\texorpdfstring{\ensuremath{\pathcolor\lozenge}}{path}}}

\newcommand{\idof}{\id\,}
\newcommand{\opof}{\op\,}
\newcommand{\isof}{\iso}
\newcommand{\pathof}{\path}

%% TODO: remove these.
\newcommand{\cid}{\id}
\newcommand{\cop}{{\opcolor\op}}
\newcommand{\ciso}{{\isocolor\iso}}
\newcommand{\cpath}{{\pathcolor\path}}

\newcommand\subtype{\mathrel{<:}}
\newcommand\fname[1]{\ensuremath{\mathrm{#1}}}
\newcommand\fn\lambda
\newcommand\fnof[1]{\fn{#1}.~}
\newcommand\kw[1]{\fname{#1}}

\newcommand\toiso{\overset\ciso\to}
\newcommand\toid{\overset\cid\to}
\newcommand\toop{\overset\cop\to}


% ====================
% Body
% ====================

\begin{document}

\begin{frame}[t]
\begin{columns}[t]
\separatorcolumn



\begin{column}{\colwidth}

  \begin{block}{Problem: Ensuring functions are monotone}

    \todo{idunno}

  \end{block}

  \begin{block}{Modes of (non-)monotonicity}
    I consider four \strong{modes}, ways a function may respect the order on its domain:
    \begin{itemize}
    \item $\cid$ is monotone, or order-preserving. For example, $(\fnof{x} x)$.
    \item $\cop$ is antitone, or order-inverting. For example, $\fname{not} : \fname{bool} \to \fname{bool}$.
    \item $\ciso$ is equivariant, preserving only equivalence. Usually, \emph{all} functions are equivariant.
    \item $\cpath$ is bivariant, or both mono- and antitone. The constant
      function $\fnof{x} 42$, for example.
    \end{itemize}

    Formally, modes alter \strong{preorderings} (reflexive, transitive
    relations), as shown in Figure~\ref{fig:mode-ops}. We say $f : A \to B$ has
    mode $T$ iff $f$ is monotone from $TA \to B$.

    %% TODO: diagrams for what each mode does to a specific ordering, like Kevin
    %% Clancy's. Maybe make sure that we have a fencepost-like example, for path?
  \end{block}
  
  \begin{block}{Approach 1: Annotate the arrows}
    We can annotate function types with their mode: $A \toid B$ for monotone
    maps; $A \toop B$ for antitone maps; $A \toiso B$ for arbitrary maps. This
    suffices for simple functions:

    \[
      \begin{array}{l}
        \fname{setMap} ~:~
        (A \toiso B) \toiso \fname{Set}~A \toid \fname{Set}~B\\
        \fname{setMap}~f~xs ~=~ \kw{do}~x \leftarrow xs\\
        \phantom{\fname{setMap}~f~xs ~=~ \kw{do}~}\fname{return}~(f\;x)
      \end{array}
    \]

    But it cannot capture more complex input-output ordering relationships:

    \[
      \begin{array}{l}
        \fname{subtractEach} ~:~
        \fname{List}~(\N \x {\opcolor \N}) \overset{???}{\to} \fname{List}~\N
        \\
        \fname{subtractEach}~ xs ~=~ \fname{map}~(\fn (x,y).~ x-y) ~xs
      \end{array}
    \]

    Annotating the arrow cannot indicate this function is \strong{monotone} in
    the first half of each $(\N \x \N)$ pair, but \strong{\opcolor antitone} in
    the second.

    %% TODO: give subtractEach a red/yellow background \& setMap a green one.
    %% or, use green/red marker to put a check mark next to setMap & an X next
    %% to subtractEach.
    %%
    %% TODO: cite Datafun and variance typing as examples of this approach.
  \end{block}

  \begin{block}{Approach 2: Modal types}
    Alternatively, we can make \strong{all} functions monotone, and apply modes
    directly to types. For example, $\opof A$ is $A$ with its ordering inverted.
    The \strong{\opcolor antitone} maps $(A \toop B)$ are exactly the
    \strong{monotone} maps $(\opof A \to B)$. Now we can give a precise type to
    \fname{subtractEach}!

    \[ \fname{subtractEach} ~:~ \fname{List}~(\N \x {\opof \N}) \to \fname{List}~\N \]

    Now that modes are type constructors, they have \strong{explicit intro and
      elim forms} (Pfenning \& Davies, 2001). This clutters the definition of
    simple functions like \fname{setMap}:

    \[
      \begin{array}{l}
        \fname{setMap} ~:~
        \iso(\iso A \to B) \to \fname{Set}~A \to \fname{Set}~B\\
        \fname{setMap}~f~xs =\\
        \quad\kw{let}~\kw{box}~g = f ~\kw{in}\\
        \quad
        \kw{do}~x \leftarrow xs\\
        \quad\phantom{\kw{do}~}\kw{let}~\kw{box}~y = x\\
        \quad\phantom{\kw{do}~}\fname{return}~(\kw{box}~(f~(\kw{box}~y)))
      \end{array}
    \]

    %% TODO: cite Pfenning-Davies in bibliography.

    %% TODO: show the Pf-D typing rules? explain Set A = Downset(\iso A)?
  \end{block}

  \begin{block}{Our approach: Modal subtyping!}
    \strong{Goal:} Handle functions which are monotone in only part of their
    input \emph{without} clunky term annotations.

    \strong{Method:} Construct and eliminate modal types \emph{implicitly} via \emph{subtyping}.

    \todo{TODO: challenges? 1. how do you \emph{introduce} a box by subtyping?\\
    2. function application.\\
    3. pattern-matching (probably leave out).}
  \end{block}

\end{column}

\separatorcolumn


\begin{column}{\colwidth}
    \begin{figure}
      \vspace{-1em}
      \begin{mathpar}
        \begin{array}{llc}
        a \le b : \idof A &\iff& a \le b : A\\
        a \le b : \opof A &\iff& a \ge b : A\\
        a \le b : \isof A &\iff& a \le b \wedge a \ge b : A\\
        a \le b : \pathof A &\impliedby& a \le b \vee b \le a : A
        \end{array}

        \begin{tikzpicture}[scale=2.5,baseline=(current bounding box.center)]
          \node (top)  at ( 0, 1) {$\cpath$};
          \node (bot)  at ( 0,-1) {$\ciso$};
          \node (-1)   at (-1, 0) {$\cid$};
          \node (1)    at ( 1, 0) {$\cop$};
          \draw (top) -- (-1) -- (bot) -- (1) -- (top);
        \end{tikzpicture}

        \begin{array}{cr|cccc}
          \multicolumn{2}{c|}{\multirow{2}{*}{$UT$}}
          & \multicolumn{4}{c}{T}\\
          && \cid & \cop & \ciso & \cpath\\\hline
          \multirow{4}{*}{$U$}
          & \cid & \cid & \cop & \ciso & \cpath\\
          & \cop & \cop & \cid & \ciso & \cpath\\
          & \ciso & \ciso & \ciso & \ciso & \cpath\\
          & \cpath & \cpath & \cpath & \ciso & \cpath
        \end{array}
      \end{mathpar}

      \caption{\ Modes, the mode lattice, and mode composition}
      \label{fig:mode-ops}
    \end{figure}

  \begin{block}{Problem: Function application is ambiguous!}
    Consider applying a function $f : \iso(A \to B)$. Which of these two valid
    subtypings should we apply to determine the expected argument and return
    types?

    \begin{align*}
      \iso(A \to B) &\subtype A \to B & \text{because $\iso C \subtype C$ for any $C$}\\
      \iso(A \to B) &\subtype \iso A \to \iso B
      & \text{by functoriality of $\iso$}
    \end{align*}

    Neither type is more specific, and both usages occur in practice.
  \end{block}

  \begin{block}{Future work}
    \heading{Is modal subtyping useful elsewhere?}

    \heading{Hybridize with SMT-based approaches}
    \todo{LiquidHaskell and other lightweight verification techniques. Probably
      necessary for complex code, but can we verify simple code without
      appealing to an SMT solver? Future work: a hybrid approach.}
  \end{block}

\end{column}

\separatorcolumn


\begin{column}{\colwidth}
  \begin{block}{wat}
    testing
  \end{block}

  %% \begin{block}{A block containing some math}

  %%   Nullam non est elit. In eu ornare justo. Maecenas porttitor sodales lacus,
  %%   ut cursus augue sodales ac.

  %%   $$
  %%   \int_{-\infty}^{\infty} e^{-x^2}\,dx = \sqrt{\pi}
  %%   $$

  %%   Interdum et malesuada fames $\{1, 4, 9, \ldots\}$ ac ante ipsum primis in
  %%   faucibus. Cras eleifend dolor eu nulla suscipit suscipit. Sed lobortis non
  %%   felis id vulputate.

  %%   \heading{A heading inside a block}

  %%   Praesent consectetur mi $x^2 + y^2$ metus, nec vestibulum justo viverra
  %%   nec. Proin eget nulla pretium, egestas magna aliquam, mollis neque. Vivamus
  %%   dictum $\mathbf{u}^\intercal\mathbf{v}$ sagittis odio, vel porta erat
  %%   congue sed. Maecenas ut dolor quis arcu auctor porttitor.

  %%   \heading{Another heading inside a block}

  %%   Sed augue erat, scelerisque a purus ultricies, placerat porttitor neque.
  %%   Donec $P(y \mid x)$ fermentum consectetur $\nabla_x P(y \mid x)$ sapien
  %%   sagittis egestas. Duis eget leo euismod nunc viverra imperdiet nec id
  %%   justo.

  %% \end{block}

  %% \begin{block}{Nullam vel erat at velit convallis laoreet}

  %%   Class aptent taciti sociosqu ad litora torquent per conubia nostra, per
  %%   inceptos himenaeos. Phasellus libero enim, gravida sed erat sit amet,
  %%   scelerisque congue diam. Fusce dapibus dui ut augue pulvinar iaculis.

  %%   \begin{table}
  %%     \centering
  %%     \begin{tabular}{l r r c}
  %%       \toprule
  %%       \textbf{First column} & \textbf{Second column} & \textbf{Third column} & \textbf{Fourth} \\
  %%       \midrule
  %%       Foo & 13.37 & 384,394 & \alpha \\
  %%       Bar & 2.17 & 1,392 & \beta \\
  %%       Baz & 3.14 & 83,742 & \delta \\
  %%       Qux & 7.59 & 974 & \gamma \\
  %%       \bottomrule
  %%     \end{tabular}
  %%     \caption{A table caption.}
  %%   \end{table}

  %%   Donec quis posuere ligula. Nunc feugiat elit a mi malesuada consequat. Sed
  %%   imperdiet augue ac nibh aliquet tristique. Aenean eu tortor vulputate,
  %%   eleifend lorem in, dictum urna. Proin auctor ante in augue tincidunt
  %%   tempor. Proin pellentesque vulputate odio, ac gravida nulla posuere
  %%   efficitur. Aenean at velit vel dolor blandit molestie. Mauris laoreet
  %%   commodo quam, non luctus nibh ullamcorper in. Class aptent taciti sociosqu
  %%   ad litora torquent per conubia nostra, per inceptos himenaeos.

  %%   Nulla varius finibus volutpat. Mauris molestie lorem tincidunt, iaculis
  %%   libero at, gravida ante. Phasellus at felis eu neque suscipit suscipit.
  %%   Integer ullamcorper, dui nec pretium ornare, urna dolor consequat libero,
  %%   in feugiat elit lorem euismod lacus. Pellentesque sit amet dolor mollis,
  %%   auctor urna non, tempus sem.

  %% \end{block}

  %% \begin{block}{References}
  %%   \nocite{*}
  %%   \footnotesize{\bibliographystyle{plain}\bibliography{poster}}
  %% \end{block}

\end{column}

\separatorcolumn
\end{columns}
\end{frame}

\end{document}

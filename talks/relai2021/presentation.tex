%% TODO: aspect ratio
\documentclass[aspectratio=141,dvipsnames]{beamer}

%% \usepackage{amssymb,amsmath,amsthm,latexsym}
%% \usepackage{url,hyperref}
\usepackage{multirow}           % \multirow, \multicolumn
\usepackage{nccmath}           % fleqn environment
\usepackage{mathpartir}        % \mathpar, \infer
\usepackage{booktabs}           % \midrule
%% \usepackage{colonequals}
\usepackage{tikz,tikz-cd}      % Hasse & commutative diagrams.
\usepackage{accents}           % \underaccent
\usepackage{pdfpages}           % \includepdf
\usepackage{mathtools}          % \dblcolon, vertically centered colon
\usepackage{pgfplots}\pgfplotsset{compat=1.5} % the perf graph
\usepackage{censor}             % \censor
\usepackage{array}              % alignment options in tabular
\usepackage{pbox}               % \pbox
\usepackage[normalem]{ulem}     % \sout

\mathtoolsset{centercolon} % not sure this works w/ Euler
\usetikzlibrary{calc}

\usepackage{iftex}
\ifPDFTeX\PassOptionsToPackage{spacing=true,tracking=true,letterspace=40,}{microtype}\else\fi
\usepackage{microtype}
\frenchspacing


%% Format fiddling.
\linespread{1.0608}
\setlength\arraycolsep{.2em}
\def\arraystretch{1.05}

\let\olditemize\itemize
\renewcommand\itemize{\olditemize\setlength\itemsep{1ex}}

\colorlet{standout}{Orange!10!Red!80!black}
\newcommand\standout{\color{standout}}

\setbeamertemplate{navigation symbols}{}
%% TODO: figure out how to get this to work
\colorlet{bgcolor}{yellow!6!white}
\setbeamercolor{background canvas}{bg=bgcolor}

\setbeamertemplate{footline}[frame number]
\setbeamerfont{footline}{size=\small,}
\setbeamerfont{frametitle}{shape=\scshape}
\setbeamercolor{frametitle}{fg=standout}
%\setbeamertemplate{footline}{\hfil\insertpagenumber\vspace{2ex}}
%% \setbeamerfont{headline}{size=\scriptsize}
%% \setbeamertemplate{headline}{\vspace{2ex}\hfill\insertpagenumber\hspace*{2ex}}


%% font fiddling
\usepackage[LY1,OT1,T1]{fontenc}
\ifPDFTeX\else\usepackage[no-math]{fontspec}\fi
\usefonttheme{professionalfonts}
\renewcommand{\familydefault}{\rmdefault}

%\usepackage[scaled=0.9543]{XCharter}
%\usepackage[llscale=1.0699300699300702,scale=1.0699300699300702,p,mono=false]{libertine}
%% \usepackage[semibold,scaled=0.9663]{sourceserifpro}
%% \usepackage[osf,semibold,scaled=0.9663]{sourcesanspro}
%% \usepackage[scaled=1.00438,varqu,var0]{inconsolata}

\usepackage{fontchoice}
\providecommand\displayfamily\rmfamily
\setmonofont{inconsolata}[Scale=MatchLowercase,Scale=1.00438,]

\usepackage[euler-digits]{eulervm}
\usepackage[bb=boondox]{mathalfa} % or bb=ams
%\usepackage[euler-digits]{eulervm}


%% ===== COMMANDS =====
\providecommand\strong[1]{{\bfseries#1}}
\newcommand\textsfit[1]{\textsf{\itshape#1}}

\newcommand\conj{,\,}
\newcommand\N{\mathbb{N}}
\newcommand\x\times
\providecommand\G{}\renewcommand\G\Gamma
\newcommand\D\Delta
\newcommand\fn{\ensuremath{\lambda}}
\newcommand\isa{\hspace{.1em}:\hspace{.1em}}
\newcommand\fnof[1]{\fn{#1}.\;}
\newcommand\fa[1]{(\forall {#1})\ }
\newcommand\iso{{\texorpdfstring{\ensuremath{\square}}{box}}}
\newcommand\desugars{\,\xrightarrow{\textsf{desugars}}\,}

\newcommand{\setfor}[2]{\{{#1} \mathrel{|} {#2}\}}

\newcommand\kw\textbf
\newcommand\n\textit
\newcommand\tpname\text
%\newcommand\kw[1]{\textsfit{#1}}\newcommand\n\textsfit\newcommand\tpname\textsf
\renewcommand\c\textsf

\newcommand\tset{\tpname{set}\,}
\newcommand\tmap[2]{\tpname{map}\,(#1,#2)}
\newcommand\tbool{\tpname{bool}}
\newcommand\tunit{\ensuremath{1}}
\newcommand\tnode{\tpname{node}}
%\newcommand\mto{\overset{\textsf{\Large +}}{\to}}
\newcommand\mto{\overset{\boldsymbol+}{\to}}
\newcommand\mvar[1]{{\mvarcolor #1}}
\newcommand\mvarcolor{\color{purple}}

%\newcommand\efor[1]{\kw{for}\hspace{.3em} {#1} \mathrel{\kw{join}} }
\newcommand\eforloop[1]{\kw{for}\hspace{.33em} {#1} \hspace{.33em}}
\newcommand\eforjoin{\kw{join}\hspace{.33em}}
\newcommand\eforwhen[1]{\kw{when}\hspace{.33em} {#1} \hspace{.33em}}
\newcommand\efor[1]{\eforloop{#1} \eforjoin}
\newcommand\eforvar[2]{\efor{{#1} \in {#2}}}
\newcommand\ewhen[1]{\kw{when}~({#1})~}
\newcommand\efixis[1]{\kw{fix}~{#1}~\kw{is}\;}
%% \renewcommand\efixis[1]{\kw{solve}~{#1} ~\kw{is}~}
%% \renewcommand\efixis[1]{\kw{solve}~{#1} =}
\newcommand\eset[1]{\{{#1}\}}
\newcommand\esetfor[2]{\eset{{#1} ~|~ {#2}}}
\newcommand\elet[1]{\kw{let}~{#1}~\kw{in}~}
\newcommand\eletfix[2]{\kw{let rec}~ {#1} = {#2}}
\newcommand\eletfixin[2]{\eletfix{#1}{#2} ~\kw{in}~}
\newcommand\zero{\ensuremath{\boldsymbol 0}}
\newcommand\join{\sqcup}
\newcommand\efalse{\text{false}}
\newcommand\etrue{\text{true}}

\newcommand\naive{na\"ive}
\newcommand\Naive{Na\"ive}

\let\oldcup\cup
\let\oldsqcup\sqcup
\renewcommand\cup{\mathrel{\oldcup}}
\renewcommand\sqcup{\mathrel{\oldsqcup}}

\newcommand\hilite{\color{Rhodamine}}
\newcommand\hi[1]{{\hilite#1}}
%\newcommand\hilitetime{\color{Orange}}\newcommand\hiti[1]{{\hilitetime#1}}

\newcommand\ensuretext[1]{\ifmmode\text{#1}\else{#1}\fi}
\newcommand\todocolor{\color{OrangeRed}}
\newcommand\todo[1]{\PackageWarning{fixme}{#1}{\todocolor\ensuretext{\bfseries\sffamily[{#1}]}}}
\newcommand\XXX{\todo{XXX}}


\title{Semi\naive\ evaluation in\\ Datafun and beyond}
%\subtitle{bottom-up logic \,$\color{Gray}\bowtie$\, \color{black}higher-order functions}
\author{michael arntzenius}
\date{December 2021}

\newcommand\interlude{\Huge\standout\displayfamily}


\begin{document}
  \Large

  \begin{frame}[plain,noframenumbering]
    \displayfamily
    \centering
    \vfill
    {\standout\Huge\itshape\inserttitle\par}\vspace{3ex}
    {\scshape michael arntzenius}\par\vspace{1ex}
    {\addfontfeatures{Numbers=Tabular}2021–12–10}
    \vfill
  \end{frame}


  \begin{frame}
    \begin{enumerate}
    \setlength\itemsep{3ex}
    %% \setlength\itemsep{\baselineskip}
    \item \textbf{Monotone fixed points}
      \\[1ex]
      
      {\small \emph{Datafun: A Functional Datalog}\\
        {Arntzenius and Krishnaswami}, \scshape icfp 2016 \par}
      
    \item \textbf{Incremental} monotone fixed points
      \\[1ex]

      {\small \emph{Semi\naive\ Evaluation for a Higher-Order Functional
          Language}\\
        {Arntzenius and Krishnaswami}, \scshape popl 2020 \par}
      
    \item \textbf{Concurrent} incremental monotone fixed points

      {\small \itshape work-in-progress}
    \end{enumerate}
  \end{frame}


  \begin{frame}
    \textsc{pop quiz}\\ How can you compute each of the following?
    \vspace{1ex}
    \begin{itemize}
    \item Graph reachability
    \item Regular expression matching
    \item Abstract interpretation
    \end{itemize}

    \vspace{1ex}

    \textsc{answer}\\ \itshape Iterate a monotone map to its fixed point.

    \vspace{\baselineskip}

%% They're all expressible as, and computable by, iterating a
%%     monotone map until it reaches a fixed point.
  \end{frame}


  \begin{frame}{}
    \centering \strong{Datalog!} 
    \begin{fleqn}[0em]
    \begin{minipage}[t][.35\paperheight]{\paperwidth}
      \[\begin{array}{l}
      \n{reach}(\c{start}).\\
      \n{reach}(Y) \gets \n{reach}(X)\conj \n{edge}(X,Y).\\[1ex]\pause
      \n{reach2}(\c{start}).\\
      \n{reach2}(Y) \gets \n{reach2}(X)\conj \n{edge2}(X,Y).\\[1ex]
      \n{reach3}(\c{start}).\\
      \n{reach3}(Y) \gets \n{reach3}(X)\conj \n{isnt-this-tedious}(X,Y).
      \\[1ex]
      \n{reach4}(\c{start}).\\
      \n{reach4}(Y) \gets \n{reach4}(X)\conj \n{yes-it-is-rather}(X,Y).
      \\[1ex]
      \n{reach5}(\c{start}).\\
      \n{reach5}(Y) \gets \n{reach5}(X)\conj \n{yes-it-is-rather}(X,Y).\\
      \end{array}\]
    \end{minipage}
    \end{fleqn}
  \end{frame}


  \begin{frame}{}{}
    {\centering\strong{\hspace{.1em}Datalog?}\par}
    \begin{fleqn}[0em]
      \[
      \begin{array}{l}
        \n{distance}(\c{start}, 0).\\
        \n{distance}(Y, D+L) \gets \n{distance}(X, D) \conj \n{edge}(X,Y,L).
      \end{array}
      \]
    \end{fleqn}
    \pause

    This finds \emph{all} distances to each node, not the shortest!
    \vspace{1.75\baselineskip}
  \end{frame}

  \begin{frame}
    \LARGE
    \begin{center}
      \begin{tabular}{@{}lcl@{}}
        \scshape\color{Blue} datalog && \scshape\color{Blue} datafun
        \\[.5ex]
        \color{Red!80!black} repeat yourself &$\to$&
        \color{Green!80!black} functions
        \\
        \color{Red!80!black} sets only &$\to$&
        \color{Green!80!black} semilattice types
      \end{tabular}
    \end{center}
  \end{frame}

  %% \begin{frame}
  %%   \LARGE
  %%   \begin{center}
  %%     \setlength\tabcolsep{1em}
  %%     \begin{tabular}{@{}lcc@{}}
  %%       \scshape\color{Blue} datalog
  %%       & \color{Red!80!black} repeat yourself
  %%       & \color{Red!80!black} sets only
  %%       \pause
  %%       \\
  %%       & $\downarrow$ & $\downarrow$
  %%       \\
  %%       \scshape\color{Blue} datafun
  %%       & \color{Green!80!black} functions
  %%       & \color{Green!80!black} semilattice types
  %%     \end{tabular}
  %%   \end{center}
  %% \end{frame}


  \newcommand\nodecolor[1]{{\alt<2->{\color{ForestGreen}}{}#1}}
  \newcommand\edgecolor[1]{{\alt<2->{\color{RoyalBlue}}{}#1}}

  \begin{frame}
    \centering
    \begin{fleqn}
      \textsc{datalog}
      \[\begin{array}{l}
      \n{reach}(\c{start}).\\
      \n{reach}(Y) \gets \n{reach}(X)\conj \n{edge}(X,Y).
      \end{array}
      \]
      \vspace{0pt}\pause

      \textsc{datafun}
      \[
      \begin{array}{l}
      \n{reachable} \isa
      \nodecolor{\tnode} \to
      \edgecolor{\tset{(\tnode \x \tnode)}}
      \to {\alt<3->{\mvarcolor}{}\tset \tnode}\\
      %% \n{reachable} \;\n{start} \;\n{edge} = \pause
      \n{reachable}
      \alt<2>{\hphantom{{}\isa{}}}{\;} \nodecolor{\n{start}}
      \alt<2>{\hphantom{i\to{}}}{\;} \edgecolor{\n{edge}}
      = \pause
      \\\quad
      %% \eletfix{\mvar R}{\uncover<4->{\eset{\n{start}} \cup
      %%   \esetfor{y}{x \in \mvar R, (x,y) \in \n{edge}}}}
      %% \\\quad
      %% \kw{in}~\mvar R
      \efixis{\mvar R}
      \\\qquad\pause
      \eset{\n{start}}
      \cup
      \esetfor{y}{x \in \mvar R, (x,y) \in \n{edge}}
      \end{array}\]
    \end{fleqn}
  \end{frame}


  %% TODO: connect this up somehow, either arrows or hilighting in different
  %% colors.
  %\newcommand\setcolor[1]{{\alt<3->{\color{ForestGreen}}{}#1}}
  \newcommand\functioncolor[1]{{\alt<2->{\color{ForestGreen}}{}#1}}
  \newcommand\fixptcolor[1]{{\alt<3->{\mvarcolor}{}#1}}
  \newcommand\setcolor[1]{{\alt<4->{\color{RoyalBlue}}{}#1}}

  \begin{frame}
    \[
      \begin{array}{l}
      \functioncolor{\n{reachable} \;\n{start} \;\n{edge} = }
      \\\quad
      \fixptcolor{\efixis{R}}
      \\\qquad
      \setcolor{\eset{\n{start}} \cup
        \esetfor{y}{x \in {\color<5->{purple} R}, (x,y) \in \n{edge}}}
    \end{array}\]

    \pause
    \strong{Datafun}\textsuperscript{\sffamily\scshape[icfp 2016, popl 2020]}
    is:
    \begin{itemize}\setlength\itemsep{.5ex}
    \item a \functioncolor{pure functional language} \pause
    \item with \fixptcolor{monotone fixed points} \pause
    \item and \setcolor{sets \& set/semilattice comprehensions} \pause
    \item where \emph{the type system tracks monotonicity}.
    \end{itemize}
    \vspace{\baselineskip}

    %% \strong{Datafun}\textsuperscript{\sffamily\scshape[icfp 2016]} is a
    %% simply-typed \fn-calculus where \emph{types are posets} and \emph{all functions are
    %% monotone},
    %% %
    %% extended with\vspace{1ex}
    %% \begin{itemize}
    %% \item a finite set datatype \& set comprehensions;
    %% \item a bottom-up monotone fixed point operator;
    %% \item and a \emph{discreteness} comonad, $\iso$.
    %% \end{itemize}

  \end{frame}


  \newcommand\discolor{\color{RoyalBlue}}

  \begin{frame}{monotonicity types}
    \centering

    \setlength\tabcolsep{1em}
    \begin{tabular}{clc}
      \textbf{Type} & \textbf{Meaning} & \textbf{Ordering} %& \textbf{Join}
      \\\midrule
%        $\N$ & naturals & $0 < 1 < 2 < \hdots$ \\
      $\tbool$ & booleans & $\efalse < \etrue$ %& logical or
      \\
      $\tset A$      & finite subsets of $A$ & ${\subseteq}$ %& union
      \\
%        $\tmap A B$ & finite maps from $A$ to $B$ & \multicolumn{1}{c}{pointwise}\\
      $A \to B$   & functions & \multicolumn{1}{c}{pointwise}\\
      $A \mto B$  & \textbf{monotone} functions & \multicolumn{1}{c}{pointwise}
    \end{tabular}

    \pause
    \[\begin{array}{l}
      \n{member} \isa {\discolor \N \to{}} {\mvarcolor\tset{\N} \mto{}} \tbool\\
      \n{member}\; {\discolor x}\; \mvar{S} = \exists({\discolor y} \in \mvar{S})\;
      {\discolor x} = {\discolor y}
    \end{array}\]

    \pause
    \pbox{.8\textwidth}{
    variables are \emph{discrete} {\discolor $x,y,z$}\\
    \phantom{variables are }\hspace{-2.28em}or \emph{monotone}
    {\mvarcolor $X,Y,Z$}
    }

  \end{frame}


  %% TODO: reduce number of examples I'm using here.
  %% start with exists query
  %% then do set comprehensions
  \begin{frame}{semilattice comprehensions}
    How does this expression \emph{run}?

    \begin{fleqn}
      \[\def\arraystretch{1.5}
      \setlength\arraycolsep{.4em}
      \begin{array}{rl}
        & \exists (y \in \eset{1,2})\; x = y
        \\\pause
        \desugars& \eforvar{y}{\eset{1,2}} x = y
        \\\pause
        \text{\small which is a big lattice join:}
        &\displaystyle \bigsqcup_{y \in \eset{1,2}} (x = y)
        \\\pause
        \text{\small i.e. boolean disjunction:}
        %% &\displaystyle \bigvee_{y \in \eset{1,2}} (x = y)
        %% \\\pause =
        & (x = 1) \vee (x = 2)
      \end{array}\]
    \end{fleqn}

  \end{frame}

  \begin{frame}{semilattice comprehensions}

    %% \large
    \begin{fleqn}
      \[
      \begin{array}{rcl}
        \exists (y \in S)\; x = y
        &\desugars&
        \eforvar{y}{S} x = y
        \\[3ex]
        \setfor{y}{x \in R, (x,y) \in \n{edge}}
        &\desugars&
        \eforloop{x_1 \in R}
        \\
        &&\quad \eforloop{(x_2,y) \in \n{edge}}
        \\
        &&\qquad
        \eforwhen{x_1 = x_2} \eforjoin \eset{y}
        \\[2ex]
      \end{array}
      \]
    \end{fleqn}

    %% \pause
    %% All of relational algebra is derivable given $(\eforloop{x \in S} \eforjoin e)$!

  \end{frame}


  \newcommand\tdist{\tpname{dist}}
  
  \begin{frame}{fixed points}
    \setlength\tabcolsep{1em}
    \large
    \begin{tabular}{@{}clc@{}}
      \textbf{Type} & \textbf{Meaning} & \textbf{Ordering} %& \textbf{Join}
      \\\midrule
      $\tset A$      & finite subsets of $A$ & ${\subseteq}$ %& union
      \\
      $\tmap A B$ & dictionaries w/ $A$ keys and $B$ values
      & pointwise
      \\
      $\tdist$ & distances under minimum
      & $\infty < ... < 1 < 0$
    \end{tabular}

    \vspace{\baselineskip}
    \begin{fleqn}
      \[\begin{array}{l}
      \n{shortest} \isa \tnode \to \tset{(\tnode \x \tnode \x \tpname{dist})}
      \to \tmap{\tnode}{\tpname{dist}}
      \\
      \n{shortest} \;\n{start} \;\n{edge} =
      \\\quad
      \efixis{\mvar R} \pause
      \\\qquad
      \phantom{\sqcup{}}
      \eset{\n{start} \mapsto \text{\liningnums{0}}}
      %\sqcup {}
      \\
      \qquad \sqcup
      (\eforloop{(x \mapsto \mvar{D}) \in \mvar R}\\
      \qquad\phantom{\sqcup (} \quad \eforloop{(x,y,l) \in \n{edge}}\\
      \qquad\phantom{\sqcup (} \qquad \eforjoin \eset{y \mapsto \mvar{D} + l})
      \end{array}
      \]
    \end{fleqn}

  \end{frame}


  \begin{frame}

    \begin{center}
      \begin{tabular}{@{}lcl@{}}
        \scshape\color{Blue} datalog && \scshape\color{Blue} datafun
        \\[.5ex]
        \color{gray} repeat yourself &&
        \color{gray} functions
        \\
        \color{gray} sets only &&
        \color{gray} semilattice types
        \\
        stratified recursion
        & $\to$
        & monotone fixed points
        \\
        conjunctive queries
        & $\to$
        & set/semilattice comprehensions
      \end{tabular}
    \end{center}

    %% \begin{center}
    %%   \setlength\tabcolsep{1em}
    %%   \begin{tabular}{@{}lcc@{}}
    %%     \scshape\color{Blue} datalog
    %%     & \color{Red!80!black} repeat yourself
    %%     & \color{Red!80!black} sets only
    %%     \\
    %%     & $\downarrow$ & $\downarrow$
    %%     \\
    %%     \scshape\color{Blue} datafun
    %%     & \color{Green!80!black} functions
    %%     & \color{Green!80!black} semilattice types
    %%   \end{tabular}
    %% \end{center}

    %% \vspace{3ex}

    %% To make Datalog into a higher-order functional language:

    %% \begin{enumerate}
    %% \item Handle \emph{fixed points} by tracking monotonicity with types.
    %% \item Generalize \emph{relational algebra} using semilattice comprehensions.
    %% \end{enumerate}

  \end{frame}


  \begin{frame}
    \interlude
    \begin{center}
      {\upshape\scshape ii}
      \\
      \emph{incremental}
      \\
      monotone fixed points
      %% \\
      %% (or, \emph{semi\naive\ evaluation})
    \end{center}
  \end{frame}


  \begin{frame}
    \[\n{step}\;R =
    {\color<-2>{RoyalBlue}\{\n{start}\}}
    \cup
    {\color<-2>{Red}\setfor{y}{x \in R, (x,y) \in \n{edge}}}
    \]

    \pause\begin{align*}
      R_0 &= \emptyset
      & \uncover<3->{dR_0} &\uncover<3->{= }\uncover<4->{\color<-5>{RoyalBlue}\{\n{start}\}}
      \\
      R_{i+1} &= \makebox[4.3em][l]{$\alt<6->{\hi{R_i \cup dR_i}}{\n{step}\;R_i}$}
      & \uncover<3->{dR_{i+1}} &\uncover<3->{= }\uncover<5->{\color<-5>{Red} \setfor{y}{x \in dR_i, (x,y) \in \n{edge}}}
    \end{align*}

    \centering\alt<3->{
      \includegraphics[height=45mm]{imgs/focus-1240.jpg}
    }{
      \includegraphics[height=45mm]{imgs/focus-0946.jpg}
    }
    \vspace{1ex}
  \end{frame}


  \begin{frame}{}{}\LARGE\centering
    \setlength\parskip{2ex}

    {\scshape semi\naive\ evaluation}

    {means}

    {computing the \strong{change} between iterations}
  \end{frame}


  \newcommand\chgto\rightsquigarrow
  \newcommand\chgat[4]{{\ensuremath{{#2} \dblcolon_{#1} {#3} \chgto {#4}}}}
  \newcommand\chg[3]{{\ensuremath{{#1} \dblcolon {#2} \chgto {#3}}}}

  \begin{frame}{incremental $\lambda$-calculus \textsuperscript{\normalshape\sffamily[%
          Cai et al \textsc{pldi 2014},
          Giarrusso's thesis \textsc{2017}]}}
    %% \large

    Each type $A$ has a type $\standout\Delta A$ of \emph{increasing changes} and a relation $\standout\chgat{A}{dx} x y$ meaning ``$dx : \Delta A$ changes $x : A$ into $y : A$''.
    %
    \begin{mathpar}
      \Delta(\tset{A}) = \tset{A}

      \chgat{\tset A}{dx}{x}{x \cup dx}
    \end{mathpar}

    \pause To incrementalize, we transform programs to \emph{push changes
    forward}, finding the change $\standout\delta(e)$ to an
    expression with respect to its free monotone variables, e.g.
    %
    \begin{mathpar}
      \delta(S \cup T) = dS \cup dT

      \delta(S \setminus t) = dS \setminus t
      \\%\pause
      \standout
      \delta(\eforvar{x}{s} e)?
    \end{mathpar}
  \end{frame}

  %% \begin{frame}%{change in datafun}
  %%   \large

  %%   Every type $A$ gets a type $\standout\Delta A$ of \emph{increasing} changes,
  %%   e.g.
  %%   %
  %%   \begin{mathpar}
  %%     \Delta(\tset{A}) = \tset{A}
  %%     %% \Delta(A \x B) = \Delta A \x \Delta B
  %%   \end{mathpar}

  %%   \pause
  %%   We formalize this via a ternary relation $\standout\chgat{A}{dx}{x}{y}$, meaning ``$dx$ changes $x$ into $y$'', e.g.

  %%   \[
  %%   \chgat{\tset A}{dx}{x}{x \cup dx}
  %%   \]

  %%   \pause
  %%   To incrementalize, we \emph{push changes through the program},
  %%   finding the change $\standout\delta(e)$ to an expression, e.g.
  %%   %
  %%   \begin{mathpar}
  %%     \delta(s \cup t) = ds \cup dt

  %%     \delta(s \setminus t) = ds \setminus t
  %%     \\%\pause
  %%     \standout
  %%     \delta(\eforvar{x}{s} e)?
  %%   \end{mathpar}
  %% \end{frame}


  \begin{frame}
    \begin{fleqn}[1em]
      \begin{align*}
        \delta(\eforvar{x}{S}{e})
        &=
        \phantom{{}\join{}}
        (\eforvar x {dS} e)
        \\
        &\phantom{{}={}} {}\join
        (\eforvar x {S \cup dS}
        \delta(e))
      \end{align*}

      \pause
      For example:

      \[
      \begin{array}{rl}
        & \delta(\eforvar{x}{S} \n{member} \;x \;T)
        \\[.5ex]\pause
        =& %\phantom{\join{}}
        (\eforvar x {dS} \n{member} \;x \;T)\\
        & \join (\eforvar x {S \cup dS}
        {\standout\delta(\n{member} \;x \;T)})
        \\[.5ex]\pause
        =& %\phantom{\join{}}
        (\eforvar x {dS} \n{member} \;x \;T)\\
        &
        \join (\eforvar x {S \cup dS}
        {\standout \n{member} \;x \;dT})
        \\[.5ex]\pause
        \approx& (dS \mathrel{\cap?} T) \vee ((S \cup dS) \mathrel{\cap?} dT)
      \end{array}
      \]
    \end{fleqn}
  \end{frame}


  \begin{frame}
    \LARGE
    \def\arraystretch{1.25}
    \[\begin{array}{rcl}
      \D(\tset A) &=& \tset A
      \\\pause
      \D(A \x B) &=& \D A \x \D B
      \\\pause
      \D(A \mto B) &=& \standout\boldsymbol ?
    \end{array}\]
  \end{frame}


  \newcommand\underexplain[4]{%
    \underaccent{%
      \parbox[t][#1]{#2}{\sffamily\centering\vspace{4pt}\small#3}
    }{#4}}

  \newcommand\chgatcov[4]{{\uncover<2->{{#2} \dblcolon_{#1}} {#3} \chgto {#4}}}
  \begin{frame}{}
    \LARGE
    \begin{mathpar}
      \D(A \mto B)
      = \underexplain{}{1.5cm}{original input}{A}
      \to \underexplain{}{1.2cm}{change to input}{\D A}
      \mto \underexplain{}{1.4cm}{change in output}{\D B}
      \vspace{1ex}
      \\\pause
      \chgat{A \to B}{df}{f}{\alt<6->{\hi f}{g}} \iff \pause
      \infer{
        \chgat{A}{dx}{x}{y}
      }{
        \uncover<4->{\chgat{B}{\uncover<5->{df\;x\;dx}}{f\;x}{\alt<6->{\hi f}{g}\;y}}
      }
    \end{mathpar}

    \setbeamercovered{invisible} % I don't know why/whether we need this.
    \vspace{3ex}
    \centering\pause\pause\pause
    In this case, we call $df$ a
    \strong{derivative} of $f$.

    \vspace{\baselineskip}
  \end{frame}


  \begin{frame}
      \begin{align*}
        \n{step} &\isa \,\tset A\, \mto \,\tset A\,
        %% & \uncover<2->{R_i} &\uncover<2->{= \n{step}^i \;\emptyset}
        \\
        \n{step}' &\isa
        \underexplain{}{2cm}{known world}{\,\tset A\,}
        \to \underexplain{}{1.8cm}{frontier}{\,\tset A\,}
        \mto \underexplain{}{1.8cm}{new frontier}{\,\tset A\,}
        %% & \uncover<3->{dR_i} &\uncover<3->{\dblcolon R_i \chgto \n{step}\;R_i}
      \end{align*}

      \textsc{strategy}
      \begin{align*}
        \underexplain{}{2cm}{known world}{{R_i}} &= \n{step}^i\;\emptyset
        &\uncover<2->{\chg{\underexplain{}{1.5cm}{frontier}{{dR_i}}}{R_i}{\n{step}\;R_i}}
      \end{align*}
      \pause\pause

      %\setbeamercovered{transparent}
      \textsc{implementation}
      \begin{align*}
        R_0 &= \emptyset %\pause
        & dR_0 &= \uncover<4->{\n{step}\;\emptyset} %\pause
        %\dblcolon \emptyset \chgto \n{step}\;\emptyset
        \\
        R_{i+1} &= \uncover<5->{R_i \cup dR_i} %\pause
        & dR_{i+1} &= \uncover<6->{\n{step}' \;R_i \;dR_i}
        %\dblcolon \n{step}\;R_i \chgto \n{step}\;(R_i \cup dR_i)
      \end{align*}
      \vspace{\baselineskip}
  \end{frame}


  \begin{frame}

    \begin{center}
      \begin{tabular}{@{}lcl@{}}
        \scshape\color{Blue} datalog && \scshape\color{Blue} datafun
        \\[.5ex]
        \color{gray} repeat yourself &&
        \color{gray} functions
        \\
        \color{gray} sets only &&
        \color{gray} semilattice types
        \\
        \color{gray} stratified recursion
        && \color{gray} monotone fixed points
        \\
        \color{gray} conjunctive queries
        && \color{gray} set/semilattice comprehensions
        \\
        semi\naive\ evaluation
        & $\to$
        & incrementalize the inner loops
      \end{tabular}
    \end{center}
    \vspace{2ex}

    {\small 
    \emph{Semi\naive\ Evaluation for a Higher-Order Functional
      Language}\\
         {Arntzenius and Krishnaswami}, \scshape popl 2020 \par}
    
  \end{frame}

  %% \begin{frame}{key ideas}
  %%   \large
  %%   \begin{center}
  %%     \setlength\tabcolsep{.67em}
  %%     \begin{tabular}{@{}lccc@{}}
  %%       \scshape\color{Blue} datalog
  %%       & \color{Red!80!black} repeat yourself
  %%       & \color{Red!80!black} sets only
  %%       & semi\naive\ evaluation
  %%       \\
  %%       & $\downarrow$ & $\downarrow$
  %%       \\
  %%       \scshape\color{Blue} datafun
  %%       & \color{Green!80!black} functions
  %%       & \color{Green!80!black} semilattice types
  %%       & incremental evaluation
  %%     \end{tabular}
  %%   \end{center}

  %%   To make Datalog into a higher-order functional language:

  %%   \begin{enumerate}
  %%   \item Handle \emph{fixed points} by tracking monotonicity with types.
  %%   \item Generalize \emph{relational algebra} using semilattice comprehensions.

  %%   \item \standout
  %%     To make fixed points efficient,  \emph{incrementalize} the inner loop by \emph{pushing changes through}.\\[1ex]

  %%   {\small\itshape (\strong{Many} details omitted here, see the {\scshape popl} 2020 paper for more.)}
  %%   \end{enumerate}

  %% \end{frame}


  \begin{frame}
    \interlude
    \begin{center}
      {\upshape\scshape iii}
      \\
      \emph{concurrent} incremental
      \\
      monotone fixed points
    \end{center}
  \end{frame}

\end{document}

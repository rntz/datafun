\documentclass{article}

\usepackage[papersize={130mm,175mm},scale=0.9]{geometry}
\pagestyle{empty}
\usepackage{mathpazo}
\linespread{1.12}
\usepackage{pgfplots}\pgfplotsset{compat=1.5}
\newcommand\naive{na\"ive}
\newcommand\Naive{Na\"ive}

\begin{document}

  %% \begin{tikzpicture}[baseline=(current bounding box.center)]
  %%   \begin{loglogaxis}[
  %%       title={\scshape transitive closure with self-loops},
  %%       xlabel={Number of nodes},
  %%       ylabel={Time (seconds)},
  %%       height=132.88pt, width=213pt, % golden
  %%       %height=140pt, width=210pt, % 2:3
  %%       %% legend entries={semi\naive,{semi\naive\ minimizing}},
  %%       %% legend entries={w/o minimizing diffs, minimizing diffs},
  %%       legend cell align=left,
  %%       legend pos = north west,
  %%       legend style={
  %%         font=\footnotesize,
  %%         draw=none,
  %%         nodes={inner sep=3pt,}
  %%       },
  %%       %% xtick = {120, 160, ..., 320},
  %%       %% ytick = {0, 60, ..., 260},
  %%     ]
  %%     \addplot [color=blue,mark=*] table [x=n,y=minimized] {trans-loop-extended.dat};
  %%     \addplot [color=red,mark=square*] table [x=n,y=normal_loopy] {trans-loop-extended.dat};
  %%     \addplot [color=black,mark=triangle*] table [x=n,y=minimized_loopy] {trans-loop-extended.dat};
  %%     \addplot [color=green,mark=x] table [x=n,y=normal] {trans-loop-extended.dat};
  %%   \end{loglogaxis}
  %% \end{tikzpicture}

  %% \clearpage


\begin{figure}[t!]
  \centering
  \begin{tikzpicture}[baseline=(current bounding box.center)]
    \begin{axis}[
        xlabel={Number of nodes},
        ylabel={Time (seconds)},
        height=140pt, width=220pt,
        % If we include the explicit table, we can make it this large without
        % enlarging the figure.
        height=172pt, width=280pt,
        legend entries={\naive,semi\naive\ raw,semi\naive\ optimized},
        legend cell align=left,
        legend pos = north west,
        legend style={
          draw=none,
          nodes={inner sep=3pt,}
        },
      ]
      \addplot [color=red,mark=square*] table [x=n,y=naive] {trans.dat};
      \addplot [color=black,mark=triangle*] table [x=n,y=seminaive_raw] {trans.dat};
      \addplot [color=blue, mark=*] table [x=n,y=seminaive] {trans.dat};
    \end{axis}
  \end{tikzpicture}
  \caption{Transitive closure on a linear graph}
\end{figure}

\begin{figure}[t!]
  \centering
  \begin{tikzpicture}[baseline=(current bounding box.center)]
    \begin{axis}[
        xlabel={Length of string},
        ylabel={Time (seconds)},
        height=140pt, width=220pt,
        % If we include the explicit table, we can make it this large without
        % enlarging the figure.
        height=172pt, width=280pt,
        legend entries={\naive,semi\naive\ raw,semi\naive\ optimized},
        legend cell align=left,
        legend pos = north west,
        legend style={
          draw=none,
          nodes={inner sep=3pt,}
        },
      ]
      \addplot [color=red, mark=square*] table [x=n,y=naive] {astar.dat};
      \addplot [color=black,mark=triangle*] table [x=n,y=seminaive_raw] {astar.dat};
      \addplot [color=blue, mark=*] table [x=n,y=seminaive] {astar.dat};
    \end{axis}
  \end{tikzpicture}
  \caption{Finding all occurrences of the regex \texttt{/a*/} in \texttt{a}\textsuperscript{\emph{n}}}
\end{figure}

\noindent
If these graphs look similar, that's no accident. Finding all matches of
\texttt{/a*/} is the same as finding the reflexive, transitive closure of the
set of index-pairs which match \texttt{/a/}; and on the string
\texttt{a}\textsuperscript{\itshape n}, the matching index-pairs of \texttt{/a/}
form a linear graph.


\begin{figure}
  \centering
  \begin{tikzpicture}[baseline=(current bounding box.center)]
    \begin{axis}[
        xlabel={Number of nodes},
        ylabel={Time (seconds)},
        height=140pt, width=220pt,
        % If we include the explicit table, we can make it this large without
        % enlarging the figure.
        height=172pt, width=280pt,
        legend entries={seminaive simple, semi\naive\ optimized},
        legend cell align=left,
        legend pos = north west,
        legend style={
          draw=none,
          nodes={inner sep=3pt,}
        },
      ]
      \addplot [color=green, mark=x] table [x=n,y=seminaive_simple] {astar.dat};
      \addplot [color=blue, mark=*] table [x=n,y=seminaive] {astar.dat};
    \end{axis}
  \end{tikzpicture}
  \caption{Matching regex \texttt{a*}}
\end{figure}


%% \clearpage
%% \begin{figure}
%%   \centering
%%   \begin{tikzpicture}[baseline=(current bounding box.center)]
%%     \begin{axis}[
%%         xlabel={Length of string},
%%         ylabel={Seconds},
%%         height=140pt, width=220pt,
%%         % If we include the explicit table, we can make it this large without
%%         % enlarging the figure.
%%         height=172pt, width=280pt,
%%         legend entries={\naive,semi\naive},
%%         legend cell align=left,
%%         legend pos = north west,
%%         legend style={
%%           draw=none,
%%           nodes={inner sep=3pt,}
%%         },
%%       ]
%%       \addplot table [x=n,y=naive] {astar2.dat};
%%       \addplot table  [x=n,y=seminaive] {astar2.dat};
%%     \end{axis}
%%   \end{tikzpicture}
%%   \caption{Finding all \emph{leading} occurrences of the regex \texttt{/a*/} in \texttt{a}\textsuperscript{\emph{n}}}
%% \end{figure}

%% \noindent By \emph{leading occurrences}, I mean occurrences which start at the
%% beginning of the string.


\end{document}

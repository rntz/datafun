\documentclass{article}

\usepackage[papersize={12cm,175mm},scale=0.9]{geometry}
\pagestyle{empty}
\usepackage{mathpazo}
\linespread{1.12}
\usepackage{pgfplots}\pgfplotsset{compat=1.5}
\newcommand\naive{na\"ive}
\newcommand\Naive{Na\"ive}

\begin{document}

\begin{figure}[t!]
  \centering
  \begin{tikzpicture}[baseline=(current bounding box.center)]
    \begin{axis}[
        xlabel={Number of nodes},
        ylabel={Seconds},
        height=140pt, width=220pt,
        % If we include the explicit table, we can make it this large without
        % enlarging the figure.
        height=172pt, width=280pt,
        legend entries={\naive,semi\naive},
        legend cell align=left,
        legend pos = north west,
        legend style={
          draw=none,
          nodes={inner sep=3pt,}
        },
      ]
      \addplot [color=red,mark=square*] table [x=n,y=naive] {trans.dat};
      \addplot [color=blue,mark=*] table  [x=n,y=seminaive] {trans.dat};
    \end{axis}
  \end{tikzpicture}
  \caption{Transitive closure on a linear graph}
\end{figure}

\begin{figure}[t!]
  \centering
  \begin{tikzpicture}[baseline=(current bounding box.center)]
    \begin{axis}[
        xlabel={Length of string},
        ylabel={Seconds},
        height=140pt, width=220pt,
        % If we include the explicit table, we can make it this large without
        % enlarging the figure.
        height=172pt, width=280pt,
        legend entries={\naive,semi\naive},
        legend cell align=left,
        legend pos = north west,
        legend style={
          draw=none,
          nodes={inner sep=3pt,}
        },
      ]
      \addplot [color=red,mark=square*] table [x=n,y=naive] {regex.dat};
      \addplot [color=blue,mark=*] table  [x=n,y=seminaive] {regex.dat};
    \end{axis}
  \end{tikzpicture}
  \caption{Finding all occurrences of the regex \texttt{/a+/} in \texttt{a}\textsuperscript{\emph{n}}}
\end{figure}

\noindent
\textbf{NB.} If these graphs look suspiciously similar, that's no accident.
Finding all matches of \texttt{/a+/} is the same as finding the \emph{transitive
  closure} of all matches of \texttt{/a/}; on the string
\texttt{a}\textsuperscript{\itshape n}, the matches of \texttt{/a/} form a
\emph{linear graph}.

\end{document}
